\levelThreeTitle{Appel de fonction}

\levelFourTitle{Formel - Calcul de plus faible précondition}

Lorsqu'une fonction est appelée, le contrat de cette fonction est utilisé pour
déterminer la précondition de l'appel, mais il est important de garder en tête deux
aspects important pour exprimer le calcul de plus faible précondition.


Premièrement, la postcondition d'une fonction $f$ qui serait appelée dans un
programme n'est pas nécessairement directement la précondition calculée pour le code
qui suit l'appel à $f$. Par exemple, si nous avons un programme :
\CodeInline{x = f() ; c }, et si $wp(\texttt{c}, Q) = 0 \leq x \leq 10$ alors que
la postcondition de la fonction \CodeInline{f} est $1 \leq x \leq 9$, nous avons
besoin d'exprimer l'affaiblissement de la précondition réelle de \CodeInline{c}
vers celle que l'on a calculé. Pour cela, nous renvoyons à la
section~\ref{l3:statements-basic-consequence}, l'idée est simplement de vérifier
que la postcondition de la fonction implique la précondition calculée.



Deuxièmement, en C, une fonction peut avoir des effets de bord. Par conséquent, les
valeurs référencées en entrée de fonction ne restent pas nécessairement les mêmes
après l'appel à la fonction, et le contrat devrait exprimer certaines propriétés à
propos des valeurs avant et après l'appel. Donc, si nous avons des labels (C) dans
la postcondition, nous devons faire les remplacements qui s'imposent par rapport au
lieu d'appel.



Pour définir le calcul de plus faible précondition associée aux fonctions, introduisons
d'abord quelques notations pour rendre les explications plus claires. Pour cela,
considérons l'exemple suivant :

\begin{CodeBlock}{c}
/*@ requires \valid(x) && *x >= 0 ;
    assigns *x ;
    ensures *x == \old(*x)+1 ; */
void inc(int* x);

void foo(int* a){
  L1:
  inc(a) ;
  L2:
}
\end{CodeBlock}


Le calcul de plus faible précondition de l'appel de fonction nous demande de
considérer le contrat de la fonction appelée (ici, dans \CodeInline{foo}, quand
nous appelons la fonction \CodeInline{inc}). Bien sûr, avant l'appel de la fonction,
nous devons vérifier sa précondition, qui fait donc partie de la plus faible
précondition. Mais nous devons aussi considérer la postcondition de la fonction,
sinon cela voudrait dire que nous ne prenons pas en compte son effet sur l'état du
programme.


Par conséquent, il est important de noter que dans la précondition, l'état mémoire
considéré est bien celui pour lequel la plus faible précondition doit être vraie,
tandis que pour la postcondition, ce n'est pas le cas : l'état considéré est celui
qui suit l'appel, alors que dans la postcondition, lorsque nous parlons des valeurs
avant l'appel nous devons explicitement ajouter le mot-clé
\CodeInline{\textbackslash{}old}. Par exemple, pour le contrat de \CodeInline{inc}
lorsque nous l'appelons depuis \CodeInline{foo}, \CodeInline{*x} dans la
précondition est \CodeInline{*a} au label \CodeInline{L1}, alors que \CodeInline{*x}
dans la postcondition est \CodeInline{*a} au label \CodeInline{L2}. Par conséquent,
la pré et la postcondition doivent être considérées de manières légèrement différentes
lorsque nous devons parler des positions mémoire mutables. Notons que pour la valeur
du paramètre \CodeInline{x} lui-même, ce n'est pas le cas : cette valeur ne peut pas
être modifiée par l'appel (du point de vue de l'appelant).



Maintenant, définissons le calcul de plus faible précondition d'un appel de fonction.
Pour cela, notons :

\begin{itemize}
\item $\vec{v}$ un vecteur de valeurs $v_1, ..., v_n$ et $v_i$ la $i^{ème}$ valeur,
\item $\vec{t}$ les arguments fournis à la fonction lors de l'appel,
\item $\vec{x}$ les paramètres dans la définition de la fonction,
\item $\vec{a}$ les valeurs modifiées (vues de l'extérieur, une fois instanciées),
\item $here(x)$ une valeur en postcondition,
\item $old(x)$ une valeur en précondition.
\end{itemize}

Nous nommons $\texttt{f:Pre}$ la précondition de la fonction, et $\texttt{f:Post}$,
la postcondition.

\begin{center}
\begin{tabular}{rl}
  $wp( f(\vec{t}), Q ) :=$ & $\texttt{f:Pre}[x_i \leftarrow t_i]$ \\
  $\wedge$ & $\forall \vec{v}, \quad (
              \texttt{f:Post}[x_i \leftarrow t_i,
                              here(a_j) \leftarrow v_j,
                              old(a_j) \leftarrow a_j] \Rightarrow
              Q[here(a_j) \leftarrow v_j])$
\end{tabular}
\end{center}

Nous pouvons détailler les étapes du raisonnement dans les différentes parties de
cette formule.


Premièrement, notons que dans les pré et postconditions, chaque paramètre $x_i$
est remplacé par l'argument correspondant ($[x_i \leftarrow t_i]$), comme nous
l'avons dit juste avant, nous n'avons pas de question d'état mémoire à considérer
ici puisque ces valeurs ne peuvent pas être changées par l'appel de fonction. Par
exemple dans le contrat de \CodeInline{inc}, chaque occurrence de \CodeInline{x}
serait remplacée par l'argument \CodeInline{a}.


Ensuite, dans la partie de notre formule qui correspond à la postcondition, nous
pouvons voir que nous introduisons un $\forall \vec{v}$. Le but ici est de modéliser
la possibilité que la fonction change la valeur des positions mémoire spécifiées par
la clause \CodeInline{assigns} du contrat. Donc, pour chaque position potentiellement
modifiée $a_j$ (qui est, pour notre exemple d'appel à \CodeInline{inc},
\CodeInline{*(\&a)}), nous générons une valeur $v_j$ qui représente sa valeur après
l'appel. Mais si nous voulons vérifier que la postcondition nous donne le bon
résultat, nous ne pouvons pas accepter \textit{toute valeur} pour les positions mémoire
modifiées, nous voulons celles \textit{qui permettent de satisfaire la postcondition}.



Nous utilisons donc ces valeurs pour transformer la postcondition de la fonction et
pour vérifier qu'elle implique la postcondition reçue en entrée du calcul de plus
faible précondition. Nous faisons cela en remplaçant, pour chaque position mémoire
modifiée $a_j$, sa valeur $here$ avec la valeur $v_j$ qu'elle obtient après l'appel
($here(a_j) \leftarrow v_j$). Finalement, nous devons remplacer chaque valeur $old$
par sa valeur avant l'appel, et pour chaque $old(a_j)$, cette valeur est
simplement $a_j$ ($old(a_j) \leftarrow a_j$).



\levelFourTitle{Formel - Exemple}


Illustrons tout cela sur un exemple en appliquant le calcul de plus faible
précondition sur ce petit code, en supposant le contrat précédemment proposé pour
la fonction \CodeInline{swap}.

\begin{CodeBlock}{c}
  int a = 4 ;
  int b = 2 ;

  swap(&a, &b) ;

  //@ assert a == 2 && b == 4 ;
\end{CodeBlock}


Nous pouvons appliquer le calcul de plus faible précondition :

\begin{tabular}{l}
  $wp(a = 4; b = 2; swap(\&a, \&b), a = 2 \wedge b = 4) = $\\
  $\quad wp(a = 4, wp(b = 2; swap(\&a, \&b), a = 2 \wedge b = 4)) = $\\
  $\quad wp(a = 4, wp(b = 2, wp(swap(\&a, \&b), a = 2 \wedge b = 4)))$
\end{tabular}


Et considérons séparément :
$$wp(swap(\&a, \&b), a = 2 \wedge b = 4)$$


Par la clause \CodeInline{assigns}, nous savons que les valeurs modifiées par la
fonction sont $*(\&a) = a$ et $*(\&b) = b$. (nous raccourcissons $here$ avec $H$
et $old$ avec $O$).

\begin{tabular}{rl}
  $\quad \quad \texttt{swap:Pre}[x \leftarrow \&a,\ y \leftarrow \&b]$ & \\
  $\quad \wedge \forall v_a, v_b,(\texttt{swap:Post}$ & $ [ x \leftarrow \&a,\ y \leftarrow \&b, $ \\
                               & $H(*(\&a)) \leftarrow v_a,\ H(*(\&b)) \leftarrow v_b,$ \\
                               & $O(*(\&a)) \leftarrow *(\&a),\ O(*(\&b)) \leftarrow *(\&b)])$\\
  \multicolumn{2}{r}{$\quad \quad \Rightarrow (H(a) = 2 \wedge H(b) = 4)[H(a)) \leftarrow v_a, H(b)) \leftarrow v_b])$}
\end{tabular}


Pour la précondition, nous obtenons :
$$valid(\&a) \wedge valid(\&b)$$
Pour la postcondition, commençons par écrire l'expression depuis laquelle nous
travaillerons avant de faire les remplacements (et sans la syntaxe de remplacement
pour rester concis) :
$$H(*x) = O(*y) \wedge H(*y) = O(*x) \Rightarrow H(a) = 2 \wedge H(b) = 4$$
Remplaçons d'abord les pointeurs ($x \leftarrow \&a,\ y \leftarrow \&b$) :
$$H(*(\&a)) = O(*(\&b)) \wedge H(*(\&b)) = O(*(\&a)) \Rightarrow H(a) = 2 \wedge H(b) = 4$$
Puis les valeurs $here$, avec les valeurs quantifiées $v_i$ ($H(a)) \leftarrow v_a, H(b)) \leftarrow v_b$) :
$$v_a = O(*(\&b)) \wedge v_b = O(*(\&a)) \Rightarrow v_a = 2 \wedge v_b = 4$$
Et les valeurs $old$, avec les valeurs avant l'appel ($O(*(\&a)) \leftarrow *(\&a),\ O(*(\&b)) \leftarrow *(\&b)$) :
$$v_a = *(\&b) \wedge v_b = *(\&a) \Rightarrow v_a = 2 \wedge v_b = 4$$
Nous pouvons maintenant simplifier cette formule en :
$$v_a = b \wedge v_b = a \Rightarrow v_a = 2 \wedge v_b = 4$$


Donc, $wp(swap(\&a, \&b), a = 2 \wedge b = 4)$ est :
$$P: valid(\&a) \wedge valid(\&b) \wedge \forall v_a, v_b, \quad v_a = b \wedge v_b = a \Rightarrow v_a = 2 \wedge v_b = 4$$
nous pouvons immédiatement simplifier cette formule en constatant que les propriétés
de validité sont trivialement vraies (puisque les variables sont allouées sur la pile
juste avant) :
$$P: \forall v_a, v_b, \quad v_a = b \wedge v_b = a \Rightarrow v_a = 2 \wedge v_b = 4$$


Maintenant, calculons $wp(a = 4, wp(b = 2, P)))$, en remplaçant d'abord $b$
par $2$ par la règle d'affectation :
$$\forall v_a, v_b, \quad v_a = 2 \wedge v_b = a \Rightarrow v_a = 2 \wedge v_b = 4$$
et ensuite $a$ par $4$ par la même règle :
$$\forall v_a, v_b, \quad v_a = 2 \wedge v_b = 4 \Rightarrow v_a = 2 \wedge v_b = 4$$


Cette dernière propriété est trivialement vraie, le programme est vérifié.


\levelFourTitle{Contrats de fonction : check et admit}


Tout comme les assertions et les invariants de boucle, les clauses
\CodeInline{requires} et \CodeInline{ensures} ont des variantes
\CodeInline{check} et \CodeInline{admit} permettant de seulement vérifier ou
admettre ces propriétés.


Commençons par la clause \CodeInline{ensures}. Quand une fonction \CodeInline{f}
a une clause \CodeInline{ensures}, elle doit être prouvée correcte lorsque l'on
vérifie \CodeInline{f}, en revanche lorsque l'on appelle \CodeInline{f}, cette
propriété est admise. Maintenant, si \CodeInline{f} a une clause
\CodeInline{check ensures}, nous devons la vérifier quand nous vérifions
\CodeInline{f}, mais nous l'\textit{ignorons} lorsque \CodeInline{f} est appelée.
À l'inverse, quand \CodeInline{f} a une clause \CodeInline{admit ensures}, nous
ne la vérifions pas quand nous vérifions \CodeInline{f}, mais elle est admise
quand \CodeInline{f} est appelée. Dans l'exemple suivant :

\CodeBlockInput[5]{c}{check-admit-ensures.c}

\image{ensures-CA}

Nous ne nous pouvons pas prouver la clause \CodeInline{E1} parce que
\CodeInline{*x} est affectée à 1, mais nous pouvons prouver \CodeInline{A1}, car
quand \CodeInline{f} est appelée, nous admettons que \CodeInline{E1} est vraie.
Nous pouvons prouver la clause \CodeInline{E2} parce que \CodeInline{*y} est
affectée à 10 qui est effectivement supérieur ou égal à 10. En revanche, nous ne
pouvons pas prouver \CodeInline{A2} parce que la postcondition est ignorée à
l'appel. Finalement, tandis que \CodeInline{*z} est affectée à 20, nous ne
vérifions pas que \CodeInline{E3} est vraie (et elle ne l'est pas), pour autant,
WP n'émet pas de \textit{warning} à ce sujet : elle est admise, donc lors de
l'appel à \CodeInline{f}, nous pouvons prouver que \CodeInline{z} est plus grand
que 30, même si elle ne l'est pas.


Maintenant, présentons le comportement pour la clause \CodeInline{requires}.
Quand une fonction \CodeInline{f} a une clause \CodeInline{requires}, elle est
supposée vraie lorsque l'on vérifie \CodeInline{f}, et doit être vérifiée
lorsque \CodeInline{f} est appelée. Quand c'est une clause
\CodeInline{check requires}, la clause est vérifiée au point d'appel, mais elle
n'est pas supposée vraie lorsque nous vérifions \CodeInline{f}. Finalement,
quand \CodeInline{f} a une clause \CodeInline{admit requires}, elle est supposée
vraie lors de la vérification de \CodeInline{f} et l'on n'essaie pas de la
vérifier au point d'appel. Dans l'exemple suivant :

\CodeBlockInput[5]{c}{check-admit-requires.c}

\image{requires-CA}

Nous ne pouvons pas prouver que \CodeInline{R1} est vraie au point d'appel, mais
nous admettons quand même qu'elle est vraie dans \CodeInline{callee}, donc nous
pouvons prouver \CodeInline{A1}. Ensuite, même si \CodeInline{R2} est vérifiée
au point d'appel, puisque nous l'ignorons pendant la vérification de
\CodeInline{callee}, nous ne pouvons pas prouver que \CodeInline{A2} est vraie.
Finalement, nous n'essayons pas de vérifier \CodeInline{R3} au point d'appel
(et WP n'émet pas de \textit{warning} à ce sujet même si elle est fausse), mais
nous pouvons prouver que \CodeInline{A3} est vérifiée, car nous supposons que
\CodeInline{R3} est vraie.


Il y a deuxième aspect important avec le comportement des clauses
\CodeInline{requires}. Le lecteur attentif aura surement remarqué que dans
l'exemple précédent, le \CodeInline{check \textbackslash{}false} est vérifié.
La raison est que les clauses \CodeInline{requires} normales et
\CodeInline{admit} sont aussi supposée vraie localement après leur vérification.
Par conséquent, même si dans cet exemple ces clauses sont fausses, nous
introduisons localement « faux » dans le contexte de preuve.


\levelFourTitle{Que devrions-nous garder en tête à propos des contrats de fonction ?}


Les fonctions sont absolument nécessaires pour programmer modulairement, et le calcul
de plus faible précondition est pleinement compatible avec cette idée, permettant de
raisonner localement à propos de chaque fonction et donc de composer les preuves juste
de la même manière que nous composons les appels de fonction.


Comme pense-bête, nous devrions toujours garder en tête le schéma suivant :


\begin{CodeBlock}{c}
  /*@       requires bar_R
      check requires bar_CR ;
      admit requires bar_AR ;

      assigns ... ;

            ensures bar_E
      check ensures bar_CE ;
      admit ensures bar_AE ;
  */
  void bar(...) ;

  /*@       requires foo_R
      check requires foo_CR ;
      admit requires foo_AR ;

      assigns ... ;

            ensures foo_E
      check ensures foo_CE ;
      admit ensures foo_AE ;
  */
  type foo(parameters...){
    // Ici, nous supposons que foo_R et foo_AR sont vraies


    // Ici, nous devons prouver que bar_R et bar_CR sont vraies
    bar(some parameters ...) ;
    // Ici, nous supposons que bar_E et bar_AE sont vraies


    // Ici, nous devons prouver que foo_E et foo_CE sont vraies
    return ... ;
  }
  \end{CodeBlock}


Notons qu'à propos du dernier commentaire, en calcul de plus faible précondition,
l'idée est plutôt de montrer que notre précondition est assez forte pour assurer
que le code nous amène à la postcondition. Cependant, premièrement, cette vision
est plus facile à comprendre et deuxièmement, un greffon comme WP (et comme n'importe
quel outil réaliste pour la preuve de programme) ne suit pas strictement un calcul
de plus faible précondition, mais une manière fortement optimisée de calculer les
conditions de vérification qui ne suit pas exactement les mêmes règles.


\levelFourTitle{Le cas particulier de la clause \CodeInline{exits}}


En C, il est possible d'appeler la fonction \CodeInline{exit} pour arrêter
l'exécution du programme avec un code d'erreur particulier. Dans un tel cas, le
code qui suit l'appel ne sera pas exécuté et l'instruction \CodeInline{return}
ne sera jamais atteinte. Par conséquent, lorsqu'une fonction \textit{quitte},
la postcondition est toujours vérifiée, car elle inatteignable (de la même
manière que lorsqu'une fonction n'appelle pas \CodeInline{exit}, la clause
\CodeInline{exits} peut être n'importe quoi) :


\CodeBlockInput{c}{exit-1.c}


Ici, nous avons indiqué que \CodeInline{\textbackslash{}true} doit être vrai
quand la fonction quitte, mais en fait, nous pouvons indiquer n'importe quelle
propriété d'intérêt. Par exemple, la valeur qui a été passée à
\CodeInline{exit} :


\CodeBlockInput{c}{exit-2.c}


Bien sûr, quand on appelle une fonction qui peut quitter, ce risque est propagé :


\CodeBlockInput{c}{exit-3.c}


\paragraph*{[Formel] Calcul de WP}


Lorsque nous voulons prouver la postcondition d'une fonction, nous commençons le
calcul depuis la postcondition. Donc, nous commençons depuis l'unique instruction
\CodeInline{return} créée par Frama-C (bien que ce soit exactement comme
commencer le calcul par toutes les instructions \CodeInline{return} du programme
d'origine), et nous raisonnons en arrière le long du programme. Lorsque nous
rencontrons un appel de fonction, nous utilisons son contrat à l'aide de la
règle précédemment expliquée et la postcondition que nous considérons à ce
point de raisonnement est celle  que l'on trouve dans les clauses
\CodeInline{ensures}.


Pour prouver les clauses \CodeInline{exits}, c'est différent, mais pas tant que
cela. Au lieu de commencer le calcul depuis les instructions \CodeInline{return},
nous commençons un nouveau calcul à partir de chaque appel de fonction
(puisqu'elle pourrait appeler \CodeInline{exit}) et à cet appel, nous utilisons
la clause \CodeInline{exits} comme postcondition. Ensuite, lorsque nous
continuons le raisonnement arrière, nous utilisons les clauses
\CodeInline{ensures} comme d'habitude puisque si nous avons atteint l'appel de
fonction depuis lequel nous avons démarré le calcul, cela veut dire que les
appels qui ont précédé n'ont pas appelé \CodeInline{exit}.

\levelThreeTitle{Fonctions récursives}


De la même manière que nous pouvons prouver n'importe quoi à propos de la
postcondition d'une fonction qui contient une boucle infinie, il est très simple
de prouver n'importe quoi à propos de la postcondition d'une fonction qui
produit une récursion infinie :


\CodeBlockInput{c}{trick.c}


\image{recursive-trick}


Nous pouvons voir que la fonction et l'assertion sont prouvées. Et, effectivement,
la preuve est correcte : nous considérons une correction partielle (puisque nous ne
pouvons pas prouver la terminaison), et cette fonction ne termine pas. Toute
assertion suivant cette fonction est vraie : elle est inatteignable. Mais à
nouveau, nous avons un chien de garde : la clause \CodeInline{terminates} n'est
pas prouvée.


Une question que l'on peut alors se poser est : que peut-on faire dans un tel
cas ? Nous pouvons à nouveau utiliser une mesure (comme dans la
Section~\ref{l4:statements-loops-variant-measure}) pour borner la profondeur de
récursion. En ACSL, c'est le rôle de la clause \CodeInline{decreases} :


\CodeBlockInput[1][6]{c}{decreases.c}


Tout comme la clause \CodeInline{loop variant}, la clause \CodeInline{decreases}
exprime une notion de mesure. Donc, une expression entière positive (ou une
expression équipée d'une relation) qui décroît strictement. Tandis que la clause
\CodeInline{loop variant} représente une borne supérieure sur le nombre
d'itérations, la clause \CodeInline{decreases} représente une borne supérieure
sur la profondeur de récursion (et pas le nombre d'appels de fonctions) :


\CodeBlockInput[14][20]{c}{decreases.c}


Notons que, comme pour la clause \CodeInline{loop variant} où nous vérifions les
propriétés de l'expression seulement lorsqu'une nouvelle itération peut
apparaître, les propriétés de l'expression d'une clause \CodeInline{decreases}
ne sont vérifiés que lorsque la fonction est à nouveau appelée. Ce qui signifie
que lorsque l'on atteint la profondeur maximale d'appel, l'expression peut être
négative :


\CodeBlockInput[8][12]{c}{decreases.c}


Nous pouvons voir la condition de vérification générée pour l'exemple précédent
en désactivant la simplification du but (option \CodeInline{-wp-no-let}, ici,
nous avons supprimé les informations redondantes de la capture d'écran) :


\image{go_negative}


Ici, la condition \CodeInline{n - 10 < n} est reformulée en
\CodeInline{n\_1 (= n) <= 9 + n} à cause de la normalisation des formules.


Lorsque nous sommes en train de prouver la correction de la clause
\CodeInline{decreases} d'une fonction (pour un appel de cette fonction),
l'expression est évaluée pour deux entités : la fonction sous preuve et
l'instruction d'appel. L'état d'évaluation de l'expression pour la fonction
sous preuve est \CodeInline{Pre}, l'état d'évaluation de l'expression pour
l'appel est \CodeInline{Here}. La valeur de l'expression lors de l'appel doit
être plus petite que la valeur de l'expression associée à l'état \CodeInline{Pre}.


\CodeBlockInput[22][30]{c}{decreases.c}


Bien sûr, des fonctions récursives peuvent être mutuellement récursives. Par
conséquent, la clause \CodeInline{decreases} peut être utilisée pour borner la
profondeur d'appels récursifs dans cette situation. Cependant, nous ne voulons
le faire \textit{que} pour les fonctions qui sont effectivement dans l'ensemble de
fonctions mises en jeu dans la récursion. Pour cela, WP calcule les composantes
fortement connexes depuis l'ensemble complet des fonctions, dans la terminologie
ACSL, on appelle ces composants des \textit{clusters}.


Nous pouvons donc préciser un peu comment est vérifiée la correction d'une clause
\CodeInline{decreases}. Lorsqu'une fonction est (mutuellement) récursive, sa
spécification doit être équipée avec une telle clause afin de prouver qu'elle
termine. Vérifier que la clause \CodeInline{decreases} d'une fonction
\CodeInline{f} donne une mesure de sa profondeur de récursion consiste à vérifier,
pour chaque appel à une fonction \textit{qui appartient au même cluster que
\CodeInline{f}}, que l'expression fournie par la clause est effectivement positive
et décroissante. Et par conséquent, aucune condition de vérification n'est
générée lorsqu'une fonction récursive n'appartenant pas au même \textit{cluster}
est appelée :


\CodeBlockInput[32][48]{c}{decreases.c}


\begin{Information}
  Notons que le calcul des \textit{clusters} est basé sur un critère syntaxique.
  Pour le moment, nous ne parlons par des pointeurs de fonctions dans ce tutoriel,
  le lecteur peut se référer à la documentation de l'option
  \CodeInline{-wp-dynamic} dans le manuel de WP.
\end{Information}


Finalement, si une fonction d'un \textit{cluster} ne fournit pas de clause
\CodeInline{decreases}, une condition de vérification
\CodeInline{\textbackslash{}false} est générée et WP émet un avertissement ;


\CodeBlockInput[50][63]{c}{decreases.c}


\image{fail-mutual-vc}


\begin{Information}
  Comme pour les clauses \CodeInline{loop variant}, il est possible de fournir
  une autre relation (voir Section~\ref{l4:statements-loops-general-measure})
  pour une clause \CodeInline{decreases}. La syntaxe est :


  \begin{CodeBlock}{c}
//@ decreases <term> for <Relation> ;
\end{CodeBlock}
\end{Information}


\levelThreeTitle{Spécifier et prouver la terminaison des fonctions}
\label{l3:statements-function-calls-terminates}


Une propriété désirable pour une fonction est souvent qu'elle doit terminer.
En effet, dans la plupart des programmes, toutes les fonctions doivent terminer,
et quand ce n'est pas le cas, il est très probable qu'en réalité, une seule
fonction puisse boucler à l'infini (donc la quasi-totalité des fonctions dans le
programme terminent).


\levelFourTitle{Syntaxe et description}


ACSL fournit une clause \CodeInline{terminates} pour spécifier qu'une fonction
doit terminer quand une propriété particulière est vérifiée en précondition. La
syntaxe est la suivante :


\begin{CodeBlock}{c}
//@ terminates condition ;
void function(void){
  // ...
}
\end{CodeBlock}


Ici, le contrat énonce que quand \CodeInline{condition} est vérifiée en
précondition, alors la fonction doit terminer. Par exemple, la fonction
\CodeInline{abs} doit toujours terminer :


\CodeBlockInput[3][8]{c}{terminates_abs.c}


alors que pour la fonction \CodeInline{main\_loop} cela peut ne pas être le cas
(notons qu'avec les options par défaut, le variant de la boucle n'est pas
vérifié, nous expliquerons pourquoi plus tard) :


\CodeBlockInput[6]{c}{main_loop.c}


Soulignons que la fonction \textit{pourrait ne pas} terminer, elle n'est pas
\textit{forcée de boucler à l'infini}. Par exemple, dans la fonction suivante,
la clause \CodeInline{terminates} est vérifiée, car
\textit{lorsque la condition de terminaison est vérifiée} (jamais), la fonction
termine (toujours) :


\CodeBlockInput[6][9]{c}{terminates.c}


\begin{Information}
  Si l'on veut vraiment vérifier qu'une fonction ne termine jamais, nous pouvons
  spécifier que la fonction ne retourne jamais, et ne quitte (\CodeInline{exits})
  jamais. C'est-à-dire : les postconditions associées à ces types de terminaison
  sont inatteignables :

  \CodeBlockInput{c}{no_terminates.c}
\end{Information}


\begin{Information}
  En ACSL, il est spécifié que lorsqu'une fonction n'a pas de clause
  \CodeInline{terminates}, le comportement par défaut
  \CodeInline{terminates \textbackslash{}true}. Ce comportement est
  automatiquement activé par WP au niveau du noyau de Frama-C. Donc, quand WP
  démarre la vérification d'une fonction, il demande au noyau de générer ces
  annotations. Ce comportement peut être désactivé
  \textbf{pour les fonctions définies par l'utilisateur} à l'aide de l'option :
  \begin{itemize}
  \item \CodeInline{-generated-spec-custom terminates:skip}
  \end{itemize}
\end{Information}


\levelFourTitle{Vérification}


Vérifier qu'une fonction termine demande de vérifier que toutes les instructions
atteignables d'une fonction terminent. Les affectations terminent trivialement,
donc nous n'avons rien de particulier à vérifier à leur sujet. Une instruction
conditionnelle termine si toutes les instructions des différentes branches
(atteignables) terminent, donc nous avons juste à vérifier que ces instructions
terminent (ou sont inatteignables). Les instructions restantes sont les boucles
et les appels de fonctions. Donc nous devons vérifier que :
\begin{itemize}
  \item toutes les boucles ont une clause \CodeInline{loop variant} (vérifiée),
  \item toutes les fonctions appelées terminent avec leurs paramètres d'entrée,
  \item toutes les fonctions récursives ont une clause \CodeInline{decreases} (vérifiée),
  \item (ou qu'il n'y a pas de boucles ou d'appels atteignables dans la fonction).
\end{itemize}


Cependant, nous devons seulement faire ces vérifications lorsque la condition de
terminaison de la fonction est vérifiée. Voyons donc maintenant quelles sont les
conditions de vérification générées et quand WP les génère.


Lorsqu'une fonction a une clause \CodeInline{terminates}, WP visite toutes les
instructions de la fonction et collecte les boucles pour lesquelles aucune
clause \CodeInline{loop variant} n'est spécifiée et les appels de fonctions. Si
aucune instruction de ce type n'est présente, la fonction termine trivialement.


\image{trivial_terminates}


S'il existe de telles instructions, leur terminaison doit être vérifiée
\textit{lorsque} la fonction doit terminer (disons, lorsque \CodeInline{T}). Donc,
les conditions de vérification sont de la forme
\CodeInline{\textbackslash{}at(T, Pre)} $\Rightarrow$ \CodeInline{<statement termination>}.
Notons que la prémisse de l'implication doit être évaluée dans l'état
\CodeInline{Pre}. Donc, dans ce code :


\CodeBlockInput[11][17]{c}{terminates.c}


Même si \CodeInline{r} a été décrémentée, la vérification de la condition de
la terminaison de \CodeInline{call(r)} est faite en prenant
\CodeInline{\textbackslash{}at(r, Pre) > 0} comme prémisse. Nous verrons dans
la section suivante la condition de vérification complète générée pour un appel.


Notons aussi que cela signifie que lorsque nous atteignons un point de programme
où \CodeInline{T} est fausse, la condition de vérification est toujours
vérifiée :


\CodeBlockInput[19][25]{c}{terminates.c}


\paragraph{Appel de fonction}


Pour vérifier qu'un appel de fonction termine, nous vérifions lorsque la
fonction est appelée, que la condition de terminaison est vraie. Par exemple, à
partir du programme suivant :


\CodeBlockInput[27][36]{c}{terminates.c}


Nous obtenons les conditions de vérification suivante (en utilisant l'option
\CodeInline{-wp-no-let} pour désactiver les simplifications du but) :


\image{call_terminates_1}


\image{call_terminates_2}


Où la terminaison du premier appel est en effet vérifiée, tandis que la
terminaison du second ne l'est pas.


\paragraph{Variant de boucle}


Lorsqu'une fonction contient une boucle qui n'a pas de clause
\CodeInline{loop variant}, sa terminaison ne peut pas être vérifiée, donc WP
nous demande de vérifier que lorsque la condition de terminaison est vérifiée,
la boucle est inatteignable.


\CodeBlockInput[38][46]{c}{terminates.c}


Dans le code précédent, la boucle n'a pas de clause \CodeInline{loop variant},
donc nous devons vérifier \CodeInline{value > 0 ==> \textbackslash{}false} à
la position de la boucle, ce qui est bon : le code est inatteignable lorsque
\CodeInline{value > 0}.


Finalement, lorsqu'une boucle a une clause \CodeInline{loop variant}, elle doit
être vérifiée \textit{seulement quand} la fonction doit terminer. Donc, si nous
reprenons l'exemple présenté au début de cette section :


\CodeBlockInput[6]{c}{main_loop.c}


Nous devons vérifier que le variant de boucle est une valeur positive et
décroissante seulement lorsque \CodeInline{debug\_steps} n'est pas $-1$. Par
contre, ce n'est pas le comportement par défaut de WP (qui cherche toujours à
montrer que les variants de boucles sont corrects par défaut), cela peut être
activé en utilisant l'option \CodeInline{-wp-variant-with-terminates} et dans
ce cas, la fonction est entièrement vérifiée :


\image{main_loop_variant_terminates}


\paragraph{Récursion}


Une fonction récursive doit avoir une clause \CodeInline{decreases} lorsque sa
spécification indique qu'elle termine. Si une telle clause est manquante, une
condition de vérification \CodeInline{\textbackslash{}false} est générée.


\CodeBlockInput[48][51]{c}{terminates.c}


Notons que ce code génère deux conditions de vérification :


\image{missing_decreases_vcs}


La première correspond à la règle associée à la terminaison d'un appel de
fonction précédemment présentée. La seconde correspond au fait que la clause
\CodeInline{decreases} est manquante et donc non vérifiée.


À nouveau en ACSL, la vérification de la clause \CodeInline{decreases} n'est
requise que lorsque la condition de terminaison est vérifiée dans l'état
\CodeInline{Pre}. Le comportement de WP sur cet aspect est similaire au cas
des variants de boucle. Par défaut, la vérification est toujours demandée, le
comportement ACSL est activé via l'option
\CodeInline{-wp-variant-with-terminates}.


\CodeBlockInput[53][60]{c}{terminates.c}


\levelThreeTitle{Exercices}


\levelFourTitle{Expliquer les échecs de preuve}


Dans le programme suivant, quelques conditions de vérification ne sont pas
vérifiées :


\CodeBlockInput{c}{ex-1-proof-failures.c}


Expliquer pourquoi elles ne sont pas vérifiées et proposer une manière de
corriger la spécification pour que tout soit vérifié.


\levelFourTitle{Expliquer les résultats des preuves de terminaison}


Dans le programme suivant :


\CodeBlockInput[5]{c}{ex-2-terminates.c}


Expliquer pourquoi les clauses de terminaison sont vérifiées ou pas. Modifier
les spécifications pour que tout soit vérifié.


\levelFourTitle{Recherche}


Spécifier et prouver la fonction de recherche récursive suivante :


\CodeBlockInput[5]{c}{ex-3-search.c}


La spécification doit inclure la condition de terminaison.


\levelFourTitle{Somme des entiers}


Le programme suivant calcule la somme des entiers entre \CodeInline{fst} et
\CodeInline{lst} :


\CodeBlockInput[5]{c}{ex-4-sum.c}


Prouver que la fonction termine. La preuve que la fonction calcule bien la
somme des entiers et l'absence d'erreurs à l'exécution ne sont pas demandées.


\levelFourTitle{Puissance}


Le programme suivant calcule \CodeInline{x} à la puissance \CodeInline{n} :


\CodeBlockInput[5]{c}{ex-5-power.c}


Prouver que la fonction termine. La preuve que la fonction calcule bien la
puissance et l'absence d'erreurs à l'exécution ne sont pas demandées.

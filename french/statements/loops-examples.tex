\levelThreeTitle{Écrire des annotations de boucle}


L'écriture des annotations de boucles demande un effort important pendant la
vérification. Il n'y a pas de manière parfaite de faire ce travail. En
particulier, trouver le bon invariant et la bonne manière de l'exprimer, est
principalement une question d'expérience. Néanmoins, quelques idées simples
peuvent aider, afin d'éviter certaines erreurs communes qui peuvent faire perdre
beaucoup de temps pendant le processus.


Avant toute chose, nous devons exprimer une clause \CodeInline{loop assigns}
correcte. Elle est généralement facile à écrire (il suffit de regarder les
instructions d'affectation et les appels de fonctions), et si elle est
incorrecte nous pourrions prouver des propriétés fausses, ce qui nous ferait
perdre du temps. Il est plutôt déconseillé de chercher à fournir des clauses
\CodeInline{loop assigns} trop précises. WP ne peut pas utiliser efficacement
des clauses comme \CodeInline{array[x .. y]} lorsque \CodeInline{x} ou
\CodeInline{y} sont elles-mêmes modifiées, on préférera donc des bornes qui
les incluent tout en étant constante pendant l'exécution de la boucle. Si nous
avons ensuite besoin d'informations plus précises pendant la preuve des
invariants, il sera plus facile de fournir des invariants supplémentaires pour
cela.


Ensuite, nous bornons les différentes variables qui sont assignées, en
particulier les indices. Ces invariants sont généralement faciles à deviner,
exprimer et vérifier. Nous mettons ces invariants au début de la liste des
clauses \CodeInline{loop invariant}, puisque, comme nous l'avons expliqué
dans les sections~\ref{l3:statements-loops-multi-inv}
et~\ref{l3:statements-loops-inv-kinds}, l'ordre des invariants est important
et ces propriétés très simples peuvent être propagées par WP dans les autres
invariants pour simplifier la condition de vérification à prouver.


Pour la plupart des boucles, exceptées celles qui reposent sur une condition
complexe, une fois que cette étape est réalisée, il est facile de fournir une
clause \CodeInline{loop variant}. Les variables que nous venons de borner sont
une bonne piste : nous avons juste à regarder la valeur qu'elles vont atteindre
en fin d'exécution. Par exemple, si dans la boucle, on a une variable
\CodeInline{i} qui va de 0 à \CodeInline{n}, \CodeInline{n - i} est un bon
candidat. Pour des boucles plus complexes, nous pouvons utiliser du code
fantôme (que l'on présente en section~\ref{l2:acsl-logic-definitions-ghost-code})
pour rendre explicite une mesure utilisable pour un variant de boucle.


Ensuite, nous ajoutons nos invariants « principaux » à la boucle, c'est-à-dire
ceux qui sont liés à la postcondition de la boucle (qui peut être aussi la
postcondition de la fonction). Pour cela, nous pouvons utiliser la postcondition
elle-même comme un guide. Si nous avons quelque chose comme
\CodeInline{ensures P(n);} et une boucle qui itère \CodeInline{i} de 0 à
\CodeInline{n}, il y a fort à parier que \CodeInline{loop invariant P(i);} soit
un bon invariant de boucle. Notons que dans certaines situations, c'est la
clause \CodeInline{assumes} d'un comportement qui se révèle être intéressante,
typiquement quand le résultat de l'exécution est une simple valeur qui dépend
de l'état en précondition, nous verrons cela dans un exemple plus tard. Ces
invariants doivent généralement être positionnés en fin de liste des invariants
de boucle.


Nous pouvons (optionnellement) avoir besoin de « glu » pour prouver les
invariants « principaux ». Par exemple, si nous avons volontairement fourni
une clause \CodeInline{loop assigns} imprécise, nous pourrions avoir besoin
d'un invariant de boucle pour expliquer que certaines parties de la mémoire
n'ont pas été modifiées, ou nous pourrions avoir besoin d'expliquer aux
prouveurs que, parce que certaines propriétés sont vraies, nous pouvons déduire
que l'invariant « principal » est vérifié, etc. Ces invariants doivent être
placés avant les invariants « principaux », mais après les invariants simples
qui bornent les variables. Ordonner ces invariants pourrait ne pas être si
direct, généralement nous passerons par un processus d'essai et erreur.


Dans la suite de cette section, nous illustrons cette approche avec quelques
exemples. Même si dans ce processus, il est fortement conseillé de lancer la
preuve entre chaque étape, nous n'irons généralement pas jusqu'à ce niveau de
détail dans les exemples futurs.


\levelThreeTitle{Exemple avec un tableau en lecture seule}
\label{l3:statements-loops-examples-ro}


S'il y a une structure de données que nous traitons avec les boucles, c'est bien
le tableau. C'est une bonne base d'exemples pour les boucles, car ils permettent
rapidement de présenter des invariants intéressants et surtout, ils nous
permettront d'introduire des constructions très importantes d'ACSL.


Prenons par exemple la fonction qui cherche une valeur dans un tableau. Pour le
moment, concentrons nous sur le contrat de la fonction :


\CodeBlockInput[5]{c}{search-1.c}


Cet exemple est suffisamment fourni pour introduire des notations importantes.


D'abord, comme nous l'avons déjà mentionné, le prédicat
\CodeInline{\textbackslash{}valid\_read} (de même que
\CodeInline{\textbackslash{}valid}) nous permet de spécifier non seulement la
validité d'une adresse en lecture, mais également celle de tout un ensemble
d'adresses contiguës. C'est la notation que nous avons utilisée dans cette
expression :


\begin{CodeBlock}{c}
//@ requires \valid_read(a + (0 .. length-1));
\end{CodeBlock}


Cette précondition nous atteste que les adresses \CodeInline{a+0},
\CodeInline{a+1}, \ldots{}, \CodeInline{a+length-1} sont valides en lecture.


Nous avons également introduit deux notations qui vont nous être très utiles, à
savoir \CodeInline{\textbackslash{}forall} ($\forall$) et \CodeInline{\textbackslash{}exists} ($\exists$), les
quantificateurs de la logique. Le premier nous servant à annoncer que pour tout
élément, la propriété suivante est vraie. Le second pour annoncer qu'il existe
un élément tel que la propriété est vraie. Si nous commentons les deux lignes en
questions, nous pouvons les lire de cette façon :


\begin{CodeBlock}{c}
/*@
//pour tout "off" de type "size_t", tel que SI "off" est compris entre 0 et "length"
//                                 ALORS la case "off" de "a" est différente de "element"
\forall size_t off ; 0 <= off < length ==> a[off] != element;

//il existe "off" de type "size_t", tel que "off" soit compris entre 0 et "length"
//                                 ET que la case "off" de "a" vaille "element"
\exists size_t off ; 0 <= off < length && a[off] == element;
*/
\end{CodeBlock}


Si nous devions résumer leur utilisation, nous pourrions dire que sur un certain
ensemble d'éléments, une propriété est vraie, soit à propos d'au moins l'un
d'eux, soit à propos de la totalité d'entre eux. Un schéma qui reviendra
typiquement dans ce cas est que nous restreindrons cet ensemble à travers une
première propriété (ici : \CodeInline{0 <= off < length}) puis nous voudrons prouver la
propriété réelle qui nous intéresse à propos d'eux. \textbf{Mais il y a une
différence fondamentale entre l'usage de \CodeInline{exists} et celui de \CodeInline{forall}}.


Avec \CodeInline{\textbackslash{}forall type a ; p(a) ==> q(a)}, la restriction
(\CodeInline{p}) est suivie par une implication. Pour tout élément, s'il
satisfait une première propriété (\CodeInline{p}), alors il doit vérifier la
seconde propriété \CodeInline{q}. Si nous mettions un ET comme pour le « il
existe » (que nous expliquerons ensuite), cela voudrait dire que nous voulons
que tout élément vérifie à la fois les deux propriétés. Parfois, cela peut être
ce que nous voulons exprimer, mais cela ne correspond alors plus à l'idée de
restreindre un ensemble dont nous voulons montrer une propriété particulière.


Avec \CodeInline{\textbackslash{}exists type a ; p(a) \&\& q(a)}, la restriction
(\CodeInline{p}) est suivie par une conjonction, nous voulons qu'il existe un
élément tel que cet élément est dans un certain état (défini par \CodeInline{p}),
tout en satisfaisant l'autre propriété \CodeInline{q}. Si nous mettions une
implication comme pour le « pour tout », alors une telle expression devient
toujours vraie à moins que \CodeInline{p} soit une tautologie ! Pourquoi ?
Existe-t-il « a » tel que p(a) implique q(a) ? Prenons n'importe quel « a » tel
que p(a) est faux, l'implication devient vraie.


Notons que dans cet exemple, la clause \CodeInline{assume} du comportement
\CodeInline{in} est exactement la négation de la clause \CodeInline{assumes}
du comportement \CodeInline{notin}, c'est la raison pour laquelle les clauses
\CodeInline{disjoint} et \CodeInline{complete} sont prouvées, en fait nous
aurions pu l'exprimer comme suit :


\begin{CodeBlock}{c}
  /*@ ...
    behavior in:
      assumes !(\forall size_t off ; 0 <= off < length ==> array[off] != element);
    ...
  */
\end{CodeBlock}


Parlons des annotations de la boucle. La première étape est d'ajouter la clause
\CodeInline{loop assigns}. Ici, elle est simple : la boucle ne modifie que la
variable \CodeInline{i}. La valeur de cette variable doit donc être bornée, elle
va de \CodeInline{0} à \CodeInline{length}, c'est notre premier invariant :
\CodeInline{0 <= i <= length}. Ceci nous donne également le variant de boucle :
\CodeInline{length - i}. Maintenant, nous pouvons fournir notre invariant
« principal ». Ici, il est relié à la clause \CodeInline{assumes}, et pas à la
clause \CodeInline{ensures}. En particulier, la partie intéressante de la
fonction est qu'à moins que l'on rencontre l'élément recherché, il ne se trouve
pas dans le tableau, nous exploitons cela en partant de la clause
\CodeInline{assumes} du comportement \CodeInline{notin} :
\begin{CodeBlock}{c}
  //@ \forall size_t off ; 0 <= off < length ==> array[off] != element;
\end{CodeBlock}
La variable qui atteint \CodeInline{length} à la fin de la boucle est
\CodeInline{i}, donc :
\begin{CodeBlock}{c}
  //@ loop invariant \forall size_t off ; 0 <= off < i ==> array[off] != element;
\end{CodeBlock}
est certainement un bon candidat. Cela nous amène aux annotations de boucle
suivantes :


\CodeBlockInput[19]{c}{search-0.c}


Et effectivement, notre invariant de boucle final définit l'action de notre
boucle, elle indique à WP ce que la boucle fera (ou apprendra dans le cas
présent) tout au long de son exécution. Ici en l'occurrence, cette formule nous
dit qu'à chaque tour, nous savons que pour toute case entre 0 et la prochaine
que nous allons visiter (\CodeInline{i} exclue), elle stocke une valeur
différente de l'élément recherché.


Le but de WP associé à la préservation de cet invariant est un peu compliqué, il
n'est pour nous pas très intéressant de se pencher dessus. En revanche, la
preuve de l'établissement de cet invariant est intéressante :


\image{trivial}


Nous constatons que cette propriété, pourtant complexe, est prouvée par
Qed sans aucun problème. Si nous regardons sur quelles parties du programme la
preuve se base, nous voyons l'instruction \CodeInline{i = 0} surlignée, et c'est
bien la dernière instruction que nous effectuons sur \CodeInline{i} avant de commencer
la boucle. Et donc effectivement, si nous faisons le remplacement dans la formule
de l'invariant :


\begin{CodeBlock}{c}
//@ loop invariant \forall size_t j; 0 <= j < 0 ==> array[j] != element;
\end{CodeBlock}


« Pour tout j, supérieur ou égal à 0 et inférieur strict à 0 », cette partie est
nécessairement fausse. Notre implication est donc nécessairement vraie.


\levelThreeTitle{Exemples avec tableaux mutables}


Nous allons voir deux exemples avec la manipulation de tableaux en mutation.
L'un avec une modification totale, l'autre en modification sélective.



\levelFourTitle{Remise à zéro}


Regardons la fonction effectuant la remise à zéro d'un tableau.



\CodeBlockInput{c}{reset.c}


Cette fois, la boucle modifie le contenu du tableau, donc nous devons fournir
cette information dans la clause \CodeInline{loop assigns}. Notons que nous
pouvons utiliser la notation \CodeInline{n .. m} pour indiquer quelle partie du
tableau a été modifiée. De plus, nous ne disons pas que la boucle assigne le
contenu depuis \CodeInline{0} jusqu'à \CodeInline{i-1} (comme \CodeInline{i}
est modifiée, WP ne peut pas exploiter cette écriture) mais depuis \CodeInline{0}
jusqu'à \CodeInline{length-1}, c'est moins précis, mais cela peut être utilisé
par WP en dehors de la boucle. Finalement, cette fois, nous relions directement
l'invariant à la postcondition, le but de la fonction est de réinitialiser le
tableau de 0 jusqu'à \CodeInline{length}, à une itération donnée, la boucle l'a
déjà fait entre 0 et \CodeInline{i}.


\levelFourTitle{Chercher et remplacer}
\label{l4:statements-loops-ex-search-and-replace}


Le dernier exemple qui nous intéresse est l'algorithme « chercher et remplacer ».
C'est un algorithme qui modifie sélectivement des valeurs dans une
certaine plage d'adresses. Il est toujours un peu difficile de guider l'outil
dans ce genre de cas car, d'une part, nous devons garder « en mémoire » ce qui est modifié
et ce qui ne l'est pas et, d'autre part, parce que l'induction repose sur ce fait.


À titre d'exemple, la première spécification et boucle annotée, que nous pouvons
réaliser pour cette fonction ressemblerait à ceci (ce qui suit sensiblement le
même processus que dans l'exemple précédent) :


\CodeBlockInput{c}{search_and_replace-0.c}


Nous utilisons la fonction logique \CodeInline{\textbackslash{}at(v, Label)} qui
nous donne la valeur de la variable \CodeInline{v} au point de programme
\CodeInline{Label}. Si nous regardons l'utilisation qui en est faite ici, nous
voyons que dans l'invariant de boucle, nous cherchons à établir une relation
entre les anciennes valeurs du tableau et leurs potentielles nouvelles valeurs :


\begin{CodeBlock}{c}
/*@
  loop invariant \forall size_t j; 0 <= j < i && \at(array[j], Pre) == old
                   ==> array[j] == new;
  loop invariant \forall size_t j; 0 <= j < i && \at(array[j], Pre) != old
                   ==> array[j] == \at(array[j], Pre);
*/
\end{CodeBlock}


Pour tout élément que nous avons visité, s'il valait la valeur qui doit être
remplacée, alors il vaut la nouvelle valeur, sinon il n'a pas changé. Alors que
nous nous reposions sur la clause \CodeInline{assigns} pour ce genre de propriété
dans les exemples précédents, ici nous ne pouvons pas le faire. Même si ACSL nous
permettrait d'exprimer cette propriété de manière très précise, WP ne pourrait pas
vraiment en tirer parti, dû à la manière dont cette clause est traitée. Nous devons
donc utiliser un invariant et conserver une approximation des positions mémoire
affectées.


En fait, si nous essayons de prouver l'invariant, WP n'y parvient pas, parce que
la clause \CodeInline{assigns} n'est pas assez précise. Dans cette situation,
nous fournissons un invariant supplémentaire pour détailler dans la plage modifiée
quelles sont les positions en mémoire qui n'ont pas encore été modifiées par la
boucle à une itération donnée :


\begin{CodeBlock}{c}
for(size_t i = 0; i < length; ++i){
    //@assert array[i] == \at(array[i], Pre); // échec de preuve
    if(array[i] == old) array[i] = new;
}
\end{CodeBlock}


Nous pouvons donc ajouter cette information comme invariant :


\CodeBlockInput[13][26]{c}{search_and_replace-1.c}


Et cette fois, la preuve passera.


\levelThreeTitle{Exercices}


Pour tous ces exercices, utiliser la commande suivante pour démarrer la vérification :

\begin{CodeBlock}{bash}
frama-c-gui -wp -wp-rte -warn-unsigned-overflow -warn-unsigned-downcast your-file.c
\end{CodeBlock}


\levelFourTitle{Fonctions sans modification : For all, Exists, ...}


Actuellement, les pointeurs de fonction ne sont pas directement supportés par WP.
Considérons que nous avons une fonction :


\CodeBlockInput[7][13]{c}{ex-1-forall-exists.c}


Écrire un corps (au choix) pour cette fonction et un contrat l'accompagnant.
Ensuite, écrire les fonctions suivantes avec leurs contrats pour prouver leur
correction. Notons qu'il n'est pas possible d'utiliser une fonction C dans un
contrat, la propriété que choisie pour la fonction \CodeInline{pred}
devra donc être inlinée dans la spécification des différentes fonctions.


\begin{itemize}
\item \CodeInline{forall\_pred} retourne vrai si et seulement si \CodeInline{pred}
  est vraie pour tous les éléments ;
\item \CodeInline{exists\_pred} retourne vrai si et seulement si \CodeInline{pred}
  est vraie pour au moins un élément ;
\item \CodeInline{none\_pred} retourne vrai si et seulement si \CodeInline{pred}
  est fausse pour tous les éléments ;
\item \CodeInline{some\_not\_pred} retourne vrai si et seulement si \CodeInline{pred}
  est fausse pour au moins un élément.
\end{itemize}


Les deux dernières fonctions devraient être écrites en appelant seulement les deux
premières.


\levelFourTitle{Fonction sans modification : Égalité de plages de valeurs}


Écrire, spécifier et prouver la fonction \CodeInline{equal} qui retourne vrai
si et seulement si deux plages de valeurs sont égales. Écrire, en utilisant la
fonction \CodeInline{equal}, le code de \CodeInline{different} qui retourne vrai
si et seulement si deux plages de valeurs sont différentes, votre postcondition
devrait contenir un quantificateur existentiel.


\CodeBlockInput[7][13]{c}{ex-2-equal.c}


\levelFourTitle{Fonction sans modification : recherche dichotomique}
\label{l4:statements-loops-ex-bsearch}


La fonction suivante cherche la position d'une valeur fournie en entrée dans
un tableau, en supposant que le tableau est trié. D'abord, considérons que la
longueur du tableau est fournie en tant qu'\CodeInline{int} et utilisons des valeurs de ce
même type pour gérer les indices. Nous pouvons noter qu'il y a deux comportements
à cette fonction : soit la valeur existe dans le tableau (et le résultat est
l'indice de cette valeur) ou pas (et le résultat est -1).


\CodeBlockInput[5]{c}{ex-3-binary-search.c}


Cette fonction est un petit peu complexe à prouver, voici quelques conseils
pour en venir à bout. D'abord, la longueur de la fonction est reçue en utilisant
un type \CodeInline{int}, donc nous devons poser une restriction sur cette longueur en
précondition pour qu'elle soit cohérente. Ensuite, l'un des invariants de la
boucle devrait indiquer les bornes des valeurs \CodeInline{low} et
\CodeInline{up}, mais nous pouvons noter que pour chacune d'elles, l'une des
bornes n'est pas nécessaire. Finalement, la seconde propriété invariante
devrait indiquer que si l'un des indices du tableau correspond à la valeur
recherchée, alors cet indice devrait être correctement borné.


\textbf{Plus dur :} Modifier cette fonction de façon à recevoir \CodeInline{len}
comme un \CodeInline{size\_t}. Il faut modifier légèrement l'algorithme et
la spécification/les invariants. Conseil : s'arranger pour que \CodeInline{up}
soit une borne exclue de la recherche.


\levelFourTitle{Fonction avec modification : addition de vecteurs}


Écrire, spécifier et prouver la fonction qui ajoute deux vecteurs dans un
troisième. Fixer des contraintes arbitraires sur les valeurs d'entrée pour
gérer le débordement des entiers. Considérer que le vecteur est résultant est
spatialement séparé des vecteurs d'entrée. En revanche, le même vecteur devrait
pouvoir être utilisé pour les deux vecteurs d'entrée.


\CodeBlockInput[7][9]{c}{ex-4-add-vectors.c}


\levelFourTitle{Fonction avec modification : inverse}


Écrire, spécifier et prouver la fonction qui inverse un vecteur en place.
Prendre garde à la partie non modifiée du vecteur à une itération donnée de la
boucle. Utiliser la fonction \CodeInline{swap} précédemment prouvée.


\CodeBlockInput[7][11]{c}{ex-5-reverse.c}


\levelFourTitle{Fonction avec modification : copie}


Écrire, spécifier et prouver la fonction \CodeInline{copy} qui copie une plage
de valeur dans un autre tableau, en commençant pas la première cellule du
tableau. Considérer (et spécifier) d'abord que les deux plages sont entièrement
séparées.


\CodeBlockInput[7][9]{c}{ex-6-copy.c}


\textbf{Plus dur :} Les vraies fonctions \CodeInline{copy} et
\CodeInline{copy\_backward}.


En fait, une séparation aussi forte n'est pas nécessaire. Pour faire une copie
en partant du début, la précondition réelle doit simplement garantir que si les
deux plages se chevauchent en mémoire, le début de la destination ne doit pas être
dans la plage source :


\begin{CodeBlock}{c}
//@ requires \separated(&src[0 .. len-1], dst) ;
\end{CodeBlock}


Essentiellement, en copiant des éléments dans cet ordre, nous pouvons les
décaler depuis la fin d'une plage vers le début. En revanche, cela signifie que
nous devons être plus précis dans notre contrat : nous ne garantissons plus une
égalité avec le tableau source, mais avec les \emph{anciennes} valeurs du tableau
source. Nous devons également être plus précis dans notre invariant, d'abord en
spécifiant aussi la relation avec l'état précédent de la mémoire, et ensuite en
ajoutant un invariant qui nous dit que le tableau source n'est pas modifié entre
le \CodeInline{i}$^{ème}$ élément visité et le dernier.


Finalement, il est aussi possible d'écrire une fonction qui copie les éléments de
la fin vers le début. Dans ce cas, à nouveau, les plages de valeurs peuvent se
chevaucher, mais la condition n'est pas exactement la même. Écrire, spécifier et
prouver la fonction \CodeInline{copy\_backward} qui copie les éléments dans le
sens inverse.

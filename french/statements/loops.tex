Les boucles ont besoin d'un traitement de faveur dans la vérification déductive
de programmes. Ce sont les seules structures de contrôle qui vont nécessiter un
travail conséquent de notre part. Nous ne pouvons pas y échapper car sans les
boucles nous pouvons difficilement prouver des programmes intéressants.



Avant de s'intéresser à la spécification des boucles, il est légitime de se
poser la question suivante : pourquoi les boucles sont-elles compliquées ?



\levelThreeTitle{Induction et invariance}
\label{l3:statements-loops-invariant}


La nature des boucles rend leur analyse difficile. Lorsque nous faisons nos
raisonnements arrières, il nous faut une règle capable de dire à partir de la
postcondition quelle est la précondition d'une certaine séquence
d'instructions. Problème : nous ne pouvons \textit{a priori} pas déduire combien de
fois la boucle s'exécutera et donc, nous ne pouvons pas non
plus savoir combien de fois les variables ont été modifiées.



Nous procéderons en raisonnant par induction. Nous devons trouver une
propriété qui est vraie avant de commencer la boucle et qui, si elle est vraie
au début d'un tour de boucle, sera vraie à la fin (et donc par extension, au
début du tour suivant). Quand la boucle termine, nous gagnons la connaissance
que la condition de boucle est fausse qui, en conjonction avec l'invariant,
doit nous permettre de prouver que la postcondition de la boucle est vérifiée.



Ce type de propriété est appelé un invariant de boucle. Un invariant de boucle
est une propriété qui doit être vraie avant et après chaque tour d'une boucle,
et plus précisément chaque fois que l'on évalue la condition de la boucle.
Par exemple, pour la boucle :



\begin{CodeBlock}{c}
for(int i = 0 ; i < 10 ; ++i){ /* */ }
\end{CodeBlock}



La propriété $0 \leq \texttt{i} \leq 10$ est un invariant de la boucle. La
propriété  $-42 \leq \texttt{i} \leq 42$ est également un invariant de la boucle
(qui est nettement plus imprécis néanmoins). La propriété $0 < \texttt{i} \leq 10$
n'est pas un invariant, elle n'est pas vraie à l'entrée de la boucle. La propriété
$0 \leq \texttt{i} < 10$ \textbf{n'est pas un invariant de la boucle}, elle n'est
pas vraie à la fin du dernier tour de la boucle qui met la valeur $\texttt{i}$ à
$10$.



Le raisonnement produit par l'outil pour vérifier un invariant $I$ sera donc :



\begin{itemize}
\item vérifions que $I$ est vrai au début de la boucle (établissement),
\item vérifions que si $I$ est vrai avant de commencer un tour, alors $I$ est vrai après (préservation).
\end{itemize}


\levelFourTitle{Formel - Règle d'inférence}


En notant l'invariant $I$, la règle d'inférence correspondant à la boucle est
définie comme suit :
$$\dfrac{\{I \wedge B \}\ c\ \{I\}}{\{I\}\ while(B)\{c\}\ \{I \wedge \neg B\}}$$


Et le calcul de plus faible précondition est le suivant :
$$wp(while (B) \{ c \}, Post) := I \wedge ((B \wedge I) \Rightarrow wp(c, I)) \wedge ((\neg B \wedge I) \Rightarrow Post)$$


Détaillons cette formule :



\begin{itemize}
\item (1) le premier $I$ correspond à l'établissement de l'invariant, c'est
vulgairement la « précondition » de la boucle,
\item la deuxième partie de la conjonction ($(B \wedge I) \Rightarrow wp(c, I)$)
correspond à la vérification du travail effectué par le corps de la boucle :

\begin{itemize}
\item la précondition que nous connaissons du corps de la boucle (notons $KWP$,
« \textit{Known WP} ») est ($KWP = B \wedge I$). Soit le fait que nous sommes
rentrés dedans ($B$ est vrai), et que l'invariant est respecté à ce moment
($I$, qui est vrai avant de commencer la boucle par (1), et dont veut
vérifier qu'il sera vrai en fin de bloc de la boucle (2)),
\item (2) ce qu'il nous reste à vérifier c'est que $KWP$ implique la
précondition réelle* du bloc de code de la boucle
  ($KWP \Rightarrow wp(c, Post)$). Ce que nous voulons en fin de bloc,
  c'est avoir maintenu l'invariant $I$ ($B$ n'est peut-être plus vrai en
  revanche). Donc
$KWP \Rightarrow wp(c, I)$, soit $(B \wedge I) \Rightarrow wp(c, I)$,
\item * cela correspond à une application de la règle de conséquence expliquée
précédemment.
\end{itemize}
\item finalement, la dernière partie ($(\neg B \wedge I) \Rightarrow Post$)
nous dit que le fait d'être sorti de la boucle ($\neg B$), tout en ayant
maintenu l'invariant $I$, doit impliquer la postcondition voulue pour la
boucle.
\end{itemize}


Dans ce calcul, nous pouvons noter que la fonction $wp$ ne nous donne aucune
indication sur le moyen d'obtenir l'invariant $I$. Nous allons donc devoir
spécifier manuellement de telles propriétés à propos de nos boucles.



\levelFourTitle{Retour à l'outil}


Il existe des outils capables d'inférer des invariants (pour peu qu'ils soient
simples, les outils automatiques restent limités). Ce n'est pas le cas de WP.
Il nous faut donc écrire nos invariants à la main. Trouver et écrire les
invariants des boucles de nos programmes sera toujours la partie la plus difficile
de notre travail lorsque nous chercherons à prouver des programmes.



En effet, si en l'absence de boucle, la fonction de calcul de plus faible
précondition peut nous fournir automatiquement les propriétés vérifiables de nos
programmes, ce n'est pas le cas pour les invariants de boucle pour lesquels
nous n'avons pas accès à une procédure automatique de calcul. Nous devons donc
trouver et formuler correctement ces invariants, et selon l'algorithme, celui-ci
peut parfois être très subtil.



Pour indiquer un invariant à une boucle, nous ajoutons les annotations suivantes
en début de boucle :



\CodeBlockInput[5]{c}{first_loop-1.c}



\begin{Warning}
\textbf{RAPPEL} : L'invariant est bien : i \textbf{$\leq$} 30 !
\end{Warning}


Pourquoi ? Parce que tout au long de la boucle \CodeInline{i} sera bien compris entre
0 et 30 \textbf{inclus}. 30 est même la valeur qui nous permettra de sortir de la
boucle. Plus encore, une des propriétés demandées par le calcul de plus faible
préconditions sur les boucles est que lorsque l'on invalide la condition de la
boucle, par la connaissance de l'invariant, on peut prouver la postcondition
(Formellement : $(\neg B \wedge I) \Rightarrow Post$).



La postcondition de notre boucle est $\texttt{i} = 30$ et doit être impliquée par
$\neg \texttt{i} < 30 \wedge 0 \leq \texttt{i} \leq 30$. Ici, cela fonctionne
bien :
$$\texttt{i} \geq 30 \wedge 0 \leq \texttt{i} \leq 30 \Rightarrow \texttt{i} = 30$$
Si l'invariant excluait l'égalité à 30, la postcondition ne serait pas atteignable.



Une nouvelle fois, nous pouvons jeter un œil à la liste des buts dans « \textit{WP
Goals} » :



\image{i_30-1}


Nous remarquons bien que WP décompose la preuve de l'invariant en deux parties :
l'établissement de l'invariant et sa préservation. WP produit exactement le
raisonnement décrit plus haut pour la preuve de l'assertion. Dans les versions
récentes de Frama-C, Qed est devenu particulièrement puissant, et l'obligation de
preuve générée ne le montre pas (affichant simplement « \textit{True} »). En utilisant
l'option \CodeInline{-wp-no-simpl} au lancement, nous pouvons quand même voir
l'obligation correspondante :



\image{i_30-2}


Mais notre spécification est-elle suffisante ?



\CodeBlockInput{c}{first_loop-2.c}



Voyons le résultat :



\image{i_30-3}


Il semble que non.



\levelThreeTitle{La clause « assigns » ... pour les boucles}


En fait, à propos des boucles, WP ne considère vraiment \textit{que} ce que lui
fournit l'utilisateur pour faire ses déductions. Et ici l'invariant ne nous dit
rien à propos de l'évolution de la valeur de \CodeInline{h}. Nous pourrions signaler
l'invariance de toute variable du programme mais ce serait beaucoup d'efforts.
ACSL nous propose plus simplement d'ajouter des annotations \CodeInline{assigns} pour
les boucles. Toute autre variable est considérée invariante. Par exemple :


\CodeBlockInput{c}{first_loop-3.c}



Cette fois, nous pouvons établir la preuve de correction de la boucle. Par contre,
rien ne nous prouve sa terminaison. L'invariant de boucle n'est pas suffisant pour
effectuer une telle preuve. Par exemple, dans notre programme, si nous réécrivons
la boucle comme ceci :



\begin{CodeBlock}{c}
/*@
  loop invariant 0 <= i <= 30;
  loop assigns i;
*/
while(i < 30){

}
\end{CodeBlock}



L'invariant est bien vérifié également, mais nous ne pourrons jamais prouver
que la boucle se termine : elle est infinie.



\levelThreeTitle{Correction partielle et correction totale - Variant de boucle}


En vérification déductive, il y a deux types de correction, la correction
partielle et la correction totale. Dans le premier cas, la formulation est
« si la précondition est validée et \textbf{si} le calcul termine, alors la
postcondition est validée ». Dans le second cas, « si la précondition est
validée, alors le calcul termine et la postcondition est validée ». WP
s'intéresse par défaut à de la preuve de correction partielle :



\CodeBlockInput{c}{infinite.c}


Si nous demandons la vérification de ce code en activant le contrôle de RTE,
nous obtenons ceci :



\image{infinite}


L'assertion « FAUX » est prouvée ! La raison est simple : la non-terminaison de
la boucle est triviale : la condition de la boucle est « VRAI » et aucune instruction
du bloc de la boucle ne permet d'en sortir puisque le bloc ne contient pas de code du
tout. Comme nous sommes en correction partielle, et que le calcul ne termine pas, nous
pouvons prouver n'importe quoi au sujet du code qui suit la partie non terminante. Si,
en revanche, la non-terminaison est non-triviale, il y a peu de chances que l'assertion
soit prouvée.



\begin{Information}
À noter qu'une assertion inatteignable est toujours prouvée comme vraie de cette
manière :
\image{goto_end}

C'est également le cas lorsque l'on sait trivialement qu'une instruction
produit nécessairement une erreur d'exécution (par exemple en déréférençant
la valeur \CodeInline{NULL}), comme nous avions déjà pu le constater avec l'exemple
de l'appel à \CodeInline{abs} avec la valeur \CodeInline{INT\_MIN}.
\end{Information}


Pour prouver la terminaison d'une boucle, nous utilisons la notion de variant de
boucle. Le variant de boucle n'est pas une propriété mais une valeur. C'est une
expression faisant intervenir des éléments modifiés par la boucle et donnant une
borne supérieure sur le nombre d'itérations restant à effectuer à un tour de la
boucle. C'est donc une expression supérieure à 0 et strictement décroissante d'un
tour de boucle à l'autre (cela sera également vérifié par induction par WP).



Si nous reprenons notre programme précédent, nous pouvons ajouter le variant
de cette façon :



\CodeBlockInput{c}{first_loop-4.c}



Une nouvelle fois nous pouvons regarder les buts générés :



\image{i_30-4}


Le variant nous génère bien deux obligations au niveau de la vérification :
assurer que la valeur est positive, et assurer qu'elle décroît strictement pendant
l'exécution de la boucle. Si nous supprimons la ligne de code qui incrémente
\CodeInline{i}, WP ne peut plus prouver que la valeur \CodeInline{30 - i} décroît strictement.



Il est également bon de noter qu'être capable de donner un variant de boucle
n'induit pas nécessairement d'être capable de donner le nombre exact d'itérations
qui doivent encore être exécutées par la boucle, car nous n'avons pas toujours une
connaissance aussi précise du comportement de notre programme. Nous pouvons par
 exemple avoir un code comme celui-ci :



\CodeBlockInput{c}{random_loop.c}



Ici, à chaque tour de boucle, nous diminuons la valeur de la variable \CodeInline{i} par une
valeur dont nous savons qu'elle se trouve entre 1 et \CodeInline{i}. Nous pouvons donc bien
assurer que la valeur de \CodeInline{i} est positive et décroît strictement, mais nous ne
pouvons pas dire combien de tours de boucles vont être réalisés pendant une
exécution.


Le variant n'est donc bien qu'une borne supérieure sur le nombre d'itérations
de la boucle.


Notons aussi que le variant de boucle n'a besoin d'être positif qu'au début de l'exécution
du bloc de la boucle. Donc, dans le code suivant :


\begin{CodeBlock}{c}
int i = 5 ;
while(i >= 0){
  i -= 2 ;
}
\end{CodeBlock}


Même si \CodeInline{i} peut être négatif lorsque la boucle termine, cette valeur est
bien un variant de la boucle puisque nous n'exécutons pas le corps de la boucle à
nouveau.


\levelThreeTitle{Lier la postcondition et l'invariant}


Supposons le programme spécifié suivant. Notre but est de prouver que le retour
de cette fonction est l'ancienne valeur de \CodeInline{a} à laquelle nous avons ajouté 10.



\CodeBlockInput{c}{add_ten-0.c}



Le calcul de plus faibles préconditions ne permet pas de sortir de la boucle des
informations qui ne font pas partie de l'invariant. Dans un programme comme :



\begin{CodeBlock}{c}
/*@
    ensures \result == \old(a) + 10;
*/
int add_ten(int a){
    ++a;
    ++a;
    ++a;
    //...
    return a;
}
\end{CodeBlock}



en remontant les instructions depuis la postcondition, on conserve toujours les
informations à propos de \CodeInline{a}. À l'inverse, comme mentionné plus tôt, en dehors
de la boucle WP, ne considérera que les informations fournies par notre
invariant. Par conséquent, notre fonction \CodeInline{add\_ten} ne peut pas être prouvée
en l'état : l'invariant ne mentionne rien à propos de \CodeInline{a}. Pour lier notre
postcondition à l'invariant, il faut ajouter une telle information. Par
exemple :



\CodeBlockInput{c}{add_ten-1.c}



\begin{Information}
Ce besoin peut apparaître comme une contrainte très forte. Il ne l'est en fait pas
tant que cela. Il existe des analyses fortement automatiques capables de
calculer les invariants de boucles. Par exemple, sans spécification, une
interprétation abstraite calculera assez facilement \CodeInline{0 <= i <= 10} et
\CodeInline{\textbackslash{}old(a) <= a <= \textbackslash{}old(a)+10}. En revanche, il est souvent bien plus difficile
de calculer les relations qui existent entre des variables différentes qui
évoluent dans le même programme, par exemple l'égalité mentionnée par notre
invariant ajouté.
\end{Information}


\levelThreeTitle{Terminaison prématurée de boucle}


Un invariant de boucle doit être vrai chaque fois que la condition de la boucle est
évaluée. En fait, cela signifie aussi qu'elle doit être vraie avant chaque itération,
et après chaque itération \textbf{complète}. Illustrons cette idée importante avec un
exemple.


\CodeBlockInput{c}{first_loop-non-term-1.c}


Dans cette fonction, quand la boucle atteint l'indice 30, elle effectue une opération
\CodeInline{break} avant que la condition de la boucle soit à nouveau testée.
L'invariant que nous avons écrit est bien sûr vérifié, mais nous pouvons en fait le
restreindre encore.



\CodeBlockInput{c}{first_loop-non-term-2.c}



Ici, nous pouvons voir que nous avons exclu 30 de la plage des valeurs de
\CodeInline{i} et la fonction est correctement vérifiée par WP. Cette propriété
est particulièrement intéressante, car elle ne s'applique pas qu'à l'invariant.
Aucune des propriétés de la boucle n'ont besoin d'être vérifiées pendant l'itération
qui mène au \CodeInline{break}. Par exemple, nous pouvons écrire ce code qui est
également vérifié :



\CodeBlockInput{c}{first_loop-non-term-3.c}


Nous voyons que nous pouvons écrire la variable \CodeInline{h} même si elle
n'est pas listée dans la clause \CodeInline{loop assigns}, et que nous pouvons
donner la valeur 42 à \CodeInline{i} alors que l'invariant l'interdirait, et aussi
que nous pouvons rendre l'expression du variant négative. En fait, tout se passe
exactement comme si nous avions écrit :


\CodeBlockInput{c}{first_loop-non-term-4.c}



C'est un schéma très pratique. Il correspond à tout algorithme qui cherche, à l'aide
d'une boucle, une condition vérifiée par un élément particulier dans une structure
de données et s'arrête quand cet élément est trouvé afin d'effectuer certaines
opérations qui ne sont finalement pas vraiment des opérations de la boucle. D'un
point de vue vérification, cela nous permet de simplifier le contrat associé à
une boucle : nous savons que l'opération (potentiellement complexe) ) réalisée
juste avant de sortir de la boucle ne nécessite pas d'être prise en compte dans
l'invariant.



\levelThreeTitle{Exercice}


\levelFourTitle{Invariant de boucle}

Écrire un invariant de boucle pour la boucle suivante et prouver qu'il est respecté
en utilisant la commande :


\begin{CodeBlock}{text}
frama-c -wp your-file.c
\end{CodeBlock}


\CodeBlockInput[6][10]{c}{ex-1-invariant.c}


Est-ce que la propriété $-100 \leq x \leq 100$ est un invariant correct ?
Expliquer pourquoi.



\levelFourTitle{Loop variant}


Écrire un invariant et un variant corrects pour la boucle suivante et prouver
l'ensemble à l'aide de la commande :

\begin{CodeBlock}{text}
frama-c -wp your-file.c
\end{CodeBlock}


\CodeBlockInput[6][10]{c}{ex-2-variant.c}


Si le variant ne donne pas précisément le nombre d'itérations restantes, ajouter
une variable qui enregistre exactement le nombre d'itérations restantes et l'utiliser
comme variant. Il est possible qu'un invariant supplémentaire soit nécessaire.



\levelFourTitle{Loop assigns}


Écrire une clause \CodeInline{loop assigns} pour cette boucle, de manière à ce
que l'assertion ligne 8 soit prouvée ainsi que la clause \CodeInline{loop assigns}.
Ignorons les erreurs à l'exécution dans cet exercice.



\CodeBlockInput[6][13]{c}{ex-3-assigns.c}


Lorsque la preuve réussit, supprimer la clause \CodeInline{loop assigns} et
trouver un autre moyen d'assurer que l'assertion soit vérifiée en utilisant des
annotations différentes (notons que vous pouvons avoir besoin d'un label C dans
le code). Que peut-on déduire à propos de la clause \CodeInline{loop assigns} ?


\levelFourTitle{Terminaison prématurée}


Écrire un contrat de boucle pour cette boucle qui permette de prouver les
assertions aux lignes 9 et 10 ainsi que le contrat de boucle.


\CodeBlockInput[6][15]{c}{ex-4-early.c}

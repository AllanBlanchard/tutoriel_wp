\levelThreeTitle{Affectation}


L'affectation est l'opération la plus basique que l'on puisse avoir dans un
langage (mise à part l'opération « ne rien faire » qui manque singulièrement
d'intérêt). Le calcul de plus faible précondition associée est le suivant :
$$wp(x = E , Post) := Post[x \leftarrow E]$$
où la notation $P[x \leftarrow E]$ signifie « la propriété $P$ où $x$ est remplacé
par $E$ ». Ce qui correspond ici à « la postcondition $Post$ où $x$ a été
remplacé par $E$ ». Dans l'idée, pour que la formule en postcondition d'une
affectation de $x$ à $E$ soit vraie, il faut qu'en remplaçant chaque occurrence de
$x$ dans la formule par $E$, on obtienne une propriété qui est vraie. Par exemple :



\begin{CodeBlock}{c}
// { P }
x = 43 * c ;
// { x = 258 }
\end{CodeBlock}

$$P = wp(x = 43*c , \{x = 258\}) = \{43*c = 258\}$$


La fonction $wp$ nous permet donc de calculer la plus faible précondition de
l'opération ($\{43*c = 258\}$), ce que l'on peut réécrire sous la forme d'un
triplet de Hoare :



\begin{CodeBlock}{c}
// { 43*c = 258 }
x = 43 * c ;
// { x = 258 }
\end{CodeBlock}



Pour calculer la précondition de l'affectation, nous avons remplacé chaque
occurrence de $x$ dans la postcondition, par la valeur $E = 43*c$ affectée.
Si notre programme était de la forme :
\begin{CodeBlock}{c}
int c = 6 ;
// { 43*c = 258 }
x = 43 * c ;
// { x = 258 }
\end{CodeBlock}
nous pourrions alors fournir la formule « $43*6 = 258$ » à notre prouveur automatique
afin qu'il détermine si cette formule peut effectivement être satisfaite. Ce à quoi
il répondrait évidemment « oui », puisque cette propriété est très simple à vérifier.
En revanche, si nous avions donné la valeur 7 pour \CodeInline{c}, le prouveur nous répondrait
que non, une telle formule n'est pas vraie.



Nous pouvons donc écrire la règle d'inférence pour le triplet de Hoare de
l'affectation, où l'on prend en compte le calcul de plus faible précondition :
$$\dfrac{}{\{Q[x \leftarrow E] \}\quad x = E \quad\{ Q \}}$$


Nous noterons qu'il n'y a pas de prémisse à vérifier. Cela veut-il dire que le
triplet est nécessairement vrai ? Oui. Mais cela ne dit pas si la précondition
est satisfaite par le programme où se trouve l'instruction, ni que cette
précondition est possible. C'est ce travail qu'effectuent ensuite les prouveurs
automatiques.



Par exemple, nous pouvons demander la vérification de la ligne suivante avec
Frama-C :



\begin{CodeBlock}{c}
int a = 42;
//@ assert a == 42;
\end{CodeBlock}



Ce qui est, bien entendu, prouvé directement par Qed car c'est une simple
application de la règle de l'affectation.



\begin{Information}
Notons que d'après la norme C, l'opération d'affectation est une expression
et non une instruction. C'est ce qui nous permet par exemple d'écrire
\CodeInline{if( (a = foo()) == 42)}. Dans Frama-C, une affectation sera toujours une
instruction. En effet, si une affectation est présente au sein d'une
expression plus complexe, le module de création de l'arbre de syntaxe abstraite
du programme analysé effectue une étape de normalisation qui crée
systématiquement une instruction séparée.
\end{Information}



\levelFourTitle{Affectation de valeurs pointées}


En C, grâce aux (à cause des ?) pointeurs, nous pouvons avoir des programmes avec
des alias, à savoir que deux pointeurs peuvent pointer vers la même position en
mémoire. Notre calcul de plus faible précondition doit donc considérer ce genre
de cas. Par exemple, nous pouvons regarder ce triplet de Hoare :


\begin{CodeBlock}{c}
//@ assert p = q ;
*p = 1 ;
//@ assert *p + *q == 2 ;
\end{CodeBlock}


Ce triplet de Hoare est correct, puisque \CodeInline{p} et \CodeInline{q} sont en
alias, modifier la valeur \CodeInline{*p} modifie aussi la valeur \CodeInline{*q},
par conséquent, ces deux expressions s'évaluent à $1$ et la postcondition est
vraie. Cependant, regardons ce que nous obtenons en appliquant le calcul de plus
faible précondition précédemment défini pour l'affectation sur cet exemple :


\begin{tabular}{ll}
$wp(*p = 1, *p + *q = 2)$ & $= (*p + *q = 2)[*p \leftarrow 1]$\\
                          & $= (1 + *q = 2)$
\end{tabular}


Nous obtenons la plus faible précondition : \CodeInline{1 + *q == 2}, et donc
nous pourrions déduire que la plus faible précondition est \CodeInline{*q == 1}
(ce qui est vrai), mais nous ne sommes pas en mesure de conclure que le programme
est correct, car rien dans notre formule ne nous indique quelque chose comme :
\CodeInline{p == q ==> *q == 1}. En fait, ici, nous voudrions être capable de
calculer une plus faible précondition de la forme :


\begin{tabular}{ll}
$wp(*p = 1, *p + *q = 2)$ & $= (1 + *q = 2 \vee q = p)$\\
                          & $= (*q = 1 \vee q = p)$
\end{tabular}


Pour cela, nous devons faire attention à la notion d'aliasing. Une manière
commune de le faire est de considérer que la mémoire est une variable particulière
du programme (nommons-la $M$) sur laquelle nous pouvons effectuer deux opérations :
obtenir la valeur d'un emplacement particulier $m$ en mémoire (qui nous retourne une
expression) et changer la valeur à une position mémoire $l$ pour y placer une nouvelle
valeur $v$ (qui nous retourne la nouvelle mémoire obtenue).


Notons :


\begin{itemize}
\item $get(M,l)$ par la notation $M[l]$
\item $set(M,l,v)$ par la notation $M[l \mapsto v]$
\end{itemize}


Ces deux opérations peuvent être définies comme suit :


\begin{tabular}{ll}
  $M[l1 \mapsto v][l2] =$ & if $l1   =  l2$ then $v$ \\
                          & if $l1 \neq l2$ then $M[l2]$
\end{tabular}


Si aucune valeur n'est associée à la position mémoire envoyée à $get$,
la valeur est indéfinie (la mémoire est une fonction partielle). Bien sûr, au début
d'une fonction, cette mémoire peut être remplie avec un ensemble de positions mémoire
pour lesquelles nous savons que la valeur a été précédemment définie (par exemple
parce que la spécification de la fonction nous l'indique).


Maintenant, nous pouvons changer légèrement notre calcul de plus faible précondition
pour le cas particulier des affectations à travers un pointeur. Pour cela, nous
considérons que nous avons dans le programme une variable implicite $M$ qui modélise
la mémoire, et nous définissons l'affectation d'une position en mémoire comme une mise
à jour de cette variable, de telle manière à ce que cette position contienne maintenant
l'expression fournie lors de l'affectation :

$$wp(*x = E, Q) = Q[M \leftarrow M[x \mapsto E]]$$

Évaluer une valeur pointée $*x$ dans une formule consiste maintenant à utiliser
l'opération $get$ pour demander la valeur à la mémoire. Nous pouvons donc
appliquer notre calcul de plus faible précondition à notre programme précédent :
\begin{tabular}{lll}
  $wp(*p = 1, *p + *q = 2)$
  & $= (*p + *q = 2)[M \leftarrow M[p \mapsto 1]]$ & (1)\\
  & $= (M[p] + M[q] = 2)[M \leftarrow M[p \mapsto 1]]$ & (2)\\
  & $= (M[p \mapsto 1][p] + M[p \mapsto 1][q] = 2)$ & (3)\\
  & $= (1 + M[p \mapsto 1][q] = 2)$ & (4)\\
  & $= (1 + (\texttt{if}\ q = p\ \texttt{then}\ 1\ \texttt{else}\ M[q]) = 2)$ & (5)\\
  & $= (\texttt{if}\ q = p\ \texttt{then}\ 1+1 = 2\ \texttt{else}\ 1+M[q] = 2)$ & (6)\\
  & $= (q = p \vee M[q] = 1)$ & (7)
\end{tabular}
\begin{enumerate}
\item nous devons appliquer la règle pour l'affectation de valeur pointée, mais
  pour cela, nous devons d'abord introduire $M$,
\item nous remplaçons nos accès de pointeurs par un appel à $get$ sur $M$,
\item nous appliquons le remplacement demandé par la règle d'affectation,
\item nous utilisons la définition de $get$ pour $p$ ($M[p \mapsto 1][p] = 1$),
\item nous utilisons la définition de $get$ pour $q$\\
  ($M[p \mapsto 1][q] = \texttt{if}\ q = p\ \texttt{then}\ 1\ \texttt{else}\ M[q]$)
\item nous appliquons quelques simplifications sur la formule \dots
\item \dots et nous pouvons finalement conclure que $M[q] = 1$ ou $p = q$,
\end{enumerate}
et comme dans notre programme, nous savons que $p = q$, nous pouvons conclure que
le programme est correct.


Le plugin WP ne fonctionne par exactement comme cela. En particulier, cela dépend du
modèle mémoire sélectionné pour réaliser la preuve, qui fait différentes hypothèses à
propos de la manière dont la mémoire est organisée. Pour le modèle mémoire que nous
utilisons, le modèle mémoire typé, WP crée différentes variables pour la mémoire.
Cela dit, regardons tout de même les conditions de vérification générées par WP pour la
postcondition de la fonction \CodeInline{swap} :


\image{memory-model}

Nous pouvons voir, au début de la définition de cette condition de vérification qu'une
variable \CodeInline{Mint\_0} représentant une mémoire pour les valeurs de type entier
a été créée, et que cette mémoire est mise à jour et accédée à l'aide des opérateurs que
nous avons définis précédemment (voir par exemple la définition de la variable
\CodeInline{x\_2}).


\levelThreeTitle{Séquence d'instructions}


Pour qu'une instruction soit valide, il faut que sa précondition nous
permette, par cette instruction, de passer à la postcondition voulue.
Maintenant, nous avons besoin d'enchaîner ce processus d'une
instruction à une autre. L'idée est alors que la postcondition assurée par la
première instruction soit compatible avec la précondition demandée par la
deuxième et que ce processus puisse se répéter pour la troisième instruction,
etc.



La règle d'inférence correspondant à cette idée, utilisant les triplets de
Hoare est la suivante :
$$\dfrac{\{P\}\quad S1 \quad \{R\} \ \ \ \{R\}\quad S2 \quad \{Q\}}{\{P\}\quad S1 ;\ S2 \quad \{Q\}}$$



Pour vérifier que la séquence d'instructions $S1;\ S2$ (NB : où $S1$ et $S2$
peuvent elles-mêmes être des séquences d'instructions), nous passons par une
propriété intermédiaire qui est à la fois la précondition de $S2$ et la
postcondition de $S1$. Cependant, rien ne nous indique pour l'instant
comment obtenir les propriétés $P$ et $R$.



Le calcul de plus faible précondition $wp$ nous dit simplement que la
propriété intermédiaire $R$ est trouvée par calcul de plus faible précondition
de la deuxième instruction. Et que la propriété $P$ est trouvée en calculant la
plus faible précondition de la première instruction. La plus faible précondition
de notre liste d'instruction est donc déterminée comme ceci :
$$wp(S1;\ S2 , Post) := wp(S1, wp(S2, Post) )$$



Le plugin WP de Frama-C fait ce calcul pour nous, c'est pour cela que nous
n'avons pas besoin d'écrire les assertions entre chaque ligne de code.



\begin{CodeBlock}{c}
int main(){
  int a = 42;
  int b = 37;

  int c = a+b; // i:1
  a -= c;      // i:2
  b += a;      // i:3

  //@assert b == 0 && c == 79;
}
\end{CodeBlock}



\levelFourTitle{Arbre de preuve}


Notons que lorsque nous avons plus de deux instructions, nous pouvons simplement
considérer que la dernière instruction est la seconde instruction de notre règle
et que toutes les instructions qui la précèdent forment la première « instruction ».
De cette manière, nous remontons bien les instructions une à une dans notre
raisonnement, par exemple avec le programme précédent :


\begin{center}
\begin{tabular}{ccc}
  $\{P\}\quad i_1 ; \quad \{Q_{-2}\}$ & $\{Q_{-2}\}\quad i_2 ; \quad \{Q_{-1}\}$ & \\
  \cline{1-2}
  \multicolumn{2}{c}{$\{P\}\quad i\_1 ; \quad i\_2 ; \quad \{Q_{-1}\}$} & $\{Q_{-1}\} \quad i_3 ; \quad \{Q\}$\\
  \hline
  \multicolumn{3}{c}{$\{P\}\quad i\_1 ; \quad i\_2 ; \quad i\_3; \quad \{ Q \}$}
\end{tabular}
\end{center}

Nous pouvons par calcul de plus faibles préconditions construire la propriété
$Q_{-1}$ à partir de $Q$ et $i_3$, ce qui nous permet de déduire $Q_{-2}$, à
partir de $Q_{-1}$ et $i_2$ et finalement $P$ avec $Q_{-2}$ et $i_1$.



Nous pouvons maintenant vérifier des programmes comprenant plusieurs
instructions ; il est temps d'y ajouter un peu de structure.



\levelThreeTitle{Règle de la conditionnelle}


Pour qu'un branchement conditionnel soit valide, il faut que la postcondition
soit atteignable par les deux banches, depuis la même précondition, à ceci
près que chacune des branches aura une information supplémentaire : le fait
que la condition était vraie dans un cas et fausse dans l'autre.



Comme avec la séquence d'instructions, nous aurons donc deux points à vérifier
(pour éviter de confondre les accolades, j'utilise la syntaxe
$if\ B\ then\ S1\ else\ S2$) :
$$\dfrac{\{P \wedge B\}\quad S1\quad \{Q\} \quad \quad \{P \wedge \neg B\}\quad S2\quad \{Q\}}{\{P\}\quad if\quad B\quad then\quad S1\quad else\quad S2 \quad \{Q\}}$$



Nos deux prémisses sont donc la vérification que lorsque nous avons la
précondition et que nous passons dans la branche \CodeInline{if}, nous atteignons bien la
postcondition, et que lorsque nous avons la précondition et que nous passons
dans la branche \CodeInline{else}, nous obtenons bien également la postcondition.



Le calcul de précondition de $wp$ pour la conditionnelle est le suivant :
$$wp(if\ B\ then\ S1\ else\ S2 , Post) := (B \Rightarrow wp(S1, Post)) \wedge (\neg B \Rightarrow wp(S2, Post))$$


À savoir que $B$ doit impliquer la précondition la plus faible de $S1$, pour
pouvoir l'exécuter sans erreur vers la postcondition, et que $\neg B$ doit
impliquer la précondition la plus faible de $S2$ (pour la même raison).



\levelFourTitle{Bloc \CodeInline{else} vide}


En suivant cette définition, si le \CodeInline{else} ne fait rien, alors la règle
d'inférence est de la forme suivante, en remplaçant $S2$ par une instruction
« ne rien faire ».
$$\dfrac{\{P \wedge B\}\quad S1\quad \{Q\} \quad \quad \{P \wedge \neg B\}\quad skip\quad \{Q\}}{\{P\}\quad if\quad B\quad then\quad S1\quad else\quad skip \quad \{Q\}}$$



Le triplet pour le \CodeInline{else} est :
$$\{P \wedge \neg B\}\quad skip\quad \{Q\}$$



Ce qui veut dire que nous devons avoir :
$$P \wedge \neg B \Rightarrow Q$$



En résumé, si la condition du \CodeInline{if} est fausse, cela veut dire que la
postcondition de l'instruction conditionnelle globale est déjà vérifiée avant de
rentrer dans le \CodeInline{else} (puisqu'il ne fait rien).



Par exemple, nous pourrions vouloir remettre une configuration $c$ à une valeur
par défaut si elle a mal été configurée par un utilisateur du programme :



\begin{CodeBlock}{c}
int c;

// ... du code ...

if(c < 0 || c > 15){
  c = 0;
}
//@ assert 0 <= c <= 15;
\end{CodeBlock}



Soit :



$wp(if \neg (c \in [0;15])\ then\ c := 0, \{c \in [0;15]\})$



$:= (\neg (c \in [0;15])\Rightarrow wp(c := 0, \{c \in [0;15]\})) \wedge (c \in [0;15]\Rightarrow wp(skip, \{c \in [0;15]\}))$



$= (\neg (c \in [0;15]) \Rightarrow 0 \in [0;15]) \wedge (c \in [0;15] \Rightarrow c \in [0;15])$



$= (\neg (c \in [0;15]) \Rightarrow true) \wedge true$



La formule est bien vérifiable : quelle que soit l'évaluation de $\neg (c \in [0;15])$ l'implication sera vraie.


\levelThreeTitle{Bonus Stage - Règle de conséquence}
\label{l3:statements-basic-consequence}



Parfois, il peut être utile pour la preuve de renforcer une postcondition ou
d'affaiblir une précondition. Si la première sera souvent établie par nos soins
pour faciliter le travail du prouveur, la seconde est plus souvent vérifiée
par l'outil à l'issue du calcul de plus faible précondition.



La règle d'inférence en logique de Hoare est la suivante :
$$\dfrac{P \Rightarrow WP \quad \{WP\}\quad c\quad \{SQ\} \quad SQ \Rightarrow Q}{\{P\}\quad c \quad \{Q\}}$$



(Nous noterons que les prémisses, ici, ne sont pas seulement des triplets de
Hoare mais également des formules à vérifier)



Par exemple, si notre postcondition est trop complexe, elle risque de générer
une plus faible précondition trop compliquée et de rendre le calcul des
prouveurs difficile. Nous pouvons alors créer une postcondition intermédiaire
$SQ$, plus simple, mais plus restreinte et impliquant la vraie postcondition.
C'est la partie $SQ \Rightarrow Q$.



Inversement, le calcul de précondition générera généralement une formule
compliquée et souvent plus faible que la précondition que nous souhaitons
accepter en entrée. Dans ce cas, c'est notre outil qui s'occupera de vérifier
l'implication entre ce que nous voulons et ce qui est nécessaire pour que notre
code soit valide. C'est la partie $P \Rightarrow WP$.



Nous pouvons par exemple illustrer cela avec le code qui suit. Notons bien qu'ici,
le code pourrait tout à fait être prouvé par l'intermédiaire de WP sans ajouter des
affaiblissements et renforcements de propriétés, car le code est très simple, il
s'agit juste d'illustrer la règle de conséquence.



\begin{CodeBlock}{c}
/*@
  requires P: 2 <= a <= 8;
  ensures  Q: 0 <= \result <= 100 ;
  assigns  \nothing ;
*/
int constrained_times_10(int a){
  //@ assert P_imply_WP: 2 <= a <= 8 ==> 1 <= a <= 9 ;
  //@ assert WP:         1 <= a <= 9 ;

  int res = a * 10;

  //@ assert SQ:         10 <= res <= 90 ;
  //@ assert SQ_imply_Q: 10 <= res <= 90 ==> 0 <= res <= 100 ;

  return res;
}
\end{CodeBlock}



(À noter ici : nous avons omis les contrôles de débordement d'entiers).



Ici, nous voulons avoir un résultat compris entre 0 et 100. Mais nous savons que
le code ne produira pas un résultat sortant des bornes 10 à 90. Donc nous
renforçons la postcondition avec une assertion que \CodeInline{res}, le résultat, est compris
entre 0 et 90 à la fin. Le calcul de plus faible précondition, sur cette propriété,
et avec l'affectation \CodeInline{res = 10*a} nous produit une plus faible précondition
\CodeInline{1 <= a <= 9} et nous savons finalement que \CodeInline{2 <= a <= 8} nous donne cette garantie.



Quand une preuve a du mal à être réalisée sur un code plus complexe, écrire des
assertions produisant des postconditions plus fortes, mais qui forment des formules
plus simples peut souvent nous aider. Notons que dans le code précédent, les lignes
\CodeInline{P\_imply\_WP} et \CodeInline{SQ\_imply\_Q} ne sont jamais utiles, car c'est le raisonnement par
défaut produit par WP, elles sont juste présentes pour l'illustration.



\levelThreeTitle{Bonus Stage - Règle de constance}
\label{l3:statements-basic-constancy}


Certaines séquences d'instructions peuvent concerner et faire intervenir des
variables différentes. Ainsi, il peut arriver que nous initialisions et manipulions
un certain nombre de variables, que nous commencions à utiliser certaines d'entre
elles, puis les délaissions au profit d'autres pendant un temps. Quand un
tel cas apparaît, nous avons envie que l'outil ne se préoccupe que des variables
qui sont susceptibles d'être modifiées pour avoir des propriétés les plus légères
possibles.



La règle d'inférence qui définit ce raisonnement est la suivante :
$$\dfrac{\{P\}\quad c\quad \{Q\}}{\{P \wedge R\}\quad c\quad \{Q \wedge R\}}$$
où $c$ ne modifie aucune variable entrant en jeu dans $R$. Ce qui nous dit : « pour
vérifier le triplet, débarrassons-nous des parties de la formule qui concernent des
variables qui ne sont pas manipulées par $c$ et prouvons le nouveau triplet ».
Cependant, il faut prendre garde à ne pas supprimer trop d'information, au risque
de ne plus pouvoir prouver nos propriétés.



Par exemple, nous pouvons imaginer le code suivant (une nouvelle fois, nous omettons
les contrôles de débordements au niveau des entiers) :



\begin{CodeBlock}{c}
/*@
  requires a > -99 ;
  requires b > 100 ;
  ensures  \result > 0 ;
  assigns  \nothing ;
*/
int foo(int a, int b){
  if(a >= 0){
    a++ ;
  } else {
    a += b ;
  }
  return a ;
}
\end{CodeBlock}



Si nous regardons le code du bloc \CodeInline{if}, il ne fait pas intervenir la variable
\CodeInline{b}, donc nous pouvons omettre complètement les propriétés à propos de  \CodeInline{b} pour
réaliser la preuve que \CodeInline{a} sera bien supérieur à 0 après l'exécution du bloc :



\begin{CodeBlock}{c}
/*@
  requires a > -99 ;
  requires b > 100 ;
  ensures  \result > 0 ;
  assigns  \nothing ;
*/
int foo(int a, int b){
  if(a >= 0){
    //@ assert a >= 0; //et rien à propos de b
    a++ ;
  } else {
    a += b ;
  }
  return a ;
}
\end{CodeBlock}



En revanche, dans le bloc \CodeInline{else}, même si \CodeInline{b} n'est pas modifiée, établir
des propriétés seulement à propos de \CodeInline{a} rendrait notre preuve impossible (en
tant qu'humains). Le code serait :



\begin{CodeBlock}{c}
/*@
  requires a > -99 ;
  requires b > 100 ;
  ensures  \result > 0 ;
  assigns  \nothing ;
*/
int foo(int a, int b){
  if(a >= 0){
    //@ assert a >= 0; // et rien à propos de b
    a++ ;
  } else {
    //@ assert a < 0 && a > -99 ; // et rien à propos de b
    a += b ;
  }
  return a ;
}
\end{CodeBlock}



Dans le bloc \CodeInline{else}, n'ayant que connaissance du fait que \CodeInline{a} est compris
entre -99 et 0, et ne sachant rien à propos de \CodeInline{b}, nous pourrions
difficilement savoir si le calcul \CodeInline{a += b} produit une valeur supérieure
stricte à 0 pour \CodeInline{a}.



Naturellement ici, WP prouvera la fonction sans problème, puisqu'il transporte
de lui-même les propriétés qui lui sont nécessaires pour la preuve. En fait,
l'analyse des variables qui sont nécessaires ou non (et l'application, par
conséquent de la règle de constance) est réalisée directement par WP.



Notons finalement que la règle de constance est une instance de la règle de
conséquence :
$$\dfrac{P \wedge R \Rightarrow P \quad \{P\}\quad c\quad \{Q\} \quad Q \Rightarrow Q \wedge R}{\{P \wedge R\}\quad c\quad \{Q \wedge R\}}$$



Si les variables de $R$ n'ont pas été modifiées par l'opération (qui par contre,
modifie les variables de $P$ pour former $Q$), alors effectivement
$P \wedge R \Rightarrow P$ et $Q \Rightarrow Q \wedge R$.



\levelThreeTitle{Exercices}


\levelFourTitle{Une série d'affectations}


Calculer à la main la plus faible précondition du programme suivant :


\begin{CodeBlock}{c}
/*@
  requires -10 <= x <= 0 ;
  requires 0 <= y <= 5 ;
  ensures -10 <= \result <= 10 ;
*/
int function(int x, int y){
  int res ;
  y += 10 ;
  x -= 5 ;
  res = x + y ;
  return res ;
}
\end{CodeBlock}


En utilisant la bonne règle d'inférence, en déduire que le programme est
conforme au contrat fixé pour cette fonction.


\levelFourTitle{Branche « \textit{then} » vide dans une conditionnelle}


Nous avons précédemment montré que lorsqu'une structure conditionnelle a une branche
« \textit{else} » vide, cela signifie que la conjonction de la précondition et de la négation
de la condition de notre structure conditionnelle est suffisante pour prouver la
postcondition de la structure conditionnelle.
Pour les deux questions qui suivent, nous avons uniquement besoin des règles d'inférence
et pas du calcul de plus faible précondition.


Montrer que lorsque, au lieu de la branche « \textit{else} », c'est la branche « \textit{then} » qui est
vide, la postcondition de structure conditionnelle est vérifiée par la conjonction de
la précondition et de la condition de notre structure conditionnelle (puisque
la branche « \textit{else} » est la seule à potentiellement modifier l'état de
la mémoire).


Montrer que si les deux branches sont vides, la structure conditionnelle est juste une
instruction \textit{skip}.


\levelFourTitle{Court-circuit (\textit{Short circuit})}


Les compilateurs C implémentent le court-circuit pour les conditions (c'est d'ailleurs
imposé par le standard C). Par exemple, cela signifie qu'un code comme (\textbf{sans
bloc « \textit{else} »}) :


\begin{CodeBlock}{c}
if(cond1 && cond2){
  // code
}
\end{CodeBlock}


peut être réécrit comme :


\begin{CodeBlock}{c}
if(cond1){
  if(cond2){
    // code
  }
}
\end{CodeBlock}


Montrer que sur ces deux morceaux de code, le calcul de plus faible précondition
génère une plus faible précondition pour tout code qui se trouverait dans le bloc
« \textit{then} ». Notons que nous supposons que les conditions sont pures (ne modifient aucune
position en mémoire).



\levelFourTitle{Un plus gros programme}


Calculer à la main la plus faible précondition du programme suivant :


\begin{CodeBlock}{c}
/*@
  requires -5 <= y <= 5 ;
  requires -5 <= x <= 5 ;
  ensures  -15 <= \result <= 25 ;
*/
int function(int x, int y){
  int res ;

  if(x < 0){
    x = 0 ;
  }

  if(y < 0){
    x += 5 ;
  } else {
    x -= 5 ;
  }

  res = x - y ;

  return res ;
}
\end{CodeBlock}



En utilisant la bonne règle d'inférence, en déduire que le programme est
conforme au contrat fixé pour cette fonction.

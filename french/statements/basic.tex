\levelThreeTitle{Affectation}


L'affectation est l'opération la plus basique que l'on puisse avoir dans un 
langage (mise à part l'opération « ne rien faire » qui manque singulièrement 
d'intérêt). Le calcul de plus faible pré-condition associé est le suivant :




\begin{center}
$wp(x = E , Post) := Post[x \leftarrow E]$


\end{center}


Où la notation $P[x \leftarrow E]$ signifie « la propriété $P$ où $x$ est remplacé
par $E$ ». Ce qui correspond ici à « la post-condition $Post$ où $x$ a été
remplacé par $E$ ». Dans l'idée, pour que la formule en post-condition d'une 
affectation de $x$ à $E$ soit vraie, il faut qu'elle soit vraie en remplaçant 
chaque occurrence de $x$ dans la formule par $E$. Par exemple :



\begin{CodeBlock}{c}
// { P }
x = 43 * c ;
// { x = 258 }
\end{CodeBlock}




\begin{center}
$P = wp(x = 43*c , \{x = 258\}) = \{43*c = 258\}$


\end{center}


La fonction $wp$ nous permet donc de calculer la plus faible pré-condition de
l'opération ($\{43*c = 258\}$), ce que l'on peut réécrire sous la forme d'un
triplet de Hoare :



\begin{CodeBlock}{c}
// { 43*c = 258 }
x = 43 * c ;
// { x = 258 }
\end{CodeBlock}



Pour calculer la pré-condition de l'affectation, nous avons remplacé chaque 
occurrence de $x$ dans la post-condition, par la valeur $E = 43*c$ affectée.
Si notre programme était de la forme:



\begin{CodeBlock}{c}
int c = 6 ;
// { 43*c = 258 }
x = 43 * c ;
// { x = 258 }
\end{CodeBlock}



Nous pourrions alors fournir la formule « $43*6 = 258$ » à notre prouveur automatique
afin qu'il détermine si cette formule peut effectivement être satisfaite. Ce à quoi
il répondrait évidemment « oui » puisque cette propriété est très simple à vérifier.
En revanche, si nous avions donné la valeur 7 pour \CodeInline{c}, le prouveur nous répondrait
que non, une telle formule n'est pas vraie.



Nous pouvons donc écrire la règle d'inférence pour le triplet de Hoare de 
l'affectation, où l'on prend en compte le calcul de plus faible pré-condition :




\begin{center}
$\dfrac{}{\{Q[x \leftarrow E] \}\quad x = E \quad\{ Q \}}$


\end{center}


Nous noterons qu'il n'y a pas de prémisse à vérifier. Cela veut-il dire que le
triplet est nécessairement vrai ? Oui. Mais cela ne dit pas si la pré-condition 
est respectée par le programme où se trouve l'instruction, ni que cette 
pré-condition est possible. C'est ce travail qu'effectuent ensuite les prouveurs
automatiques.



Par exemple, nous pouvons demander la vérification de la ligne suivante avec 
Frama-C :



\begin{CodeBlock}{c}
int a = 42;
//@ assert a == 42;
\end{CodeBlock}



Ce qui est, bien entendu, prouvé directement par Qed car c'est une simple 
application de la règle de l'affectation.



\begin{Information}
Notons que d'après la norme C, l'opération d'affectation est une expression
et non une instruction. C'est ce qui nous permet par exemple d'écrire 
\CodeInline{if( (a = foo()) == 42)}. Dans Frama-C, une affectation sera toujous une
instruction. En effet, si une affectation est présente au sein d'une 
expression plus complexe, le module de création de l'arbre de syntaxe abstraite
du programme analysé effectue une étape de normalisation qui crée 
systématiquement une instruction séparée.
\end{Information}


\levelThreeTitle{Séquence d'instructions}


Pour qu'une instruction soit valide, il faut que sa pré-condition nous 
permette, par cette instruction, de passer à la post-condition voulue. 
Maintenant, nous avons besoin d'enchaîner ce processus d'une 
instruction à une autre. L'idée est alors que la post-condition assurée par la
première instruction soit compatible avec la pré-condition demandée par la 
deuxième et que ce processus puisse se répéter pour la troisième instruction, 
etc.



La règle d'inférence correspondant à cette idée, utilisant les triplets de 
Hoare est la suivante:




\begin{center}
$\dfrac{\{P\}\quad S1 \quad \{R\} \ \ \ \{R\}\quad S2 \quad \{Q\}}{\{P\}\quad S1 ;\ S2 \quad \{Q\}}$


\end{center}


Pour vérifier que la séquence d'instructions $S1;\ S2$ (NB : où $S1$ et $S2$ 
peuvent elles-mêmes être des séquences d'instructions), nous passons par une 
propriété intermédiaire qui est à la fois la pré-condition de $S2$ et la 
post-condition de $S1$. Cependant, rien ne nous indique pour l'instant 
comment obtenir les propriétés $P$ et $R$.



Le calcul de plus faible pré-condition $wp$ nous dit simplement que la 
propriété intermédiaire $R$ est trouvée par calcul de plus faible pré-condition
de la deuxième instruction. Et que la propriété $P$ est trouvée en calculant la
plus faible pré-condition de la première instruction. La plus faible pré-condition
de notre liste d'instruction est donc déterminée comme ceci :




\begin{center}
$wp(S1;\ S2 , Post) := wp(S1, wp(S2, Post) )$


\end{center}


Le plugin WP de Frama-C fait ce calcul pour nous, c'est pour cela que nous 
n'avons pas besoin d'écrire les assertions entre chaque ligne de code.



\begin{CodeBlock}{c}
int main(){
  int a = 42;
  int b = 37;

  int c = a+b; // i:1
  a -= c;      // i:2
  b += a;      // i:3

  //@assert b == 0 && c == 79;
}
\end{CodeBlock}



\levelFourTitle{Arbre de preuve}


Notons que lorsque nous avons plus de deux instructions, nous pouvons simplement
considérer que la dernière instruction est la seconde instruction de notre règle
et que toutes les instructions qui la précède forment la première « instruction ». 
De cette manière nous remontons bien les instructions une à une dans notre
raisonnement, par exemple avec le programme précédent :


\begin{center}
\begin{tabular}{ccc}
  $\{P\}\quad i_1 ; \quad \{Q_{-2}\}$ & $\{Q_{-2}\}\quad i_2 ; \quad \{Q_{-1}\}$ & \\
  \cline{1-2}
  \multicolumn{2}{c}{$\{P\}\quad i\_1 ; \quad i\_2 ; \quad \{Q_{-1}\}$} & $\{Q_{-1}\} \quad i_3 ; \quad \{Q\}$\\
  \hline
  \multicolumn{3}{c}{$\{P\}\quad i\_1 ; \quad i\_2 ; \quad i\_3; \quad \{ Q \}$}
\end{tabular}
\end{center}

Nous pouvons par calcul de plus faibles pré-conditions construire la propriété
$Q_{-1}$ à partir de $Q$ et $i_3$, ce qui nous permet de déduire $Q_{-2}$, à 
partir de $Q_{-1}$ et $i_2$ et finalement $P$ avec $Q_{-2}$ et $i_1$.



Nous pouvons maintenant vérifier des programmes comprenant plusieurs 
instructions, il est temps d'y ajouter un peu de structure.



\levelThreeTitle{Règle de la conditionnelle}


Pour qu'un branchement conditionnel soit valide, il faut que la post-condition
soit atteignable par les deux banches, depuis la même pré-condition, à ceci 
près que chacune des branches aura une information supplémentaire : le fait 
que la condition était vraie dans un cas et fausse dans l'autre.



Comme avec la séquence d'instructions, nous aurons donc deux points à vérifier
(pour éviter de confondre les accolades, j'utilise la syntaxe 
$if\ B\ then\ S1\ else\ S2$) :




\begin{center}
$\dfrac{\{P \wedge B\}\quad S1\quad \{Q\} \quad \quad \{P \wedge \neg B\}\quad S2\quad \{Q\}}{\{P\}\quad if\quad B\quad then\quad S1\quad else\quad S2 \quad \{Q\}}$


\end{center}


Nos deux prémisses sont donc la vérification que lorsque nous avons la 
pré-condition et que nous passons dans la branche \CodeInline{if}, nous atteignons bien la
post-condition, et que lorsque nous avons la pré-condition et que nous passons
dans la branche \CodeInline{else}, nous obtenons bien également la post-condition.



Le calcul de pré-condition de $wp$ pour la conditionnelle est le suivant :




\begin{center}
$wp(if\ B\ then\ S1\ else\ S2 , Post) := (B \Rightarrow wp(S1, Post)) \wedge (\neg B \Rightarrow wp(S2, Post))$


\end{center}


À savoir que $B$ doit impliquer la pré-condition la plus faible de $S1$, pour 
pouvoir l'exécuter sans erreur vers la post-condition, et que $\neg B$ doit 
impliquer la pré-condition la plus faible de $S2$ (pour la même raison).



\levelFourTitle{Bloc \CodeInline{else} vide}


En suivant cette définition, si le \CodeInline{else} ne fait rien, alors la règle
d'inférence est de la forme suivante, en remplaçant $S2$ par une instruction
« ne rien faire ».




\begin{center}
$\dfrac{\{P \wedge B\}\quad S1\quad \{Q\} \quad \quad \{P \wedge \neg B\}\quad skip\quad \{Q\}}{\{P\}\quad if\quad B\quad then\quad S1\quad else\quad skip \quad \{Q\}}$


\end{center}


Le triplet pour le \CodeInline{else} est :




\begin{center}
$\{P \wedge \neg B\}\quad skip\quad \{Q\}$


\end{center}


Ce qui veut dire que nous devons avoir :




\begin{center}
$P \wedge \neg B \Rightarrow Q$


\end{center}


En résumé, si la condition du \CodeInline{if} est fausse, cela veut dire que la 
post-condition de l'instruction conditionnelle globale est déjà vérifiée avant de 
rentrer dans le \CodeInline{else} (puisqu'il ne fait rien).



Par exemple, nous pourrions vouloir remettre une configuration $c$ à une valeur 
par défaut si elle a mal été configurée par un utilisateur du programme :



\begin{CodeBlock}{c}
int c;

// ... du code ...

if(c < 0 || c > 15){
  c = 0;
}
//@ assert 0 <= c <= 15;
\end{CodeBlock}



Soit :



$wp(if \neg (c \in [0;15])\ then\ c := 0, \{c \in [0;15]\})$



$:= (\neg (c \in [0;15])\Rightarrow wp(c := 0, \{c \in [0;15]\})) \wedge (c \in [0;15]\Rightarrow wp(skip, \{c \in [0;15]\}))$



$= (\neg (c \in [0;15]) \Rightarrow 0 \in [0;15]) \wedge (c \in [0;15] \Rightarrow c \in [0;15])$



$= (\neg (c \in [0;15]) \Rightarrow true) \wedge true$



La formule est bien vérifiable : quelle que soit l'évaluation de $\neg (c \in [0;15])$ l'implication sera vraie.



\levelTwoTitle{Bonus Stage - Conséquence et constance}


\levelThreeTitle{Règle de conséquence}


Parfois, il peut être utile pour la preuve de renforcer une post-condition ou 
d'affaiblir une pré-condition. Si la première sera souvent établie par nos soins
pour faciliter le travail du prouveur, la seconde est plus souvent vérifiée 
par l'outil à l'issu du calcul de plus faible pré-condition.



La règle d'inférence en logique de Hoare est la suivante :




\begin{center}
$\dfrac{P \Rightarrow WP \quad \{WP\}\quad c\quad \{SQ\} \quad SQ \Rightarrow Q}{\{P\}\quad c \quad \{Q\}}$


\end{center}


(Nous noterons que les prémisses, ici, ne sont pas seulement des triplets de
Hoare mais également des formules à vérifier)



Par exemple, si notre post-condition est trop complexe, elle risque de générer
une plus faible pré-condition trop compliquée et de rendre le calcul des 
prouveurs difficile. Nous pouvons alors créer une post-condition intermédiaire
$SQ$, plus simple, mais plus restreinte et impliquant la vraie post-condition. 
C'est la partie $SQ \Rightarrow Q$.



Inversement, le calcul de pré-condition générera généralement une formule 
compliquée et souvent plus faible que la pré-condition que nous souhaitons
accepter en entrée. Dans ce cas, c'est notre outil qui s'occupera de vérifier 
l'implication entre ce que nous voulons et ce qui est nécessaire pour que notre
code soit valide. C'est la partie $P \Rightarrow WP$.



Nous pouvons par exemple illustrer cela avec le code qui suit. Notons bien qu'ici,
le code pourrait tout à fait être prouvé par l'intermédiaire de WP sans ajouter des
affaiblissements et renforcements de propriétés car le code est très simple, il 
s'agit juste d'illustrer la règle de conséquences.



\begin{CodeBlock}{c}
/*@
  requires P: 2 <= a <= 8;
  ensures  Q: 0 <= \result <= 100 ;
  assigns  \nothing ;
*/
int constrained_times_10(int a){
  //@ assert P_imply_WP: 2 <= a <= 8 ==> 1 <= a <= 9 ;
  //@ assert WP:         1 <= a <= 9 ;

  int res = a * 10;

  //@ assert SQ:         10 <= res <= 90 ;
  //@ assert SQ_imply_Q: 10 <= res <= 90 ==> 0 <= res <= 100 ;

  return res;
}
\end{CodeBlock}



(À noter ici : nous avons omis les contrôles de débordement d'entiers).



Ici, nous voulons avoir un résultat compris entre 0 et 100. Mais nous savons que
le code ne produira pas un résultat sortant des bornes 10 à 90. Donc nous 
renforçons la post-condition avec une assertion que \CodeInline{res}, le résultat, est compris
entre 0 et 90 à la fin. Le calcul de plus faible pré-condition, sur cette propriété,
et avec l'affectation \CodeInline{res = 10*a} nous produit une plus faible pré-condition 
\CodeInline{1 <= a <= 9} et nous savons finalement que \CodeInline{2 <= a <= 8} nous donne cette garantie.



Quand une preuve a du mal à être réalisée sur un code plus complexe, écrire des
assertions produisant des post-conditions plus fortes mais qui forment des formules
plus simples peut souvent nous aider. Notons que dans le code précédent, les lignes
\CodeInline{P\_imply\_WP} et \CodeInline{SQ\_imply\_Q} ne sont jamais utiles car c'est le raisonnement par
défaut produit par WP, elles sont juste présentes pour l'illustration.



\levelThreeTitle{Règle de constance}


Certaines séquences d'instructions peuvent concerner et faire intervenir des 
variables différentes. Ainsi, il peut arriver que nous initialisions et manipulions
un certain nombre de variables, que nous commencions à utiliser certaines d'entre 
elles, puis que nous les délaissions au profit d'autres pendant un temps. Quand un
tel cas apparaît, nous avons envie que l'outil ne se préoccupe que des variables 
qui sont susceptibles d'être modifiées pour avoir des propriétés les plus légères 
possibles.



La règle d'inférence qui définit ce raisonnement est la suivante :




\begin{center}
$\dfrac{\{P\}\quad c\quad \{Q\}}{\{P \wedge R\}\quad c\quad \{Q \wedge R\}}$


\end{center}


Où $c$ ne modifie aucune variable entrant en jeu dans $R$. Ce qui nous dit : « pour 
vérifier le triplet, débarrassons nous des parties de la formule qui concerne des
variables qui ne sont pas manipulées par $c$ et prouvons le nouveau triplet ». 
Cependant, il faut prendre garde à ne pas supprimer trop d'informations, au risque
de ne plus pouvoir prouver nos propriétés.



Par exemple, nous pouvons imaginer le code suivant (une nouvelle fois, nous omettons
les contrôles de débordements au niveau des entiers) :



\begin{CodeBlock}{c}
/*@
  requires a > -99 ;
  requires b > 100 ;
  ensures  \result > 0 ;
  assigns  \nothing ;
*/
int foo(int a, int b){
  if(a >= 0){
    a++ ;
  } else {
    a += b ;
  }
  return a ;
}
\end{CodeBlock}



Si nous regardons le code du bloc \CodeInline{if}, il ne fait pas intervenir la variable
\CodeInline{b}, donc nous pouvons omettre complètement les propriétés à propos de  \CodeInline{b} pour
réaliser la preuve que \CodeInline{a} sera bien supérieur à 0 après l'exécution du bloc :



\begin{CodeBlock}{c}
/*@
  requires a > -99 ;
  requires b > 100 ;
  ensures  \result > 0 ;
  assigns  \nothing ;
*/
int foo(int a, int b){
  if(a >= 0){
    //@ assert a >= 0; //et rien à propos de b
    a++ ;
  } else {
    a += b ;
  }
  return a ;
}
\end{CodeBlock}



En revanche, dans le bloc \CodeInline{else}, même si \CodeInline{b} n'est pas modifiée, établir
des propriétés seulement à propos de \CodeInline{a} rendrait notre preuve impossible (en
tant qu'humains). Le code serait :



\begin{CodeBlock}{c}
/*@
  requires a > -99 ;
  requires b > 100 ;
  ensures  \result > 0 ;
  assigns  \nothing ;
*/
int foo(int a, int b){
  if(a >= 0){
    //@ assert a >= 0; // et rien à propos de b
    a++ ;
  } else {
    //@ assert a < 0 && a > -99 ; // et rien à propos de b
    a += b ;
  }
  return a ;
}
\end{CodeBlock}



Dans le bloc \CodeInline{else}, n'ayant que connaissance du fait que \CodeInline{a} est compris
entre -99 et 0, et ne sachant rien à propos de \CodeInline{b}, nous pourrions 
difficilement savoir si le calcul \CodeInline{a += b} produit une valeur supérieure
strict à 0 pour \CodeInline{a}.



Naturellement ici, WP prouvera la fonction sans problème, puisqu'il transporte
de lui-même les propriétés qui lui sont nécessaires pour la preuve. En fait,
l'analyse des variables qui sont nécessaires ou non (et l'application, par 
conséquent de la règle de constance) est réalisée directement par WP.



Notons finalement que la règle de constance est une instance de la règle de 
conséquence :




\begin{center}
$\dfrac{P \wedge R \Rightarrow P \quad \{P\}\quad c\quad \{Q\} \quad Q \Rightarrow Q \wedge R}{\{P \wedge R\}\quad c\quad \{Q \wedge R\}}$


\end{center}


Si les variables de $R$ n'ont pas été modifiées par l'opération (qui par contre, 
modifie les variables de $P$ pour former $Q$), alors effectivement 
$P \wedge R \Rightarrow P$ et $Q \Rightarrow Q \wedge R$.

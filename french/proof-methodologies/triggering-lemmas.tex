Il y a différents niveaux d'automatisation dans la vérification de programmes,
depuis les outils complètement automatiques, comme les interpréteurs abstraits
qui ne nécessitent aucune aide de la part de l'utilisateur (ou en tout cas, très peu),
jusqu'aux outils interactifs comme les assistants de preuve, où les preuves sont
principalement écrites à la main et l'outil est juste là pour vérifier que nous
le faisons correctement.


Les outils comme WP (et beaucoup d'autres comme Why3, Spark, ...) visent à
maximiser l'automatisation. Cependant, plus les propriétés que nous voulons
prouver sont complexes, plus il sera difficile d'obtenir automatiquement toute
la preuve. Par conséquent, nous devons souvent aider les outils pour terminer
la vérification. Nous faisons souvent cela en fournissant plus d'annotations
pour aider le processus de génération des conditions de vérification. Ajouter un
invariant de boucle est par exemple une manière de fournir les informations
nécessaires pour produire le raisonnement par induction permettant son analyse,
alors que les prouveurs automatiques sont plutôt mauvais à cet exercice.


Cette technique de vérification a été appelée « \textit{auto-active} ». Ce mot est la
contraction de « \textit{automatic} » et « \textit{interactive} ». Elle est automatique au sens
où la majorité de la preuve est effectuée par des outils automatiques, mais
elle est aussi en partie interactive puisqu'en tant qu'utilisateurs, nous
fournissons manuellement de l'information aux outils.


Dans cette section, nous allons voir plus en détails comment nous pouvons
utiliser des assertions pour guider la preuve. En ajoutant des assertions, nous
créons une certaine base de connaissances (des propriétés que nous savons vraies),
qui sont collectées par le générateur de conditions de vérification pendant le calcul
de WP et qui sont données aux prouveurs automatiques qui ont par conséquent plus
d'information et peuvent potentiellement prouver des propriétés plus complexes.


\levelThreeTitle{Contexte de preuve}


Pour comprendre exactement le bénéfice que représente l'ajout d'assertions dans
les annotations d'un programme, commençons par regarder de plus près les
conditions de vérification générées par WP à partir du code source annoté et comment
les assertions sont prises en compte. Pour cela, nous allons utiliser le prédicat
suivant (qui ressemble furieusement au théorème de Pythagore) :



\CodeBlockInput[5][8]{c}{context.c}



Regardons d'abord cet exemple :


\CodeBlockInput[10][20]{c}{context.c}



Ici, nous avons spécifié une précondition suffisamment complexe pour que WP ne
puisse par directement deviner les valeurs en entrée de fonction. En fait, ces
valeurs sont exactement : \CodeInline{*x == 3}, \CodeInline{*y == 4} et
\CodeInline{*z == 5}. Maintenant, si nous regardons la condition de vérification
générée pour notre première assertion, nous pouvons voir ceci (il faut bien
sélectionner la vue « \textit{Full Context} » ou « \textit{Raw obligation} » -
elles ne sont pas exactement identiques, mais assez similaires, la première est
juste légèrement plus jolie) :


\image{context_e1_a1}


Nous y voyons les différentes contraintes que nous avons formulées comme préconditions
de fonction (notons que les valeurs ne sont pas exactement les mêmes, et que quelques
propriétés supplémentaires ont été générées). Maintenant, regardons plutôt
la condition de vérification générée pour la seconde assertion (notons que nous avons
édité les captures d'écran restantes de cette section pour nous concentrer sur ce
qui est important, les autres propriétés pouvant être ignorées dans notre cas) :


\image{context_e1_a2}


Ici, nous pouvons voir que dans le contexte utilisable pour la preuve de la seconde
assertion, WP a collecté et ajouté la première assertion et en a fait une supposition.
WP considère que les solveurs SMT peuvent supposer que cette propriété est vraie.
Cela signifie que les prouveurs peuvent l'utiliser, mais également qu'elle doit être
prouvée pour que la condition de vérification actuelle soit complètement vérifiée.


Notons que WP ne collecte que ce qu'il trouve sur les différents chemins d'exécution
qui permettent d'atteindre l'assertion. Par exemple, si nous modifions le code de
telle manière à ce que le chemin qui mène à la seconde assertion saute le chemin qui
passe par la première, celle-ci n'apparaît pas dans le contexte de la seconde assertion.


\CodeBlockInput[30][35]{c}{context.c}


\image{context_e1p_a2}


Maintenant, modifions un peu notre exemple de manière à illustrer comment les
assertions peuvent changer la manière de prouver un programme. Par exemple, nous
pouvons modifier les différentes positions mémoire (en doublant chaque valeur) et
vérifier que le triangle résultant est rectangle.


\CodeBlockInput[38][51]{c}{context.c}



\image{context_e2}



Ici, le solveur déroulera probablement le prédicat et vérifiera directement
que la propriété qui y est définie est vraie. En effet, depuis la condition de
vérification, il n'y a pas vraiment d'autres informations qui pourrait nous
amener à obtenir une preuve. Maintenant, ajoutons de l'information dans les
annotations :



\CodeBlockInput[53][72]{c}{context.c}



Nous prouvons d'abord que si nous multiplions par 2 chacune des valeurs, le
prédicat est vrai pour les nouvelles valeurs. Le solveur prouvera d'abord
la même propriété, bien sûr, mais ce n'est pas ce que nous voulons montrer ici.
Nous ajoutons ensuite que chaque valeur a été multipliée par 2. Maintenant, nous
pouvons regarder la condition de vérification générée pour la dernière assertion :


\image{context_e3}



Alors que nous devons prouver exactement la même propriété qu'avant (avec un peu
de renommage), nous pouvons voir que nous avons une autre manière de la prouver.
En effet, en combinant ces propriétés :


\begin{CodeBlock}{text}
(* Assertion *)
Have: P_rectangle(x_10, x_8, x_5).
(* Assertion *)
Have: x_9 = x_10.
(* Assertion *)
Have: x_6 = x_8.
(* Assertion *)
Have: x_3 = x_5.
\end{CodeBlock}


Il est facile de déduire :


\begin{CodeBlock}{text}
Prove: P_rectangle(x_9, x_6, x_3).
\end{CodeBlock}


En remplaçant simplement les valeurs \CodeInline{x\_9}, \CodeInline{x\_6} et
\CodeInline{x\_3}. Donc le solveur pourrait utiliser ceci pour faire la preuve
sans avoir à déplier le prédicat. Cependant, il ne le fera pas forcément : les
solveurs SMT sont basés sur des méthodes heuristiques, nous pouvons juste leur
fournir des propriétés et espérer qu'ils les utiliseront.


Ici la propriété est simple à prouver, donc il n'était pas vraiment nécessaire
d'ajouter ces assertions (et donc de faire plus d'efforts pour faire la même
chose). Cependant, dans d'autre cas, comme nous allons le voir maintenant, nous
devons donner la bonne information au bon endroit de façon à ce que les prouveurs
trouvent les informations dont ils ont besoin pour finir les preuves.


\levelThreeTitle{Déclencher les lemmes}


Nous utilisons souvent des assertions pour exprimer des propriétés qui
correspondent aux prémisses d'un lemme ou à ses conclusions. En faisant cela,
nous maximisons les chances que les prouveurs automatiques « reconnaissent »
que ce que nous avons écrit correspond à un lemme en particulier et qu'il
devrait l'utiliser.


Illustrons cela avec l'exemple suivant. Nous utilisons des axiomes et non des
lemmes parce qu'ils sont considérés de la même manière par WP lorsque nous nous
intéressons à la preuve d'une propriété qui en dépend. Regardons d'abord notre
définition axiomatique. Nous définissons deux prédicats \CodeInline{P} et
\CodeInline{Q} à propos d'une position mémoire particulière \CodeInline{x}.
Nous avons deux axiomes : \CodeInline{ax\_1} qui énonce que si \CodeInline{P(x)}
est vraie, alors \CodeInline{Q(x)} est vraie, et un second axiome \CodeInline{ax\_2}
qui énonce que si la position mémoire pointée ne change pas entre deux labels
(ce que l'on représente par le prédicat \CodeInline{eq}) et que \CodeInline{P(x)}
est vraie pour le premier label, alors elle est vraie pour le second.



\CodeBlockInput[5][19]{c}{trigger_1_1.c}


Et nous voulons prouver le programme suivant :


\CodeBlockInput[21][31]{c}{trigger_1_1.c}


Cependant, nous pouvons voir que la preuve échoue sur la condition de vérification
suivante (nous avons, à nouveau, retiré les éléments qui ne sont pas intéressants
pour notre explication) :


\image{trigger_1_1}


D'après cela, notre prouveur automatique semble incapable d'utiliser l'un des
axiomes de notre définition : soit il ne peut pas montrer qu'après l'appel
\CodeInline{g(y)}, \CodeInline{P(x)} est toujours vraie, soit il le peut, et
dans ce cas, cela veut dire qu'il n'arrive pas à montrer que cela implique
\CodeInline{Q(x)}. Essayons donc d'ajouter une assertion pour vérifier qu'il
arrive à montrer \CodeInline{P(x)} après l'appel :


\CodeBlockInput[24][32]{c}{trigger_1_1p.c}


\image{trigger_1_1p}


Il semble que malgré le fait qu'il est clair que \CodeInline{*x} n'a pas changé
pendant l'appel \CodeInline{g(y)}, et donc que \CodeInline{eq\{Pre, Here\}(x)}
est vraie après l'appel, puisque la propriété n'est pas directement fournie dans
notre condition de vérification, le prouveur automatique n'utilise pas l'axiome
\CodeInline{ax\_2} correspondant. Fournissons donc cette information au prouveur
automatique :


\CodeBlockInput[24][32]{c}{trigger_1_2.c}



Maintenant, tout est prouvé. Si nous regardons la condition de vérification générée,
nous pouvons voir que l'information nécessaire est bien fournie, ce qui permet
au prouveur automatique d'en faire usage :


\image{trigger_1_2}


\levelThreeTitle{Un exemple plus complexe : du tri à nouveau}


Travaillons maintenant avec un exemple plus complexe utilisant une définition
axiomatique réelle. Cette fois, nous nous intéresserons à montrer la correction
d'un tri par insertion :


\CodeBlockInput[5]{c}{insert_sort.c}


La fonction \CodeInline{insertion\_sort} visite chaque valeur, du début du tableau
jusqu'à la fin. Pour chaque valeur $v$, elle est insérée (en utilisant la fonction
\CodeInline{insert}) à la bonne place dans la plage des valeurs déjà triées (et qui
se trouvent dans le début du tableau), en décalant les éléments jusqu'à
rencontrer un élément qui est plus petit que $v$, ou le début du tableau.



Nous voulons prouver la même postcondition que ce que nous avons déjà prouvé pour
le tri par sélection, c'est-à-dire : nous voulons créer une permutation triée des
valeurs originales. À nouveau, chaque itération de la boucle doit assurer que la
nouvelle configuration obtenue est une permutation des valeurs originales, et que
la plage de valeurs allant du début à la cellule actuellement visitée est triée.
Toutes ces propriétés sont garanties par la fonction \CodeInline{insert}. Si l'on
regarde cette fonction de plus près, nous voyons qu'elle enregistre la
valeur à insérer (qui se trouve à la fin de la plage de valeurs) dans une variable
\CodeInline{value}, et en commençant à la fin de la plage, décale itérativement
les valeurs rencontrées jusqu'à rencontrer une valeur plus petite que la valeur à
insérer ou la première cellule du tableau, et insère ensuite la valeur. Pour cette
preuve, nous activons les options \CodeInline{-warn-unsigned-overflow} et
\CodeInline{-warn-unsigned-downcast}.




Tout d'abord, fournissons un contrat et un invariant pour la fonction de tri
par insertion. Le contrat est équivalent à celui que nous avions fourni pour le
tri par sélection. Notons cependant que l'invariant est plus faible : nous
n'avons pas besoin que les valeurs restant à trier soient plus grandes que les
valeurs déjà visitées : nous insérons chaque valeur à la bonne position.



\CodeBlockInput[67][85]{c}{insert_sort-contract.c}



Maintenant, nous pouvons fournir un contrat à la fonction d'insertion. La
fonction requière que la plage de valeurs considérée soit triée du début jusqu'à
l'avant-dernière valeur. En échange elle doit garantir que la plage finale soit
triée et soit une permutation des valeurs originales :



\CodeBlockInput[47][65]{c}{insert_sort-contract.c}



Ensuite, nous devons fournir un invariant utilisable pour expliquer le comportement
de la boucle de la fonction \CodeInline{insert}. Cette fois, nous pouvons voir
qu'avec notre précédente définition de la notion de permutation, nous sommes un
peu embêtés. En effet, notre définition inductive de la permutation spécifie trois
cas : une plage de valeur est une permutation d'elle-même, ou deux (et seulement
deux) valeurs ont été changées, ou finalement la permutation d'une permutation est
une permutation. Mais aucun de ces cas ne peut être appliqué à notre fonction
d'insertion puisque la plage obtenue ne l'est pas par une succession d'échanges de
valeurs et les deux autres cas ne peuvent évidemment pas s'appliquer ici.



Nous avons donc besoin d'une meilleure définition pour notre permutation. Nous
pouvons constater que ce dont nous avons vraiment besoin, c'est une manière de
dire « chaque valeur qui était dans le tableau est toujours dans le tableau et
si plusieurs valeurs étaient équivalentes, le nombre d'occurrences de ces valeurs
ne change pas ». En fait, nous n'avons besoin que de la dernière partie de cette
définition pour exprimer notre permutation. Une permutation est une plage de valeurs
telles que pour toute valeur, le nombre d'occurrences de cette valeur dans cette
plage ne change pas d'un point de programme à un autre :



\CodeBlockInput[42][46]{c}{insert_sort-proved.c}



En partant de cette définition, nous sommes capables de fournir des lemmes qui
nous permettront de raisonner efficacement à propos des permutations, à
supposer que certaines propriétés sont vraies à propos du tableau entre deux
points de programme. Par exemple, nous pourrions définir le cas \CodeInline{Swap}
de notre définition inductive précédente en utilisant un lemme. C'est bien
entendu aussi le cas pour notre plage de valeur « décalée ».



Déterminons quels sont les lemmes requis en considérant d'abord la fonction
\CodeInline{insert\_sort}. La seule propriété non prouvée est l'invariant qui
exprime que le tableau est une permutation du tableau original. Comment
pouvons-nous le déduire ? (Nous nous intéresserons aux preuves de ces lemmes
plus tard).



Nous pouvons observer deux faits : la première plage du tableau (de
\CodeInline{beg} à \CodeInline{i+1}) est une permutation de la même plage au
début de l'itération (par le contrat de la fonction \CodeInline{insert}). La
seconde partie (de \CodeInline{i+1} à \CodeInline{end}) est inchangée, donc
c'est aussi une permutation. Essayons d'utiliser quelques assertions pour voir
parmi ces propriétés ce qui peut être prouvé et ce qui ne peut pas l'être.
Tandis que la première propriété est bien prouvée, nous pouvons voir que la
seconde ne l'est pas :



\begin{CodeBlock}{c}
  /*@
    loop invariant beg+1 <= i <= end ;
    loop invariant sorted(a, beg, i) ;
    loop invariant permutation{Pre, Here}(a,beg,end);
    loop assigns a[beg .. end-1], i ;
    loop variant end-i ;
  */
  for(size_t i = beg+1; i < end; ++i) {
    //@ ghost L:
    insert(a, beg, i);
    //@ assert permutation{L, Here}(a, beg, i+1); // PROVED
    //@ assert permutation{L, Here}(a, i+1, end); // NOT PROVED
  }
\end{CodeBlock}


Nous avons donc besoin d'un premier lemme pour cette propriété. Définissons
deux prédicats \CodeInline{shifted} et \CodeInline{unchanged}, le second
étant utilisé pour définir le premier (nous verrons pourquoi un peu plus
tard) et exprimer qu'une plage inchangée est une permutation :


\CodeBlockInput[53][60]{c}{insert_sort-proved.c}

\CodeBlockInput[72][75]{c}{insert_sort-proved.c}


Maintenant, nous pouvons vérifier que ces deux sous-tableaux sont des
permutations, nous faisons cela en ajoutant une assertion qui montre que la
plage allant de \CodeInline{i+1} à \CodeInline{end} est inchangée, afin de
déclencher l'usage de notre lemme \CodeInline{unchanged\_is\_permutation}.


\CodeBlockInput[134][147]{c}{insert_sort-proved.c}


Comme ces deux sous-parties du tableau sont des permutations, le tableau
global est une permutation des valeurs initialement présentes au début de
l'itération. Cependant, cela n'est pas prouvé directement, nous ajoutons
donc aussi un lemme pour cela :


\CodeBlockInput[80][86]{c}{insert_sort-proved.c}


Maintenant, nous pouvons déduire qu'une itération de la boucle produit
une permutation en ajoutant cette conclusion comme une assertion :


\begin{CodeBlock}{c}
    //@ ghost L: ;
    insert(a, beg, i);
    //@ assert permutation{L, Here}(a, beg, i+1);
    //@ assert unchanged{L, Here}(a, i+1, end);
    //@ assert permutation{L, Here}(a, i+1, end);
    //@ assert permutation{L, Here}(a, beg, end); // PROVED
\end{CodeBlock}


Finalement, nous devons ajouter une information supplémentaire, la permutation
d'une permutation est aussi une permutation. Cette fois, nous n'avons pas
besoin d'une assertion supplémentaire. Le contexte contient :
\begin{itemize}
\item \CodeInline{permutation\{Pre, L\}(a, beg, end)} (invariant)
\item \CodeInline{permutation\{L, Here\}(a, beg, end)} (assertion)
\end{itemize}
qui est suffisant pour conclure \CodeInline{permutation\{Pre, Here\}(a, beg, end)}
à la fin du bloc de la boucle en utilisant le lemme suivant :


\CodeBlockInput[47][52]{c}{insert_sort-proved.c}


Maintenant, nous pouvons regarder de plus près notre fonction d'insertion en
nous intéressant d'abord à comment obtenir la connaissance que la fonction
produit une permutation.


Elle décale les différents éléments vers la gauche jusqu'à rencontrer le
début du tableau ou un élément plus petit que l'élément à insérer qui est
initialement à la fin de la plage de valeur et inséré à la position ainsi atteinte.
Les cellules du début du tableau jusqu'à la position d'insertion restent
inchangées, c'est donc une permutation. Nous avons un lemme pour cela, mais nous
devons ajouter cette connaissance que le début du tableau ne change pas comme
un invariant de la boucle pour pouvoir déclencher le lemme après celle-ci. La
seconde partie du tableau est une permutation parce que nous faisons « tourner »
les éléments, nous avons besoin d'un lemme pour exprimer cela et d'indiquer dans
l'invariant de boucle que les éléments sont décalés par la boucle. Finalement,
l'union de deux permutations est une permutation et nous avons déjà un lemme
pour cela.


Tout d'abord, donnons un invariant pour la permutation :
\begin{itemize}
    \item nous fournissons les bornes de \CodeInline{i},
    \item nous indiquons que la première partie du tableau est inchangée,
    \item nous indiquons que la seconde partie est décalée vers la gauche,
\end{itemize}
et nous ajoutons quelques assertions pour vérifier quelques propriétés d'intérêt :
\begin{itemize}
    \item d'abord, pour déclencher \CodeInline{unchanged\_permutation}, nous
          plaçons une première assertion pour énoncer que la première partie
          du tableau est inchangée, ce qui nous permet de prouver que ...
    \item la seconde assertion, qui nous dit que la première partie du tableau
          est une permutation de l'originale, et que l'on utilise en combinaison
          avec ...
    \item la troisième assertion qui nous dit que la seconde partie du tableau
          est une permutation de l'originale (qui nous permet de déclencher
          l'usage de \CodeInline{union\_permutation} et de prouver la postcondition).
\end{itemize}


\begin{CodeBlock}{c}
  /*@
    loop invariant beg <= i <= last ;
    loop invariant \forall integer k ; beg <= k <= i    ==> a[k] == \at(a[k], Pre) ;
    loop invariant \forall integer k ; i+1 <= k <= last ==> a[k] == \at(a[k-1], Pre) ;

    loop assigns i, a[beg .. last] ;
    loop variant i ;
  */
  while(i > beg && a[i - 1] > value){
    a[i] = a[i - 1] ;
    --i ;
  }

  a[i] = value ;

  //@ assert unchanged{Pre, Here}(a, beg, i) ;   // PROVED
  //@ assert permutation{Pre, Here}(a, beg, i) ; // PROVED

  //@ assert rotate_left{Pre, Here}(a, i, last+1) ; //PROVED
  //@ assert permutation{Pre, Here}(a, i, last+1) ; // NOT PROVED
\end{CodeBlock}


Pour la dernière assertion, nous avons besoin d'un lemme à propos de la
rotation des éléments :


\CodeBlockInput[61][65]{c}{insert_sort-proved.c}
\CodeBlockInput[76][79]{c}{insert_sort-proved.c}


Nous devons également aider un peu les prouveurs automatiques pour montrer
que l'ensemble des valeurs est trié après l'insertion. Pour cela, nous fournissons
un nouvel invariant pour montrer que les valeurs « décalées » sont plus grandes
que la valeur à insérer. Puis, nous ajoutons également des assertions pour
montrer que le tableau est trié avant l'insertion, et que toutes les valeurs
avant la cellule où nous insérons sont plus petites que la valeur insérée, et
que la plage est en conséquence triée après l'insertion. Cela nous amène à
la fonction \CodeInline{insert} complètement annotée suivante :



\CodeBlockInput[88][124]{c}{insert_sort-proved.c}



En tout, nous avons six lemmes à prouver :
\begin{itemize}
\item \CodeInline{l\_occurrences\_union},
\item \CodeInline{shifted\_maintains\_occ},
\item \CodeInline{unchanged\_is\_permutation},
\item \CodeInline{rotate\_left\_is\_permutation},
\item \CodeInline{union\_permutation},
\item \CodeInline{transitive\_permutation}.
\end{itemize}


Tandis que les preuves Coq de ces lemmes sont en dehors des préoccupations de ce
tutoriel (et nous verrons plus tard que dans le cas particulier de cette preuve
nous pouvons nous en débarrasser), donnons quelques indications pour obtenir une
preuve de ces lemmes (notons que les scripts Coq de ces preuves sont disponibles
sur le répertoire GitHub de ce livre).


Pour prouver \CodeInline{l\_occurrences\_union}, nous raisonnons par induction
sur la taille de la seconde partie du tableau. Le cas de base est trivial : si
la taille est 0, nous avons immédiatement l'égalité puisque \CodeInline{split == to}.
Maintenant, nous devons prouver que si l'égalité est vraie pour une plage de taille
$i$, elle est vraie pour une plage de taille $i+1$. Puisque nous savons que c'est
le cas jusqu'à $i$ par hypothèse d'induction, nous analysons simplement les
différents cas pour le dernier élément de la plage (au rang $i$) : soit cet élément
est celui que nous comptons, soit il ne l'est pas. Quoi qu'il en soit, cela ajoute
la même valeur des deux côtés de l'égalité.


Pour \CodeInline{shifted\_maintains\_occ}, nous raisonnons par induction sur la
plage complète. Le premier cas est trivial (la plage est vide). Pour le cas
d'induction, nous avons juste à montrer que la valeur ajoutée a été décalée, et
qu'elle est donc la même.


La propriété \CodeInline{unchanged\_is\_permutation} peut être prouvée par les
solveurs SMT grâce au fait que nous avons exprimé \CodeInline{unchanged} en
utilisant \CodeInline{shifted}, le prouveur peut donc directement instancier le
lemme précédent. Si ce n'est pas le cas, la preuve peut être réalisée en
instanciant \CodeInline{shifted\_maintains\_occ} avec la valeur 0 pour la
propriété de décalage.


Pour prouver \CodeInline{rotate\_left\_is\_permutation}, nous séparons la plage
pour \CodeInline{L1} en deux sous-plages \CodeInline{beg .. beg+1} et
\CodeInline{beg+1 .. end} et la plage pour \CodeInline{L2} en deux sous-plages
\CodeInline{beg .. end-1} et \CodeInline{end-1 .. end} en utilisant la propriété
\CodeInline{l\_occurrences\_union}. Nous montrons que le nombre d'occurrences
dans \CodeInline{beg+1 .. end} pour \CodeInline{L1} et \CodeInline{beg .. end-1}
pour \CodeInline{L2} n'a pas changé grâce à \CodeInline{shifted\_maintains\_occ}
et que le nombre d'occurrences dans \CodeInline{beg .. beg+1} pour
\CodeInline{L1} et \CodeInline{end-1 .. end} pour \CodeInline{L2} est le même
en analysant par cas (et en utilisant le fait que la valeur correspondante est
la même).


Pour prouver \CodeInline{union\_permutation}, nous instancions le lemme
\CodeInline{l\_occurrences\_union}. Finalement, le lemme
\CodeInline{transitive\_permutation} est prouvé automatiquement par transitivité
de l'égalité.


\levelThreeTitle{Comment utiliser correctement les assertions ?}


Il n'y a pas de guide précis à propos de quand utiliser des assertions ou non.
La plupart du temps nous les utilisons d'abord pour comprendre pourquoi certaines
preuves échouent en exprimant des propriétés dont nous pensons qu'elles sont vraies
à un point particulier de programme. De plus, il n'est pas rare que les conditions
de vérification soient longues ou un peu complexes à lire directement. Bien
utiliser les assertions nécessite que l'on garde en tête les lemmes déjà exprimés
pour savoir quelle assertion utiliser pour déclencher l'usage d'un lemme nous
amenant à la propriété voulue. S'il n'y a pas de tel lemme, par exemple parce
que la preuve de la propriété voulue nécessite un raisonnement par induction à
propos d'une propriété ou d'une valeur, nous avons probablement besoin d'ajouter
un nouveau lemme.


Avec un peu d'expérience, l'utilisation des assertions et des lemmes devient de
plus en plus naturelle. Cependant, il est important de garder en tête qu'il est
facile d'abuser de cela. Plus nous ajoutons de lemmes et d'assertions, plus notre
contexte de preuve est riche et est susceptible de contenir les informations
nécessaires à la preuve. Cependant, il y a aussi un risque d'ajouter trop
d'information de telle manière à ce que le contexte de preuve finisse par contenir
des informations inutiles qui polluent le contexte de preuve, rendant le travail
des solveurs SMT plus difficile. Nous devons donc essayer de trouver le bon
compromis.


\levelThreeTitle{Exercices}


\levelFourTitle{Comprendre le contexte de preuve}


Dans la fonction suivante, la dernière assertion est prouvée automatiquement
par le solveur SMT, probablement en dépliant le prédicat pour prouver
directement la propriété. En utilisant des assertions, fournir une nouvelle
manière de prouver la dernière propriété. Dans le contexte de preuve, trouver
les propriétés générées qui peuvent amener à une autre preuve de l'assertion
et expliquer comment :


\CodeBlockInput[5]{c}{ex-1-context.c}


\levelFourTitle{Déclencher les lemmes}


Dans le programme suivant, WP échoue à prouver que la postcondition de la
fonction \CodeInline{g} est vérifiée. Ajouter la bonne assertion, au bon endroit,
de façon à ce que la preuve réussisse.


\CodeBlockInput{c}{ex-2-trigger.c}


\levelFourTitle{Déclencher les lemmes sous condition}


Dans le programme suivant, WP échoue à prouver que la postcondition de la
fonction \CodeInline{example} est vérifiée. Cependant, nous pouvons noter
que la fonction \CodeInline{g} assure indirectement que la valeur pointée
est soit augmentée soit diminuée. Ajouter deux assertions qui montrent que
le prédicat est vérifié en fonction de la valeur de \CodeInline{*x}.


\CodeBlockInput{c}{ex-3-trigger-cond.c}


Les assertions devraient ressembler à :


\begin{CodeBlock}{c}
//@ assert *x ... ==> ... ;
//@ assert *x ... ==> ... ;
\end{CodeBlock}


Une autre manière d'ajouter de l'information au contexte est d'utiliser du
code fantôme. Par exemple, la valeur de vérité d'une conditionnelle apparaît
dans le contexte d'une condition de vérification. Modifier les annotations pour
que le code ressemble à :


\begin{CodeBlock}{c}
void example(int* x){
  g(x);
  /*@ ghost
   if ( ... ){
    /@ assert ... @/
   } else {
    /@ assert ... @/
   }
  */
}
\end{CodeBlock}

Comparer la condition de vérification générée pour chaque assertion avec les
précédentes.


Finalement, nous pouvons remarquer que « la valeur pointée a été augmentée ou
diminuée » peut être exprimée en une seule assertion. Écrire l'annotation
correspondante et recommencer la preuve.


\levelFourTitle{Un exemple avec la somme}
\label{l4:proof-methodologies-triggering-lemmas-exercises-sum}


La fonction suivante incrémenter de 1 la valeur d'une cellule du tableau, donc
elle augmente aussi la valeur de la somme du contenu du tableau. Écrire un contrat
pour la fonction qui exprime ce fait :


\CodeBlockInput[5]{c}{ex-4-sum-content.c}


Pour prouver que cette fonction remplit son contrat, nous avons besoin de fournir
des assertions qui guideront la preuve. Plus précisément, nous devons montrer que
puisque toutes les valeurs avant la cellule modifiée n'ont pas été modifiées, la
somme n'a pas été modifiée pour cette partie du tableau, et de même pour les
cellules qui suivent la cellule modifiée.


Nous avons donc besoin de deux lemmes :
\begin{itemize}
    \item \CodeInline{sum\_separable} doit exprimer que nous pouvons séparer le
          tableau en deux sous parties, compter dans chaque partie et sommer les
          résultats pour obtenir la somme totale,
    \item \CodeInline{unchanged\_sum} devrait exprimer que si une plage dans un
          tableau n'a pas changé entre deux labels, la somme du contenu est la
          même.
\end{itemize}


Compléter les code des lemmes et utiliser des assertions pour assurer qu'ils sont
utilisés pour compléter la preuve. Nous ne demandons par la preuve des lemmes, les
preuves Coq sont disponibles sur le répertoire GitHub de ce livre.

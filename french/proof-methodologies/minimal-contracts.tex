Nous avons vu que la preuve d'un programme permet de vérifier deux aspects
principaux à propos de sa correction ; d'abord que le programme ne contient pas
d'erreur d'exécution, et ensuite que le programme répond correctement à sa
spécification. Cependant, il est parfois difficile d'obtenir le second aspect,
et le premier est déjà une étape intéressante pour la correction de notre programme.


En effet, les erreurs à l'exécution entraînent souvent la présence des fameux
« \textit{undefined behaviors} » dans les programmes C. Ces comportements peuvent être
des vecteurs de failles de sécurité. Par conséquent, garantir leur absence nous
protège déjà d'un grand nombre de ces vecteurs d'attaques. L'absence d'erreur à
l'exécution peut être vérifiée avec WP à l'aide d'une approche appelée
« contrats minimaux ».


\levelThreeTitle{Principe}


L'approche par contrats minimaux est guidée par l'usage du greffon RTE de Frama-C.
L'idée est simple : pour toutes les fonctions d'un module ou d'un projet,
nous générons les assertions nécessaires à vérifier l'absence d'erreurs à l'exécution,
et nous écrivons ensuite l'ensemble des spécifications (correctes) qui sont suffisantes
pour prouver ces assertions et les contrats ainsi rédigés. La plupart du temps, cela
permet d'avoir beaucoup moins de lignes de spécifications que ce qui est nécessaire
pour prouver la correction fonctionnelle du programme.


Commençons par un exemple simple avec la fonction valeur absolue.


\CodeBlockInput[5]{c}{abs-1.c}


Ici, nous pouvons générer les assertions nécessaires à prouver pour montrer
l'absence d'erreurs à l'exécution, ce qui génère ce programme :


\CodeBlockInput[5]{c}{abs-2.c}


Donc nous avons seulement besoin de spécifier la précondition qui nous dit que
\CodeInline{x} doit être plus grand que \CodeInline{INT\_MIN} :


\CodeBlockInput[3]{c}{abs-3.c}


Cette condition est suffisante pour montrer l'absence d'erreurs à l'exécution
dans cette fonction.


Comme nous le verrons plus tard, généralement une fonction est cependant utilisée
dans un contexte particulier. Il est donc probable que ce contrat ne soit en
réalité pas suffisant pour assurer la correction dans son contexte d'appel. Par
exemple, il est commun en C d'avoir des variables globales ou des pointeurs, il
est donc probable que nous devions spécifier ce qui est assigné par la fonction.
La plupart du temps, les clauses \CodeInline{assigns} ne peuvent pas être ignorées
(ce qui est prévisible dans un langage où tout est mutable par défaut). De plus,
si une personne demande la valeur absolue d'un entier, c'est probablement qu'elle
a besoin d'une valeur positive. En réalité, le contrat ressemblera probablement à
ceci :


\CodeBlockInput[3]{c}{abs-4.c}


Mais cette addition ne devrait être guidée que par la vérification du ou des
contextes dans lesquels la fonction est appelée, une fois que nous avons
prouvée l'absence d'erreur d'exécution dans cette fonction.


\levelThreeTitle{Exemple: la fonction recherche}


Maintenant que nous avons le principe en tête, travaillons avec un exemple un peu
plus complexe, Celui-ci en particulier nécessite une boucle.


\CodeBlockInput[5]{c}{search-1.c}


Lorsque nous générons les assertions liées aux erreurs à l'exécution, nous
obtenons le programme suivant :


\CodeBlockInput[5]{c}{search-2.c}


Nous devons prouver que toute cellule visitée par le programme peut être lue,
nous avons donc besoin d'exprimer comme précondition que ce tableau est
\CodeInline{\textbackslash{}valid\_read} sur la plage de valeurs correspondante.
Cependant, ce n'est pas suffisant pour terminer la preuve puisque nous avons une
boucle dans ce programme. Nous devons donc aussi fournir un invariant, nous
voulons aussi probablement prouver que la boucle termine.


Nous obtenons donc la fonction suivante, spécifiée minimalement :


\CodeBlockInput[1][15]{c}{search-3.c}



Ce contrat peut être comparé avec le contrat fourni pour la fonction de recherche
de la section~\ref{l3:statements-loops-examples-ro}, et nous pouvons voir qu'il est
beaucoup plus simple.


Maintenant imaginons que cette fonction est utilisée dans le programme suivant :


\CodeBlockInput[17][22]{c}{search-3.c}


Nous devons à nouveau fournir un invariant pour cette fonction, à nouveau en
regardant l'assertion générée par le plugin RTE :


\CodeBlockInput[30][39]{c}{search-4.c}


Nous devons donc vérifier que :

\begin{itemize}
\item le pointeur que la fonction reçoit est valide,
\item \CodeInline{*p+1} ne fait pas de débordement,
\item la précondition de la fonction \CodeInline{search} est satisfaite.
\end{itemize}


En plus du contrat de \CodeInline{foo}, nous devons fournir plus d'informations
dans le contrat de \CodeInline{search}. En effet, nous ne pourrons pas prouver
que le pointeur est valide si la fonction ne nous garantit
pas qu'il est dans la plage correspondant à notre tableau dans ce cas. De plus,
nous ne pourrons pas prouver que \CodeInline{*p} a une valeur inférieure à
\CodeInline{INT\_MAX} si la fonction peut modifier le tableau.


Cela nous amène donc au programme complet annoté suivant :


\CodeBlockInput{c}{search-5.c}


\levelThreeTitle{Avantages et limitations}


L'avantage le plus évident de cette approche est le fait qu'elle permet de garantir
qu'un programme ne contient pas d'erreurs à l'exécution dans toute fonction d'un
module ou d'un programme en (relative) isolation des autres fonctions. De plus,
cette absence d'erreurs à l'exécution est garantie pour tout usage de la fonction
dont l'appel satisfait ses préconditions. Cela permet de gagner une certaine
confiance dans un système avec une approche dont le coût est relativement
raisonnable.



Cependant, comme nous avons pu le voir, lorsque nous utilisons une fonction, cela
peut changer les connaissances que nous avons besoin d'avoir à son sujet, nécessitant
d'enrichir son contrat progressivement. Nous pouvons par conséquent atteindre un
point où nous avons prouvé la correction fonctionnelle de la fonction.



De plus, prouver l'absence d'erreur à l'exécution peut parfois ne pas être trivial
comme nous avons pu le voir précédemment avec des fonctions comme la factorielle ou
la somme des N premiers entiers, qui nécessitent de donner une quantité notable
d'information aux solveurs SMT pour montrer qu'elle ne déborde pas.



Finalement, parfois les contrats minimaux d'une fonction ou d'un module sont
simplement la spécification fonctionnelle complète. Et dans ce cas, effectuer la
vérification d'absence d'erreur à l'exécution correspond à réaliser la vérification
fonctionnelle complète du programme. C'est communément le cas lorsque nous devons
travailler avec des structures de données complexes où les propriétés dont nous
avons besoin pour montrer l'absence d'erreurs à l'exécution dépendent du
comportement fonctionnel des fonctions, maintenant des invariants non triviaux
à propos de la structure de donnée.


\levelThreeTitle{Exercices}


\levelFourTitle{Exemple simple}


Prouver l'absence d'erreurs à l'exécution dans le programme suivant en utilisant
une approche par contrats minimaux :


\CodeBlockInput[5]{c}{ex-1-simple.c}


\levelFourTitle{Inverse}


Prouver l'absence d'erreurs à l'exécution dans la fonction \CodeInline{reverse}
suivante et ses dépendances en utilisant une approche par contrats minimaux.
Notons que la fonction \CodeInline{swap} doit également être spécifiée par
contrats minimaux. Ne pas oublier d'ajouter les options
\CodeInline{-warn-unsigned-overflow} et \CodeInline{-warn-unsigned-downcast}.


\CodeBlockInput[5]{c}{ex-2-reverse.c}



\levelFourTitle{Recherche dichotomique}


Prouver l'absence d'erreurs à l'exécution dans la fonction \CodeInline{bsearch}
suivante en utilisant une approche par contrats minimaux. Ne pas oublier d'ajouter
les options \CodeInline{-warn-unsigned-overflow} et
\CodeInline{-warn-unsigned-downcast}.


\CodeBlockInput[5]{c}{ex-3-binary-search.c}


\levelFourTitle{Tri}


Prouver l'absence d'erreurs à l'exécution dans la fonction \CodeInline{sort}
et ses dépendances en utilisant une approche par contrats minimaux. Notons que
ces dépendances doivent également être spécifiées par contrats minimaux. Ne pas
oublier d'ajouter les options \CodeInline{-warn-unsigned-overflow} et
\CodeInline{-warn-unsigned-downcast}.


\CodeBlockInput[5]{c}{ex-4-sort.c}

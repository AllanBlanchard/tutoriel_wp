Maintenant que  nous avons présenté les fonctionnalités les plus importantes
d'ACSL pour la preuve de programme, intéressons nous plus spécifiquement à la
manière de prouver un programme avec Frama-C et WP. Nous allons présenter
différentes approches qui peuvent être utilisées, selon la cible de vérification,
le type de propriétés que l'on cherche à montrer et les fonctionnalités d'ACSL
que nous utilisons.

%% \begin{levelTwo}
%%   {Data-structure API}
%%   {data-structure-api}
%% \end{levelTwo}

\begin{levelTwo}
  {Absence d'erreurs à l'exécution: contrats minimaux}
  {minimal-contracts}
\end{levelTwo}

\begin{levelTwo}
  {Assertions de guidage et déclenchement de lemmes}
  {triggering-lemmas}
\end{levelTwo}

\begin{levelTwo}
  {Plus de code fantôme, fonctions lemmes et macros lemmes}
  {lemma-functions}
\end{levelTwo}



\horizontalLine
\newpage


A mesure que nous essayons de prouver des propriétés plus complexes,
particulièrement quand les programmes contiennent des boucles, et des
structures de données complexes, il y a un part de « \textit{trial and error} »
pour comprendre ce qui manque aux prouveurs pour établir la preuve.


Il peut manquer des hypothèses. Dans ce cas, nous pouvons essayer d'ajouter
des assertions pour guider les prouveurs, ou écrire du code ghost avec les
bons invariants, ce qui permet d'effectuer une part du raisonnement nous
même lorsque c'est trop difficile pour les solveurs SMT.


Avec un peu d'expérience, il est possible de lire le contenu des
conditions de vérification ou essayer de faire la preuve soi-même avec
l'assistant de preuve Coq pour voir si la preuve semble réalisable.
Parfois, le prouveur a juste besoin de plus de temps, dans ce cas, nous
pouvons augmenter (parfois beaucoup) le temps de \textit{timeout}. Bien
sûr la propriété peut parfois être trop difficile pour le prouveur et
le code ghost ne pas être adapté, dans ce cas, il sera nécessaire de
terminer la preuve nous mêmes.


Finalement, l'implémentation peut être incorrecte, et dans ce cas,
nous devons la corriger. A ce moment là, nous utiliserons du test et
non de la preuve, car un test nous permettra de mettre en évidence la
présence du bug et de l'analyser.

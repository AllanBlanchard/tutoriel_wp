\begin{Warning}
  Le code source de ce
  tutoriel est disponible sur GitHub, de même que les solutions aux
  différents exercices (incluant quelques preuves Coq de certaines
  propriétés).

  Si vous trouvez des erreurs, n'hésitez pas à créer une \textit{issue} ou
  une \textit{pull request} sur :

  \externalLink{https://github.com/AllanBlanchard/tutoriel\_wp}{https://github.com/AllanBlanchard/tutoriel_wp}

  ou à poster sur le sujet de la bêta sur Zeste de Savoir :

  \externalLink{https://zestedesavoir.com/forums/sujet/7725/introduction-a-la-preuve-de-programmes-c-avec-frama-c-et-son-greffon-wp/}{https://zestedesavoir.com/forums/sujet/7725/introduction-a-la-preuve-de-programmes-c-avec-frama-c-et-son-greffon-wp/}
\end{Warning}

\begin{Information}
Le choix de certains exemples et d'une partie de l'organisation dans le présent
tutoriel est le même que celui du
\externalLink{tutoriel présenté à TAP 2013}
             {https://frama-c.com/download/publications/tutorial_tap2013_slides.pdf}
par Nikolai Kosmatov, Virgile Prevosto et Julien Signoles du CEA List du fait de
son cheminement didactique. Il contient également des exemples tirés de
\textit{\externalLink{ACSL By Example}{https://github.com/fraunhoferfokus/acsl-by-example}}
de Jochen Burghardt, Jens Gerlach, Kerstin Hartig, Hans Pohl et Juan Soto du
Fraunhofer. Pour les aspects formels, je me suis reposé sur le cours
à propos de Why3 donné par Andrei Paskevich
\textit{\externalLink{à l'EJCP 2018}{https://ejcp2018.sciencesconf.org/file/441131}}.
Le reste vient de mon expérience personnelle avec Frama-C et WP.

\horizontalLine

Les versions des outils utilisés dans ce tutoriel sont les suivantes :
\begin{itemize}
\item Frama-C 21.1 Scandium
\item Why3 1.3.1
\item Alt-Ergo 2.3.2
\item Coq 8.11.2 (pour les scripts proposés, Coq n'est pas utilisé dans le tutoriel)
\item Z3 4.8.4 (utilisés dans un exemple, il n'est pas absolument nécessaire)
\end{itemize}
Selon les versions utilisées par le lecteur, quelques différences pourraient
apparaître avec ce qui est prouvé et ce qui ne l'est pas. Quelques fonctionnalités
ne sont disponibles que dans les versions récentes de Frama-C.

Les scripts Coq devraient fonctionner au moins de la version 8.7.2 à 8.11.2.

\horizontalLine

Le seul prérequis pour ce cours est d'avoir une connaissance basique du
langage C, au moins jusqu'à la notion de pointeur.


\end{Information}

\newpage


Malgré son ancienneté, le C est un langage de programmation encore largement
utilisé. Il faut dire qu'il n'existe, pour ainsi dire, aucun langage qui soit
disponible sur une aussi large variété de plateformes (matérielles et
logicielles) différentes, que son orientation bas-niveau et les années
d'optimisations investies dans ses compilateurs permettent de générer à
partir de programmes C des exécutables très performants (à condition bien sûr
que le code le permette), et qu'il possède un nombre d'experts (et donc une
base de connaissances) très conséquent.



De plus, de très nombreux systèmes reposent sur des quantités phénoménales de
code historiquement écrit en C, qu'il faut maintenir et corriger, car ils
coûteraient bien trop chers à re-développer.



Mais toute personne qui a déjà codé en C sait également que c'est un langage
très difficile à maîtriser parfaitement. Les raisons sont multiples, mais les
ambiguïtés présentes dans sa norme et la permissivité extrême qu'il offre au
développeur, notamment en ce qui concerne les accès à la mémoire, font que
créer un programme C robuste est très difficile même pour un programmeur
chevronné.



Pourtant, C est souvent choisi comme langage de prédilection pour la
réalisation de systèmes demandant un niveau critique de sûreté (aéronautique,
ferroviaire, armement, ...) où il est apprécié pour ses performances, sa
maturité technologique et la prévisibilité de sa compilation.



Dans ce genre de cas, les besoins de couverture par le test deviennent
colossaux. Et, plus encore, la question « avons-nous suffisamment testé ? »
devient une question à laquelle il est de plus en plus difficile de répondre.
C'est là qu'intervient la preuve de programme. Plutôt que tester toutes les
entrées possibles et (in)imaginables, nous allons prouver « mathématiquement »
qu'aucun problème ne peut apparaître à l'exécution.



L'objet de ce tutoriel est d'utiliser Frama-C, un logiciel développé au
CEA List, et WP, son greffon de preuve déductive, pour s'initier à la preuve
de programmes C. Au-delà de l'usage de l'outil en lui-même, le but de ce tutoriel
est de vous convaincre qu'il est possible d'écrire des programmes sans erreurs de
programmation, mais également de sensibiliser à des notions simples
permettant de mieux comprendre et de mieux écrire les programmes.



\begin{Information}
Merci aux différents bêta-testeurs pour leurs remarques constructives :

\begin{itemize}
\item \externalLink{Taurre}{https://zestedesavoir.com/membres/voir/Taurre/}
\item \externalLink{barockobamo}{https://zestedesavoir.com/membres/voir/barockobamo/}
\item \externalLink{Vayel}{https://zestedesavoir.com/membres/voir/Vayel/}
\item \externalLink{Aabu}{https://zestedesavoir.com/membres/voir/Aabu/}
\end{itemize}
Ainsi qu'aux validateurs qui ont encore permis d'améliorer la qualité de ce tutoriel :

\begin{itemize}
\item \externalLink{Taurre}{https://zestedesavoir.com/membres/voir/Taurre/} (oui, encore lui)
\item \externalLink{Saroupille}{https://zestedesavoir.com/membres/voir/Saroupille/}
\item \externalLink{Aabu}{https://zestedesavoir.com/membres/voir/Aabu/} (oui, encore lui aussi)
\end{itemize}

Un grand merci à Jens Gerlach pour son aide lors de la traduction anglaise du tutoriel.

Merci finalement aux reviewers occasionnels sur GitHub :
\begin{itemize}
  \item \externalLink{Alex Lyr}{https://github.com/AlexLyrr}
  \item \externalLink{Rafael Bachmann}{https://github.com/barafael}
  \item \externalLink{Daniel Rocha}{https://github.com/danroc}
  \item \externalLink{@GaoTamanrasset}{https://github.com/GaoTamanrasset}
  \item \externalLink{André Maroneze}{https://github.com/maroneze}
  \item \externalLink{@MSoegtropIMC}{https://github.com/MSoegtropIMC}
  \item \externalLink{@rtharston}{https://github.com/rtharston}
  \item \externalLink{@TrigDevelopment}{https://github.com/TrigDevelopment}
  \item \externalLink{Quentin Santos}{https://github.com/qsantos}
  \item \externalLink{Ricardo M. Correia}{https://github.com/wizeman}
  \item \externalLink{Basile Desloges}{https://github.com/zilbuz}
\end{itemize}
pour leurs relectures et corrections.

\end{Information}

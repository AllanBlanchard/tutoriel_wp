\begin{levelTwo}
  {}
  {introduction}
\end{levelTwo}


\begin{levelTwo}
  {Concepts de base}
  {basic}
\end{levelTwo}

\begin{levelTwo}
  {Les boucles}
  {loops}
\end{levelTwo}

\begin{levelTwo}
  {Plus d'exemples sur les boucles}
  {loops-examples}
\end{levelTwo}

\begin{levelTwo}
  {Appels de fonction}
  {function-calls}
\end{levelTwo}

\horizontalLine


\newpage


Dans cette partie nous avons pu voir comment se traduisent les affectations et
les structures de contrôle d'un point de vue logique. Nous nous sommes beaucoup 
attardés sur les boucles parce que c'est là que se trouvent la majorité des 
difficultés lorsque nous voulons spécifier et prouver un programme par 
vérification déductive, les annotations ACSL qui leur sont spécifiques nous 
permettent d'exprimer le plus précisément possible leur comportement.



Pour la suite, nous allons nous attarder plus précisément sur les constructions
que nous offre le langage ACSL du côté de la logique. Elles sont très 
importantes parce que ce sont elles qui vont nous permettre de nous abstraire
du code pour avoir des spécifications plus compréhensibles et plus aisément 
prouvables.

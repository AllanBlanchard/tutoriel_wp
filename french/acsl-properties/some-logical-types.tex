ACSL fournit différents types logiques qui permettent d'écrire des
propriétés dans un monde plus abstrait, plus mathématique. Parmi les types qui
peuvent être utiles, certains sont dédiés aux nombres et permettent d'exprimer
des propriétés ou des fonctions sans avoir à nous soucier des contraintes dues
à la taille en mémoire des types primitifs du C. Ces types sont \CodeInline{integer}
et \CodeInline{real}, qui représentent respectivement les entiers mathématiques et
les réels mathématiques (pour ces derniers, la modélisation est aussi proche que
possible de la réalité, mais la notion de réel ne peut pas être parfaitement
représentée).



Par la suite, nous utiliserons souvent des entiers à la place des classiques
\CodeInline{int} du C. La raison est simplement que beaucoup de propriétés sont
vraies quelle que soit la taille de l'entier (au sens C, cette fois) en entrée.



En revanche, nous ne parlerons pas de \CodeInline{real} VS \CodeInline{float/double}, parce que
cela induirait que nous parlions de preuve de programmes avec du calcul en virgule
flottante et que nous n'en parlerons pas ici. Par contre, ce tutoriel en parle :
\externalLink{Introduction à l'arithmétique flottante}{https://zestedesavoir.com/tutoriels/570/introduction-a-larithmetique-flottante/}.

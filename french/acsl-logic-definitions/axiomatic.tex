
Les axiomes sont des propriétés dont nous énonçons qu'elles sont vraies quelle
que soit la situation. C'est un moyen très pratique d'énoncer des notions
complexes qui pourront rendre le processus très efficace en abstrayant
justement cette complexité. Évidemment, comme toute propriété exprimée comme un
axiome est supposée vraie, il faut également faire très attention à ce que nous
définissons, car si nous introduisons une propriété fausse dans les notions que
nous supposons vraies alors ... nous saurons tout prouver, même ce qui est faux.



\levelThreeTitle{Syntaxe}


Pour introduire une définition axiomatique, nous utilisons la syntaxe suivante :



\begin{CodeBlock}{c}
/*@
  axiomatic Name_of_the_axiomatic_definition {
    // ici nous pouvons définir ou déclarer des fonctions et prédicats

    axiom axiom_name { Label0, ..., LabelN }:
      // property ;

    axiom other_axiom_name { Label0, ..., LabelM }:
      // property ;

    // ... nous pouvons en mettre autant que nous voulons
  }
*/
\end{CodeBlock}



Nous pouvons par exemple définir cette axiomatique :



\begin{CodeBlock}{c}
/*@
  axiomatic lt_plus_lt{
    axiom always_true_lt_plus_lt:
      \forall integer i, j; i < j ==> i+1 < j+1 ;
  }
*/
\end{CodeBlock}



et nous pouvons voir que dans Frama-C, la propriété est bien supposée
vraie\footnote{In section \ref{l2:acsl-logic-definitions-answers}, nous
présentons un axiome \emph{extrêmement} utile.} :


\image{first-axiom}


\levelFourTitle{Lien avec la notion de lemme}


Les lemmes et les axiomes nous permettront d'exprimer les mêmes types de
propriétés, à savoir des propriétés exprimées sur des variables quantifiées (et
éventuellement des variables globales, mais cela reste assez rare puisqu'il est
difficile de trouver une propriété qui soit globalement vraie à leur sujet tout
en étant intéressante). Outre ce point commun, il faut également savoir que
comme les axiomes, en dehors de leur définition, les lemmes sont considérés
vrais par WP.



La seule différence entre lemme et axiome du point de vue de la preuve est donc
que nous devrons fournir une preuve que le premier est valide alors que l'axiome
est toujours supposé vrai.



\levelThreeTitle{Définition de fonctions ou prédicats récursifs}


Les définitions axiomatiques de fonctions ou de prédicats récursifs sont
particulièrement utiles, car elles permettent d'empêcher les prouveurs de
dérouler la récursion quand c'est possible.



L'idée est alors de ne pas définir directement la fonction ou le prédicat, mais
plutôt de la déclarer puis de définir des axiomes spécifiant son comportement.
Si nous reprenons par exemple la factorielle, nous pouvons la définir
axiomatiquement de cette manière :



\begin{CodeBlock}{c}
/*@
  axiomatic Factorial{
    logic integer factorial(integer n);

    axiom factorial_0:
      \forall integer i; i <= 0 ==> 1 == factorial(i) ;

    axiom factorial_n:
      \forall integer i; i > 0 ==> i * factorial(i-1) == factorial(i) ;
  }
*/
\end{CodeBlock}



Dans cette définition axiomatique, notre fonction n'a pas de corps. Son
comportement étant défini par les axiomes ensuite définis. Excepté ceci,
rien ne change, en particulier, notre fonction peut être utilisée dans nos
spécifications, exactement comme avant.


Une petite subtilité
est qu'il faut prendre garde au fait que si les axiomes énoncent des propriétés
à propos du contenu d'une ou plusieurs zones mémoires pointées, il faut
spécifier ces zones mémoires en utilisant la notation \CodeInline{reads} au niveau de
la déclaration. Si nous oublions une telle spécification, le prédicat, ou la
fonction, sera considéré comme énoncé à propos du pointeur et non à propos de la
zone mémoire pointée. Une modification de celle-ci n'entraînera donc pas
l'invalidation d'une propriété connue axiomatiquement.


Par exemple, si nous prenons la définition inductive que nous avons rédigée pour
\CodeInline{zeroed} dans le chapitre précédent, nous pouvons l'écrire à l'aide
d'une définition axiomatique qui prendra cette forme :



\CodeBlockInput[5][16]{c}{reset-0.c}


Notons la présence de la clause \CodeInline{reads[b .. e-1]} qui spécifie la
position mémoire dont le
prédicat dépend. Tandis que nous n'avons pas besoin de spécifier les positions
mémoires « lues » par une définition inductive, nous devons spécifier ces propriétés
pour les propriétés définies axiomatiquement.


\levelThreeTitle{Consistance}


En ajoutant des axiomes à notre base de connaissances, nous pouvons produire des
preuves plus complexes, car certaines parties de cette preuve, mentionnées par
les axiomes, ne nécessiteront plus de preuves qui allongeraient le processus
complet. Seulement, en faisant cela \textbf{nous devons être extrêmement prudents}.
En effet, la moindre hypothèse fausse introduite dans la base pourraient rendre
tous nos raisonnements futiles. Notre raisonnement serait toujours correct, mais
basé sur des connaissances fausses, il ne nous apprendrait donc plus rien de correct.



L'exemple le plus simple à produire est le suivant :



\CodeBlockInput{c}{false.c}



Tout est prouvé, y compris que le déréférencement de l'adresse 0 est OK :



\image{false-axiom}[Preuve de tout un tas de choses fausses]


Évidemment cet exemple est extrême, nous n'écririons pas un tel axiome. Le
problème est qu'il est très facile d'écrire une axiomatique subtilement fausse
lorsque nous exprimons des propriétés plus complexes, ou que nous commençons à
poser des suppositions sur l'état global d'un système.



Quand nous commençons à créer de telles définitions, ajouter de temps en
temps une preuve ponctuelle de « \textit{false} » dont nous voulons qu'elle échoue permet
de s'assurer que notre définition n'est pas inconsistante. Mais cela ne fait pas
tout ! Si la subtilité qui crée le comportement faux est suffisamment cachée, les
prouveurs peuvent avoir besoin de beaucoup d'informations autre que l'axiomatique
elle-même pour être menés jusqu'à l'inconsistance, donc il faut toujours être
vigilant !



Notamment parce que, par exemple, la mention des valeurs lues par une fonction
ou un prédicat défini axiomatiquement est également importante pour la
consistance de l'axiomatique. En effet, comme mentionné précédemment, si nous
n'exprimons pas les valeurs lues dans le cas de l'usage d'un pointeur, la
modification d'une valeur du tableau n'invalide pas une propriété que l'on
connaîtrait à propos du contenu du tableau par exemple. Dans un tel cas, la
preuve passe, mais l'axiome n'exprimant pas le contenu, nous ne prouvons rien.



Par exemple, si nous reprenons l'exemple de mise à zéro, nous pouvons modifier
la définition de notre axiomatique en retirant la mention des valeurs dont
dépendent le prédicat : \CodeInline{reads a[b .. e-1]}. La preuve passera toujours,
mais n'exprimera plus rien à propos du contenu des tableaux considérés.
Par exemple, la fonction suivante :
\CodeBlockInput[16][25]{c}{all-zeroes-bad.c}
est prouvée correcte alors qu'une valeur a changé dans le tableau et donc elle
n'est plus 0.


Notons qu'à la différence des définitions inductives, où Why3 nous permet de contrôler
que ce que nous écrivons est relativement bien défini, nous n'avons pas de mécanisme
de ce genre pour les définitions axiomatiques. Ces axiomes sont simplement traduits
comme axiomes aussi du côté de Why3 et sont donc supposés vrais.


\levelThreeTitle{Exemple : comptage de valeurs}


Dans cet exemple, nous cherchons à prouver qu'un algorithme compte bien les
occurrences d'une valeur dans un tableau. Nous commençons par définir
axiomatiquement la notion de comptage dans un tableau :



\CodeBlockInput[3][22]{c}{occurrences_of.c}


Nous avons trois cas à gérer :
\begin{itemize}
\item la plage de valeur concernée est vide : le nombre d'occurrences est 0 ;
\item la plage de valeur n'est pas vide et le dernier élément est celui recherché :
le nombre d'occurrences est le nombre d'occurrences dans la plage actuelle que
l'on prive du dernier élément, plus 1 ;
\item la plage de valeur n'est pas vide et le dernier élément n'est pas celui
recherché : le nombre d'occurrences est le nombre d'occurrences dans la plage
privée du dernier élément.
\end{itemize}


Par la suite, nous pouvons écrire la fonction C exprimant ce comportement et la
prouver :



\CodeBlockInput[24][42]{c}{occurrences_of.c}



Une alternative au fait de spécifier que dans ce code \CodeInline{result} est au
maximum \CodeInline{i} est d'exprimer un lemme plus général à propos de la valeur
du nombre d'occurrences, dont nous savons qu'elle est comprise entre 0 et
la taille maximale de la plage de valeurs considérée :



\begin{CodeBlock}{c}
/*@
lemma l_occurrences_of_range{L}:
  \forall int v, int* array, integer from, to:
    from <= to ==> 0 <= l_occurrences_of(v, a, from, to) <= to-from;
*/
\end{CodeBlock}



La preuve de ce lemme ne pourra pas être déchargée par un solveur automatique. Il
faudra faire cette preuve interactivement avec Coq par exemple. Exprimer des
lemmes généraux prouvés manuellement est souvent une bonne manière d'ajouter des
outils aux prouveurs pour manipuler plus efficacement les axiomatiques, sans
ajouter formellement d'axiomes qui augmenteraient nos chances d'introduire des
erreurs. Ici, nous devrons quand même réaliser les preuves des propriétés
mentionnées.


\levelThreeTitle{Exemple : la fonction \CodeInline{strlen}}


Dans cette section, prouvons la fonction C \CodeInline{strlen}:


\CodeBlockInput[5]{c}{strlen-0.c}


Premièrement, nous devons fournir un contrat adapté. Supposons que nous avons
une fonction logique \CodeInline{strlen} qui retourne la longueur d'une chaîne
de caractères, à savoir ce que nous attendons de notre fonction C.


\begin{CodeBlock}{c}
/*@
  logic integer strlen(char const* s) = // on verra plus tard
*/
\end{CodeBlock}


Nous voulons recevoir une chaîne C valide en entrée et nous voulons en
calculer la longueur, une valeur qui correspond à celle fournie par la
fonction logique \CodeInline{strlen} appliquée à cette chaîne. Bien sûr,
cette fonction n'affecte rien. Définir ce qu'est une chaîne valide n'est
pas si simple. En effet, précédemment dans ce tutoriel, nous avons uniquement
travaillé avec des tableaux, en recevant en entrée à la fois un pointeur
vers le tableau et la taille du tableau. Cependant, tel que
c'est généralement fait en C, nous supposons que la chaîne termine avec
un caractère \CodeInline{'\textbackslash{}0'}. Cela signifie que nous
avons en fait besoin de la fonction logique \CodeInline{strlen} pour
définir ce qu'est une chaîne valide. Utilisons d'abord cette définition
(notons que nous utilisons \CodeInline{\textbackslash{}valid\_read},
car nous ne modifions pas la chaîne) pour fournir un contrat pour
\CodeInline{strlen} :


\CodeBlockInput[14][24]{c}{strlen-step.c}



Définir la fonction logique \CodeInline{strlen} n'est pas si simple. En effet,
nous devons calculer la fonction d'une chaîne en trouvant le caractère
\CodeInline{'\textbackslash{}0'}, et nous espérons le trouver, mais en fait,
nous pouvons facilement imaginer une chaîne qui n'en contiendrait pas. Dans
ce cas, nous voudrions avoir une valeur d'erreur, mais il est impossible de
garantir que le calcul termine : une fonction logique ne peut donc pas être
utilisée pour exprimer cette propriété.



Définissons donc cette fonction axiomatiquement. D'abord définissons ce qui
est lu par la fonction, à savoir : toute position mémoire depuis le pointeur
jusqu'à une plage infinie d'adresses. Ensuite, considérons deux cas : la chaîne
est finie, ou elle ne l'est pas, ce qui nous amène à deux axiomes :
\CodeInline{strlen} retourne une valeur positive qui correspond à l'indice
du premier caractère \CodeInline{'\textbackslash{}0'} et retourne une valeur
négative s'il n'y a pas de tel caractère.



\CodeBlockInput[4][15]{c}{strlen-proved.c}



Maintenant, nous pouvons être plus précis dans notre définition de
\CodeInline{\textbackslash{}valid\_read\_string}, une chaîne valide est une
chaîne telle qu'est valide depuis le premier indice jusqu'à \CodeInline{strlen}
de la chaîne et telle que cette valeur est plus grande que 0 (puisqu'une
chaîne infinie n'est pas valide) :


\CodeBlockInput[27][30]{c}{strlen-proved.c}


Avec cette nouvelle définition, nous pouvons avancer et fournir un invariant
utilisable pour la boucle de la fonction \CodeInline{strlen}. Il est plutôt
simple : \CodeInline{i} est compris entre 0 et \CodeInline{strlen(s)}, pour
toute valeur entre 0 et \CodeInline{i}, elles sont différentes de
\CodeInline{'\textbackslash{}0'}. Cette boucle n'affecte que \CodeInline{i}
et le variant correspond à la distance entre \CodeInline{i} et
\CodeInline{strlen(s)}. Cependant, si nous essayons de prouver cette fonction,
la preuve échoue. Pour avoir plus d'information, nous pouvons relancer la
preuve avec la vérification d'absence de RTE, avec la vérification de
non-débordement des entiers non signés :



\image{strlen-failed}


Nous pouvons voir que le prouveur échoue à montrer la propriété liée à la
plage de valeur de \CodeInline{i}, et que \CodeInline{i} peut excéder la
valeur maximale d'un entier non signé. Nous pouvons essayer de fournir une
limite à la valeur de \CodeInline{strlen(s)} en précondition.


\CodeBlockInput[33][33]{c}{strlen-proved.c}


Cependant, c'est insuffisant. La raison et que si nous avons défini la
valeur de \CodeInline{strlen(s)} comme l'index du premier
\CodeInline{'\textbackslash{}0'} dans le tableau, l'inverse n'est pas
vrai : savoir que la valeur de \CodeInline{strlen(s)} est positive n'est
pas suffisant pour déduire que la valeur à l'indice correspondant est
\CodeInline{'\textbackslash{}0'}. Nous étendons donc notre définition
axiomatique avec une autre proposition indiquant cette propriété (nous
ajoutons également une autre proposition à propos des valeurs qui
précèdent cet indice même si ici, ce n'est pas nécessaire) :



\CodeBlockInput[17][23]{c}{strlen-proved.c}


Cette fois la preuve réussit. Frama-C fournit ses propres headers pour
la bibliothèque standard, et cela inclut une définition axiomatique pour
la fonction logique \CodeInline{strlen}. Elle peut être trouvée dans le
dossier de Frama-C, sous le dossier \CodeInline{libc}, le fichier est
nommé \CodeInline{\_\_fc\_string\_axiomatic.h}. Notons que cette définition
a beaucoup plus d'axiomes pour déduire plus de propriétés à propos de
\CodeInline{strlen}.


\levelThreeTitle{Cluster de blocs axiomatiques}


La plupart des annotations globales (fonctions logiques et prédicats, lemmes, ...)
peuvent être définies à deux niveaux différents : à \emph{top-level}, le niveau
des fonctions, des variables globales, etc. (sauf pour les axiomes et les
fonctions et prédicats abstraits) ou dans des blocs axiomatiques. Les
annotations à \emph{top-level} sont toujours chargées dans le contexte des
obligations de preuve, ce n'est pas le cas des annotations des blocs
axiomatiques.


Dans l'exemple suivant :


\CodeBlockInput[5]{c}{cluster.c}


Puisque la clause \CodeInline{ensures} n'utilise que le prédicat \CodeInline{P}
qui est défini dans le bloc axiomatique \CodeInline{X}, WP ne charge que
l'axiome \CodeInline{x}. À l'inverse, si l'on remplace \CodeInline{P(p)} par
\CodeInline{Q(p)} dans la clause \CodeInline{ensures}, WP charge l'axiomatique
\CodeInline{Y} et donc l'axiome \CodeInline{y}, qui utilise également le
prédicat \CodeInline{P}. Par conséquent, le bloc axiomatique \CodeInline{X} est
également chargé. La clôture transitive des blocs axiomatiques ainsi chargés
forme un \emph{cluster} de définitions axiomatiques.


Ce comportement peut être utilisé pour éviter de fournir trop de lemmes aux
solveurs SMT. Cela peut permettre d'améliorer les performances de la preuve
dans certaines situations. Nous présenterons plus de détails à propos du guidage
de la preuve avec des lemmes dans la
section~\ref{l2:proof-methodologies-triggering-lemmas}.


\levelThreeTitle{Exercices}



\levelFourTitle{Comptage d'occurrences}


Le programme suivant ne peut pas être prouvé avec la définition axiomatique
que nous avons défini précédemment à propos du comptage d'occurrences :


\CodeBlockInput[18][30]{c}{ex-1-occurrences_of.c}


Ré-exprimer la définition axiomatique dans une forme qui permet de prouver le
programme.


\levelFourTitle{Plus grand diviseur commun}


Exprimer la fonction logique qui calcule le plus grand diviseur commun à l'aide
d'une définition axiomatique et écrire le contrat de la fonction \CodeInline{gcd}
puis la prouver ;


\CodeBlockInput[5]{c}{ex-2-gcd.c}


\levelFourTitle{Somme des N premiers entiers}


Exprimer la fonction logique qui calcule la somme des N premiers entiers à l'aide
d'une définition axiomatique. Écrire le contrat de la fonction \CodeInline{sum\_n}
et la prouver :


\CodeBlockInput[5]{c}{ex-3-sum-N-first-ints.c}


\levelFourTitle{Permutation}


Reprendre l'exemple à propos du tri par sélection
(section~\ref{l3:acsl-logic-definitions-inductive-sort}). Ré-exprimer le
prédicat de permutation comme une définition axiomatique. Prendre garde à la
clause \CodeInline{reads} (en particulier, noter que le prédicat relie deux
labels mémoire).


\CodeBlockInput[5]{c}{ex-4-sort.c}

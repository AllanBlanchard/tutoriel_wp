Le principe d'un contrat de fonction est de poser les conditions selon
lesquelles la fonction s'exécutera. On distinguera deux parties :
\begin{itemize}
\item \textbf{la précondition}, c'est-à-dire ce que doit assurer le code
      appelant à propos des variables passées en paramètres et de l'état de
      la mémoire globale pour que la fonction s'exécute correctement ;
\item \textbf{la postcondition}, c'est-à-dire ce que s'engage à fournir la
      fonction en retour à propos de l'état de la mémoire et de la valeur de
      retour.
\end{itemize}


Ces propriétés sont exprimées en langage ACSL dont la syntaxe est relativement
simple pour qui a déjà fait du C, puisqu'elle reprend la syntaxe des expressions
booléennes du C. Cependant, elle ajoute également :
\begin{itemize}
\item certaines constructions et connecteurs logiques qui ne sont pas présents
originellement en C pour faciliter l'écriture ;
\item des prédicats pré-implémentés pour exprimer des propriétés souvent utiles
en C (par exemple, la validité d'un pointeur) ;
\item ainsi que des types plus généraux que les types primitifs du C,
typiquement les types entiers ou réels.
\end{itemize}


Nous introduirons au fil du tutoriel les notations présentes dans le
langage ACSL.



Les spécifications ACSL sont introduites dans nos codes source par
l'intermédiaire d'annotations placées dans des commentaires. Syntaxiquement,
un contrat de fonction est intégré dans les sources de la manière suivante :



\begin{CodeBlock}{c}
/*@
  //contrat
*/
void foo(int bar){

}
\end{CodeBlock}



Notons bien le \CodeInline{@} à la suite du début du bloc de commentaire, c'est lui qui
fait que ce bloc devient un bloc d'annotations pour Frama-C et pas un simple
bloc de commentaires à ignorer.



Maintenant, regardons comment sont exprimés les contrats, à commencer par la
postcondition, puisque c'est ce que nous attendons en priorité de notre
programme (nous nous intéresserons ensuite aux préconditions).



\levelThreeTitle{Postcondition}


La postcondition d'une fonction est précisée avec la clause \CodeInline{ensures}.
Nous travaillerons avec la fonction suivante qui donne la valeur absolue
d'un entier reçu en entrée.
Une de ses postconditions est que le résultat (que nous notons avec le
mot-clé \CodeInline{\textbackslash{}result}) est supérieur ou égal à 0.



\CodeBlockInput[5]{c}{abs-1.c}



(Notons le \CodeInline{;} à la fin de la ligne de spécification comme en C).



Mais ce n'est pas tout, il faut également spécifier le comportement général
attendu d'une fonction renvoyant la valeur absolue. À savoir : si la valeur
est positive ou nulle, la fonction renvoie la même valeur, sinon elle renvoie
l'opposé de la valeur.



Nous pouvons spécifier plusieurs postconditions, soit en les composants avec
un \CodeInline{\&\&} comme en C, soit en introduisant une nouvelle clause \CodeInline{ensures},
comme illustré ci-dessous.



\CodeBlockInput[5]{c}{abs-2.c}



Cette spécification est l'opportunité de présenter un connecteur logique
très utile que propose ACSL mais qui n'est pas présent en C :
l'implication $A \Rightarrow B$, que l'on écrit en ACSL \CodeInline{A ==> B}.
La table de vérité de l'implication est la suivante :



\begin{longtabu}{|c|c|c|} \hline
$A$ & $B$ & $A \Rightarrow B$ \\ \hline
$F$ & $F$ & $V$ \\ \hline
$F$ & $V$ & $V$ \\ \hline
$V$ & $F$ & $F$ \\ \hline
$V$ & $V$ & $V$ \\ \hline
\end{longtabu}



Ce qui veut dire qu'une implication $A \Rightarrow B$ est vraie dans deux cas :
soit $A$ est fausse (et dans ce cas, il ne faut pas se préoccuper de $B$), soit
$A$ est vraie et alors $B$ doit être vraie aussi. Notons que cela signifie que
$A \Rightarrow B$ est équivalente à $\neg A \vee B$. L'idée étant finalement
« je veux savoir si dans le cas où $A$ est vrai, $B$ l'est aussi. Si $A$ est
faux, je considère que l'ensemble est vrai ». Par exemple, « s'il pleut, je veux
vérifier que j'ai un parapluie, s'il ne pleut pas, ce n'est pas un problème
de savoir si j'en ai un ou pas, tout va bien ».



Sa cousine l'équivalence $A \Leftrightarrow B$ (écrite \CodeInline{A <==> B} en ACSL)
est plus forte. C'est la conjonction de l'implication dans les deux sens :
$(A \Rightarrow B) \wedge (B \Rightarrow A)$. Cette formule n'est vraie que
dans deux cas : $A$ et $B$ sont vraies toutes les deux, ou fausses
toutes les deux (c'est donc la négation du ou-exclusif). Pour continuer avec
notre petit exemple, « je ne veux plus seulement savoir si j'ai un parapluie
quand il pleut, je veux être sûr de n'en avoir que dans le cas où il pleut ».



\begin{Information}
Profitons en pour rappeler l'ensemble des tables de vérités des opérateurs
usuels en logique du premier ordre ($\neg$ = \CodeInline{!}, $\wedge$ = \CodeInline{\&\&},
$\vee$ = \CodeInline{||}) :

\begin{longtabu}{|c|c|c|c|c|c|c|} \hline
$A$ & $B$ & $\neg A$ & $A \wedge B$ & $A \vee B$ & $A \Rightarrow B$ & $A \Leftrightarrow B$ \\ \hline
$F$ & $F$ & $V$ & $F$ & $F$ & $V$ & $V$ \\ \hline
$F$ & $V$ & $V$ & $F$ & $V$ & $V$ & $F$ \\ \hline
$V$ & $F$ & $F$ & $F$ & $V$ & $F$ & $F$ \\ \hline
$V$ & $V$ & $F$ & $V$ & $V$ & $V$ & $V$ \\ \hline
\end{longtabu}
\end{Information}


Revenons à notre spécification. Quand nos fichiers commencent à être longs et
contenir beaucoup de spécifications, il peut être commode de nommer les
propriétés que nous souhaitons vérifier. Pour cela, nous indiquons un nom (les
espaces ne sont pas autorisées) suivi de \CodeInline{:} avant de mettre effectivement
la propriété, il est possible de mettre plusieurs « étages » de noms pour
catégoriser nos propriétés. Par exemple, nous pouvons écrire ceci :



\CodeBlockInput[7]{c}{abs-3.c}



Dans une large part du tutoriel, nous ne nommerons pas les éléments que nous
tenterons de prouver, les propriétés seront généralement relativement simples
et peu nombreuses, les noms n'apporteraient pas beaucoup d'information.



Nous pouvons copier/coller la fonction \CodeInline{abs} et sa spécification dans un
fichier \CodeInline{abs.c} et regarder avec Frama-C si l'implémentation est conforme à la
spécification.



Pour cela, il faut lancer l'interface graphique de Frama-C (il est également
possible de se passer de l'interface graphique, cela ne sera pas présenté
dans ce tutoriel) soit par cette commande :



\begin{CodeBlock}{bash}
frama-c-gui
\end{CodeBlock}



Soit en l'ouvrant depuis l'environnement graphique.



Il est ensuite possible de cliquer sur le bouton « \textit{Create a new session from
existing C files} », les fichiers à analyser peuvent être sélectionnés par
double-clic, OK terminant la sélection. Par la suite, l'ajout d'autres
fichiers à la session s'effectue en cliquant sur Files > Source Files.



À noter également qu'il est possible d'ouvrir directement le(s) fichier(s)
depuis la ligne de commande en le(s) passant en argument(s) de \CodeInline{frama-c-gui}.



\begin{CodeBlock}{bash}
frama-c-gui abs.c
\end{CodeBlock}



\image{1-abs-1.png}


La fenêtre de Frama-C s'ouvre, dans le volet correspondant aux fichiers et aux
fonctions, nous pouvons sélectionner la fonction \CodeInline{abs}.
Pour chaque ligne \CodeInline{ensures}, il y a un cercle bleu dans la marge.
Ces cercles indiquent qu'aucune vérification n'a été tentée pour ces lignes.



Nous demandons de vérifier que le code répond à la spécification en faisant
un clic droit sur le nom de la fonction et « \textit{Prove function annotations by WP} » :



\image{1-abs-2}


Nous pouvons voir que les cercles bleus deviennent des pastilles vertes,
indiquant que la spécification est bien assurée par le programme. Il est
possible de prouver les propriétés une à une en cliquant-droit sur celles-ci
et pas sur le nom de la fonction. Nous pouvons également voir que deux
annotations supplémentaires ont été générées par Frama-C pour notre fonction :
\begin{itemize}
  \item \CodeInline{exits \textbackslash{}false}
  \item \CodeInline{terminates \textbackslash{}true}
\end{itemize}
Le premier dit que la fonction ne doit pas appeler la fonction C
\CodeInline{exit} (ou une fonction qui appellerait elle-même transitivement la
fonction \CodeInline{exit} à un certain point), et qu'elle finira normalement
son exécution en atteignant l'instruction \CodeInline{return}. La seconde dit
que la fonction ne doit pas s'exécuter à l'infini. Ces annotations sont générées
par défaut, cependant il est également possible d'ajouter sa propre clause
\CodeInline{exits} ou \CodeInline{terminates} pour changer son comportement.
Nous expliquerons cela plus tard, pour l'instant, ignorons simplement ces
annotations.



Le code est-il vraiment sans erreurs ? WP nous permet de nous
assurer que le code répond à la spécification, mais il ne fait pas de contrôle
d'erreur à l'exécution (\textit{runtime error}, abrégé RTE) si nous ne le demandons
pas. Un autre \textit{plugin} de Frama-C, appelé sobrement RTE, peut être
utilisé pour générer des annotations ACSL qui peuvent ensuite être vérifiées par
d'autres \textit{plugins}.
Son but est
d'ajouter des contrôles dans le programme pour les erreurs d'exécutions
possibles (débordements d'entiers, déréférencements de pointeurs invalides,
division par 0, etc).



Pour activer ce contrôle, nous devons activer l'option dans la configuration
de WP. Pour cela, il faut d'abord cliquer sur le bouton de configuration des
\textit{plugins} :


\image{plugin-options}


Et ensuite cocher l'option \CodeInline{-wp-rte} dans les options liées à WP :


\image{select-rte}


Il est également possible de demander à WP d'ajouter ces
contrôles par un clic droit sur le nom de la fonction puis
« Insert wp-rte guards ».


\begin{Information}
  À partir de ce point du tutoriel, \CodeInline{-wp-rte} devra toujours être
  activé pour traiter les exemples, sauf indication contraire.
\end{Information}


Enfin, nous relançons la vérification (nous pouvons également cliquer sur le
bouton « \textit{Reparse} » de la barre d'outils, cela aura pour effet de supprimer les
preuves déjà effectuées).



Nous voyons alors que WP échoue à prouver  l'impossibilité de
débordement arithmétique sur le calcul de -val. Et c'est bien normal parce
que -\CodeInline{INT\_MIN} ($-2^{31}$) > \CodeInline{INT\_MAX} ($2^{31}-1$).



\image{1-abs-4.png}


\begin{Information}
Il est bon de noter que le risque de dépassement est pour nous réel, car nos
machines (dont Frama-C détecte la configuration) fonctionne en
\externalLink{complément à deux}{https://fr.wikipedia.org/wiki/Compl\%C3\%A9ment\_\%C3\%A0\_deux}
pour lequel le dépassement n'est pas défini par la norme C.
\end{Information}


Ici, nous pouvons voir un autre type d'annotation ACSL. La
ligne \CodeInline{//@ assert propriete ;} nous permet de demander la vérification
d'une propriété à un point particulier du programme. Ici, l'outil l'a
insérée pour nous, car il faut vérifier que le \CodeInline{-val} ne provoque pas de
débordement, mais il est également possible d'en ajouter manuellement dans
un code.



Comme le montre cette capture d'écran, nous avons deux nouveaux codes couleur
pour les pastilles : vert + marron et orange.



La couleur verte et marron nous indique que la preuve a été effectuée, mais
qu'elle dépend potentiellement de propriétés pour lesquelles ce n'est pas le
cas.



Si  la preuve n'a pas été recommencée intégralement par rapport à la preuve
précédente, ces pastilles ont dû rester vertes, car les preuves associées ont
été réalisées avant l'introduction de la propriété nous assurant l'absence
d'erreur d'exécution, et ne se sont donc pas reposées sur la connaissance de
cette propriété puisqu'elle n'existait pas.



En effet, lorsque WP transmet une condition de vérification à un prouveur
automatique, il transmet deux types de propriétés : $G$, le but, la propriété
que l'on cherche à prouver, et $S_1$ ... $S_n$ les diverses suppositions que l'on
peut faire à propos de l'état du programme au point où l'on cherche à vérifier $G$.
Cependant, il ne reçoit pas, en retour, quelles propriétés ont été utilisées par
le prouveur pour valider $G$. Donc si $S_3$ fait partie des suppositions, et si
WP n'a pas réussi à obtenir une preuve de $S_3$, il indique que $G$ est vraie, mais
en supposant que $S_3$ est vraie, pour laquelle nous n'avons actuellement pas
établi de preuve.



La couleur orange nous signale qu'aucun prouveur n'a pu déterminer si la
propriété est vérifiable. Les deux raisons peuvent être :
\begin{itemize}
\item qu'il n'a pas assez d'information pour le déterminer ;
\item que malgré toutes ses recherches, il n'a pas pu trouver un résultat à
temps. Auquel cas, il rencontre un \textit{timeout} dont la durée est configurable
dans le volet de WP.
\end{itemize}


Dans le volet inférieur, nous pouvons sélectionner l'onglet « \textit{WP Goals} »,
celui-ci nous affiche la liste des conditions de vérification et pour chaque
prouveur indique un petit logo si la preuve a été tentée et si elle a été
réussie, échouée ou a rencontré un \textit{timeout} (logo avec les ciseaux).
Pour voir la totalité des conditions de vérification, il
faut s'assurer que « \textit{All Results} » est bien sélectionné dans le champ encadré
dans la capture.



\image{1-abs-5.png}


Le tableau est découpé comme suit, en première colonne, nous avons le nom de la
fonction où se trouve le but à prouver. En seconde colonne, nous trouvons le nom
du but. Ici par exemple notre postcondition nommée est estampillée
\CodeInline{postcondition 'positive\_value,function\_result'}, nous pouvons d'ailleurs noter
que lorsqu'une propriété est sélectionnée dans le tableau, elle est également
surlignée dans le code source. Les propriétés anonymes se voient assignées
comme nom le type de propriété voulu. En troisième colonne, nous trouvons le
modèle mémoire utilisé pour la preuve, (nous n'en parlerons pas dans ce
tutoriel). Finalement, les dernières colonnes représentent les différents
prouveurs accessibles à WP.



Dans ces prouveurs, le premier élément de la colonne est Qed. Ce n'est pas
à proprement parler un prouveur. C'est un outil utilisé par WP pour simplifier
les propriétés avant de les envoyer aux prouveurs externes. Ensuite, nous
voyons la colonne Script, les scripts fournissent une manière de terminer
les preuves à la main lorsque les prouveurs automatiques n'y arrivent pas.
Finalement, nous trouvons la colonne Alt-Ergo, qui est un prouveur automatique.
Notons que sur la propriété en question des ciseaux sont indiqués, cela
veut dire que le prouveur a été stoppé à cause d'un \textit{timeout}.


En fait, si nous double-cliquons sur la
propriété « ne pas déborder » (surlignée en bleu dans la capture précédente),
nous pouvons voir ceci (si ce n'est pas le cas, il faut s'assurer que
« \textit{Raw obligation} » est bien sélectionné dans le champ encadré en bleu) :



\image{1-abs-6.png}


C'est la condition de vérification que génère WP par rapport à notre propriété et
notre programme, il n'est pas nécessaire de comprendre tout ce qu'il s'y passe,
juste d'avoir une idée globale. Elle contient (dans la partie « \textit{Assume} ») les
suppositions que nous avons pu donner et celles que WP a pu déduire des
instructions du programme. Elle contient également (dans la partie « \textit{Prove} »)
la propriété que nous souhaitons vérifier.



Que fait WP avec ces éléments ? En fait, il les transforme en une formule
logique puis demande aux différents prouveurs s'il est possible de la
satisfaire (de trouver pour chaque variable, une valeur qui rend la formule
vraie), cela détermine si la propriété est prouvable. Mais avant d'envoyer
cette formule aux prouveurs, WP utilise un module qui s'appelle Qed et qui est
capable de faire différentes simplifications à son sujet. Parfois, comme dans
le cas des autres propriétés de \CodeInline{abs}, ces simplifications suffisent à
déterminer que la propriété est forcément vraie, auquel cas, nous ne faisons
pas appel aux prouveurs.



Lorsque les prouveurs automatiques ne parviennent pas à assurer que nos
propriétés sont bien vérifiées, il est parfois difficile de comprendre
pourquoi. En effet, les prouveurs sont généralement incapables de nous
répondre autre chose que « oui », « non » ou « inconnu », ils ne sont pas
capables d'extraire le « pourquoi » d'un « non » ou d'un « inconnu ». Il
existe des outils qui
sont capables d'explorer les arbres de preuve pour en extraire ce type
d'information, Frama-C n'en possède pas à l'heure actuelle. La lecture des
conditions de vérification peut parfois nous aider, mais cela demande un peu
d'habitude pour pouvoir les déchiffrer facilement. Finalement, le meilleur
moyen de comprendre la raison d'un échec est d'effectuer la preuve de manière
interactive avec Coq. En revanche, il faut déjà avoir une certaine habitude de
ce langage pour ne pas être perdu devant les conditions de vérification générées par
WP, étant donné que celles-ci encodent les éléments de la sémantique de C, ce
qui rend le code souvent indigeste.



Si nous retournons dans notre tableau des conditions de vérification (bouton
encadré en rouge dans la capture d'écran précédente), nous pouvons donc voir
que les hypothèses n'ont pas suffi aux prouveurs pour déterminer que la
propriété  « absence de débordement » est vraie (et nous l'avons dit : c'est
normal), il nous faut donc ajouter une hypothèse supplémentaire pour garantir
le bon fonctionnement de la fonction : une précondition d'appel.



\levelThreeTitle{Précondition}


Les préconditions de fonctions sont introduites par la clause \CodeInline{requires}.
De la même manière qu'avec \CodeInline{ensures}, nous pouvons composer nos
expressions logiques et mettre plusieurs préconditions :



\begin{CodeBlock}{c}
/*@
  requires 0 <= a < 100;
  requires b < a;
*/
void foo(int a, int b){

}
\end{CodeBlock}



Les préconditions sont des propriétés sur les entrées (et potentiellement sur
des variables globales) qui seront supposées préalablement vraies lors de
l'analyse de la fonction. La preuve que celles-ci sont effectivement validées
n'interviendra qu'aux points où la fonction est appelée.



Dans ce petit exemple, nous pouvons également noter une petite différence avec
le C dans l'écriture des expressions booléennes. Si nous voulons spécifier
que \CodeInline{a} se trouve entre 0 et 100, il n'y a pas besoin d'écrire \CodeInline{0 <= a \&\& a < 100}
(c'est-à-dire en composant les deux comparaisons avec un \CodeInline{\&\&}). Nous
pouvons simplement écrire \CodeInline{0 <= a < 100} et l'outil se chargera de faire
la traduction nécessaire.



Si nous revenons à notre exemple de la valeur absolue, pour éviter le
débordement arithmétique, il suffit que la valeur de \CodeInline{val} soit
strictement  supérieure à \CodeInline{INT\_MIN} pour garantir que le
débordement n'arrive pas.
Nous l'ajoutons donc comme précondition (à noter : il faut également
inclure l'en-tête où \CodeInline{INT\_MIN} est défini) :



\CodeBlockInput{c}{abs-4.c}



\begin{Warning}
Rappel : la fenêtre de Frama-C ne permet pas l'édition du code source.
\end{Warning}


Une fois le code source modifié de cette manière, un clic sur « \textit{Reparse} » et
nous lançons à nouveau l'analyse. Cette fois, tout est validé pour WP ; notre
implémentation est prouvée :



\image{2-abs-1.png}


Nous pouvons également vérifier qu'une fonction qui appellerait \CodeInline{abs}
satisfait bien la précondition qu'elle impose :



\CodeBlockInput[15]{c}{abs-5.c}



\image{2-foo-1.png}

Notons qu'en cliquant sur la pastille à côté de l'appel de fonction, nous
pouvons voir la liste des préconditions et voir quelles sont celles qui ne sont
pas vérifiées. Ici, nous n'avons qu'une précondition, mais quand il y en a
plusieurs, c'est très utile pour pouvoir voir quel est exactement le problème.


\image{2-foo-1-bis.png}


Pour modifier un peu l'exemple, nous pouvons essayer d'inverser les deux
dernières lignes. Auquel cas, nous pouvons voir que l'appel \CodeInline{abs(a)}
est validé par WP s'il se trouve après l'appel \CodeInline{abs(INT\_MIN)} !
Pourquoi ?



Il faut bien garder en tête que le principe de la preuve déductive est de nous
assurer que si les préconditions sont vérifiées et que le calcul termine alors
la postcondition est vérifiée.


Si nous donnons à notre fonction une valeur qui viole explicitement sa
précondition, nous pouvons déduire que n'importe quoi peut arriver, incluant
obtenir « faux » en postcondition. Plus précisément, ici, après l'appel, nous
supposons que la précondition est vraie (puisque la fonction ne peut pas
modifier la valeur reçue en paramètre), sinon la fonction n'aurait pas pu
s'exécuter correctement. Par conséquent, nous supposons que
\CodeInline{INT\_MIN < INT\_MIN} qui est trivialement faux. À partir de là,
nous pouvons  prouver tout ce que nous voulons, car ce « faux » devient une
supposition pour tout appel qui viendrait ensuite. À partir de « faux », nous
prouvons tout ce que
nous voulons, car si nous avons la preuve de « faux » alors « faux » est vrai,
de même que « vrai » est vrai. Donc tout est vrai.



En prenant le programme modifié, nous pouvons d'ailleurs regarder les conditions
de vérification générées par WP pour l'appel fautif et l'appel prouvé par
conséquent :



\image{2-foo-2.png}


\image{2-foo-3.png}


Nous pouvons remarquer que pour les appels de fonctions, l'interface graphique
surligne le chemin d'exécution suivi avant l'appel dont nous cherchons à
vérifier la précondition. Ensuite, si nous regardons l'appel \CodeInline{abs(INT\_MIN)},
nous remarquons qu'à force de simplifications, Qed a déduit que nous
cherchons à prouver « False ». Conséquence logique, l'appel suivant \CodeInline{abs(a)}
reçoit dans ses suppositions « False ». C'est pourquoi Qed est capable de déduire
immédiatement « True ».



La deuxième partie de la question est alors : pourquoi lorsque nous mettons les
appels dans l'autre sens (\CodeInline{abs(a)} puis \CodeInline{abs(INT\_MIN)}), nous obtenons
quand même une violation de la précondition sur le deuxième ? La réponse est
simplement que pour \CodeInline{abs(a)} nous ajoutons dans nos suppositions la
connaissance \CodeInline{a < INT\_MIN}, et tandis que nous n'avons pas de preuve
que c'est vrai, nous n'en avons pas non plus que c'est faux. Donc si nous obtenons
nécessairement une preuve de « faux » avec un appel \CodeInline{abs(INT\_MIN)}, ce n'est
pas le cas de l'appel \CodeInline{abs(a)}, car \CodeInline{a} peut être autre
chose que \CodeInline{INT\_MIN}.


\levelThreeTitle{Le cas particulier de la fonction \CodeInline{main}}


La fonction \CodeInline{main} n'est généralement pas appelée directement par le
programme. Cependant, elle peut avoir des préconditions. Dans un tel cas, WP
génère des conditions de vérification pour elles, mais avec une différence
importante : le contexte fournit pour la preuve n'est pas le même que pour les
autres fonctions. En effet, comme la fonction principale est appelée avant toute
autre fonction du programme, à ce point de l'exécution, les variables globales
ont nécessairement la valeur qui leur a été transmise à l'initialisation.


Illustrons cela avec l'extrait de code suivant :


\CodeBlockInput[7]{c}{main.c}


Si nous exécutons WP sur cet exemple, la clause \CodeInline{requires} de la
fonction \CodeInline{main} est prouvée, tandis qu'aucune tentative de preuve n'a
été réalisée pour la précondition de la fonction \CodeInline{f} puisqu'elle
n'est pas appelée dans le programme. La précondition de la fonction
\CodeInline{main} est vérifiée, car après l'initialisation \CodeInline{h} a la
valeur 42 qui est bien entendu entre 0 et 100. Remarquons que l'assertion dans
la fonction \CodeInline{main} est, elle aussi, prouvée puisque la valeur de
\CodeInline{h} n'est pas changée par la fonction, alors que dans la fonction
\CodeInline{f} ce n'est pas le cas puisque WP ne fait aucune supposition à
propos des valeurs des variables globales.


\image{3-main-1}


Notons que WP fait cette hypothèse à propos de la fonction \CodeInline{main}
parce que c'est le comportement par défaut de Frama-C. Ce comportement peut être
configuré à l'aide de l'option \CodeInline{-lib-entry} qui dit à Frama-C de
considérer que toutes les fonctions (incluant donc \CodeInline{main}) peuvent
être appelées dans le programme. Donc, quand nous activons cette option sur la
ligne de commande de Frama-C, pour notre exemple, aucune condition de
vérification n'est générée pour les clauses \CodeInline{requires} de
\CodeInline{main} puisqu'elle n'est pas appelée et aucune des assertions n'est
prouvée :


\image{3-main-2}


Finalement, Frama-C permet de configurer quelle fonction du programme doit être
considérée comme la fonction principale. Par exemple, si nous appelons Frama-C
avec l'option \CodeInline{-main=f} sur la ligne de commande, la clause
\CodeInline{requires} de la fonction \CodeInline{f} est prouvée, de même que
l'assertion qu'elle contient. À l'inverse, aucune vérification n'est tentée
pour la clause \CodeInline{requires} de la fonction \CodeInline{main} et
l'assertion dans la fonction \CodeInline{main} n'est pas prouvée.


\image{3-main-3}


\levelThreeTitle{Exercices}


Ces exercices ne sont pas absolument nécessaires pour lire les chapitres à
venir dans ce tutoriel, nous conseillons quand même de les réaliser. Nous
suggérons aussi fortement d'au moins lire le quatrième exercice qui introduit
une notation qui peut parfois d'avérer utile.


\levelFourTitle{Addition}


Écrire la postcondition de la fonction d'addition suivante :


\CodeBlockInput[5]{c}{ex-1-addition.c}


Lancer la commande :

\begin{CodeBlock}{bash}
frama-c-gui your-file.c -wp
\end{CodeBlock}


Lorsque la preuve que la fonction est conforme à son contrat est établie, lancer
la commande :
\begin{CodeBlock}{bash}
frama-c-gui your-file.c -wp -wp-rte
\end{CodeBlock}
qui devrait échouer. Adapter le contrat en ajoutant la bonne précondition.


\levelFourTitle{Distance}


Écrire la postcondition de la fonction distance suivante, en exprimant
la valeur de \CodeInline{b} en fonction de \CodeInline{a} et
\CodeInline{\textbackslash{}result} :


\CodeBlockInput[5]{c}{ex-2-distance.c}


Lancer la commande :


\begin{CodeBlock}{bash}
frama-c-gui your-file.c -wp
\end{CodeBlock}


Lorsque la preuve que la fonction est conforme à son contrat est établie, lancer
la commande :
\begin{CodeBlock}{bash}
frama-c-gui your-file.c -wp -wp-rte
\end{CodeBlock}
qui devrait échouer. Adapter le contrat en ajoutant la bonne précondition.


\levelFourTitle{Lettres de l'alphabet}


Écrire la postcondition de la fonction suivante, qui retourne vrai si le
caractère reçu en entrée est une lettre de l'alphabet. Utiliser la relation
d'équivalence  \CodeInline{<==>}.


\CodeBlockInput[5]{c}{ex-3-alphabet.c}


Lancer la commande :


\begin{CodeBlock}{bash}
frama-c-gui your-file.c -wp
\end{CodeBlock}


Toutes les conditions de vérification devraient être prouvées, y compris les
assertions dans la fonction main.


\levelFourTitle{Jours du mois}


Écrire la postcondition de la fonction suivante qui retourne le nombre de
jours en fonction du mois reçu en entrée (NB : nous considérons que le mois
reçu est entre 1 et 12), pour février, nous considérons uniquement le cas
où il a 28 jours, nous verrons plus tard comment régler ce problème :


\CodeBlockInput[5]{c}{ex-4-day-month.c}


Lancer la commande :


\begin{CodeBlock}{bash}
frama-c-gui your-file.c -wp
\end{CodeBlock}


Lorsque la preuve que la fonction est conforme à contrat est établie, lancer
la commande :

\begin{CodeBlock}{bash}
frama-c-gui your-file.c -wp -wp-rte
\end{CodeBlock}


Si cela échoue, adapter le contrat en ajoutant la bonne précondition.


Le lecteur aura peut-être constaté qu'écrire la postcondition est un peu
laborieux. Il est possible de simplifier cela. ACSL fournit la notion
d'ensemble mathématique et l'opérateur \CodeInline{\textbackslash{}in} qui
peut être utilisé pour vérifier si une valeur est dans un ensemble ou non.


Par exemple :

\begin{CodeBlock}{c}
//@ assert 13 \in { 1, 2, 3, 4, 5 } ; // FAUX
//@ assert 3  \in { 1, 2, 3, 4, 5 } ; // VRAI
\end{CodeBlock}


Modifier la postcondition en utilisant cette notation.


\levelFourTitle{Le dernier angle d'un triangle}


Cette fonction reçoit deux valeurs d'angle en entrée et retourne
la valeur du dernier angle composant le triangle correspondant en se
reposant sur la propriété que la somme des angles d'un triangle vaut
180 degrés. Écrire la postcondition qui exprime que la somme des trois
angles vaut 180.


\CodeBlockInput[5]{c}{ex-5-last-angle.c}


Lancer la commande :


\begin{CodeBlock}{bash}
frama-c-gui your-file.c -wp
\end{CodeBlock}


Lorsque la preuve que la fonction est conforme à son contrat est établie, lancer
la commande :

\begin{CodeBlock}{bash}
frama-c-gui your-file.c -wp -wp-rte
\end{CodeBlock}


Si cela échoue, adapter le contrat en ajoutant la bonne précondition.
Notons que la valeur de chaque angle ne peut pas être supérieure à 180
et que cela inclut l'angle résultant.

\levelThreeTitle{Assurer la conformité des logiciels}


Assurer qu'un programme a un comportement conforme à celui que nous attendons
est souvent une tâche difficile. Plus en amont encore, il est déjà complexe
d'établir sur quel critère nous pouvons estimer que le programme « fonctionne ».



\begin{itemize}
\item Les débutants « essayent » simplement leurs programmes et estiment qu'ils
fonctionnent s'ils ne plantent pas.
\item Les codeurs un peu plus habitués établissent quelques jeux de tests dont ils
connaissent les résultats et comparent les sorties de leurs programmes.
\item La majorité des entreprises établissent des bases de tests conséquentes,
couvrant un maximum de code, tests exécutés de manière systématique sur les
codes de leurs bases. Certaines font du développement dirigé par le test.
\item Les entreprises de domaines critiques, comme l'aérospatial, le ferroviaire ou
l'armement, passent par des certifications leur demandant de répondre à des
critères très stricts de codage et de couverture de code par les tests.
\end{itemize}


Et bien sûr, il existe tous les « entre-deux » dans cette liste.



Dans toutes ces manières de s'assurer qu'un programme fait ce qui est attendu,
il y a un mot qui revient souvent : \textit{test}. Nous \textit{essayons} des entrées de
programme dans le but d'isoler des cas qui poseraient problème. Nous fournissons
des entrées \textit{estimées représentatives} de l'utilisation réelle du programme
(laissant souvent de côté les usages non prévus, qui sont souvent les plus
dangereux) et
nous nous assurons que les résultats attendus sont conformes. Mais nous ne
pouvons pas \textit{tout} tester. Nous ne pouvons pas essayer \textit{toutes} les
combinaisons de \textit{toutes} les entrées possibles du programme. Toute la
difficulté réside donc dans le fait de choisir les bons tests.



Le but de la preuve de programmes est de s'assurer que, quelle que soit l'entrée
fournie au programme, si elle est conforme à la spécification, alors le programme
fera ce qui est attendu. Cependant, comme nous ne pouvons pas tout essayer, nous
allons établir formellement, mathématiquement, la preuve que le logiciel ne
peut exhiber que les comportements qui sont spécifiés et que les erreurs
d'exécution n'en font pas partie.



Une phrase très célèbre de \externalLink{Dijkstra}{https://fr.wikipedia.org/wiki/Edsger_Dijkstra} exprime très clairement la différence entre
test et preuve :



\begin{Quotation}[Dijkstra]
Program testing can be used to show the presence of bugs, but never to show
their absence!
\end{Quotation}



Le test de programme peut être utilisé pour montrer la présence de bugs, mais
jamais pour montrer leur absence.



\levelFourTitle{Le Graal du logiciel sans bug}


Dans chaque nouvelle à propos d'attaque sur des systèmes informatiques, ou
des virus, ou des bugs provoquant des crashs, il y a toujours la remarque
séculaire « le programme inviolable/incassable/sans bugs n'existe pas ». Et
il s'avère généralement que, bien qu'assez vraie, cette phrase est assez
mal comprise.



Tout d’abord, nous ne précisons pas ce que nous entendons par « sans bug ».
La création d’un logiciel fait toujours au moins intervenir deux étapes : la
rédaction de ce qui est attendu sous la forme d’une spécification (souvent
un cahier des charges) et la réalisation du logiciel répondant à cette
spécification. À cela s’ajoute la spécification de notre langage de
programmation qui nous définit la manière correcte de l’utiliser. Chacun de
ces aspects peut donner lieu à l’introduction de bugs, que nous pouvons
séparer en trois catégories :
\begin{itemize}
\item le programme n’est pas conforme, ou son comportement non défini,
      d’après la spécification du langage (par exemple, le programme accède
      en dehors d’un tableau pendant une recherche de l'indice de la valeur
      minimale) ;
\item le programme n’est pas conforme à la spécification que nous en avons
      donné (par exemple, nous avons défini que le programme doit trouver
      l'indice de la valeur minimale d’un tableau, mais en fait il ne
      regarde pas sa dernière valeur à cause d’une erreur) ;
\item la spécification ne reflète pas parfaitement « ce que nous voulons »,
      et par conséquent, le programme non plus (par exemple, nous avons
      défini que le programme doit trouver l'indice de la valeur minimale
      d’un tableau, mais nous n’avons pas spécifié que s’il y en a plusieurs,
      il faut prendre la première, parce que cela me semblait trop évident,
      mais du coup ce n’est pas ce que fait le programme).
\end{itemize}


Chacune de ces catégories peut affecter la sûreté et la sécurité de nos
programmes, qui ne sont pas des notions tout à fait équivalentes. Pour donner
une idée de la différence qui existe entre ces deux notions, nous pouvons
dire que dans le cas de la sécurité, on suppose qu’il existe une entité
capable d’attaquer (volontairement ou pas) le système, tandis que dans la
sûreté, nous voulons juste vérifier que lorsqu’il est utilisé de manière
conforme, le système se comporte correctement. Par conséquent, sans sûreté,
nous ne pouvons pas avoir la sécurité\footnote{Selon votre domaine
d’activité, le terme sûreté peut avoir un sens très différent. Plus
précisément, un système sûr serait un système qui ne doit jamais mettre la
vie d’un humain en danger. Et donc dans ce cas, la situation est inverse :
sans sécurité, nous ne pouvons pas avoir la sûreté. Dans ce tutoriel, nous
nous plaçons bien dans le cas « sûreté = le programme ne présente pas de
problème lorsqu’on l’utilise de manière conforme ».}.


Tout au long de ce tutoriel, nous montrerons comment prouver que les
implémentations de nos programmes ne contiennent pas de bugs correspondant
aux deux premières catégories définies plus haut, à savoir qu’ils sont
conformes :
\begin{itemize}
\item à la spécification de notre langage ;
\item à la spécification de ce que nous attendons d’eux.
\end{itemize}


Mais quels sont les arguments de la preuve par rapport aux tests ? D’abord,
la preuve est complète, elle n’oublie pas de cas s’ils sont présents dans la
spécification (le test serait trop coûteux s’il était exhaustif). D’autre
part, l’obligation de formaliser la spécification sous une forme logique
demande de comprendre exactement le besoin auquel nous devons répondre.

Nous pourrions dire avec cynisme que la preuve nous montre finalement que
l’implémentation « ne contient aucun bug de plus que la spécification », et
donc que nous ne traitons pas la troisième catégorie de bugs que nous avons
définie. Cependant, être sûr que le programme « ne contient aucun bug de
plus que la spécification » est déjà un sacré pas en avant par rapport à
savoir que le programme « ne contient pas beaucoup plus de bugs que la
spécification », après tout cela représente deux catégories entières de bugs
dont nous nous débarrassons, bugs qui peuvent déjà sévèrement compromettre la
sûreté et la sécurité de nos programmes. Ensuite, il existe également des
techniques pour traiter la troisième catégorie de bugs, en analysant les
spécifications en quête d’erreurs ou d’insuffisance. Par exemple, les
techniques de \emph{model checking} - vérification de modèles - permettent de
construire un modèle abstrait à partir d’une spécification et de produire
un ensemble d’états accessibles du programme d’après le modèle. En
caractérisant les états fautifs, nous sommes en mesure de déterminer si les
états accessibles contiennent des états fautifs.


\levelThreeTitle{Un peu de contexte}

En informatique, les méthodes dites \textit{formelles} permettent
 de raisonner de manière rigoureuse, mathématique, à propos des
programmes. Il existe un très large panel de méthodes formelles qui peuvent
intervenir à tous les niveaux de la conception, l'implémentation, l'analyse et
la validation des programmes ou de manière plus générale de tout système
permettant le traitement de l'information.



Ici, nous nous intéresserons à la vérification de la conformité de nos
programmes au comportement attendu. Nous utiliserons
des outils capables d'analyser le code et de nous dire si oui, ou non, notre
code correspond à ce que nous voulons exprimer. La technique que nous allons
étudier ici est une analyse statique, à opposer aux analyses dynamiques.



Le principe des analyses statiques est que nous n'exécuterons pas le programme
pour nous assurer que son fonctionnement est correct, mais nous raisonnerons sur
un modèle mathématique définissant l'ensemble des états qu'il peut atteindre.
À l'inverse, les analyses dynamiques comme le test de programmes nécessitent
d'exécuter le code analysé. Il existe également des analyses dynamiques et
formelles, comme de la génération automatique de tests ou encore des techniques
de \textit{monitoring} de code qui pourront, par exemple, instrumenter un code
source afin de vérifier à l'exécution que les allocations et désallocation de
mémoire sont faites de manière sûre.



Dans le cas des analyses statiques, le modèle utilisé est plus ou moins
abstrait selon la technique employée, c'est donc une approximation des états
possibles de notre programme. Plus l'approximation est précise, plus le modèle est
concret, plus l'approximation est large, plus il est abstrait.



Pour illustrer la différence entre modèle concret et abstrait, prenons
l'exemple d'un chronomètre simple. Une modélisation très abstraite du
comportement de notre chronomètre est la suivante :


\image{model-fr}[Modélisation très abstraite d'un chronomètre]


Nous avons bien une modélisation du comportement de notre chronomètre avec
différents états qu'il peut atteindre en fonction des actions qu'il subit.
Cependant, nous n'avons pas modélisé comment ces états sont
représentés dans le programme (Est-ce une énumération ? Une position précise
atteinte au sein du code ?), ni comment est modélisé le calcul du temps (une seule
variable en secondes ? Plusieurs variables heures, minutes, secondes ?). Nous
aurions donc bien du mal à spécifier des propriétés à propos de notre programme.
Nous pouvons ajouter des informations :
\begin{itemize}
\item état arrêté à zéro : temps = 0 s ;
\item état en marche : temps > 0 s ;
\item état arrêté : temps > 0 s.
\end{itemize}


Ce qui nous donne déjà un modèle plus concret, mais qui est toujours insuffisant
pour poser des questions intéressantes à propos de notre système comme : « est-il
possible que dans l'état arrêté, le temps continue de s'écouler ? ». Car nous
n'avons pas modélisé l'écoulement du temps par le chronomètre.



À l'inverse, avec le code source du programme, nous avons un modèle concret du
chronomètre. Le code source exprime bien le comportement du chronomètre puisque
c'est lui qui sert à produire l'exécutable. Mais ce n'est pas pour autant le plus concret ! Par exemple, l'exécutable en code machine obtenu à la fin
de la compilation est un modèle encore plus concret de notre programme.



Plus un modèle est concret, plus il décrit précisément le comportement de notre
programme. Le code source exprime le comportement plus précisément que notre
diagramme, mais il est moins précis que le code de l'exécutable. Cependant, plus
un modèle est précis, plus il est difficile d'avoir une vision globale du
comportement qu'il définit. Notre diagramme est compréhensible en un coup d'\oe{}il,
le code demande un peu plus de temps, quant à l'exécutable ... Toute personne qui
a déjà ouvert par erreur un exécutable avec un éditeur de texte sait que ce n'est
pas très agréable à lire dans son ensemble\footnote{Il existe des analyses
formelles cherchant à comprendre le fonctionnement des exécutables en code
machine, par exemple pour comprendre ce que font des logiciels malveillants ou
pour détecter des failles de sécurité introduites lors de la compilation.}.



Lorsque nous créons une abstraction d'un système, nous l'approximons, pour limiter
la quantité d'informations que nous avons à son sujet et faciliter notre
raisonnement. Une des contraintes si nous voulons qu'une vérification soit
correcte est bien sûr que nous ne devons jamais sous-approximer les comportements
du programme : nous risquerions d'écarter un comportement qui contient une erreur.
Inversement, si nous sur-approximons notre programme, nous ajoutons des exécutions
qui ne peuvent pas arriver en réalité et si nous ajoutons trop d'exécutions
inexistantes, nous pourrions ne plus être en mesure de prouver son bon
fonctionnement dans le cas où certaines d'entre elles seraient fautives.



Dans le cas de l'outil que nous utiliserons, le modèle est plutôt concret.
Chaque type d'instruction, chaque type de structure de contrôle d'un programme
se voit attribuer une sémantique, une représentation de son comportement dans
un monde purement logique, mathématique. Le cadre logique qui nous intéresse
ici, c'est la logique de Hoare adaptée pour le langage C et toutes ses
subtilités (qui rendent donc le modèle final très concret).


\levelThreeTitle{Les triplets de Hoare}


La logique de Hoare est une méthode de formalisation des programmes proposée
par \externalLink{Tony Hoare}{https://fr.wikipedia.org/wiki/Charles_Antony_Richard_Hoare}
en 1969 dans un article intitulé \textit{An Axiomatic Basis for
Computer Programming} (une base axiomatique pour la programmation des
ordinateurs). Cette méthode définit :
\begin{itemize}
\item des axiomes, qui sont des propriétés que nous admettons, comme \\
« l'action “ne rien faire” ne change pas l'état du programme »,
\item et des règles pour raisonner à propos des différentes possibilités de
compositions d'actions, par exemple « l'action “ne rien faire” puis “faire
l'action A” est équivalent à “faire l'action A” ».
\end{itemize}


Le comportement d'un programme est défini par ce que nous appelons les triplets
de Hoare :



$$\{P\} \quad C \quad \{Q\}$$



Où $P$ et $Q$ sont des prédicats (c'est-à-dire des formules logiques) qui nous disent dans
quel état se trouve la mémoire traitée par le programme. $C$ est un ensemble de
commandes définissant un programme. Cette écriture nous dit « si nous sommes
dans un état où $P$ est vrai, après exécution de $C$ et si $C$ termine,
alors $Q$ sera vrai pour le nouvel état du programme ». Dit autrement, $P$ est
une précondition suffisante pour que $C$ nous amène à la postcondition $Q$.
Par exemple, le triplet correspondant à l'action « ne rien faire »
(\CodeInline{skip}) est le suivant :




$$\{P\} \quad \texttt{skip} \quad \{P\}$$



Quand nous ne faisons rien, la postcondition est la même que la précondition.



Tout au long de ce tutoriel, nous verrons la sémantique de diverses
constructions (blocs conditionnels, boucles, etc.) dans la logique de Hoare.
Nous n'allons donc pas tout de suite rentrer dans ces détails puisque nous en
aurons l'occasion plus tard. Il n'est pas nécessaire de mémoriser ces notions
ni même de comprendre toute la théorie derrière, mais il est toujours utile
d'avoir au moins une vague idée du fonctionnement de l'outil que nous
utilisons.



Tout ceci nous donne les bases permettant de dire « voilà ce que fait cette
action » mais ne nous donne pas encore de matériel pour mécaniser la preuve.
L'outil que nous utiliserons repose sur la technique du calcul de plus
faible précondition.



\levelThreeTitle{Calcul de plus faible précondition}


Le calcul de plus faible précondition est une forme de sémantique de
transformation de prédicats, proposée par Dijkstra en 1975 dans \textit{Guarded
commands, non-determinacy and formal derivation of programs}.



Cette phrase contient pas mal de mots méchants, mais le concept est en fait très
simple. Comme nous l'avons vu précédemment, la logique de Hoare donne des
règles expliquant comment se comportent les actions d'un programme. Mais
elle ne nous dit pas comment appliquer ces règles pour établir une preuve
complète du programme.



Dijkstra reformule la logique de Hoare en expliquant comment, dans le triplet
$\{P\}C\{Q\}$, l'instruction ou le bloc d'instructions, $C$ transforme le
prédicat $P$, en $Q$. Cette forme est appelée « raisonnement vers l'avant » ou
\textit{forward-reasoning}. Nous calculons à partir d'une précondition et d'une ou
plusieurs instructions, la plus forte postcondition que nous pouvons
atteindre. Informellement, en considérant ce qui est reçu en entrée, nous
calculons ce qui sera renvoyé au plus en sortie. Si la postcondition voulue
est au plus aussi forte, alors nous avons prouvé qu'il n'y a pas de
comportements non voulus.



Par exemple :

\begin{CodeBlock}{c}
int a = 2;
a = 4;
//postcondition calculée : a == 4
//postcondition voulue   : 0 <= a <= 30
\end{CodeBlock}



Pas de problème, 4 fait bien partie des valeurs acceptables pour a.



La forme qui nous intéresse, le calcul de plus faible précondition, fonctionne
dans le sens inverse, nous parlons de « raisonnement vers l'arrière » ou
\textit{backward-reasoning}. À partir de la postcondition voulue et de
l'instruction que nous traitons, nous déduisons la précondition minimale
qui nous assure ce fonctionnement. Si notre précondition réelle est au moins
aussi forte, c'est-à-dire, qu'elle implique la plus faible précondition, alors
notre programme est valide.



Par exemple, si nous avons l'instruction (sous forme de triplet) :



$\{P\}$ $x$ $:=$ a $\{x = 42\}$



Quelle est la précondition minimale pour que la postcondition $\{x = 42\}$
soit vérifiée ? La règle nous dira que $P$ est $\{$a$=42\}$.



Nous n'allons pas nous étendre sur ces notions pour le moment, nous y
reviendrons au cours du tutoriel pour comprendre ce que font les outils que
nous utilisons. Et nos outils, parlons-en justement.

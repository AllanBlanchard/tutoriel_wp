

\image{frama-c}


\levelThreeTitle{Frama-C ? WP ?}


Frama-C (pour FRAmework for Modular Analysis of C code) est une plate-forme
 dédiée à l'analyse de programmes C créée par le CEA List et Inria. Elle est
 basée sur une architecture modulaire permettant l'utilisation de divers
 \textit{plugins}. Les \textit{plugins} fournis par défaut comprennent diverses
 analyses statiques (sans exécution du code analysé), dynamiques (avec
 exécution du code), ou combinant les deux. Ces *plugins* peuvent collaborer
 ou non, soit en communiquant directement entre eux, soit en utilisant le
 langage de spécification fourni par Frama-C.

\image{gallery}


Ce langage de spécification s'appelle ACSL (« Axel »)
pour \textit{ANSI C Specification Language} et permet d'exprimer les
propriétés que nous souhaitons vérifier sur nos programmes. Ces propriétés seront
écrites sous forme d'annotations dans les commentaires. Pour les personnes qui
auraient déjà utilisé Doxygen, ça y ressemble beaucoup, sauf que tout sera
écrit sous forme de formules logiques. Tout au long de ce tutoriel, nous parlerons
beaucoup d'ACSL donc nous ne nous étendrons pas plus à son sujet ici.



L'analyse que nous allons utiliser ici est fournie par un plugin appelé WP pour
\textit{Weakest Precondition}, un plugin de vérification déduction. Il implémente
la technique dont nous avons parlé plus tôt :
à partir des annotations ACSL et du code source, le plugin génère ce que nous
appelons des conditions de vérification (ou obligations de preuves), qui sont
des formules logiques dont nous devons vérifier si elles sont vraies ou fausses
(on parle de « satisfiabilité »). Cette vérification peut être faite de manière
manuelle ou automatique, ici nous n'utiliserons que des outils automatiques.



Nous allons en l'occurrence utiliser un solveur de formules SMT
(\externalLink{satisfiabilité modulo théorie}{https://fr.wikipedia.org/wiki/Satisfiability\_modulo\_theories},
et nous n'entrerons pas dans les détails). Ce solveur se nomme
\externalLink{Alt-Ergo}{http://alt-ergo.ocamlpro.com/}, initialement développé par le Laboratoire
de Recherche en Informatique d'Orsay, aujourd'hui maintenu par
OCamlPro.



\levelThreeTitle{Installation}


Frama-C est un logiciel développé sous Linux et macOS. Son support est donc bien
meilleur sous ces derniers. Il existe quand même de quoi faire une installation
sous Windows et en théorie l'utilisation sera sensiblement la même que sous
Linux, mais :



\begin{Warning}
\begin{itemize}
\item le tutoriel présentera le fonctionnement sous Linux et l'auteur n'a pas
expérimenté les différences d'utilisation qui pourraient exister avec
Windows ;
\item les versions récentes de Windows 10 permettent d'utiliser Windows Subsystem
  for Linux, en combinaison avec un Xserver installé sous Windows pour avoir la
  GUI ;
\item La section « Bonus » un peu plus loin dans cette partie pourrait ne pas être
accessible.
\end{itemize}
\end{Warning}


\levelFourTitle{Linux}


\levelFiveTitle{Via les gestionnaires de paquets}


Sous Debian, Ubuntu et Fedora, il existe des paquets pour Frama-C. Dans ce cas,
il suffit de taper cette ligne de commande :



\begin{CodeBlock}{bash}
apt-get/yum install frama-c
\end{CodeBlock}



Par contre, ces dépôts ne sont pas systématiquement à jour. En soi, ce n'est pas très gênant car il n'y a pas de nouvelle version de Frama-C tous les deux mois, mais il est tout de même bon de le savoir.



Les informations pour vérifier l'installation sont données dans la sous-section « Vérifier l'installation ».


\levelFiveTitle{Via opam}


La deuxième solution consiste à passer par Opam, un gestionnaire de paquets
pour les bibliothèques et applications OCaml.


D'abord, Opam doit être installé et configuré sur votre distribution (voir
leur documentation). Ensuite, il faut également que quelques paquets de votre
distribution soient présents préalablement à l'installation de Frama-C. Sur la
plupart des distributions nous pouvons demander à Opam d'aller chercher les
bonnes dépendances pour le paquet que nous voulons installer. Pour cela, nous
utilisons l'outil depext d'Opam, qu'il faut d'abord installer :


\begin{CodeBlock}{bash}
opam install depext
\end{CodeBlock}


Puis nous lui demandons de prendre les dépendances de Frama-C :


\begin{CodeBlock}{bash}
opam depext frama-c
\end{CodeBlock}


Si depext ne trouve pas les dépendances pour votre distribution, les paquets
suivant doivent être présents sur votre système :


\begin{itemize}
\item GTK2 (development library)
\item GTKSourceview 2 (development library)
\item GnomeCanvas 2 (development library)
\item autoconf
\end{itemize}

Sur les versions récentes de certaines distributions, GTK2 peut ne pas
être disponible. Dans ce cas, ou si vous voulez avoir GTK3 et pas GTK2,
les paquets GTK2, GTKSourceview2 et GnomeCanvas2 doivent être remplacés
par GTK3 et GTKSourceview3.


Enfin, du côté d'Opam, il reste à installer Frama-C et Alt-Ergo.



\begin{CodeBlock}{bash}
opam install frama-c
opam install alt-ergo
\end{CodeBlock}


Les informations pour vérifier l'installation sont données dans la sous-section « Vérifier l'installation ».



\levelFiveTitle{Via une compilation « manuelle »}


Pour installer Frama-C via une compilation manuelle, les paquets indiqués dans la
section Opam sont nécessaires (mis à part Opam lui-même bien sûr). Il faut
également une version récente d'Ocaml et de son compilateur (y compris vers
code natif). Il est aussi nécessaire d'installer Why3, en version 1.2.0,
qui est disponible sur Opam ou sur leur site web
(\externalLink{Why3}{http://why3.lri.fr/}).



Après décompression de l'archive disponible ici :
\externalLink{http://frama-c.com/download.html}{http://frama-c.com/download.html} (Source distribution).
Il faut se rendre dans le dossier et exécuter la commande :



\begin{CodeBlock}{bash}
autoconf && ./configure && make && sudo make install
\end{CodeBlock}


Les informations pour vérifier l'installation sont données dans la sous-section « Vérifier l'installation ».


\levelFourTitle{OSX}


L'installation sur OSX passe par Homebrew et Opam. L'auteur n'ayant
personnellement pas d'OSX, voici une honteuse paraphrase du guide
d'installation de Frama-C pour OSX.



Pour les utilitaires d'installation et de configuration :



\begin{CodeBlock}{bash}
> xcode-select --install
> open http://brew.sh
> brew install autoconf opam
\end{CodeBlock}



Pour l'interface graphique :



\begin{CodeBlock}{bash}
> brew install gtk+ --with-jasper
> brew install gtksourceview libgnomecanvas graphviz
> opam install lablgtk ocamlgraph
\end{CodeBlock}



Dépendances pour alt-ergo :



\begin{CodeBlock}{bash}
> brew install gmp
> opam install zarith
\end{CodeBlock}



Frama-C et prouveur Alt-Ergo :



\begin{CodeBlock}{bash}
> opam install alt-ergo
> opam install frama-c
\end{CodeBlock}


Les informations pour vérifier l'installation sont données dans la sous-section « Vérifier l'installation ».



\levelFourTitle{Windows}


Actuellement, la meilleure manière d'utiliser Frama-C sous Windows
est de passer par Windows Subsystem for Linux. Une fois que le sous-système
Linux est installé dans Windows, il suffit d'installer Opam et de suivre les
instructions d'installation fournies dans la section Linux. Notons que pour
profiter de l'interface graphique, il faudra installer un serveur X sous
Windows.


Les informations pour vérifier l'installation sont données dans la sous-section
« Vérifier l'installation ».


\levelThreeTitle{Vérifier l'installation}


Pour vérifier votre installation, commençez par mettre ce code très simple dans
un fichier « main.c » :


\CodeBlockInput{c}{verify.c}


Ensuite, depuis un terminal, dans le dossier où ce fichier a été créé,
nous pouvons lancer Frama-C avec la commande suivante :



\begin{CodeBlock}{bash}
frama-c-gui -wp -rte main.c
\end{CodeBlock}



Cette fenêtre devrait s'ouvrir.



\image{verif_install-1}


En cliquant sur \CodeInline{main.c} dans le volet latéral gauche pour le sélectionner,
nous pouvons voir le contenu du fichier \CodeInline{main.c} modifié et des pastilles
vertes sur différentes lignes comme ceci :



\image{verif_install-2}


Si c'est le cas, tant mieux, sinon il faudra d'abord vérifier que rien n'a été
oublié au cours de l'installation (par exemple : l'oubli de bibliothèques graphiques
ou encore l'oubli de l'installation d'Alt-Ergo). Si tout semble correct, divers forums
pourront vous fournir de l'aide
(\externalLink{Forum de Zeste de Savoir}{https://zestedesavoir.com/forums/savoirs/programmation/}
 \externalLink{StackOverflow - Frama-C}{https://stackoverflow.com/questions/tagged/frama-c}).



\begin{Warning}
L'interface graphique de Frama-C ne permet pas l'édition du code source.
\end{Warning}


\begin{Information}
Pour les daltoniens, il est possible de lancer Frama-C avec un mode où les
pastilles de couleurs sont remplacées par des idéogrammes noirs et blancs :

\begin{CodeBlock}{bash}
frama-c-gui -gui-theme colorblind
\end{CodeBlock}
\end{Information}


\levelThreeTitle{(Bonus) Installation de prouveurs supplémentaires}


Cette partie est purement optionnelle, rien de ce qui est ici ne sera
indispensable pendant le tutoriel. Cependant, lorsque l'on commence à
s'intéresser vraiment à la preuve, il est possible de toucher assez rapidement
aux limites du prouveur pré-intégré Alt-Ergo et d'avoir besoin d'autres
prouveurs. Pour des propriétés simples, tous les prouveurs jouent à armes
égales, pour des propriétés complexes, chaque prouveur à ses domaines de
prédilecion.

\levelFourTitle{Why3}


Why3 est une plateforme pour la preuve déductive développée par le LRI à Orsay.
Elle embarque en outre un langage de programmation et de spécification ainsi
qu'un module permettant le dialogue avec une large variété de prouveurs
automatiques et interactifs. C'est en cela qu'il est utile dans le cas de
Frama-C et WP. WP utilise Why3 comme \textit{backend} pour dialoguer avec les
prouveurs externes.


Nous pouvons retrouver sur ce même site
\externalLink{la liste des prouveurs}{http://why3.lri.fr/\#provers} qu'il supporte.
Il est vivement conseillé d'avoir \externalLink{Z3}{https://github.com/Z3Prover/z3/wiki},
développé par Microsoft Research, et \externalLink{CVC4}{http://cvc4.cs.stanford.edu/web/},
développé par des personnes de divers organismes de recherche (New York
University, University of Iowa, Google, CEA List). Ces deux prouveurs sont très
efficaces et relativement complémentaires.


De nouveaux prouveurs peuvent être installés n'importe quand après
l'installation de Frama-C. Cependant la liste des prouveurs vus par Why3
doit être mise à jour :


\begin{CodeBlock}{bash}
why3 config --full-config
\end{CodeBlock}


puis activée dan Frama-C. Dans le panneau latéral, dans la partie WP,
cliquons sur « Provers... » :


\image{provers}


Puis « Detect » dans la fenêtre qui apparaît. Une fois que c'est fait,
les prouveurs peuvent être activés grâce au bouton qui se trouve à côté
de leur nom.


\image{detect-and-select}


\levelFourTitle{Coq}


Coq, développé par l'organisme de recherche Inria, est un assistant de
preuve. C'est-à-dire que nous écrivons nous-même les preuves dans un
langage dédié, et la plateforme se charge de vérifier (par typage) que
cette preuve est valide.



Pourquoi aurait-on besoin d'un tel outil ? Il se peut parfois que les
propriétés que nous voulons prouver soient trop complexes pour un prouveur
automatique, typiquement lorsqu'elles nécessitent des raisonnements par
induction avec des choix minutieux à chaque étape. Auquel cas, WP pourra
générer des conditions de vérification traduites en Coq et nous laisser écrire
la preuve en question.



Pour apprendre à utiliser Coq,
\externalLink{ce tutoriel}{http://www.cis.upenn.edu/~bcpierce/sf/current/index.html}
est très bon.



\begin{Information}
Si Frama-C est installé par l'intermédiaire du gestionnaire de
paquets, il peut arriver que celui-ci ait directement intégré Coq.
\end{Information}


Pour plus d'informations à propos de Coq et de son installation, voir \externalLink{The Coq Proof Assistant}{https://coq.inria.fr/}.


Le support actuel de Coq dans Frama-C est déprécié mais toujours
accessible. Comme nous fournissons dans ce tutoriel quelques scripts de preuve
Coq qui peuvent être utilisés, nous fournissons ces instructions afin que
ces scripts puissent être visionnés et testés par les utilisateurs. Pour lancer
Frama-C avec le support de Coq, nous utilisons la commande :


\begin{CodeBlock}{bash}
  frama-c -wp-provers=native:coq
\end{CodeBlock}

Dans cette partie nous allons voir trois notions importantes d'ACSL :
\begin{itemize}
\item les définitions inductives,
\item les définitions axiomatiques,
\item le code fantôme.
\end{itemize}


Dans certaines configurations, ces trois notions sont absolument nécessaires
pour faciliter le processus de spécification et de preuve. Soit en forçant
l'abstraction de certaines propriétés, soit en explicitant des informations qui
sont autrement implicites et plus difficiles à prouver.



Le risque de ces trois notions est qu'elles peuvent rendre notre preuve inutile
si nous faisons une erreur dans leur usage. Les définitions inductives et
axiomatiques introduisent le risque de faire entrer « faux » dans nos
hypothèses, ou d'écrire des définitions imprécises. Le code fantôme, s'il ne
respecte pas certaines propriétés, ouvre le risque de modifier le programme
vérifié, nous faisant ainsi prouver un autre programme que celui que nous
voulons prouver.


\begin{levelTwo}
  {Définitions inductives}
  {inductive}
\end{levelTwo}


\begin{levelTwo}
  {Définitions axiomatiques}
  {axiomatic}
\end{levelTwo}


\begin{levelTwo}
  {Code fantôme}
  {ghost-code}
\end{levelTwo}


\begin{levelTwo}
  {Contenu caché}
  {answers}
\end{levelTwo}


\horizontalLine
\newpage


Dans cette partie, nous avons vu des constructions plus avancées du langage ACSL
qui nous permettent d'exprimer et de prouver des propriétés plus complexes à
propos de nos programmes.



Mal utilisées, ces fonctionnalités peuvent fausser nos analyses, il faut donc se
montrer attentif lorsque nous manipulons ces constructions et ne pas hésiter à
les relire ou encore à exprimer des propriétés à vérifier à leur sujet afin de
s'assurer que nous ne sommes pas en train d'introduire des incohérences dans
notre programme ou nos hypothèses de travail.

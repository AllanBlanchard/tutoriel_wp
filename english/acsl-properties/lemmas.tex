Lemmas are general properties about predicates or functions. Once these
properties are expressed, their proof can be performed (one time) and
the provers will then be able to use this result to perform other proofs
without requiring to perform again all steps needed to perform the
original proof if it appears in a much longer proof about an other
property.




For example, lemmas allow us to express properties about recursive
functions in order to get easier proofs when we are interested in
proving properties that use such functions.



\levelThreeTitle{Syntax}


Again, we introduce lemmas using ACSL annotations. The syntax is
following:



\begin{CodeBlock}{c}
/*@
  lemma name_of_the_lemma { Label0, ..., LabelN }:
    property ;
*/
\end{CodeBlock}



This time, the properties we want to express do not depend on received
parameters (except for labels). So we will express these properties on
universally quantified variables. For example, we can state this lemma,
which is true, even if it is trivial:



\begin{CodeBlock}{c}
/*@
  lemma lt_plus_lt:
    \forall integer i, j ; i < j ==> i+1 < j+1;
*/
\end{CodeBlock}



This proof can be performed using WP. The property is, of course, proved
using only Qed.



\levelThreeTitle{Example: properties about linear functions}


We can come back to our linear functions and express some interesting
properties about them:



\begin{CodeBlock}{c}
/* @
  lemma ax_b_monotonic_neg:
    \forall integer a, b, i, j ;
      a <  0 ==> i <= j ==> ax_b(a, i, b) >= ax_b(a, j, b);
  lemma ax_b_monotonic_pos:
    \forall integer a, b, i, j ;
      a >  0 ==> i <= j ==> ax_b(a, i, b) <= ax_b(a, j, b);
  lemma ax_b_monotonic_nul:
    \forall integer a, b, i, j ;
      a == 0 ==> ax_b(a, i, b) == ax_b(a, j, b);
*/
\end{CodeBlock}



For these proofs, Alt-ergo, will probably not be able to discharge
generated goals. In this case, Z3 will certainly perform it. We can then
write some code examples:



\begin{CodeBlock}{c}
/*@
  requires a > 0;
  requires limit_int_min_ax_b(a,4) < x < limit_int_max_ax_b(a,4);
  assigns \nothing ;
  ensures \result == ax_b(a,x,4);
*/
int function(int a, int x){
  return a*x + 4;
}

/*@ 
  requires a > 0;
  requires limit_int_min_ax_b(a,4) < x < limit_int_max_ax_b(a,4) ;
  requires limit_int_min_ax_b(a,4) < y < limit_int_max_ax_b(a,4) ;
  assigns \nothing ;
*/
void foo(int a, int x, int y){
  int fmin, fmax;
  if(x < y){
    fmin = function(a,x);
    fmax = function(a,y);
  } else {
    fmin = function(a,y);
    fmax = function(a,x);
  }
  //@assert fmin <= fmax;
}
\end{CodeBlock}



If we do not give the lemmas provided earlier, Alt-ergo will not be able
to prove the proof that \CodeInline{fmin} is lesser or equal to
\CodeInline{fmax}. With the lemmas it is however very easy for it since the
property is the simply an instance of the lemma
\CodeInline{ax\_monotonic\_pos}, the proof is then trivial as our lemma is
considered to be true when are not currently proving it.

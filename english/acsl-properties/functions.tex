Logic functions are meant to describe functions that can only be used in
specifications. It allows us, first, to factor out those specifications and,
second, to define some operations on \CodeInline{integer} or \CodeInline{real}
with the guarantee that they cannot overflow since they involve mathematical
types.

Like predicates, they can receive different labels and values as
parameters.



\levelThreeTitle{Syntax}


To define a logic function, the syntax is the following:



\begin{CodeBlock}{c}
/*@
  logic return_type my_function{ Label0, ..., LabelN }( type0 arg0, ..., typeN argN ) =
    formula using the arguments ;
*/
\end{CodeBlock}



We can for example define a mathematical \externalLink{linear function}{https://en.wikipedia.org/wiki/Linear_function_(calculus)} using a logic function:



\CodeBlockInput[1][4]{c}{linear-0.c}



And it can be used to prove the source code of the following function:



\CodeBlockInput[6][12]{c}{linear-0.c}



\image{linear-1}


This code is indeed proved but some runtime-errors seems to be possible.
We can add a constraint in the precondition so that the result can be
stored into a C integer:


\CodeBlockInput[8][15]{c}{linear-1.c}


Some runtime errors are still possible. Indeed, while the bound provided
for \CodeInline{x} by our logic function is defined for the complete
computation, it does not say anything about the value obtained in the
intermediate computation. For example, the fact that
\CodeInline{3 * x + 4} is not lower than \CodeInline{INT\_MIN} does not
guarantee that this is the case for \CodeInline{3 * x}. We can imagine
two ways to solve this problem, and this choice should be guided by the
project in which this function would be used.


Either we further restrict the input values:


\CodeBlockInput[17][25]{c}{linear-1.c}


Or we can modify the source code so that the risk of overflow does not
appear anymore:


\CodeBlockInput[27][37]{c}{linear-1.c}


Note that, as in specifications, computations are done using mathematical
integers. We then do not need to care about some overflow risk when using
the \CodeInline{ax\_b} function:


\CodeBlockInput[39][41]{c}{linear-1.c}


which is correctly discharged by WP, which does not generate any alarm
related to overflows:


\image{linear-math}


\levelThreeTitle{Recursive functions and limits of logic functions}


Logic functions (as well as predicates) can be recursively defined.
However, such an approach will soon reveal some limitations in their use for
program proof. Indeed, when the automatic solver reasons about such logic
properties, if such a function is met, it is necessary to evaluate it.
SMT solvers are not meant to be efficient for this task, thus it is generally
costly, producing excessively long proof resolutions and eventually timeouts.

Below is a concrete example with the factorial function, using a
recursive logic function and a non-recursive C implementation.



\CodeBlockInput{c}{facto-0.c}



Without checking overflows, this function can be easily proved. If
we add runtime error checking, we see that there is a possibility of
overflow on the multiplication.



With the \CodeInline{int} type, the maximum value for which we can compute
factorial is 12. Any higher and we get an overflow. We can then add this
precondition:



\CodeBlockInput[9][14]{c}{facto-1.c}



If we ask for a proof on this input, Alt-ergo will probably fail,
whereas CVC5 can compute the proof in less than a second. The reason is
that in this case, the heuristics used by CVC5 consider it
a good idea to spend a bit more time evaluating the
function.



So logic functions can be defined recursively, but without some extra
help, we are rapidly limited by the fact that provers need to
perform evaluation or to ``reason'' by induction, two tasks for which
they are not efficient. This can limit our possibilities for program
proofs, but we will see later that we can get rid of these problems.


\levelThreeTitle{Exercises}



\levelFourTitle{Distance}


Specify and prove the following program:



\CodeBlockInput[5]{c}{ex-1-distance.c}



For this, define two logic functions \CodeInline{abs} and \CodeInline{distance}.
Use these functions to write the specification of the function.



\levelFourTitle{Square}


Write the body of the \CodeInline{square} function. Specify and prove the
program. Use a \CodeInline{square} logic function.


\CodeBlockInput[5]{c}{ex-2-square.c}


Pay attention to the types of the variables and do not over-constrain the input of
the function. Furthermore, when verifying the absence of runtime errors, do not
forget to provide the options \CodeInline{-warn-unsigned-overflow} and
\CodeInline{-warn-unsigned-downcast}.


\levelFourTitle{Iota}


Here is a possible implementation of the iota function:


\CodeBlockInput[5]{c}{ex-3-iota.c}


Write a logic function that returns the input value increased by one. Prove
that after the execution of \CodeInline{iota}, the first value of the array is
the input value and that each value of the array corresponds to the value that
precedes it increased by one (using the previously-defined logic function).



\levelFourTitle{Vector add}



In the following program, the \CodeInline{vec\_add} function adds the second
vector in input into the first one. Write a contract for the function
\CodeInline{show\_the\_difference} that expresses, for each value of the vector
\CodeInline{v1} the difference between the pre and the postcondition. For this,
define a logic function \CodeInline{diff} that returns the difference between
the value of a memory location at a label \CodeInline{L1} and the value at a
label \CodeInline{L2}.


\CodeBlockInput[5]{c}{ex-4-vec-inc.c}



Re-express the \CodeInline{unchanged} predicate using the logic function you
have defined.



\levelFourTitle{The sum of the N first integers}
\label{l4:acsl-properties-functions-n-first-ints}


The following function computes the sum of the N first integers. Write a
recursive logic function that returns the sum of the N first integers and
write a specification for the C function expressing that it computes the same
value as provided by the logic function.


\CodeBlockInput[5]{c}{ex-5-n-first-ints.c}


Try to verify the absence of runtime errors. The integer overflow is not so
simple to get rid of. However, write a precondition that should be enough to
prove the function (remember that the sum of the N first integers can be
expressed with a really simple formula ...). It will certainly not be enough
to directly prove the absence of overflow, but we will see how to provide such
an information in the next section.

ACSL provides different logic types that allow us to write properties in a more
abstract, mathematical world. Among the types that can be useful, some are
dedicated to numbers, and allow us to express properties or functions without
having to think about constraints due to the size of
the representation of primitive C types in memory. These types are
\CodeInline{integer} and \CodeInline{real}, which respectively represent
mathematical integers and reals (that are modeled to be as close to their
mathematical ideals as possible, but this notion cannot be perfectly handled).

From now on, we will often use integers instead of classical C
\CodeInline{int}s. The reason is simply that a lot of properties and
definitions are true regardless the size of the machine integer we have
as input.

On the other hand, we will not talk about the differences that exist
between \CodeInline{real} and \CodeInline{float/double}. This would require
speaking about precise floating-point calculations, and proofs of programs
that rely on such calculations would require an entire dedicated
tutorial.

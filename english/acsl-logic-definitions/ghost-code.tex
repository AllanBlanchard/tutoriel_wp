
The previous techniques we have seen in this chapter are meant to make the
specification more abstract. The role of ghost code is the opposite, here, we
find in fact a way to enrich our specification with information expressed as
concrete C code. The idea is to add variables and source code that is not part
of the actual program to verify and is thus only visible for the verification
tool. It is used to make explicit some logic properties that, else, would only
be known implicitly.


\levelThreeTitle{Syntax}


Ghost code is added using annotations that contain C code
introduced using the \CodeInline{ghost} keyword:


\begin{CodeBlock}{c}
/*@
  ghost
  // C code
*/
\end{CodeBlock}


In ghost code, we write normal C code. Some subtleties will be explained
later in this section. For now, let us have a look to the basic elements
we can write in ghost code.


We can declare variables:


\begin{CodeBlock}{c}
//@ ghost int ghost_glob_var = 0;

void foo(int a){
  //@ ghost int ghost_loc_var = a;
}
\end{CodeBlock}


These variables can be modified via instructions and conditional structures:


\begin{CodeBlock}{c}
//@ ghost int ghost_glob_var = 0;

void foo(int a){
  //@ ghost int ghost_loc_var = a;
  /*@ ghost
    for(int i = 0 ; i < 10 ; i++){
      ghost_glob_var += i ;
      if(i < 5) ghost_local_var += 2 ;
    }
  */
}
\end{CodeBlock}


We can declare ghost labels that can be used into annotations (or to perform
a \CodeInline{goto} from the ghost code itself, under some conditions that
we will explain later):


\begin{CodeBlock}{c}
void foo(int a){
  //@ ghost Ghost_label: ;
  a = 28 ;
  //@ assert ghost_loc_var == \at(a, Ghost_label) == \at(a, Pre);
}
\end{CodeBlock}


A conditional structure \CodeInline{if} can be extended with a ghost
\CodeInline{else} if it does not have one originally. For example:


\begin{CodeBlock}{c}
void foo(int a){
  //@ ghost int a_was_ok = 0 ;
  if(a < 5){
    a = 5 ;
  } /*@ ghost else {
    a_was_ok = 1 ;
  } */
}
\end{CodeBlock}


A function can be provided with ghost parameters that can be used
to give more information for the verification of the function. For
example, if we imagine the verification of a function that receives
a linked list, we could provide a ghost parameter that represents
the length of the list:


\begin{CodeBlock}{c}
void function(struct list* l) /*@ ghost (int length) */ {
  // visit the list
  /*@ variant length ; */
  while(l){
    l = l->next ;
    //@ ghost length--;
  }
}
void another_function(struct list* l){
  //@ ghost int length ;

  // ... do something to compute the length

  function(l) /*@ ghost(n) */ ; // we call 'function' with the ghost argument
}
\end{CodeBlock}


Note that when a function takes ghost parameters, all calls to this
function must provide the corresponding ghost arguments.


An entire function can be ghost. For example, in the previous example, we
could have used a ghost function to compute the length of the list:


\begin{CodeBlock}{c}
/*@ ghost
  /@ ensures \result == logic_length_of_list(l) ; @/
  int compute_length(struct list* l){
    // does the right computation
  }
*/

void another_function(struct list* l){
  //@ ghost int length ;

  //@ ghost length = compute_length(l) ;
  function(l) /*@ ghost(n) */ ; // we call function with the ghost parameter
}
\end{CodeBlock}


Here, we can see a specific syntax for the contract of the ghost function.
Indeed, it is often useful to write some contracts or assertions in ghost
code. As it must be specified in code that is already in C comments, we
must use a specific syntax to provide ghost contracts or assertions.
We open ghost annotations with the syntax \CodeInline{/@} and close
them with \CodeInline{@/}. Of course, it applies to loops in ghost code
for example:


\begin{CodeBlock}{c}
void foo(unsigned n){
 /*@ ghost
   unsigned i ;

   /@
     loop invariant 0 <= i <= n ;
     loop assigns i ;
     loop variant n - i ;
   @/
   for(i = 0 ; i < n ; ++i){

   }
   /@ assert i == n ; @/
 */
}
\end{CodeBlock}


\levelThreeTitle{Ghost code validity, what Frama-C checks}


Frama-C verifies several properties about the ghost code we write:
\begin{itemize}
\item the ghost code cannot modify the control flow graph of the program,
\item the normal code cannot access to the ghost memory,
\item the ghost code can only write the ghost memory.
\end{itemize}

Verifying these properties, our goal is to guarantee that for any
program, for any input, its observable behavior is the same with or
without the ghost code.


\begin{Warning}
  Before Frama-C 21 Scandium, most of these properties were not
  verified by the Frama-C kernel. Thus, if we work with a previous
  version, we have to ensure that they are verified ourselves.
\end{Warning}


If some of these properties are not verified, it would mean that
the ghost code can change the behavior of the verified program.
Let us have a closer look to each of these constraints.


\levelFourTitle{Maintain the same control flow}


The control flow of a program is the order in which the instructions
are executed by the program. If the ghost code changes this order, or
let the program ignore some instructions of the original program, then
the behavior of the program is not the same, and thus, we are not
verifying the same program.


For example, this function compute the sum of the $n$ first integers :


\CodeBlockInput[6]{c}{control-flow-1.c}


By introducing, in ghost code, the instruction \CodeInline{break} in
the body of the loop, the program does not have the same behavior
anymore: instead of visiting all $i$s between $0$ and $n+1$, we stop
immediately in the first iteration of the loop. Consequently, this
program is rejected by Frama-C:


\begin{CodeBlock}{text}
[kernel:ghost:bad-use] file.c:4: Warning:
  Ghost code breaks CFG starting at:
  /*@ ghost break; */
  x += i;
\end{CodeBlock}


It is important to note that when a ghost code changes the control
flow, the instruction that is pointed by Frama-C is the starting
point of the ghost code. For example, if we introduce a conditional
around our \CodeInline{break} instruction:


\CodeBlockInput[6]{c}{control-flow-2.c}


The error message will point to the enclosing \CodeInline{if}:


\begin{CodeBlock}{text}
[kernel:ghost:bad-use] file.c:4: Warning:
  Ghost code breaks CFG starting at:
  /*@ ghost if (i < 3) break; */
  x += i;
\end{CodeBlock}


Note that verifying that the control flow is not changed by ghost
code is done in a purely syntactic way. For example, if the
\CodeInline{break} is unreachable in any execution of the program,
an error will still be triggered:


\CodeBlockInput[6]{c}{control-flow-3.c}


\begin{CodeBlock}{text}
[kernel:ghost:bad-use] file.c:4: Warning:
  Ghost code breaks CFG starting at:
  /*@ ghost if (i > n) break; */
  x += i;
\end{CodeBlock}


Finally, we can remark that there exists two ways to change the control
flow. The first one is to use jumps (thus \CodeInline{break},
\CodeInline{continue}, or \CodeInline{goto}), the second is to
introduce a non-terminating code. For this last one, unless the non-termination
is trivial, the kernel cannot verify that the control
flow is changed (and thus, it never does, this is delegated to plugins).
We will treat this question in
section~\ref{l3:acsl-logic-definitions-what-remains}.


\levelFourTitle{Access to the memory}


The ghost code is an observer of the normal code. Consequently, normal
code is not authorized to access ghost code, neither to its memory,
nor to the functions. Ghost code can read the memory of the normal code,
but cannot write it. Currently, ghost code cannot call normal functions,
we will give more details about this restriction later in this section.


Normal code is not allowed to read ghost code for a very simple reason:
if the normal code tries to access ghost variables,
it would simply not compile: the compiler does not see the variables
declared in annotations. For example:
\CodeBlockInput[6]{c}{non-ghost-access-ghost.c}
cannot be compiled:
\begin{CodeBlock}{text}
# gcc -c file.c
file.c: In function ‘sum_42’:
file.c:5:10: error: ‘r’ undeclared (first use in this function)
    5 |     x += r;
      |          ^
\end{CodeBlock}
and is thus also refused by Frama-C:
\begin{CodeBlock}{text}
[kernel] file.c:5: User Error:
Variable r is a ghost symbol. It cannot be used in non-ghost context. Did you forget a /*@ ghost ... /?
  3       //@ ghost int r = 42 ;
  4       for(int i = 0; i < n; ++i){
  5         x += r;
                 ^
  6       }
  7       return x;
\end{CodeBlock}


In ghost code, normal variables cannot be modified. Indeed, it
would imply that we can change the result of a program by just
adding ghost code. For example, in the following code:


\CodeBlockInput[6]{c}{ghost-access-non-ghost-1.c}


The result would not be the same with or without the ghost code.
Frama-C thus forbids such a code:


\begin{CodeBlock}{text}
[kernel:ghost:bad-use] file.c:5: Warning:
  'x' is a non-ghost lvalue, it cannot be assigned in ghost code
\end{CodeBlock}


Note that this verification is done thanks to the types of the different
variables. A variable declared in normal code has a normal status, while
a variable declared in ghost code has a ghost status. Consequently, even
if the ghost code does not really change the behavior of the program, any
write on a normal variable in ghost code is forbidden:


\CodeBlockInput[8]{c}{ghost-access-non-ghost-2.c}


\begin{CodeBlock}{text}
[kernel:ghost:bad-use] file.c:9: Warning:
  'x' is a non-ghost lvalue, it cannot be assigned in ghost code
[kernel:ghost:bad-use] file.c:10: Warning:
  'x' is a non-ghost lvalue, it cannot be assigned in ghost code
\end{CodeBlock}


This also applies to the \CodeInline{assigns} clause when it is
in ghost code:


\CodeBlockInput[6]{c}{assigns-clause-1.c}


\begin{CodeBlock}{text}
[kernel:ghost:bad-use] file.c:4: Warning:
  'x' is a non-ghost lvalue, it cannot be assigned in ghost code
[kernel:ghost:bad-use] file.c:11: Warning:
  'x' is a non-ghost lvalue, it cannot be assigned in ghost code
\end{CodeBlock}


On the other hand, normal functions and loops contracts can (and
must) specify assigned ghost memory locations. For example, if we
fix the previous program by making \CodeInline{x} ghost, first, our
previous \CodeInline{assigns} clauses are allowed, but we can also
specify that the function \CodeInline{foo} writes the ghost global
variable \CodeInline{x}:


\CodeBlockInput{c}{assigns-clause-2.c}


\levelFourTitle{Typing of ghost elements}


We should now provide some more details about the typing of variables
declared in ghost code. For example, it is sometimes useful to create
a ghost array to store some information:


\CodeBlockInput{c}{ghost-array-1.c}


Here, we use indices to access our arrays, but we could for example
access their content using pointers:


\CodeBlockInput[6]{c}{ghost-array-2.c}


However, we immediately see that Frama-C does not like our way to do it:


\begin{CodeBlock}{text}
[kernel:ghost:bad-use] file.c:3: Warning:
  Invalid cast of 'even' from 'int \ghost *' to 'int *'
[kernel:ghost:bad-use] file.c:6: Warning:
  '*pe' is a non-ghost lvalue, it cannot be assigned in ghost code
\end{CodeBlock}


In particular, the first message indicates that we try to cast
a pointer to \CodeInline{int \textbackslash{}ghost} into a pointer
to \CodeInline{int}. Indeed, when a variable is declared in ghost
code, only the variable is considered ghost. Thus, for a pointer,
the pointed memory is not considered ghost (and thus here, while
\CodeInline{pe} is ghost, the memory pointed by \CodeInline{pe} is
not). To solve this problem, Frama-C provides the
\CodeInline{\textbackslash{}ghost} qualifier that can be used to
add a ghost status to a type:


\CodeBlockInput{c}{ghost-array-3.c}


On some aspects, the \CodeInline{\textbackslash{}ghost} qualifier
looks like the \CodeInline{const} keyword. However, it does not
behave exactly the same way for two main reasons.


First, while the following \CodeInline{const} definition is allowed,
it is not allowed to write something similar with the
\CodeInline{\textbackslash{}ghost} qualifier:


\begin{CodeBlock}{c}
int const * * const p ;
//@ ghost int \ghost * * p ;
\end{CodeBlock}


\begin{CodeBlock}{text}
[kernel:ghost:bad-use] file.c:2: Warning:
  Invalid type for 'p': indirection from non-ghost to ghost
\end{CodeBlock}


Declaring a const pointer to a mutable location that contains pointers
to some constant locations can make sense. On the other hand, it does
make sense for the \CodeInline{\textbackslash{}ghost} qualifier. That
would mean that some normal memory contains pointers to ghost memory
locations, and we do not want to allow this.


Second, we can assign a pointer to mutable locations to a pointer to
const locations:


\begin{CodeBlock}{c}
int a[10] ;
int const * p = a ;
\end{CodeBlock}


This code is valid as we only restrict our capacity to write memory
locations when we initialize (or assign) \CodeInline{p} to \CodeInline{\&a[0]}.
On the other hand, both the two following pointer initialization
(or equivalent assignments) are forbidden with the
\CodeInline{\textbackslash{}ghost} qualifier:


\begin{CodeBlock}{c}
int non_ghost_int ;
//@ ghost int ghost_int ;

//@ ghost int \ghost * p = & non_ghost_int ;
//@ ghost int * q = & ghost_int ;
\end{CodeBlock}


While the reason why we refuse the first initialization is quite direct:
it would allow writing into the normal memory from ghost code through
\CodeInline{p}, the reason why we refuse the second one is not that
intuitive. And, indeed, we must use workarounds to create a problem with
this conversion:


\CodeBlockInput[6]{c}{why-not-const.c}


Here, we make a conversion that seems to be reasonable. Indeed, we give
the address of a pointer to some ghost memory location to a function
that takes a pointer to a normal memory, thus we only restrict our ability
to access to the pointed memory. However, by this function call, the function
\CodeInline{assign} assigns the current value of \CodeInline{q} (which is
\CodeInline{\&x}) to \CodeInline{p}. Thus, via the last instruction, we
can write \CodeInline{x} in ghost code, which should be forbidden.
Consequently, such a conversion is never allowed.


Finally, a ghost code cannot currently call a non ghost function for
similar reasons. Some particular cases could be allowed, but this is
not currently supported by Frama-C.


\levelThreeTitle{Ghost code validity, what remains to be verified}
\label{l3:acsl-logic-definitions-what-remains}


Except for the restrictions mentioned in the previous section,
ghost code is just normal C code. That means that if we want to
verify the original program, we must take care of at least two more
aspects:
\begin{itemize}
\item absence of runtime errors,
\item ghost code termination.
\end{itemize}


The first case does not require more attention than the rest of the
code. Indeed, absence of runtime errors in ghost code will be treated
by the RTE plugin as for normal code.


As we said in section~\ref{l3:statements-loops-variant}, there are two
kinds of correctness: partial and total correctness, the second one allowing
to prove that a program terminates. For normal code, showing that the code
terminates is not always something that we want. On the other hand, if we
use some ghost code during the verification, it is absolutely necessary to
prove total correctness of this ghost code. Indeed, if it is non-terminating
(if it contains an infinite loop for example), it could allow us to prove
anything about the program.


\begin{CodeBlock}{c}
/*@ ensures \false ; */
void foo(void){
  /*@ ghost
    while(1){}
  */
}
\end{CodeBlock}



\levelThreeTitle{Make a logical state explicit}


The goal of ghost code is to make explicit some information that would be else
implicit. For example, in the verification of the sort algorithm, we used it
to create a label in the program that is not visible by the compiler but that
we can use in the verification. The fact that the values were swapped between the
two labels was implicitly provided by the contract of the function, adding this
ghost label allows us to write an explicit assertion of this fact.



Let us take a more complex example where we more clearly create explicit
knowledge about the program. Here, we want to prove that the
following function returns the value of the maximal sum of subarrays of
a given array. A subarray of an array \CodeInline{a} is a contiguous subset
of values of \CodeInline{a}. For example, for an array \CodeInline{\{ 0 , 3 , -1 , 4 \}},
some subarrays can be
\CodeInline{\{\}}, \CodeInline{\{ 0 \}}, \CodeInline{\{ 3 , -1 \}},
\CodeInline{\{ 0 , 3 , -1 , 4 \}}, ... Note that as we allow
empty arrays for subarrays, the sum is at least 0. In the previous
array, the subarray with the maximal sum is \CodeInline{\{ 3 , -1 , 4 \}},
the function would then return 6.



\CodeBlockInput[7]{c}{max_subarray-0.c}



In order to specify the previous function, we need an axiomatic
definition about sum. This is not too complex, the careful reader
can express the needed axioms as an exercise:



\CodeBlockInput[7][9]{c}{max_subarray-1.c}



Some correct axioms are available at: \ref{l2:acsl-logic-definitions-answers}



The specification of the function is the following:



\CodeBlockInput[18][23]{c}{max_subarray-1.c}



For any bounds, the value returned by the function must be greater or
equal to the sum of the elements between these bounds, and there must
exist some bounds such that the returned value is exactly the sum of the
elements between these bounds. About this specification, when we want to
add the loop invariant, we realize that we miss some information.
We want to express what are the values \CodeInline{max} and \CodeInline{cur},
and what are the relations between them, but we cannot do it!


Basically, in order to prove our postcondition, we need to know that there
exists some bounds
\CodeInline{low} and \CodeInline{high} such that the computed sum corresponds to
these bounds. However, in our code, we do not have anything that expresses
it from a logic point of view, and we cannot \emph{a priori} make the
link between this logic formalization. We then use ghost code to
record these bounds and express the loop invariant.


We first need two variables to record the bounds
of the maximum sum range, let us call them \CodeInline{low} and
\CodeInline{high}. Every time we find a range where the sum is greater
than the current one, we update our ghost variables. These bounds
then correspond to the sum currently stored by \CodeInline{max}. That
induces that we need other bounds: the ones that correspond to the sum
stored by the variable \CodeInline{cur} from which we build the bounds
corresponding to \CodeInline{max}. For these bounds, we only add a
single ghost variable: the current low bound \CodeInline{cur\_low}, the high
bound being the variable \CodeInline{i} of the loop.



\CodeBlockInput[18][54]{c}{max_subarray-1.c}



The invariant \CodeInline{BOUNDS} expresses how the different bounds are
ordered during the computation. The invariant \CodeInline{REL} expresses
what the variables \CodeInline{cur} and \CodeInline{max} mean depending on the
bounds. Finally, the invariant \CodeInline{POST} allows us to create a link
between the invariant and the postcondition of the function.



The reader can verify that this function is indeed correctly proved
without RTE verification. If we add RTE verification, the overflow on
the variable \CodeInline{cur}, that is the sum, seems to be possible (and it
is indeed the case).



Here, we do not try to fix this because it is not the topic of this
example. The way we can prove the absence of RTE here strongly depends
on the context where we use this function. A possibility is to strongly
restrict the contract, forcing some properties about values and the size
of the array. For example, we could limit the maximal size of the array
and bound each value of the different cells. Another possibility would be to
add an error value in case of overflow
(\(-1\) for example), and to specify that when an overflow is produced,
this value is returned.



\levelThreeTitle{Exercises}


\levelFourTitle{Ghost code validity}


In these example functions, and without running Frama-C, explain what is
wrong with the ghost code. When Frama-C should reject the code, explain
why. Note that one can execute Frama-C without checking ghost code validity
using the option \CodeInline{-kernel-warn-key ghost=inactive}.


\CodeBlockInput[5]{c}{ex-1-ghost-reject.c}


\levelFourTitle{Two times}


This program computes \CodeInline{2 * x} using a loop. Use a ghost variable
\CodeInline{i} to express as an invariant that the value of \CodeInline{r}
is \CodeInline{i * 2} and complete the proof.


\CodeBlockInput[5]{c}{ex-2-times-2.c}



\levelFourTitle{Playing with arrays}


In this function, we receive an array, and we have a loop where we do nothing
except that we have indicated that it assigns the content of the array. However,
we would like to prove in postcondition that the array has not been modified.



\CodeBlockInput[5]{c}{ex-3-array.c}


Without modifying the \CodeInline{assigns} clause of the loop and without using
the keyword \CodeInline{\textbackslash{}at}, prove that the function does not
modify the array. For this, complete the ghost code and the loop invariant, by
assuring that the array \CodeInline{g} represents the old value of
\CodeInline{a}.


Once it is done, create a ghost function that performs the copy and use it in
\CodeInline{foo} to perform the same proof.


\levelFourTitle{Search and replace}


The following program performs a search and replace:


\CodeBlockInput[5]{c}{ex-4-replace.c}


Assuming that the function \CodeInline{replace} requires \CodeInline{old} and
\CodeInline{new} to be different, write a contract for \CodeInline{replace}
and prove that the function satisfies it.


Now, we would like to know which cells have been updated by the function.
Add a ghost parameter to \CodeInline{replace} so that it can receive a
second array that will record the updated (or not) cells. Adding also the
following code after the call to replace:


\begin{CodeBlock}{c}
/*@ ghost
  /@ loop invariant 0 <= i <= 40 ;
     loop assigns i;
     loop variant 40 - i ;
  @/
  for(size_t i = 0 ; i < 40 ; ++i){
    if(updated[i]){
      /@ assert a[i] != \at(a[\at(i, Here)], L); @/
    } else {
      /@ assert a[i] == \at(a[\at(i, Here)], L); @/
    }
  }
*/
\end{CodeBlock}


Everything should be proved.

Axioms are properties we state to be true no matter the situation. It is
a good way to state complex properties that will allow the proof process
to be more efficient by abstracting their complexity. Of course, as any
property expressed as an axiom is assumed to be true, we have to be very
careful when we use them to defined properties: if we introduce a false
property in our assumptions, ``false'' becomes ``true'' and we can then
prove anything.



\levelThreeTitle{Syntax}


Axiomatic definitions are introduced using this syntax:



\begin{CodeBlock}{c}
/*@
  axiomatic Name_of_the_axiomatic_definition {
    // here we can define or declare functions and predicates

    axiom axiom_name { Label0, ..., LabelN }:
      // property ;

    axiom other_axiom_name { Label0, ..., LabelM }:
      // property ;

    // ... we can put as many axioms we need
  }
*/
\end{CodeBlock}



For example, we can write this axiomatic block:



\begin{CodeBlock}{c}
/*@
  axiomatic lt_plus_lt{
    axiom always_true_lt_plus_lt:
      \forall integer i, j; i < j ==> i+1 < j+1 ;
  }
*/
\end{CodeBlock}



And we can see that in Frama-C, this property is actually assumed to be
true\footnote{In section \ref{l2:acsl-logic-definitions-answers}, we
present an {\em extremely} useful axiom.}:



\image{first-axiom}




\levelFourTitle{Link with lemmas}



Lemmas and axioms allows to express the same kinds of properties.
Namely, properties expressed about quantified variables (and possibly
global variables, but it is quite rare since it is often difficult to
find a global property about such variables being both true and
interesting). Apart this first common point, we can also notice that
when we are not considering the definition of the lemma itself, lemmas
are assumed to be true by WP exactly as axioms are.




The only difference between lemmas and axioms from a proof point of view
is that we must provide a proof that each lemma is true, whereas an
axiom is always assumed to be true.



\levelThreeTitle{Recursive function or predicate definitions}


Axiomatic definitions of recursive functions and predicates are
particularly useful since they will prevent provers from unrolling the
recursion when it is possible.



The idea is then not to define directly the function or the predicate
but to declare it and then to define axioms that specify its
behavior. If we come back to the factorial function, we can define it
axiomatically as follows:



\begin{CodeBlock}{c}
/*@
  axiomatic Factorial{
    logic integer factorial(integer n);

    axiom factorial_0:
      \forall integer i; i <= 0 ==> 1 == factorial(i) ;

    axiom factorial_n:
      \forall integer i; i > 0 ==> i * factorial(i-1) == factorial(i) ;
  }
*/
\end{CodeBlock}



In this axiomatic definition, our function does not have a body. Its
behavior is only defined by the axioms we have stated about it.
Except this, nothing changes, in particular the logic function can
be used in our specification just as before.



A small subtlety that we must take care of is the fact that if some
axioms state properties about the content of some pointed memory cells,
we have to specify considered memory blocks using the \texttt{reads}
notation in the declaration. If we omit such a specification, the
predicate or function will be considered to be stated about the received
pointers and not about pointer memory blocks. So, if the code modifies
the content of an array for which we had proven that the predicate or
function gives some result, this result will not be considered to be
potentially different.



For example, if we take the inductive property we stated for "zeroed" in the
previous chapter, we can write it using an axiomatic definition, and it will be
written like this:



\CodeBlockInput[5][16]{c}{reset-0.c}



Notice the \CodeInline{reads[b .. e-1]} that specifies the memory location
on which the predicate depends. While it is not necessary to specify what are
the memory locations read in an inductive definition, we have to specify such
an information for axiomatically defined properties.


\levelThreeTitle{Consistency}


By adding axioms to our knowledge base, we can produce more complex
proofs since some part of these proofs, expressed by axioms, do not need
to be proved (they are already specified to be true) shortening
the proof process. However, using axiomatic definitions, \textbf{we must
be extremely careful}. Indeed, even a small error could introduce false
in the knowledge base, making our whole reasoning futile. Our reasoning
would still be correct, but relying on false knowledge, it would only
learn incorrect things.



The simplest example is the following:



\CodeBlockInput{c}{false.c}



And everything is proved, comprising the fact that the dereferencing of
0 is valid:



\image{false-axiom}


Of course, this example is extreme, we would not write such an axiom.
The problem is in fact that it is really easy to write an axiomatic
definition that is subtly false when we express more complex properties,
or adding assumptions about the global state of the system.







When we start to create axiomatic definitions, it is worth adding
assertions or postconditions requiring a proof of false that we expect
to fail to ensure that the definition is not inconsistent. However, it
is often not enough! If the subtlety that creates the inconsistency is
hard enough to find, provers could need a lot of information other than
the axiomatic definition itself to be able to find and use the
inconsistency, we then need to always be careful!




More specifically, specifying the values read by a function or a
predicate is important for the consistency of an axiomatic definition.
Indeed, as previously explained, if we do not specify what is read when
a pointer is received, an update of a value in the array does not
invalidate a property known about the content of the array. In such a
case, the proof is performed but since the axiom does not talk about the
content of the array, we do not prove anything.




For example, in the function that resets an array to 0, if we modify the
axiomatic definition, removing the specification of the values that are
read by the predicate (\CodeInline{reads a[b .. e-1]}), the proof
will still be performed, but will not prove anything about the content
of the arrays. For example, the following function:
\CodeBlockInput[16][25]{c}{all-zeroes-bad.c}
is proved to be correct, while we obviously changed a value in the array
and the value is not 0 anymore.



Note that unlike inductive definitions, where Why3 provides us a way to control
that what we write in ACSL is relatively well-defined, we do not have such a
mechanism for axiomatic definitions. Basically, even with Why3 such a definition
is translated into a list of axioms that are thus assumed.



\levelThreeTitle{Example: counting occurrences of a value}


In this example, we want to prove that an algorithm actually counts the
occurrences of a value in an array. We start by axiomatically defining
what is the number of occurrences of a value in an array:



\CodeBlockInput[3][22]{c}{occurrences_of.c}



We have three different cases:
\begin{itemize}
\item
  the considered range of values is empty: the number of occurrences is
  0,
\item
  the considered range of values is not empty and the last element is
  the one we are searching for: the number of occurrences is the number of
  occurrences in the current range without the last element, plus 1,
\item
  the considered range of values is not empty and the last element is
  not the one we are searching for: the number of occurrences is the number
  of occurrences in the current range without the last element.
\end{itemize}
Then, we can write the C function that computes the number of
occurrences of a value in an array and prove it:



\CodeBlockInput[24][42]{c}{occurrences_of.c}



An alternative way to specify, in this code, that \texttt{result} is at
most \texttt{i}, is to express a more general lemma about the number of
occurrences of a value in an array, since we know that it is
comprised between 0 and the size of maximum considered range of values:



\begin{CodeBlock}{c}
/*@
lemma l_occurrences_of_range{L}:
  \forall int v, int* in, integer from, to:
    from <= to ==> 0 <= l_occurrences_of(v, in, from, to) <= to-from;
*/
\end{CodeBlock}



An automatic solver cannot discharge this lemma. It would be necessary
to prove it interactively using Coq, for example. By expressing generic
manually proved lemmas, we can often add useful tools to provers to
manipulate more efficiently our axiomatic definitions, without directly
adding new axioms that would augment the chances to introduce errors.
Here, we still have to realize the proof of the lemma to get a complete
proof.


\levelThreeTitle{Example: The \texttt{strlen} function}


In this section, let us prove the C \CodeInline{strlen} function:


\CodeBlockInput[5]{c}{strlen-0.c}


First, we have to provide a suitable contract. Let us suppose that we
have a logic function \CodeInline{strlen} that returns the length of a
string, that is to say, what we expect of our C function:


\begin{CodeBlock}{c}
/*@
  logic integer strlen(char const* s) = // let's see later
*/
\end{CodeBlock}



Basically, we want to receive a valid string in input, and we want to
compute a value that equals to the result of our logic function
\CodeInline{strlen} applied to this string, of course this function
does not assign anything. Defining what is a valid string is not that
simple. Indeed, previously in this tutorial, we only worked with
arrays, receiving in input both the array and the size of the array,
however here, and as it is common in C, we suppose that the string ends
with a character \CodeInline{'\textbackslash{}0'}. That means that we
basically need the \CodeInline{strlen} function to define what is a
valid string. Let us first use this definition (note that we use the
\CodeInline{\textbackslash{}valid\_read} variant of pointer validity
since we do not expect the function to modify the string) and provide
a contract for \CodeInline{strlen}:



\CodeBlockInput[14][24]{c}{strlen-step.c}



Defining the logic function \CodeInline{strlen} is a bit tricky. Indeed,
we want to compute the length of a string by finding the character
\CodeInline{'\textbackslash{}0'}, we expect to find it but in fact, we
can easily imagine that we receive a string of infinite length. In this case,
we would like to return an error value, but it is basically impossible to
guarantee that the computation terminates, thus a logic function cannot
be used to express this property.



Thus, let us define this function axiomatically. First, let us define what
is read by the function, which is: any memory cell from the pointer to an
infinite range of address. Then we consider two cases: the string is
finite, or it is not, that leads to two axioms: \CodeInline{strlen} returns
a positive value that corresponds to the index of the first
\CodeInline{'\textbackslash{}0'} character, and returns a negative value if
no such value exists.



\CodeBlockInput[4][15]{c}{strlen-proved.c}



And now, we can be more precise for our definition of
\CodeInline{\textbackslash{}valid\_read\_string}, a valid string is a
string such that it is valid from the first index to \CodeInline{strlen}
of the string and, such that this value is greater than 0 (since an
infinite string is not a valid string):




\CodeBlockInput[27][30]{c}{strlen-proved.c}




With this definition we can now go further and provide a suitable
invariant to the loop of the \CodeInline{strlen} function. It is quite
simple: \CodeInline{i} ranges between 0 and \CodeInline{strlen(s)}, for
all values met before the iteration \CodeInline{i}, they are not
\CodeInline{'\textbackslash{}0'}. This loop assigns \CodeInline{i} and
the variant corresponds to the distance between \CodeInline{i} and
\CodeInline{strlen(s)}. However, if we try to produce the proof of
correctness of the function, it fails. And to get more information
we can try a verification asking RTE with the verification that unsigned
integers do not overflow:



\image{strlen-failed}



We see that the prover fails to discharge the verification condition
related to the range of \CodeInline{i}, and that
\CodeInline{i} can exceed the maximum of an unsigned int. One could try
to provide a limit to the value of \CodeInline{strlen(s)} in precondition:



\CodeBlockInput[33][33]{c}{strlen-proved.c}



However, it is not enough, and the reason is that while we have defined
that the value of \CodeInline{strlen(s)} is defined to be the index of
the first \CodeInline{'\textbackslash{}0'} in the array, the converse is
not true: knowing that the value of \CodeInline{strlen(s)} is positive
is not enough to deduce that the value at the corresponding index is
\CodeInline{'\textbackslash{}0'}. Thus we extend the axiomatic definition
with another proposition that gives us this fact (we also add one for the
values that precede the \CodeInline{strlen(s)} index even if here, it is
not necessary):



\CodeBlockInput[17][23]{c}{strlen-proved.c}




And this time the proof succeeds. Frama-C provides its own standard
library headers, and they include an
axiomatic definition for the \CodeInline{strlen} logic function. It can
be found in the installation directory of Frama-C, under the directory
\CodeInline{libc}, the file is named \CodeInline{\_\_fc\_string\_axiomatic.h}.
Note that this definition include more axioms in order to be able to
deduce more properties about \CodeInline{strlen}.



\levelThreeTitle{Exercises}


\levelFourTitle{Occurrence counting}


The following program cannot be proved with the axiomatic definition we
previously defined about occurrences counting:


\CodeBlockInput[18][30]{c}{ex-1-occurrences_of.c}


Re-express the axiomatic definition in a form that allows to prove the
program.


\levelFourTitle{Greatest Common Divisor}


Express the logic function that allows to compute the greatest common divisor as
an axiomatic definition, write the contract of the \CodeInline{gcd} function and
prove it:


\CodeBlockInput[5]{c}{ex-2-gcd.c}


\levelFourTitle{Sum of the N first integers}


Express the logic function that allows to compute the sum of the N first
integers as an axiomatic definition. Write the contract of the following
\CodeInline{sum\_n} function and prove it:


\CodeBlockInput[5]{c}{ex-3-sum-N-first-ints.c}


\levelFourTitle{Permutation}


Take back the example about selection sort
(section~\ref{l3:acsl-logic-definitions-inductive-sort}). Re-express the
permutation predicate using an axiomatic definition. Take care of the
\CodeInline{reads} clause (in particular, note that the predicate relates
two memory labels).


\CodeBlockInput[5]{c}{ex-4-sort.c}

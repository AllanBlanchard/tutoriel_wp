It is time to enter the heart of the matter. Rather than starting with
basic notions of the C language, as we would do for a tutorial about C,
we will start with functions. First because it is necessary to be able
to write functions before starting this tutorial (to be able to prove
that a code is correct, being able to write it correct is required), and
then because it will allow us to directly prove some programs.



After this part about functions, we will on the opposite focus on simple
notions like assignments or conditonal structures, to understand how our
tool really works.



In order to be able to prove that a code is valid, we first need to
specify what we expect of it. Building the proof of our program consists
in ensuring that the code we wrote corresponds to the specification that
describes its job. As we previously said, Frama-C provides the ACSL
language to let the developer write contracts about each function (but
that is not its only purpose, as we will see later).



\levelTwoTitle{Contract definition}

Le principe d'un contrat de fonction est de poser les conditions selon
lesquelles la fonction s'exécutera. On distinguera deux parties :


\begin{itemize}
\item \textbf{la précondition}, c'est-à-dire ce que doit respecter le code
      appelant à propos des variables passées en paramètres et de l'état de
      la mémoire globale pour que la fonction s'exécute correctement ;
\item \textbf{la postcondition}, c'est-à-dire ce que s'engage à respecter la
      fonction en retour à propos de l'état de la mémoire et de la valeur de
      retour.
\end{itemize}


Ces propriétés sont exprimées en langage ACSL dont la syntaxe est relativement
simple pour qui a déjà fait du C, puisqu'elle reprend la syntaxe des expressions
booléennes du C. Cependant, elle ajoute également :



\begin{itemize}
\item certaines constructions et connecteurs logiques qui ne sont pas présents
originellement en C pour faciliter l'écriture ;
\item des prédicats pré-implémentés pour exprimer des propriétés souvent utiles
en C (par exemple, la validité d'un pointeur) ;
\item ainsi que des types plus généraux que les types primitifs du C,
typiquement les types entiers ou réels.
\end{itemize}


Nous introduirons au fil du tutoriel les notations présentes dans le
langage ACSL.



Les spécifications ACSL sont introduites dans nos codes source par
l'intermédiaire d'annotations placées dans des commentaires. Syntaxiquement,
un contrat de fonction est intégré dans les sources de la manière suivante :



\begin{CodeBlock}{c}
/*@
  //contrat
*/
void foo(int bar){

}
\end{CodeBlock}



Notons bien le \CodeInline{@} à la suite du début du bloc de commentaire, c'est lui qui
fait que ce bloc devient un bloc d'annotations pour Frama-C et pas un simple
bloc de commentaires à ignorer.



Maintenant, regardons comment sont exprimés les contrats, à commencer par la
postcondition, puisque c'est ce que nous attendons en priorité de notre
programme (nous nous intéresserons ensuite aux préconditions).



\levelThreeTitle{Postcondition}


La postcondition d'une fonction est précisée avec la clause \CodeInline{ensures}.
Nous travaillerons avec la fonction suivante qui donne la valeur absolue
d'un entier reçu en entrée.
Une de ses postconditions est que le résultat (que nous notons avec le
mot-clé \CodeInline{\textbackslash{}result}) est supérieur ou égal à 0.



\CodeBlockInput[5]{c}{abs-1.c}



(Notons le \CodeInline{;} à la fin de la ligne de spécification comme en C).



Mais ce n'est pas tout, il faut également spécifier le comportement général
attendu d'une fonction renvoyant la valeur absolue. À savoir : si la valeur
est positive ou nulle, la fonction renvoie la même valeur, sinon elle renvoie
l'opposé de la valeur.



Nous pouvons spécifier plusieurs postconditions, soit en les composants avec
un \CodeInline{\&\&} comme en C, soit en introduisant une nouvelle clause \CodeInline{ensures},
comme illustré ci-dessous.



\CodeBlockInput[5]{c}{abs-2.c}



Cette spécification est l'opportunité de présenter un connecteur logique
très utile que propose ACSL mais qui n'est pas présent en C :
l'implication $A \Rightarrow B$, que l'on écrit en ACSL \CodeInline{A ==> B}.
La table de vérité de l'implication est la suivante :



\begin{longtabu}{|c|c|c|} \hline
$A$ & $B$ & $A \Rightarrow B$ \\ \hline
$F$ & $F$ & $V$ \\ \hline
$F$ & $V$ & $V$ \\ \hline
$V$ & $F$ & $F$ \\ \hline
$V$ & $V$ & $V$ \\ \hline
\end{longtabu}



Ce qui veut dire qu'une implication $A \Rightarrow B$ est vraie dans deux cas :
soit $A$ est fausse (et dans ce cas, il ne faut pas se préoccuper de $B$), soit
$A$ est vraie et alors $B$ doit être vraie aussi. Notons que cela signifie que
$A \Rightarrow B$ est équivalente à $\neg A \vee B$. L'idée étant finalement
« je veux savoir si dans le cas où $A$ est vrai, $B$ l'est aussi. Si $A$ est
faux, je considère que l'ensemble est vrai ». Par exemple, « s'il pleut, je veux
vérifier que j'ai un parapluie, s'il ne pleut pas, ce n'est pas un problème
de savoir si j'en ai un ou pas, tout va bien ».



Sa cousine l'équivalence $A \Leftrightarrow B$ (écrite \CodeInline{A <==> B} en ACSL)
est plus forte. C'est la conjonction de l'implication dans les deux sens :
$(A \Rightarrow B) \wedge (B \Rightarrow A)$. Cette formule n'est vraie que
dans deux cas : $A$ et $B$ sont vraies toutes les deux, ou fausses
toutes les deux (c'est donc la négation du ou-exclusif). Pour continuer avec
notre petit exemple, « je ne veux plus seulement savoir si j'ai un parapluie
quand il pleut, je veux être sûr de n'en avoir que dans le cas où il pleut ».



\begin{Information}
Profitons en pour rappeler l'ensemble des tables de vérités des opérateurs
usuels en logique du premier ordre ($\neg$ = \CodeInline{!}, $\wedge$ = \CodeInline{\&\&},
$\vee$ = \CodeInline{||}) :

\begin{longtabu}{|c|c|c|c|c|c|c|} \hline
$A$ & $B$ & $\neg A$ & $A \wedge B$ & $A \vee B$ & $A \Rightarrow B$ & $A \Leftrightarrow B$ \\ \hline
$F$ & $F$ & $V$ & $F$ & $F$ & $V$ & $V$ \\ \hline
$F$ & $V$ & $V$ & $F$ & $V$ & $V$ & $F$ \\ \hline
$V$ & $F$ & $F$ & $F$ & $V$ & $F$ & $F$ \\ \hline
$V$ & $V$ & $F$ & $V$ & $V$ & $V$ & $V$ \\ \hline
\end{longtabu}
\end{Information}


Revenons à notre spécification. Quand nos fichiers commencent à être longs et
contenir beaucoup de spécifications, il peut être commode de nommer les
propriétés que nous souhaitons vérifier. Pour cela, nous indiquons un nom (les
espaces ne sont pas autorisées) suivi de \CodeInline{:} avant de mettre effectivement
la propriété, il est possible de mettre plusieurs « étages » de noms pour
catégoriser nos propriétés. Par exemple, nous pouvons écrire ceci :



\CodeBlockInput[7]{c}{abs-3.c}



Dans une large part du tutoriel, nous ne nommerons pas les éléments que nous
tenterons de prouver, les propriétés seront généralement relativement simples
et peu nombreuses, les noms n'apporteraient pas beaucoup d'information.



Nous pouvons copier/coller la fonction \CodeInline{abs} et sa spécification dans un
fichier abs.c et regarder avec Frama-C si l'implémentation est conforme à la
spécification.



Pour cela, il faut lancer l'interface graphique de Frama-C (il est également
possible de se passer de l'interface graphique, cela ne sera pas présenté
dans ce tutoriel) soit par cette commande :



\begin{CodeBlock}{bash}
frama-c-gui
\end{CodeBlock}



Soit en l'ouvrant depuis l'environnement graphique.



Il est ensuite possible de cliquer sur le bouton « \textit{Create a new session from
existing C files} », les fichiers à analyser peuvent être sélectionnés par
double-clic, OK terminant la sélection. Par la suite, l'ajout d'autres
fichiers à la session s'effectue en cliquant sur Files > Source Files.



À noter également qu'il est possible d'ouvrir directement le(s) fichier(s)
depuis la ligne de commande en le(s) passant en argument(s) de \CodeInline{frama-c-gui}.



\begin{CodeBlock}{bash}
frama-c-gui abs.c
\end{CodeBlock}



\image{1-abs-1.png}


La fenêtre de Frama-C s'ouvre, dans le volet correspondant aux fichiers et aux
fonctions, nous pouvons sélectionner la fonction \CodeInline{abs}.
Pour chaque ligne \CodeInline{ensures}, il y a un cercle bleu dans la marge.
Ces cercles indiquent qu'aucune vérification n'a été tentée pour ces lignes.



Nous demandons de vérifier que le code répond à la spécification en faisant
un clic droit sur le nom de la fonction et « \textit{Prove function annotations by WP} » :



\image{1-abs-2.png}[Lancer la vérification de abs avec WP]


Nous pouvons voir que les cercles bleus deviennent des pastilles vertes,
indiquant que la spécification est bien assurée par le programme. Il est
possible de prouver les propriétés une à une en cliquant-droit sur celles-ci
et pas sur le nom de la fonction.



Mais le code est-il vraiment sans erreur pour autant ? WP nous permet de nous
assurer que le code répond à la spécification, mais il ne fait pas de contrôle
d'erreur à l'exécution (\textit{runtime error}, abrégé RTE) si nous le demandons
pas. Un autre \textit{plugin} de Frama-C, appelé sobrement RTE, peut être
utilisé pour générer des annotations ACSL qui peuvent ensuite être vérifiées par
d'autres \textit{plugins}.
Son but est
d'ajouter des contrôles dans le programme pour les erreurs d'exécutions
possibles (débordements d'entiers, déréférencements de pointeurs invalides,
division par 0, etc).



Pour activer ce contrôle, nous devons activer l'option dans la configuration
de WP. Pour cela, il faut d'abord cliquer sur le bouton de configuration des
\textit{plugins} :


\image{plugin-options}


Et ensuite cocher l'option \CodeInline{-wp-rte} dans les options liées à WP :


\image{select-rte}


Il est également possible de demander à WP d'ajouter ces
contrôles par un clic droit sur le nom de la fonction puis
« Insert wp-rte guards ».


\begin{Information}
  A partir de ce point du tutoriel, \CodeInline{-wp-rte} devra toujours être
  activé pour traiter les exemples, sauf indication contraire.
\end{Information}


Enfin, nous relançons la vérification (nous pouvons également cliquer sur le
bouton « \textit{Reparse} » de la barre d'outils, cela aura pour effet de supprimer les
preuves déjà effectuées).



Nous voyons alors que WP échoue à prouver  l'impossibilité de
débordement arithmétique sur le calcul de -val. Et c'est bien normal parce
que -\CodeInline{INT\_MIN} ($-2^{31}$) > \CodeInline{INT\_MAX} ($2^{31}-1$).



\image{1-abs-4.png}


\begin{Information}
Il est bon de noter que le risque de dépassement est pour nous réel car nos
machines (dont Frama-C détecte la configuration) fonctionne en
\externalLink{complément à deux}{https://fr.wikipedia.org/wiki/Compl\%C3\%A9ment\_\%C3\%A0\_deux}
pour lequel le dépassement n'est pas défini par la norme C.
\end{Information}


Ici, nous pouvons voir un autre type d'annotation ACSL. La
ligne \CodeInline{//@ assert propriete ;} nous permet de demander la vérification
d'une propriété à un point particulier du programme. Ici, l'outil l'a
insérée pour nous, car il faut vérifier que le \CodeInline{-val} ne provoque pas de
débordement, mais il est également possible d'en ajouter manuellement dans
un code.



Comme le montre cette capture d'écran, nous avons deux nouveaux codes couleur
pour les pastilles : vert + marron et orange.



La couleur vert + marron nous indique que la preuve a été effectuée mais
qu'elle dépend potentiellement de propriétés pour lesquelles ce n'est pas le
cas.



Si  la preuve n'a pas été recommencée intégralement par rapport à la preuve
précédente, ces pastilles ont dû rester vertes, car les preuves associées ont
été réalisées avant l'introduction de la propriété nous assurant l'absence
d'erreur d'exécution, et ne se sont donc pas reposées sur la connaissance de
cette propriété puisqu'elle n'existait pas.



En effet, lorsque WP transmet une obligation de preuve à un prouveur automatique,
il transmet deux types de propriétés : $G$, le but, la propriété
que l'on cherche à prouver, et $S_1$ ... $S_n$ les diverses suppositions que l'on
peut faire à propos de l'état du programme au point où l'on cherche à vérifier $G$.
Cependant, il ne reçoit pas, en retour, quelles propriétés ont été utilisées par
le prouveur pour valider $G$. Donc si $S_3$ fait partie des suppositions, et si
WP n'a pas réussi à obtenir une preuve de $S_3$, il indique que $G$ est vraie, mais
en supposant que $S_3$ est vraie, pour laquelle nous n'avons actuellement pas
établi de preuve.



La couleur orange nous signale qu'aucun prouveur n'a pu déterminer si la
propriété est vérifiable. Les deux raisons peuvent être :



\begin{itemize}
\item qu'il n'a pas assez d'information pour le déterminer ;
\item que malgré toutes ses recherches, il n'a pas pu trouver un résultat à
temps. Auquel cas, il rencontre un \textit{timeout} dont la durée est configurable
dans le volet de WP.
\end{itemize}


Dans le volet inférieur, nous pouvons sélectionner l'onglet « \textit{WP Goals} »,
celui-ci nous affiche la liste des obligations de preuve et pour chaque
prouveur indique un petit logo si la preuve a été tentée et si elle a été
réussie, échouée ou a rencontré un \textit{timeout} (logo avec les ciseaux).
Pour voir la totalité des obligations de preuves, il
faut s'assurer que « \textit{All Results} » est bien sélectionné dans le champ encadré
dans la capture.



\image{1-abs-5.png}


Le tableau est découpé comme suit, en première colonne nous avons le nom de la
fonction où se trouve le but à prouver. En seconde colonne nous trouvons le nom
du but. Ici par exemple notre postcondition nommée est estampillée
« postcondition 'positive\textit{value,function}result' », nous pouvons d'ailleurs noter
que lorsqu'une propriété est sélectionnée dans le tableau, elle est également
surlignée dans le code source. Les propriétés anonymes se voient assignées
comme nom le type de propriété voulu. En troisième colonne, nous trouvons le
modèle mémoire utilisé pour la preuve, (nous n'en parlerons pas dans ce
tutoriel). Finalement, les dernières colonnes représentent les différents
prouveurs accessibles à WP.



Dans ces prouveurs, le premier élément de la colonne est Qed. Ce n'est pas
à proprement parler un prouveur. C'est un outil utilisé par WP pour simplifier
les propriétés avant de les envoyer aux prouveurs externes. Ensuite, nous
voyons la colonne Script, les scripts fournissent une manière de terminer
les preuves à la main lorsque les prouveurs automatiques n'y arrivent pas.
Finalement, nous trouvons la colonne Alt-Ergo, qui est un prouveur automatique.
Notons que sur la propriété en question des ciseaux sont indiqués, cela
veut dire que le prouveur a été stoppé à cause d'un \textit{timeout}.


En fait, si nous double-cliquons sur la
propriété « ne pas déborder » (surlignée en bleu dans la capture précédente),
nous pouvons voir ceci (si ce n'est pas le cas, il faut s'assurer que
« \textit{Raw obligation} » est bien sélectionné dans le champ encadré en bleu) :



\image{1-abs-6.png}


C'est l'obligation de preuve que génère WP par rapport à notre propriété et
notre programme, il n'est pas nécessaire de comprendre tout ce qu'il s'y passe,
juste d'avoir une idée globale. Elle contient (dans la partie « \textit{Assume} ») les
suppositions que nous avons pu donner et celles que WP a pu déduire des
instructions du programme. Elle contient également (dans la partie « \textit{Prove} »)
la propriété que nous souhaitons vérifier.



Que fait WP avec ces éléments ? En fait, il les transforme en une formule
logique puis demande aux différents prouveurs s'il est possible de la
satisfaire (de trouver pour chaque variable, une valeur qui rend la formule
vraie), cela détermine si la propriété est prouvable. Mais avant d'envoyer
cette formule aux prouveurs, WP utilise un module qui s'appelle Qed et qui est
capable de faire différentes simplifications à son sujet. Parfois, comme dans
le cas des autres propriétés de \CodeInline{abs}, ces simplifications suffisent à
déterminer que la propriété est forcément vraie, auquel cas, nous ne faisons
pas appel aux prouveurs.



Lorsque les prouveurs automatiques ne parviennent pas à assurer que nos
propriétés sont bien vérifiées, il est parfois difficile de comprendre
pourquoi. En effet, les prouveurs sont généralement incapables de nous
répondre autre chose que « oui », « non » ou « inconnu », ils sont pas
incapables d'extraire le « pourquoi » d'un « non » ou d'un « inconnu ». Il
existe des outils qui
sont capables d'explorer les arbres de preuve pour en extraire ce type
d'information, Frama-C n'en possède pas à l'heure actuelle. La lecture des
obligations de preuve peut parfois nous aider, mais cela demande un peu
d'habitude pour pouvoir les déchiffrer facilement. Finalement, le meilleur
moyen de comprendre la raison d'un échec est d'effectuer la preuve de manière
interactive avec Coq. En revanche, il faut déjà avoir une certaine habitude de
ce langage pour ne pas être perdu devant les obligations de preuve générées par
WP, étant donné que celles-ci encodent les éléments de la sémantique de C, ce
qui rend le code souvent indigeste.



Si nous retournons dans notre tableau des obligations de preuve (bouton
encadré en rouge dans la capture d'écran précédente), nous pouvons donc voir
que les hypothèses n'ont pas suffi aux prouveurs pour déterminer que la
propriété  « absence de débordement » est vraie (et nous l'avons dit : c'est
normal), il nous faut donc ajouter une hypothèse supplémentaire pour garantir
le bon fonctionnement de la fonction : une précondition d'appel.



\levelThreeTitle{Précondition}


Les préconditions de fonctions sont introduites par la clause \CodeInline{requires}.
De la même manière qu'avec \CodeInline{ensures}, nous pouvons composer nos
expressions logiques et mettre plusieurs préconditions :



\begin{CodeBlock}{c}
/*@
  requires 0 <= a < 100;
  requires b < a;
*/
void foo(int a, int b){

}
\end{CodeBlock}



Les préconditions sont des propriétés sur les entrées (et potentiellement sur
des variables globales) qui seront supposées préalablement vraies lors de
l'analyse de la fonction. La preuve que celles-ci sont effectivement validées
n'interviendra qu'aux points où la fonction est appelée.



Dans ce petit exemple, nous pouvons également noter une petite différence avec
le C dans l'écriture des expressions booléennes. Si nous voulons spécifier
que \CodeInline{a} se trouve entre 0 et 100, il n'y a pas besoin d'écrire \CodeInline{0 <= a \&\& a < 100}
(c'est-à-dire en composant les deux comparaisons avec un \CodeInline{\&\&}). Nous
pouvons simplement écrire \CodeInline{0 <= a < 100} et l'outil se chargera de faire
la traduction nécessaire.



Si nous revenons à notre exemple de la valeur absolue, pour éviter le
débordement arithmétique, il suffit que la valeur de \CodeInline{val} soit
strictement  supérieure à \CodeInline{INT\_MIN} pour garantir que le
débordement n'arrive pas.
Nous l'ajoutons donc comme précondition (à noter : il faut également
inclure l'en-tête où \CodeInline{INT\_MIN} est défini) :



\CodeBlockInput{c}{abs-4.c}



\begin{Warning}
Rappel : la fenêtre de Frama-C ne permet pas l'édition du code source.
\end{Warning}


Une fois le code source modifié de cette manière, un clic sur « \textit{Reparse} » et
nous lançons à nouveau l'analyse. Cette fois, tout est validé pour WP ; notre
implémentation est prouvée :



\image{2-abs-1.png}


Nous pouvons également vérifier qu'une fonction qui appellerait \CodeInline{abs}
respecte bien la précondition qu'elle impose :



\CodeBlockInput[15]{c}{abs-5.c}



\image{2-foo-1.png}

Notons qu'en cliquant sur la pastille à côté de l'appel de fonction, nous
pouvons voir la liste des préconditions et voir quelles sont celles qui ne sont
pas vérifiées. Ici, nous n'avons qu'une précondition, mais quand il y en a
plusieurs, c'est très utile pour pouvoir voir quel est exactement le problème.


\image{2-foo-1-bis.png}


Pour modifier un peu l'exemple, nous pouvons essayer d'inverser les deux
dernières lignes. Auquel cas, nous pouvons voir que l'appel \CodeInline{abs(a)}
est validé par WP s'il se trouve après l'appel \CodeInline{abs(INT\_MIN)} !
Pourquoi ?



Il faut bien garder en tête que le principe de la preuve déductive est de nous
assurer que si les préconditions sont vérifiées et que le calcul termine alors
la postcondition est vérifiée.


Si nous donnons à notre fonction une valeur qui viole explicitement sa
précondition, nous pouvons déduire que n'importe quoi peut arriver, incluant
obtenir « faux » en postcondition. Plus précisément, ici, après l'appel, nous
supposons que la précondition est vraie (puisque la fonction ne peut pas
modifier la valeur reçue en paramètre), sinon la fonction n'aurait pas pu
s'exécuter correctement. Par conséquent, nous supposons que
\CodeInline{INT\_MIN < INT\_MIN} qui est trivialement faux. À partir de là,
nous pouvons  prouver tout ce que nous voulons, car ce « faux » devient une
supposition pour tout appel qui viendrait ensuite. À partir de « faux », nous
prouvons tout ce que
nous voulons, car si nous avons la preuve de « faux » alors « faux » est vrai,
de même que « vrai » est vrai. Donc tout est vrai.



En prenant le programme modifié, nous pouvons d'ailleurs regarder les obligations
de preuve générées par WP pour l'appel fautif et l'appel prouvé par conséquent :



\image{2-foo-2.png}


\image{2-foo-3.png}


Nous pouvons remarquer que pour les appels de fonctions, l'interface graphique
surligne le chemin d'exécution suivi avant l'appel dont nous cherchons à
vérifier la précondition. Ensuite, si nous regardons l'appel \CodeInline{abs(INT\_MIN)},
nous remarquons qu'à force de simplifications, Qed a déduit que nous
cherchons à prouver « False ». Conséquence logique, l'appel suivant \CodeInline{abs(a)}
reçoit dans ses suppositions « False ». C'est pourquoi Qed est capable de déduire
immédiatement « True ».



La deuxième partie de la question est alors : pourquoi lorsque nous mettons les
appels dans l'autre sens (\CodeInline{abs(a)} puis \CodeInline{abs(INT\_MIN)}), nous obtenons
quand même une violation de la précondition sur le deuxième ? La réponse est
simplement que pour \CodeInline{abs(a)} nous ajoutons dans nos suppositions la
connaissance \CodeInline{a < INT\_MIN}, et tandis que nous n'avons pas de preuve
que c'est vrai, nous n'en avons pas non plus que c'est faux. Donc si nous obtenons
nécessairement une preuve de « faux » avec un appel \CodeInline{abs(INT\_MIN)}, ce n'est
pas le cas de l'appel \CodeInline{abs(a)} qui peut aussi ne pas échouer.


\levelThreeTitle{Exercices}


Ces exercices ne sont pas absolument nécessaires pour lire les chapitres à
venir dans ce tutoriel, nous conseillons quand même de les réaliser. Nous
suggérons aussi fortement d'au moins lire le quatrième exercice qui introduit
une notation qui peut parfois d'avérer utile.


\levelFourTitle{Addition}


Écrire la postcondition de la fonction d'addition suivante :


\CodeBlockInput[5]{c}{ex-1-addition.c}


Lancer la commande :

\begin{CodeBlock}{bash}
frama-c-gui your-file.c -wp
\end{CodeBlock}


Lorsque la preuve que la fonction respecte son contrat est établie, lancer
la commande :

\begin{CodeBlock}{bash}
frama-c-gui your-file.c -wp -wp-rte
\end{CodeBlock}


qui devrait échouer. Adapter le contrat en ajoutant la bonne précondition.


\levelFourTitle{Distance}


Écrire la postcondition de la fonction distance suivante, en exprimant
la valeur de \CodeInline{b} en fonction de \CodeInline{a} et
\CodeInline{\textbackslash{}result} :


\CodeBlockInput[5]{c}{ex-2-distance.c}


Lancer la commande :


\begin{CodeBlock}{bash}
frama-c-gui your-file.c -wp
\end{CodeBlock}


Lorsque la preuve que la fonction respecte son contrat est établie, lancer
la commande :

\begin{CodeBlock}{bash}
frama-c-gui your-file.c -wp -wp-rte
\end{CodeBlock}


qui devrait échouer. Adapter le contrat en ajoutant la bonne précondition.


\levelFourTitle{Lettres de l'alphabet}


Écrire la postcondition de la fonction suivante, qui retourne vrai si le
caractère reçu en entrée est une lettre de l'alphabet. Utiliser la relation
d'équivalence  \CodeInline{<==>}.


\CodeBlockInput[5]{c}{ex-3-alphabet.c}


Lancer la commande :


\begin{CodeBlock}{bash}
frama-c-gui your-file.c -wp
\end{CodeBlock}


Toutes les obligations de preuve devraient être prouvées, y compris les
assertions dans la fonction main.


\levelFourTitle{Jours du mois}


Écrire la postcondition de la fonction suivante qui retourne le nombre de
jours en fonction du mois reçu en entrée (NB: nous considérons que le mois
reçu est entre 1 et 12), pour février, nous considérons uniquement le cas
où il a 28 jours, nous verrons plus tard comment régler ce problème :


\CodeBlockInput[5]{c}{ex-4-day-month.c}


Lancer la commande :


\begin{CodeBlock}{bash}
frama-c-gui your-file.c -wp
\end{CodeBlock}


Lorsque la preuve que la fonction respecte son contrat est établie, lancer
la commande :

\begin{CodeBlock}{bash}
frama-c-gui your-file.c -wp -wp-rte
\end{CodeBlock}


Si cela échoue, adapter le contrat en ajoutant la bonne précondition.


Le lecteur aura peut-être constaté qu'écrire la postcondition est un peu
laborieux. Il est possible de simplifier cela. ACSL fournit la notion
d'ensemble mathématique et l'opérateur \CodeInline{\textbackslash{}in} qui
peut être utilisé pour vérifier si une valeur est dans un ensemble ou non.


Par exemple :

\begin{CodeBlock}{c}
//@ assert 13 \in { 1, 2, 3, 4, 5 } ; // FAUX
//@ assert 3  \in { 1, 2, 3, 4, 5 } ; // VRAI
\end{CodeBlock}


Modifier la postcondition en utilisant cette notation.


\levelFourTitle{Le dernier angle d'un triangle}


Cette fonction reçoit deux valeurs d'angle en entrée et retourne
la valeur du dernier angle composant le triangle correspondant en se
reposant sur la propriété que la somme des angles d'un triangle vaut
180 degrés. Écrire la postcondition qui exprime que la somme des trois
angle vaut 180.


\CodeBlockInput[5]{c}{ex-5-last-angle.c}


Lancer la commande :


\begin{CodeBlock}{bash}
frama-c-gui your-file.c -wp
\end{CodeBlock}


Lorsque la preuve que la fonction respecte son contrat est établie, lancer
la commande :

\begin{CodeBlock}{bash}
frama-c-gui your-file.c -wp -wp-rte
\end{CodeBlock}


Si cela échoue, adapter le contrat en ajoutant la bonne précondition.
Notons que la valeur de chaque angle ne peut pas être supérieure à 180
et que cela inclut l'angle résultant.



\levelTwoTitle{Well specified function}

\levelThreeTitle{Bien traduire ce qui est attendu}


C'est certainement notre tâche la plus difficile. En soi, la programmation est
déjà un effort consistant à écrire des algorithmes qui répondent à notre
besoin. La spécification nous demande également de faire ce travail, la
différence est que nous ne nous occupons plus de préciser la manière de répondre
au besoin mais le besoin lui-même. Pour prouver que la réalisation implémente
bien ce que nous attendons, il faut donc être capable de décrire précisément le
besoin.



Changeons d'exemple et spécifions la fonction suivante :



\CodeBlockInput[4][6]{c}{max-1.c}



Le lecteur pourra écrire et prouver sa spécification. Pour la suite, nous
travaillerons avec celle-ci :



\CodeBlockInput[1][6]{c}{max-1.c}



Si nous donnons ce code à WP, il accepte sans problème de prouver la fonction.
Pour autant cette spécification est-elle suffisante ? Nous pouvons par exemple
essayer de voir si ce code est validé :



\CodeBlockInput[8][14]{c}{max-1.c}



La réponse est non. En fait, nous pouvons aller plus loin en modifiant le corps
de la fonction \CodeInline{max} et remarquer que le code suivant est également valide
quant à la spécification :



\CodeBlockInput[1][8]{c}{max-2.c}




Même si elle est correcte, notre spécification est trop permissive. Il faut que nous
soyons plus précis.
Nous attendons du résultat non seulement qu'il soit supérieur ou égal à nos
deux paramètres mais également qu'il soit exactement l'un des deux :



\CodeBlockInput[1][7]{c}{max-3.c}


Nous pouvons également prouver que cette spécification est vérifiée par notre
fonction. Mais nous pouvons maintenant prouver en plus l'assertion présente dans
notre fonction \CodeInline{foo}, et nous ne pouvons plus prouver que
l'implémentation qui retourne \CodeInline{INT\_MAX} vérifie la spécification.


\levelThreeTitle{Préconditions incohérentes}


Bien spécifier son programme est d'une importance cruciale. Typiquement,
préciser une précondition fausse peut nous donner la possibilité de prouver
FAUX :


\CodeBlockInput{c}{bad-precond.c}


Si nous demandons à WP de prouver cette fonction. Il l'acceptera sans rechigner,
car la propriété que nous lui donnons comme précondition est nécessairement fausse.
Par contre, nous aurons bien du mal à lui donner une valeur en entrée qui respecte
la précondition.


Pour cette catégorie particulière d'incohérences, une fonctionnalité utile de WP
est l'option « \textit{smoke tests} » du greffon. Ces tests préliminaires, effectués
sur notre spécification, sont utilisés pour détecter que des préconditions ne peuvent
pas être satisfaites. Par exemple, ici, nous pouvons lancer cette ligne de
commande :


\begin{CodeBlock}{bash}
  frama-c-gui -wp -wp-smoke-tests file.c
\end{CodeBlock}


et nous obtenons le résultat suivant dans l'interface graphique :


\image{2-bad-precond}


Nous pouvons voir une pastille orange et rouge à côté de la précondition de la
fonction, qui signifie que s'il existe un appel atteignable à la fonction dans
le programme, la précondition sera nécessairement violée lors de cet appel ; et
une pastille rouge dans la liste des obligations de preuve, indiquant qu'un
prouveur a réussi à montrer que la précondition est incohérente.


Notons que lorsque ces tests préliminaires réussissent, par exemple si
nous corrigeons la précondition de cette façon :


\image{2-smoke-success}


cela ne signifie pas que la précondition est nécessairement cohérente, juste
qu'aucun prouveur n'a été capable de montrer qu'elle est incohérente.


Certaines notions que nous verrons plus loin dans le tutoriel apporterons un
risque encore plus grand de créer ce genre d'incohérence. Il faut donc toujours
avoir une attention particulière pour ce que nous spécifions.



\levelThreeTitle{Pointeurs}


S'il y a une notion à laquelle nous sommes confrontés en permanence en
langage C, c'est bien la notion de pointeur. C'est une notion complexe et
l'une des principales cause de bugs critiques dans les programmes, ils ont
donc droit à un traitement de faveur dans ACSL. Pour avoir une spécification
correcte des programmes utilisant des pointeurs, il est impératif de détailler
la configuration de la mémoire que l'on considère.



Prenons par exemple une fonction \CodeInline{swap} pour les entiers :



\CodeBlockInput{c}{swap-0.c}



\levelFourTitle{Historique des valeurs}


Ici, nous introduisons une première fonction logique fournie de base par
ACSL : \CodeInline{\textbackslash{}old}, qui permet de parler de l'ancienne valeur d'un élément.
Ce que nous dit donc la spécification c'est « la fonction doit assurer que
\CodeInline{*a} soit égal à l'ancienne valeur (au sens : la valeur avant l'appel) de \CodeInline{*b}
et inversement ».



La fonction \CodeInline{\textbackslash{}old} ne peut être utilisée que dans la postcondition d'une
fonction. Si nous avons besoin de ce type d'information ailleurs, nous
utilisons \CodeInline{\textbackslash{}at} qui nous permet d'exprimer des propriétés à propos de la
valeur d'une variable à un point donné. Elle reçoit deux paramètres. Le premier
est la variable (ou position mémoire) dont nous voulons obtenir la valeur et le
second la position (sous la forme d'un label C) à laquelle nous voulons
contrôler la valeur en question.



Par exemple, nous pourrions écrire :



\CodeBlockInput[2][6]{c}{at.c}



En plus des labels que nous pouvons nous-mêmes créer, il existe 6 labels
qu'ACSL nous propose par défaut:



\begin{itemize}
\item \CodeInline{Pre}/\CodeInline{Old} : valeur avant l'appel de la fonction,
\item \CodeInline{Post} : valeur après l'appel de la fonction,
\item \CodeInline{LoopEntry} : valeur en début de boucle,
\item \CodeInline{LoopCurrent} : valeur en début du pas actuel de la boucle,
\item \CodeInline{Here} : valeur au point d'appel.
\end{itemize}


\begin{Information}
Le comportement de \CodeInline{Here} est en fait le comportement par défaut lorsque
nous parlons de la valeur d'une variable. Son utilisation avec \CodeInline{\textbackslash{}at} nous
servira généralement à s'assurer de l'absence d’ambiguïté lorsque nous parlons
de divers points de programme dans la même expression.
\end{Information}


À la différence de \CodeInline{\textbackslash{}old}, qui ne peut être utilisée que dans les
postconditions de contrats de fonction, \CodeInline{\textbackslash{}at} peut être utilisée partout.
En revanche, tous les points de programme ne sont pas accessibles selon le type
d'annotation que nous sommes en train d'écrire. \CodeInline{Old} et \CodeInline{Post} ne sont
disponibles que dans les postconditions d'un contrat, \CodeInline{Pre} et \CodeInline{Here}
sont disponibles partout. \CodeInline{LoopEntry} et \CodeInline{LoopCurrent} ne sont
disponibles que dans le contexte de boucles (dont nous parlerons plus loin dans
le tutoriel).


Notons qu'il est important de s'assurer que l'on utilise \CodeInline{\textbackslash{}old} at
\CodeInline{\textbackslash{}at} pour des valeurs qui ont du sens. C'est pourquoi par
exemple dans un contrat, toutes les valeurs reçues en entrée sont placées dans un
appel à \CodeInline{\textbackslash{}old} par Frama-C lorsqu'elles sont utilisées dans
les postconditions, la nouvelle valeur d'une variable fournie en entrée d'une
fonction n'a aucun sens pour l'appelant puisque cette valeur est inaccessible par
lui : elles sont locales à la fonction appelée. Par exemple, si nous regardons le
contrat de la fonction \CodeInline{swap} dans Frama-C, nous pouvons voir que dans
la postcondition, chaque pointeur se trouve dans un appel à \CodeInline{\textbackslash{}old} :


\image{2-old-swap}


Pour la fonction built-in \CodeInline{\textbackslash{}at}, nous devons plus
explicitement faire attention à cela. En particulier, le label transmis en entrée
doit avoir un sens par rapport à la portée de la variable que l'on lui transmet.
Par exemple, dans le programme suivant, Frama-C détecte que nous demandons la valeur
de la variable \CodeInline{x} à un point du programme où elle n'existe pas:


\CodeBlockInput[1][5]{c}{at-2.c}


\image{2-at-1}


Cependant, dans certains cas, tout ce que nous pouvons obtenir est un échec de
la preuve, parce que déterminer si la valeur existe ou non à un label particulier
ne peut être fait par une analyse purement syntaxique. Par exemple, si la variable
est déclarée mais pas définie, ou si nous demandons la valeur d'une zone mémoire
pointée :


\CodeBlockInput[7][19]{c}{at-2.c}


Ici, il est facile de remarquer le problème. Cependant, le label que nous
transmettons à la fonction \CodeInline{\textbackslash{}at} est propagé également
aux sous-expressions. Dans certains cas, des termes qui paraissent tout à fait
innocents peuvent en réalité nous donner des comportements surprenant si nous
ne gardons pas cette idée en tête. Par exemple, dans le programme suivant :


\CodeBlockInput[21][26]{c}{at-2.c}


La première assertion est prouvée, et tandis que la seconde assertion a l'air
d'exprimer la même propriété, elle ne peut pas être prouvée. La raison est
simplement qu'elle n'exprime pas la même propriété. L'expression
\CodeInline{\textbackslash{}at(x[*p], Pre)} doit être lue comme
\CodeInline{\textbackslash{}at(x[\textbackslash{}at(*p)], Pre)} parce que le
label est propagé à la sous-expression \CodeInline{*p}, pour laquelle nous ne
connaissons pas la valeur au label \CodeInline{Pre} (qui n'est pas spécifié).


Pour le moment, nous n'utiliserons pas \CodeInline{\textbackslash{}at}, mais elle peut rapidement se
montrer indispensable pour écrire des spécifications précises.



\levelFourTitle{Validité de pointeurs}


Si nous essayons de prouver le fonctionnement de \CodeInline{swap} (en activant
la vérification des RTE), notre postcondition est bien vérifiée mais WP nous
indique qu'il y a un certain nombre de possibilités de \textit{runtime-error}. Ce qui
est normal, car nous n'avons pas précisé à WP que les pointeurs que nous
recevons en entrée de fonction sont valides.



Pour ajouter cette précision, nous allons utiliser le prédicat \CodeInline{\textbackslash{}valid} qui
reçoit un pointeur en entrée :



\CodeBlockInput[3][11]{c}{swap-1.c}



À partir de là, les déréférencements qui sont effectués par la suite sont
acceptés car la fonction demande à ce que les pointeurs d'entrée soient
valides.



Comme nous le verrons plus tard, \CodeInline{\textbackslash{}valid} peut recevoir plus qu'un
pointeur en entrée. Par exemple, il est possible de lui transmettre une
expression de cette forme : \CodeInline{\textbackslash{}valid(p + (s .. e))} qui voudra dire « pour
tout \CodeInline{i} entre \CodeInline{s} et \CodeInline{e} (inclus), \CodeInline{p+i} est un pointeur valide », ce sera important
notamment pour la gestion des tableaux dans les spécifications.



Si nous nous intéressons aux assertions ajoutées par WP dans la fonction \CodeInline{swap}
avec la validation des RTEs, nous pouvons constater qu'il existe une variante
de \CodeInline{\textbackslash{}valid} sous le nom \CodeInline{\textbackslash{}valid\_read}. Contrairement au premier,
celui-ci assure qu'il est uniquement nécessaire que le pointeur puisse
être déréférencé en lecture et pas forcément en écriture, pour pouvoir
réaliser l'opération de lecture. Cette subtilité est due au fait qu'en C, le
\textit{downcast} de pointeur vers un élément \CodeInline{const} est très facile à faire mais
n'est pas forcément légal.



Typiquement, dans le code suivant :



\CodeBlockInput{c}{unref.c}



Le déréférencement de \CodeInline{p} est valide, pourtant la précondition de \CodeInline{unref}
ne sera pas validée par WP, car le déréférencement de l'adresse de \CodeInline{value}
n'est légal qu'en lecture. Un accès en écriture sera un comportement
indéterminé. Dans un tel cas, nous pouvons préciser que dans \CodeInline{unref}, le
pointeur \CodeInline{p} doit être nécessairement \CodeInline{\textbackslash{}valid\_read} et pas \CodeInline{\textbackslash{}valid}.



\levelFourTitle{Effets de bord}


Notre fonction \CodeInline{swap} est bien prouvable au regard de sa spécification et
de ses potentielles erreurs à l'exécution, mais est-elle pour autant
suffisamment spécifiée ? Pour voir cela, nous pouvons modifier légèrement le code
de cette façon (nous utilisons \CodeInline{assert} pour analyser des propriétés
ponctuelles) :



\CodeBlockInput{c}{swap-1.c}



Le résultat n'est pas vraiment celui escompté :



\image{2-swap-1}


En effet, nous n'avons pas spécifié les effets de bords autorisés pour notre
fonction. Pour cela, nous utilisons la clause \CodeInline{assigns}
qui fait partie des postconditions de la fonction. Elle nous permet de spécifier
quels éléments \textbf{non locaux} (on vérifie bien des effets de bord), sont
susceptibles d'être modifiés par la fonction.



Par défaut, WP considère qu'une fonction a le droit de modifier n'importe quel
élément en mémoire. Nous devons donc préciser ce qu'une fonction est en droit
de modifier. Par exemple pour notre fonction \CodeInline{swap}, nous pouvons
spécifier que seules les valeurs pointées par les pointeurs reçus peuvent être
modifiées :



\CodeBlockInput[3][14]{c}{swap-2.c}



Si nous rejouons la preuve avec cette spécification, la fonction et les
assertions que nous avions demandées dans le \CodeInline{main} seront validées par WP.



Finalement, il peut arriver que nous voulions spécifier qu'une fonction ne
provoque pas d'effets de bords. Ce cas est précisé en donnant \CodeInline{\textbackslash{}nothing}
à \CodeInline{assigns} :



\CodeBlockInput{c}{assigns-nothing.c}



Le lecteur pourra maintenant reprendre les exemples précédents pour y intégrer
la bonne clause \CodeInline{assigns}.



\levelFourTitle{Séparation des zones de la mémoire}


Les pointeurs apportent le risque d'\textit{aliasing} (plusieurs pointeurs ayant accès à
la même zone de mémoire). Si dans certaines fonctions, cela ne pose pas de
problème (par exemple si nous passons deux pointeurs égaux
à notre fonction \CodeInline{swap}, la spécification est toujours vérifiée par le
code source), dans d'autre cas, ce n'est pas si simple :



\CodeBlockInput{c}{incr_a_by_b-0.c}



Si nous demandons à WP de prouver cette fonction, nous obtenons le
résultat suivant :



\image{2-incr_a_by_b-1}


La raison est simplement que rien ne garantit que le pointeur \CodeInline{a} est bien
différent du pointeur \CodeInline{b}. Or, si les pointeurs sont égaux,



\begin{itemize}
\item la propriété \CodeInline{*a == \textbackslash{}old(*a) + *b} signifie en fait
\CodeInline{*a == \textbackslash{}old(*a) + *a}, ce qui ne peut être vrai que si l'ancienne valeur
pointée par \CodeInline{a} était 0, ce qu'on ne sait pas,
\item la propriété \CodeInline{*b == \textbackslash{}old(*b)} n'est pas validée car potentiellement,
nous la modifions.
\end{itemize}


\begin{Question}
Pourquoi la clause \CodeInline{assigns} est-elle validée ?

C'est simplement dû au fait, qu'il n'y a bien que la zone mémoire pointée par
\CodeInline{a} qui est modifiée étant donné que si \CodeInline{a != b} nous ne modifions bien
que cette zone et que si \CodeInline{a == b}, il n'y a toujours que cette zone, et
pas une autre.
\end{Question}


Pour assurer que les pointeurs sont bien sur des zones séparées de mémoire,
ACSL nous offre le prédicat \CodeInline{\textbackslash{}separated(p1, ..., pn)} qui reçoit en entrée
un certain nombre de pointeurs et qui nous assurera qu'ils sont deux à deux
disjoints. Ici, nous spécifierions :



\CodeBlockInput{c}{incr_a_by_b-1.c}



Et cette fois, la preuve est effectuée :



\image{2-incr_a_by_b-2}


Nous pouvons noter que nous ne nous intéressons pas ici à la preuve de
l'absence d'erreur à l'exécution, car ce n'est pas l'objet de cette section.
Cependant, si cette fonction faisait partie d'un programme complet à vérifier,
il faudrait définir le contexte dans lequel on souhaite l'utiliser et définir
les préconditions qui nous garantissent l'absence de débordement en conséquence.


\levelFourTitle{Écrire le bon contrat}


Trouver les bonnes préconditions à une fonction est parfois difficile. Il est
intéressant de noter qu'une bonne manière de vérifier qu'une spécification est
suffisamment précise est d'écrire des tests pour voir si le contrat nous permet,
depuis un code appelant, de déduire des propriétés intéressantes. En fait,
c'est exactement ce que nous avons fait pour nos exemples \CodeInline{max} et
\CodeInline{swap}. Nous avons écrit une première version de notre spécification
et du code appelant qui nous a servi à déterminer si nous pouvions prouver des
propriétés que nous estimions devoir être capables de prouver à l'aide du
contrat.



Le plus important est avant tout de déterminer le contrat sans prendre en compte
le contenu de la fonction (au moins dans un premier temps). En effet, nous
essayons de prouver une fonction, mais elle pourrait contenir un bug, donc si
nous suivons de trop près le code de la fonction, nous risquons d'introduire
dans la spécification le même bug présent dans le code, par exemple en prenant
en compte une condition erronée. C'est pour cela que
l'on souhaitera généralement que la personne qui développe le programme et la
personne qui le spécifie formellement soient différentes (même si elles ont pu
préalablement s'accorder sur une spécification textuelle par exemple).



Une fois que le contrat est posé, alors seulement, nous nous intéressons aux
spécifications dues au fait que nous sommes soumis aux contraintes de notre langage
et notre matériel. Cela concerne principalement nos préconditions. Par exemple,
la fonction valeur absolue n'a, au fond, pas  vraiment de précondition à respecter,
c'est la machine cible qui détermine qu'une condition supplémentaire doit être
respectée en raison du complément à deux. Comme nous le verrons dans le chapitre
\ref{l1:proof-methodologies}, vérifier l'absence de runtime errors peut aussi
impacter nos postconditions, pour l'instant laissons cela de côté.


\levelThreeTitle{Exercices}


\levelFourTitle{Division et reste}


Spécifier la postcondition de la fonction suivante, qui calcule le résultat
de la division de \CodeInline{a} par \CodeInline{b} et le reste de cette
division et écrit ces deux valeurs à deux positions mémoire \CodeInline{p}
et \CodeInline{q} :


\CodeBlockInput{c}{ex-1-div-rem.c}


Lancer la commande :


\begin{CodeBlock}{bash}
frama-c-gui your-file.c -wp
\end{CodeBlock}


Une fois que la fonction est prouvée, lancer :

\begin{CodeBlock}{bash}
frama-c-gui your-file.c -wp -wp-rte
\end{CodeBlock}


Si cela échoue, compléter le contrat en ajoutant la bonne précondition.


\levelFourTitle{Remettre à zéro selon une condition}


Donner un contrat à la fonction suivante qui remet à zéro la valeur
pointée par le premier paramètre si et seulement si celle pointée par le
second est vraie. Ne pas oublier d'exprimer que la valeur pointée par le second
paramètre doit rester la même :


\CodeBlockInput{c}{ex-2-reset-on-cond.c}


Lancer la commande :


\begin{CodeBlock}{bash}
frama-c-gui your-file.c -wp -wp-rte
\end{CodeBlock}


\levelFourTitle{Addition de valeurs pointées}


La fonction suivante reçoit deux pointeurs en entrée et retourne la
somme des valeurs pointées. Écrire le contrat de cette fonction :


\CodeBlockInput{c}{ex-3-add-ptr.c}


Lancer la commande :


\begin{CodeBlock}{bash}
frama-c-gui your-file.c -wp -wp-rte
\end{CodeBlock}


Une fois que la fonction et son code appelant sont prouvées, modifier
la signature de la fonction comme suit :


\begin{CodeBlock}{c}
void add(int* a, int* b, int* r);
\end{CodeBlock}


Le résultat doit maintenant être stocké à la position mémoire \CodeInline{r}.
Modifier l'appel dans la fonction \CodeInline{main} et le code de la fonction de
façon à implémenter ce comportement. Modifier le contrat de la fonction
\CodeInline{add} et recommencer la preuve. \CodeInline{*a} et \CodeInline{*b}
devraient rester inchangés.


\levelFourTitle{Maximum de valeurs pointées}


Le code suivant calcule le maximum des valeurs pointées par \CodeInline{a}
et \CodeInline{b}. Écrire le contrat de cette fonction :


\CodeBlockInput{c}{ex-4-max-ptr.c}


Lancer la commande :


\begin{CodeBlock}{bash}
frama-c-gui your-file.c -wp -wp-rte
\end{CodeBlock}


Une fois que la fonction est prouvée, modifier la signature de la
fonction comme suit :


\begin{CodeBlock}{c}
void max_ptr(int* a, int* b);
\end{CodeBlock}


La fonction doit maintenant s'assurer qu'après l'exécution, \CodeInline{*a}
contient le maximum des valeurs pointées et \CodeInline{*b} contient l'autre
valeur. Modifier le code de façon à assurer cela ainsi que le contrat.
Notons que la variable  \CodeInline{x} n'est plus nécessaire dans la fonction
\CodeInline{main} et que nous pouvons changer l'assertion en ligne 15 pour
mettre en lumière le nouveau comportement de la fonction.


\levelFourTitle{Ordonner trois valeurs}


La fonction suivante doit ordonner trois valeurs reçues en entrée dans
l'ordre croissant. Écrire le code correspondant et la spécification de la
fonction :

\CodeBlockInput{c}{ex-5-order-3.c}


Et lancer la commande :


\begin{CodeBlock}{bash}
frama-c-gui your-file.c -wp -wp-rte
\end{CodeBlock}


Il faut bien garder en tête qu'ordonner des valeurs ne consiste pas seulement
à s'assurer qu'elles sont dans l'ordre croissant et que chaque valeur doit
être l'une de celles d'origine. Toutes les valeurs d'origine doivent être
présente et en même quantité. Pour exprimer cette idée, nous pouvons nous
reposer à nouveau sur les ensembles. La propriété suivante est vraie par
exemple :


\begin{CodeBlock}{c}
//@ assert { 1, 2, 3 } == { 2, 3, 1 };
\end{CodeBlock}


Nous pouvons l'utiliser pour exprimer que l'ensemble des valeurs d'entrée et
de sortie est le même. Cependant, ce n'est pas la seule chose à prendre en
compte car un ensemble ne contient qu'une occurrence de chaque valeur. Donc,
si \CodeInline{*a == *b == 1}, alors \CodeInline{\{ *a, *b \} == \{ 1 \}}.
Par conséquent nous devons considérer trois autres cas particuliers:


\begin{itemize}
\item toutes les valeurs d'origine sont les mêmes ;
\item deux valeurs d'origine sont les mêmes, la dernière est plus grande ;
\item deux valeurs d'origine sont les mêmes, la dernière est plus petite.
\end{itemize}


Qui nous permet d'ajouter la bonne contrainte aux valeurs de sortie.


Pour la réalisation de la spécification, le programme de test suivant peut
nous aider :


\CodeBlockInput[27][43]{c}{ex-5-order-3-answer.c}


Si la spécification est suffisamment précise, chaque assertion devrait être
prouvée. Cependant, cela ne signifie pas que tous les cas ont été considérés,
il ne faut pas hésiter à ajouter d'autres tests.



\levelTwoTitle{Behaviors}

Il peut arriver qu'une fonction ait divers comportements potentiellement très
différents en fonction de l'entrée. Un cas typique est la réception d'un
pointeur vers une ressource optionnelle : si le pointeur est \CodeInline{NULL}, nous
aurons un certain comportement et un comportement complètement différent s'il ne
l'est pas.



Nous avons déjà vu une fonction qui avait des comportements différents, la
fonction \CodeInline{abs}. Nous reprendrons comme exemple. Les deux
comportements que nous pouvons isoler sont le cas où la valeur est positive et
le cas où la valeur est négative.



Les comportements nous servent à spécifier les différents cas pour les
postconditions. Nous les introduisons avec le mot-clé \CodeInline{behavior}.
Chaque comportement a un nom. Pour un comportement donné, nous trouvons
différentes hypothèses à propos de l'entrée de la fonction, elles sont
introduites à l'aide du mot clé \CodeInline{assumes} (notons que, puisqu'elles
caractérisent les entrées, le mot clé \CodeInline{\textbackslash{}old} ne peut
pas être utilisé ici). Cependant, chaque propriété exprimée par ces clauses
n'a pas \textbf{besoin} d'être vérifiée avant à l'appel, elle \textbf{peut}
être vérifiée et dans ce cas, les postconditions associées à ce comportement
s'appliquent. Ces postconditions sont à nouveau introduites à l'aide du mot
clé \texttt{ensures}. Finalement, nous pouvons également demander à WP
de vérifier le fait que les comportements sont disjoints (pour garantir
le déterminisme) et complets (pour garantir que nous couvrons toutes les
entrées possibles).



Les comportements sont disjoints si pour toute entrée de la fonction, elle ne
correspond aux hypothèses (\textit{assumes}) que d'un seul comportement. Les
comportements sont complets si les hypothèses recouvrent bien tout le domaine
des entrées.



Par exemple pour \CodeInline{abs} :



\CodeBlockInput{c}{abs.c}


Notons qu'introduire des comportements ne nous interdit pas de spécifier une
postcondition globale. Par exemple ici, nous avons spécifié que quel que soit
le comportement, la fonction doit retourner une valeur positive.


Pour comprendre ce que font précisément \CodeInline{complete} et \CodeInline{disjoint}, il est utile
d'expérimenter deux possibilités :



\begin{itemize}
\item remplacer l'hypothèse de « pos » par \CodeInline{val > 0} auquel cas les
comportements seront disjoints mais incomplets (il nous manquera le cas
\CodeInline{val == 0}) ;
\item remplacer l'hypothèse de « neg » par \CodeInline{val <= 0} auquel cas les
comportements seront complets mais non disjoints (le cas \CodeInline{val == 0}) sera
présent dans les deux comportements.
\end{itemize}


\begin{Warning}
Même si \CodeInline{assigns} est une postcondition, à ma connaissance, il n'est pas
possible de mettre des \CodeInline{assigns} pour chaque \textit{behavior}. Si nous avons
besoin d'un tel cas, nous spécifions :

\begin{itemize}
\item \CodeInline{assigns} avant les \textit{behavior} (comme dans notre exemple) avec tout
élément non-local susceptible d'être modifié,
\item en postcondition de chaque \textit{behavior} les éléments qui ne sont finalement
pas modifiés en les indiquant égaux à leur ancienne (\CodeInline{\textbackslash{}old}) valeur.
\end{itemize}
\end{Warning}


Les comportements sont très utiles pour simplifier l'écriture de spécifications
quand les fonctions ont des effets très différents en fonction de leurs
entrées. Sans eux, les spécifications passent systématiquement par des
implications traduisant la même idée mais dont l'écriture et la lecture sont
plus difficiles (nous sommes susceptibles d'introduire des erreurs).
D'autre part, la traduction de la complétude et de la disjonction devraient
être écrites manuellement, ce qui serait fastidieux et une nouvelle fois source
d'erreurs.



\levelThreeTitle{Exercices}


\levelFourTitle{Exercices précédents}


Dans les sections précédentes, reprendre les exemples :


\begin{itemize}
\item à propos du calcul de la distance entre deux entiers ;
\item « Remettre à zéro selon une condition » ;
\item « Jours du mois » ;
\item « Maximum des valeurs pointées ».
\end{itemize}


En considérant que les contrats étaient :


\CodeBlockInput{c}{ex-1-past.c}


Les réécrire en utilisant des comportements.


\levelFourTitle{Deux autres exercices simples}


Produire le code et la spécification des deux fonctions suivantes puis
les prouver. La spécification devrait faire usage des comportements.


Tout d'abord une fonction qui retourne si un caractère est une voyelle
ou une consonne, supposer (et exprimer) que la fonction reçoit une
lettre minuscule.


\CodeBlockInput[1][5]{c}{ex-2-simple.c}


Puis une fonction qui renvoie à quel quadrant d'un repère appartient
une coordonnée. Lorsque la coordonnée se trouve sur un axe, choisir
arbitrairement l'un des quadrants qu'elle touche.


\CodeBlockInput[7][9]{c}{ex-2-simple.c}


\levelFourTitle{Triangle}


Compléter les fonctions suivantes qui reçoivent la longueur des différents
côté retournent respectivement :


\begin{itemize}
\item si le triangle est scalène, isocèle, ou équilatéral ;
\item si le triangle est rectangle, acutangle ou obtusangle.
\end{itemize}


\CodeBlockInput{c}{ex-3-triangle.c}


En supposant (et exprimant) que :


\begin{itemize}
\item les valeurs reçues forment bien un triangle,
\item \CodeInline{a} est l'hypoténuse du triangle,
\end{itemize}


spécifier et prouver que les fonctions font la tâche prévue.


\levelFourTitle{Maximum des valeurs pointées}


Reprendre l'exemple « Maximum des valeurs pointées » de la section
précédente et plus précisément la version qui retourne la plus grande
valeur. En considérant que le contrat était :


\CodeBlockInput[1][10]{c}{ex-4-max-ptr.c}


\begin{enumerate}
\item Le réécrire en utilisant des comportements
\item Modifier le contrat de 1. de sorte que les comportements ne
  soient pas disjoints. Excepté cette propriété, tout le reste devrait
  être correctement prouvé
\item Modifier le contrat de 1. de sorte que les comportements ne
  soient pas complets, puis ajouter un nouveau comportement pour le
  rendre de nouveau complet
\item Modifier la fonction de 1. de façon à accepter la valeur
  \CodeInline{NULL} pour les pointeurs d'entrées, si les deux pointeurs
  sont nuls, retourner  \CodeInline{INT\_MIN}, si l'un seulement est
  nul, retourner l'autre valeur, sinon retourner le maximum des deux
  valeurs. Modifier le contrat de façon à prendre en compte tout cela
  par de nouveaux comportements. Prendre soin d'assurer que les comportements
  sont complets et disjoints.
\end{enumerate}



\levelFourTitle{Ordonner trois valeurs}


Reprendre l'exemple « Ordonner trois valeurs » de la section
précédente, en considérant que le contrat était :


\CodeBlockInput{c}{ex-5-order-3.c}


Le réécrire en utilisant des comportements. Notons que le contrat
devrait être composé d'un comportement général et de trois
comportements spécifiques. Est-ce que ces comportements sont complets ?
Sont-ils disjoints ?



\levelTwoTitle{WP Modularity}

Pour terminer cette partie nous allons parler de la composition des appels de
fonctions et commencer à entrer dans les détails de fonctionnement de WP. Nous
allons en profiter pour regarder comment se traduit le découpage de nos 
programmes en fichiers lorsque nous voulons les spécifier et les prouver avec WP.



Notre but sera de prouver la fonction \CodeInline{max\_abs} qui renvoie les maximums 
entre les valeurs absolues de deux valeurs :



\CodeBlockInput[6][11]{c}{max_abs.c}



Commençons par (sur-)découper les déclarations et définitions des différentes
fonctions dont nous avons besoin (et que nous avons déjà propuvé) en couples 
headers/source, à savoir \CodeInline{abs} et \CodeInline{max}. Cela donne pour
\CodeInline{abs} :



Fichier abs.h :

\CodeBlockInput{c}{abs.h}


Fichier abs.c :

\CodeBlockInput{c}{abs.c}



Nous découpons en mettant le contrat de la fonction dans le header. Le but de
ceci est de pouvoir, lorsque nous aurons besoin de la fonction dans un autre 
fichier, importer la spécification en même temps que la déclaration de 
celle-ci. En effet, WP en aura besoin pour montrer que les appels à cette 
fonction sont valides. D'abord pour prouver que la précondition est respectée
(et donc que l'appel est légal) et ensuite pour savoir ce qu'il peut apprendre
en retour (à savoir la postcondition) afin de pouvoir l'utiliser pour prouver
la fonction appelante.



Nous pouvons créer un fichier sous le même formatage pour la fonction \CodeInline{max}.
Dans les deux cas, nous pouvons ré-ouvrir le fichier source (pas besoin de 
spécifier les fichiers headers dans la ligne de commande) avec Frama-C et 
remarquer que la spécification est bien associée à la fonction et que nous
pouvons la prouver.



Maintenant, nous pouvons préparer le terrain pour la fonction \CodeInline{max\_abs}. 
Dans notre header :



\CodeBlockInput{c}{max_abs_uns.h}



Et dans le source :



\CodeBlockInput{c}{max_abs.c}



Et ouvrir ce dernier fichier dans Frama-C. Si nous regardons le panneau latéral, 
nous pouvons voir que les fichiers header que nous avons inclus dans le fichier 
\CodeInline{abs\_max.c} y apparaissent et que les contrats de fonction sont décorés avec des 
pastilles particulières (vertes et bleues) :



\image{max_abs}


Ces pastilles nous disent qu'en l'absence d'implémentation, les propriétés sont
supposées vraies. Et c'est une des forces de la preuve déductive de programmes 
par rapport à certaines autres méthodes formelles, les fonctions sont vérifiées
en isolation les unes des autres.



En dehors de la fonction, sa spécification est considérée comme étant 
vérifiée : nous ne cherchons pas à reprouver que la fonction fait bien son travail
à chaque appel, nous nous contenterons de vérifier que les pré-conditions sont 
réunies au moment de l'appel. Cela donne donc des preuves très modulaires et donc 
des spécifications plus facilement réutilisables. Évidemment, si notre preuve 
repose sur la spécification d'une autre fonction, cette fonction doit-elle même 
être vérifiable pour que la preuve soit formellement complète. Mais nous pouvons
également vouloir simplement faire confiance à une bibliothèque externe sans la
prouver.



Finalement, le lecteur pourra essayer de spécifier la fonction \CodeInline{max\_abs}.



La spécification peut ressembler à ceci:



\CodeBlockInput[4][14]{c}{max_abs.h}



\levelThreeTitle{Exercices}



\levelFourTitle{Jours du mois}
\label{l4:contract-modularity-ex-days-of-month}


Spécifier la fonction année bissextile qui retourne vrai si l'année reçue
en entrée est bissextile. Utiliser cette fonction pour compéter la fonction
jours du mois de façon à retourner le nombre de jour du mois reçu en entrée,
incluant le bon comportement lorsque le mois en question est février et que
l'année est bissextile.


\CodeBlockInput{c}{ex-1-days-month.c}


\levelFourTitle{Ordonner trois valeurs}


Reprendre la fonction \CodeInline{max\_ptr} dans sa version qui « ordonne »
les deux valeurs. Écrire une fonction \CodeInline{min\_ptr} qui utilise la
fonction précédente pour effectuer l'opération inverse. Utiliser ces fonctions
pour compléter les quatre fonctions qui ordonnent trois valeurs. Pour chaque
variante (ordre croissant et décroissant), l'écrire une première fois en
utilisant uniquement \CodeInline{max\_ptr} et une seconde en utilisant
\CodeInline{min\_ptr}. Écrire un contrat précis pour chacune de ces fonctions
et les prouver.


\CodeBlockInput{c}{ex-3-order-3.c}




\horizontalLine



During this part of the tutorial, we have studied how we can specify
functions using contracts, composed of a pre and a post-condition, as
well as some features ACSL provides to express those properties. We have
also seen why it is important to be precise when we specify and how the
introduction of behaviors can help us to write more understandable and
concise specification.

However, we do not have studied one important point: the specification
of loops. Before that, we should have a closer look to the way WP works.

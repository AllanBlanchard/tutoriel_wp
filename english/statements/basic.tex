
\levelThreeTitle{Assignment}


Assignment is the most basic operation one can have in an imperative
language (leaving aside the ``do nothing'' operation that is not particularly
interesting). The weakest precondition calculus associates the following
$$wp(x = E , Post) := Post[x \leftarrow E]$$


Here the notation $P[x \leftarrow E]$ means ``the property $P$ where
$x$ is replaced by $E$''. In our case this corresponds to ``the
postcondition $Post$ where $x$ is replaced by $E$''. The idea is
that the postcondition of an assignment of $E$ to $x$ can only be
true if replacing all occurrences of $x$ in the formula by $E$ leads
to a property that is true. For example:



\begin{CodeBlock}{c}
// { P }
x = 43 * c ;
// { x = 258 }
\end{CodeBlock}


$$P = wp(x = 43*c , \{x = 258\}) = \{43*c = 258\}$$


The function $wp$ allows us to compute, as weakest precondition of
the assignment provided our expected postcondition, the formula
$\{43*c = 258\}$, thus obtaining the following Hoare triple:


\begin{CodeBlock}{c}
// { 43*c = 258 }
x = 43 * c ;
// { x = 258 }
\end{CodeBlock}


In order to compute the precondition of the assignment we have replaced
each occurrence of $x$ in the postcondition by the assigned value
$E = 43*c$. If our program were of the form:
\begin{CodeBlock}{c}
int c = 6 ;
// { 43*c = 258 }
x = 43 * c ;
// { x = 258 }
\end{CodeBlock}
we could submit the formula $43*6 = 258$ to our automatic prover
in order to determine whether it is really valid. The answer would of
course be ``yes'' because the property is easy to verify. If we had,
however, given the value 7 to the variable \CodeInline{c} the prover's reply
would be ``no'' since the formula $43*7 = 258$ is not true.



Taking into account the weakest precondition calculus, we can now write
the inference rule for the Hoare triple of an assignment as
$$\dfrac{}{\{Q[x \leftarrow E] \}\quad x = E \quad\{ Q \}}$$


We note that there is no premise to verify. Does this mean that the
triple is necessarily true? Yes. However, it does not mean that the
precondition is satisfied by the program to which the assignment belongs
or that the precondition is at all possible. Here the automatic provers
come into play.



For example, we can ask Frama-C to verify the following line:
\begin{CodeBlock}{c}
int a = 42;
//@ assert a == 42;
\end{CodeBlock}
which is, of course, directly proven by Qed, since it is a simple
application of the assignment rule.



\begin{Information}
  We remark that according to the C standard, an assignment is in fact an
  expression. This allows us, for example, to write
  \CodeInline{if( (a = foo()) == 42)}.
  In Frama-C, an assignment will always be treated as a statement. Indeed,
  if an assignment occurs within a larger expression, then the Frama-C
  preprocessor, while building the abstract syntax tree, systematically
  performs a \emph{normalization step} that produces a separate assignment
  statement.
\end{Information}



\levelFourTitle{Assignment of pointed value}



In C, thanks to (because of?) pointers, we can have programs with aliases,
meaning that two pointers can point to the same memory location. Our weakest
precondition calculus should consider these cases. For example, let us consider
this simple Hoare triple:


\begin{CodeBlock}{c}
//@ assert p = q ;
*p = 1 ;
//@ assert *p + *q == 2 ;
\end{CodeBlock}



This Hoare triple is correct, since \CodeInline{p} and \CodeInline{q} are in
alias, modifying \CodeInline{*p} also modifies \CodeInline{*q}, thus both these
expressions evaluate to $1$ and the postcondition is true. However, let us apply
the weakest precondition calculus from the postcondition:



\begin{tabular}{ll}
$wp(*p = 1, *p + *q = 2)$ & $= (*p + *q = 2)[*p \leftarrow 1]$\\
                          & $= (1 + *q = 2)$
\end{tabular}



We get the weakest precondition: \CodeInline{1 + *q == 2}, and thus we could
deduce that the weakest precondition is \CodeInline{*q == 1}, which is true, but
does not allow us to conclude that the program is correct, since in our formula
we do not have anything that models that \CodeInline{p == q ==> *q == 1}. In
fact, here, we would like to be able to compute the weakest precondition like:



\begin{tabular}{ll}
$wp(*p = 1, *p + *q = 2)$ & $= (1 + *q = 2 \vee q = p)$\\
                          & $= (*q = 1 \vee q = p)$
\end{tabular}



For this, we have to take care of aliasing. A common way to do this is to
consider that the memory is one particular variable (let us name this variable
$M$) on which we can perform two operations: get the element at a particular
location $l$ in memory (which returns an expression) and set the element at a
particular location $l$ to a new value $v$ (which returns the new memory).


We denote:


\begin{itemize}
\item $get(M,l)$ with the notation $M[l]$
\item $set(M,l,v)$ with the notation $M[l \mapsto v]$
\end{itemize}


And basically, the get operation can be seen as follows:


\begin{tabular}{ll}
  $M[l1 \mapsto v][l2] =$ & if $l1   =  l2$ then $v$ \\
                          & if $l1 \neq l2$ then $M[l2]$
\end{tabular}


If there is no value associated to the location we use for a get, the value is
undefined (thus, the memory is partial function). Of course, at the beginning of
a function, the memory context can be populated with the memory locations for
which a value is known to be defined.


Now, we can change a bit the weakest precondition calculus for assignment
of pointed memory location. For this, we consider that we have an implicit
variable $M$ that models the memory, and we define the assignment of a memory
location as an update of the memory such that now the corresponding pointer points
to the written expression.
$$wp(*x = E, Q) := Q[M \leftarrow M[x \mapsto E]]$$


And evaluating a pointed value $*x$ in a formula now requires us to use the
get operator to ask the right value. Thus, we can for example compute the weakest
precondition of our previous program:

\begin{tabular}{lll}
  $wp(*p = 1, *p + *q = 2)$
  & $= (*p + *q = 2)[M \leftarrow M[p \mapsto 1]]$ & (1)\\
  & $= (M[p] + M[q] = 2)[M \leftarrow M[p \mapsto 1]]$ & (2)\\
  & $= (M[p \mapsto 1][p] + M[p \mapsto 1][q] = 2)$ & (3)\\
  & $= (1 + M[p \mapsto 1][q] = 2)$ & (4)\\
  & $= (1 + (\texttt{if}\ q = p\ \texttt{then}\ 1\ \texttt{else}\ M[q]) = 2)$ & (5)\\
  & $= (\texttt{if}\ q = p\ \texttt{then}\ 1+1 = 2\ \texttt{else}\ 1+M[q] = 2)$ & (6)\\
  & $= (q = p \vee M[q] = 1)$ & (7)
\end{tabular}
\begin{enumerate}
\item we have to apply the rule of assignment for pointers, but for this we need
  to introduce $M$,
\item we replace pointer accesses in the formula by a call to $get$ on $M$,
\item we apply the replacement asked by the assignment rule,
\item we use the definition of the $get$ operator for the expression about $p$
  ($M[p \mapsto 1][p] = 1$)
\item we use the definition of the $get$ operator for the expression about $q$\\
  ($M[p \mapsto 1][q] = \texttt{if}\ q = p\ \texttt{then}\ 1\ \texttt{else}\ M[q]$)
\item we perform some simplification to the formula ...
\item ... and finally conclude that either $M[q] = 1$ or $p = q$.
\end{enumerate}


Then in our program, since we know that $p = q$, we can conclude that the program
is correct.


The WP plugin does not exactly work like this. In particular, it depends on the
memory model chosen for the proof that will make different assumption about the
memory is organized. For the memory model we use, the typed memory model, in
fact WP creates multiple variables for memory. However, let us have a look at
the verification condition generated for the postcondition of the swap function:


\image{memory-model}


We can see, in the beginning of the verification condition, that a variable
\CodeInline{Mint\_0} representing a memory of values of integer types have been
created, and that this memory is updated and accessed using the operators we
previously introduced (see the definition of the variable \CodeInline{x\_2}).


\levelThreeTitle{Composition of statements}

For a statement to be valid, its precondition must allow us by means of
executing the said statement to reach the desired postcondition. Now we
would like to execute several statements one after another. Here the
idea is that the postcondition of the first statement is compatible with
the required precondition of the second statement and so on for the
third statement.



The inference rule that corresponds to this idea utilizes the following
Hoare triples:
$$\dfrac{\{P\}\quad S1 \quad \{R\} \ \ \ \{R\}\quad S2 \quad \{Q\}}{\{P\}\quad S1 ;\ S2 \quad \{Q\}}$$
In order to verify the composed statement $S1;\ S2$ we rely on an
intermediate property $R$ that is at the same time the postcondition
of $S1$ and the precondition of $S2$. (Please note that $S1$ and
$S2$ are not necessarily simple statements; they themselves can be
composed statements.) The problem is, however, that nothing indicates us
how to determine the properties $P$ and $R$.

The weakest-precondition calculus now says us that the intermediate
property $R$ can be computed as the weakest precondition of the second
statement. The property $P$, on the other hand, then is computed as
the weakest precondition of the first statement. In other words, the
weakest precondition of the composed statement $S1; S2$ is determined
as follows:
$$wp(S1;\ S2 , Post) := wp(S1, wp(S2, Post) )$$


The WP plugin of Frama-C performs all these computations for us. Thus,
we do not have to write the intermediate properties as ACSL assertions
between the lines of codes.



\begin{CodeBlock}{c}
int main(){
  int a = 42;
  int b = 37;

  int c = a+b; // i:1
  a -= c;      // i:2
  b += a;      // i:3

  //@assert b == 0 && c == 79;
}
\end{CodeBlock}



\levelFourTitle{Proof tree}


When we have more than two statements, we can consider the last
statement as second statement of our rule and all the preceding ones as
first statement. This way we traverse step by step backwards the
statements in our reasoning. With the previous program this looks like:


\begin{center}
\begin{tabular}{ccc}
  $\{P\}\quad i_1 ; \quad \{Q_{-2}\}$ & $\{Q_{-2}\}\quad i_2 ; \quad \{Q_{-1}\}$ & \\
  \cline{1-2}
  \multicolumn{2}{c}{$\{P\}\quad i\_1 ; \quad i\_2 ; \quad \{Q_{-1}\}$} & $\{Q_{-1}\} \quad i_3 ; \quad \{Q\}$\\
  \hline
  \multicolumn{3}{c}{$\{P\}\quad i\_1 ; \quad i\_2 ; \quad i\_3; \quad \{ Q \}$}
\end{tabular}
\end{center}

The weakest-precondition calculus allows us to construct the property
$Q_{-1}$ starting from the property $Q$ and statement $i_3$ which
in turn enables us to derive the property $Q_{-2}$ from the property
$Q_{-1}$ and statement $i_2$. Finally, $P$ can be determined from
$Q_{-2}$ and $i_1$.



Now that we can verify programs that consist of several statements it
is time to add some structure to them.



\levelThreeTitle{Conditional rule}


For a conditional statement to be true, one must be able to reach the
postcondition through both branches. Of course, for both branches, the
same precondition (of the conditional statement) must hold. In addition,
we have that in the if-branch the condition is true while in the
else-branch it is false.

We therefore have, as in the case of composed statements, two facts to
verify (in order to avoid confusion we are using here the syntax
$if\ B\ then\ S1\ else\ S2$):
$$\dfrac{\{P \wedge B\}\quad S1\quad \{Q\} \quad \quad \{P \wedge \neg B\}\quad S2\quad \{Q\}}{\{P\}\quad if\quad B\quad then\quad S1\quad else\quad S2 \quad \{Q\}}$$

Our two premises are therefore that we can both in the if-branch and the
else-branch reach the postcondition from the precondition.

The result of the weakest-precondition calculus for a conditional
statement reads as follows:
$$wp(if\ B\ then\ S1\ else\ S2 , Post) := (B \Rightarrow wp(S1, Post)) \wedge (\neg B \Rightarrow wp(S2, Post))$$
This means that the condition $B$ has to imply the weakest
precondition of $S1$ in order to safely arrive at the postcondition.
Analogously, the negation of $B$ must imply the weakest precondition
of $S2$.



\levelFourTitle{Empty \CodeInline{else}-branch}


Following this definition, we obtain for the case of an empty else-branch the
following rule by simply replacing the statement $S2$ by the empty
statement \CodeInline{skip}.
$$\dfrac{\{P \wedge B\}\quad S1\quad \{Q\} \quad \quad \{P \wedge \neg B\}\quad skip\quad \{Q\}}{\{P\}\quad if\quad B\quad then\quad S1\quad else\quad skip \quad \{Q\}}$$
The triple for \CodeInline{else} is:
$$\{P \wedge \neg B\}\quad skip\quad \{Q\}$$
which means that we need to ensure:
$$P \wedge \neg B \Rightarrow Q$$

In short, if the condition $B$ of \CodeInline{if} is false, this means
that the postcondition of the complete conditional statement is already
established before entering the else-branch (since it does not do
anything).



As an example, we consider the following code snippet where we reset a
variable $c$ to a default value in case it had not been properly
initialized by the user.



\begin{CodeBlock}{c}
int c;

// ... some code ...

if(c < 0 || c > 15){
  c = 0;
}
//@ assert 0 <= c <= 15;
\end{CodeBlock}



Let



$wp(if \neg (c \in [0;15])\ then\ c := 0, \{c \in [0;15]\})$



$:= (\neg (c \in [0;15])\Rightarrow wp(c := 0, \{c \in [0;15]\})) \wedge (c \in [0;15]\Rightarrow wp(skip, \{c \in [0;15]\}))$



$= (\neg (c \in [0;15]) \Rightarrow 0 \in [0;15]) \wedge (c \in [0;15] \Rightarrow c \in [0;15])$



$= (\neg (c \in [0;15]) \Rightarrow true) \wedge true$



The property can be verified: independent of the evaluation of
$\neg (c \in [0;15])$, the implication will hold.



\levelThreeTitle{Bonus Stage - Consequence rule}
\label{l3:statements-basic-consequence}


It can sometimes be useful to strengthen a postcondition or to weaken a
precondition. The former will often be established by us to
facilitate the work of the prover, the latter is more often verified by
the tool as the result of computing the weakest precondition.



The inference rule of Hoare logic is the following:
$$\dfrac{P \Rightarrow WP \quad \{WP\}\quad c\quad \{SQ\} \quad SQ \Rightarrow Q}{\{P\}\quad c \quad \{Q\}}$$

(We remark that the premises here are not only Hoare triples but also
formulas to verify.)

For example, if our postcondition is too complex, it may generate a
weaker precondition that is, however, too complicated, thus making the
work of provers more difficult. We can then create a simpler
intermediate postcondition $SQ$, that is, however, stricter and
implies the real postcondition. This is the part $SQ \Rightarrow Q$.

Conversely, the calculation of the precondition will usually generate a
complicated and often weaker formula than the precondition we want to
accept as input. In this case, it is our tool that will check the
implication between what we want and what is necessary for our code to
be valid. This is the part $P \Rightarrow WP$.

We can illustrate this with the following code. Note that here the code
could be proved by WP without the weakening and strengthening of
properties because the code is very simple, it is just to illustrate the
rule of consequence.



\begin{CodeBlock}{c}
/*@
  requires P: 2 <= a <= 8;
  ensures  Q: 0 <= \result <= 100 ;
  assigns  \nothing ;
*/
int constrained_times_10(int a){
  //@ assert P_imply_WP: 2 <= a <= 8 ==> 1 <= a <= 9 ;
  //@ assert WP:         1 <= a <= 9 ;

  int res = a * 10;

  //@ assert SQ:         10 <= res <= 90 ;
  //@ assert SQ_imply_Q: 10 <= res <= 90 ==> 0 <= res <= 100 ;

  return res;
}
\end{CodeBlock}



(Note: We have omitted here the control of integer overflow.)



Here we want to have a result between 0 and 100. But we know that the
code will not produce a result outside the bounds of 10 and 90. So we
strengthen the postcondition with an assertion that at the end
\CodeInline{res}, the result, is between 0 and 90. The calculation of the
weakest precondition of this property together with the assignment
\CodeInline{res = 10 * a} yields a weaker precondition
\CodeInline{1 <= a <= 9}, and we know that
\CodeInline{2 <= a <= 8} gives us the desired
guarantee.



When there are difficulties to carry out a proof on more complex code,
then it is often helpful to write assertions that produce stronger, yet
easier to verify, postconditions. Note that in the previous code, the
lines \CodeInline{P\_imply\_WP} and\CodeInline{SQ\_imply\_Q} are never used
because this is the default reasoning of WP. They are just here for
illustrating the rule.


\levelThreeTitle{Bonus Stage - Constancy rule}
\label{l3:statements-basic-constancy}


Certain sequences of instructions may concern and involve different
variables. Thus, we may initialize and manipulate a certain number of
variables, begin to use some of them for a time, before using other
variables. When this happens, we want our tool to be concerned only with
variables that are susceptible to change in order to obtain the simplest
possible properties.



The rule of inference that defines this reasoning is the following:
$$\dfrac{\{P\}\quad c\quad \{Q\}}{\{P \wedge R\}\quad c\quad \{Q \wedge R\}}$$
where $c$ does not modify any variable in $R$. In other words:
``To check the triple, let's get rid of the parts of the formula that
involve variables that are not influenced by $c$ and prove the new
triple.'' However, we must be careful not to delete too much
information, since this could mean that we are not able to prove our
properties.


As an example, let us consider the following code (here gain, we ignore
potential integer overflows):



\begin{CodeBlock}{c}
/*@
  requires a > -99 ;
  requires b > 100 ;
  ensures  \result > 0 ;
  assigns  \nothing ;
*/
int foo(int a, int b){
  if(a >= 0){
    a++ ;
  } else {
    a += b ;
  }
  return a ;
}
\end{CodeBlock}


If we look at the code of the \CodeInline{if} block, we notice that it does
not use the variable \CodeInline{b}. Thus, we can completely omit the
properties about \CodeInline{b} in order to prove that \CodeInline{a} will be
strictly greater than 0 after the execution of the block:



\begin{CodeBlock}{c}
/*@
  requires a > -99 ;
  requires b > 100 ;
  ensures  \result > 0 ;
  assigns  \nothing ;
*/
int foo(int a, int b){
  if(a >= 0){
    //@ assert a >= 0; // and nothing about b
    a++ ;
  } else {
    a += b ;
  }
  return a ;
}
\end{CodeBlock}



On the other hand, in the \CodeInline{else} block, even if \CodeInline{b} is
not modified, formulating properties only about \CodeInline{a} would render
a proof impossible for humans. The code would be:



\begin{CodeBlock}{c}
/*@
  requires a > -99 ;
  requires b > 100 ;
  ensures  \result > 0 ;
  assigns  \nothing ;
*/
int foo(int a, int b){
  if(a >= 0){
    //@ assert a >= 0; // and nothing about b
    a++ ;
  } else {
    //@ assert a < 0 && a > -99 ; // and nothing about b
    a += b ;
  }
  return a ;
}
\end{CodeBlock}



In the \CodeInline{else} block, knowing that\CodeInline{a} lies between -99 and
0, but knowing nothing about \CodeInline{b}, we could hardly know if the
operation \CodeInline{a += b} produces a result that is greater than 0.

The WP plugin will, of course, prove the function without problems,
since it produces by itself the properties that are necessary for the
proof. In fact, the analysis which variables are necessary or not (and,
consequently, the application of the constancy rule) is conducted
directly by WP.

Let us finally remark that the constancy rule is an instance of the
consequence rule
$$\dfrac{P \wedge R \Rightarrow P \quad \{P\}\quad c\quad \{Q\} \quad Q \Rightarrow Q \wedge R}{\{P \wedge R\}\quad c\quad \{Q \wedge R\}}$$


If the variables of $R$ have not been modified by the operation
(which, on the other hand, may modify the variables of $P$ to produce
$Q$), then the properties $P \wedge R \Rightarrow P$ and
$Q \Rightarrow Q \wedge R$ hold.


\levelThreeTitle{Assertion}


ACSL assertions needs to be proved (or used). Thus, they need their WP calculus
rules. There are three different kinds of assertions in ACSL:
\begin{itemize}
  \item \CodeInline{assert}
  \item \CodeInline{check}
  \item \CodeInline{admit}
\end{itemize}


Each kind of assertion has a different behavior.


Let us start with the one we have used so far: \CodeInline{assert}. An
annotation \CodeInline{assert P} has the following informal semantics: the
property \CodeInline{P} must be verified at this program point, and then the
fact that it is true at this program point is added as a hypothesis when proving
properties that come later. Note that it might not be true, but it is still
added to the proof context anyway. However, if it is not true, first, we cannot
to prove it, and second, the properties that we are proved admitting this
property are marked as ``valid under hypothesis''.


The annotation \CodeInline{check P} means that the property \CodeInline{P} must
be verified at this program point, but then it is not added to the proof context
for the proof of the annotations that follow it.


Finally, the annotation \CodeInline{admit P} means that from this program point,
the property \CodeInline{P} is considered to be true, thus added to the proof
context of the annotations that follow it, even if we do not verify it at this
program point.


Let us illustrate these different kind of annotations with this short example:


\CodeBlockInput{c}{annot-kinds.c}


\image{annot-kinds}


In the \CodeInline{check\_annot} function, neither the \CodeInline{check} nor
the \CodeInline{ensures} clauses are proved. Each one have been tried separately
(and failed).


In the \CodeInline{admit\_annot} function, the \CodeInline{admit} annotation
appear with a blue and green bullet (just like when we have a contract on a
function declaration without a body), meaning that from this point, it is
considered to be valid even it is not proved. Thus, the \CodeInline{ensures}
clause is proved, since \CodeInline{\textbackslash{}false} is assumed to be
true.


Finally, in the \CodeInline{assert\_annot} function, the \CodeInline{assert}
annotation is not proved, but from this point it is assumed to be true. Thus,
the \CodeInline{ensures} clause is proved. However, while in the
\CodeInline{admit\_annot} function it appears entirely valid (green bullet),
here, it appears valid under hypothesis: the validity depends on the validity
of a property that has not been proved yet (and will not).


We can encode these different behaviors with the following WP rules.


The rule for the \CodeInline{check} annotation is the following:
$$ wp(check\ A, Post) := A \wedge Post $$
which means that we add to the properties we must verify another property, which
is the property $A$.


The rule for the \CodeInline{admit} annotation is the following:
$$ wp(admit\ A, Post) := A \Rightarrow Post $$
which means that assuming $A$, we must verify the postcondition (of the
statement, which can be a property that has been computing using other WP
rules).


Finally, the rule for the \CodeInline{assert} combines the previous ones.
Indeed:
\begin{CodeBlock}{c}
  assert P;
\end{CodeBlock}
is equivalent to:
\begin{CodeBlock}{c}
  check P; // first, verify that P is true
  admit P; // now, we can assume that it is
\end{CodeBlock}
Thus:
$$ wp(assert\ A, Post) := wp(check\ A, wp(admit\ A, Post)) \equiv A \wedge (A \Rightarrow Post) $$


This allows to introduce some sort of cut in the proof, saying "OK, let us first
prove $A$ and once it is done, we will prove that when $A$ is true, $Post$ is
too".


One could think the \CodeInline{admit} annotation is dangerous. It is. But it
can be useful when one wants to debug a proof. In particular, in
Chapter~\ref{l1:proof-methodologies}, we will see that \CodeInline{assert}
annotations can be used to guide program proof. During the process,
\CodeInline{admit} can be useful to try annotations without having to
immediately prove them or to consider some annotations as valid to make the
design phase faster. More rarely, it can be useful to explicit some hypotheses
about hardware, or software platform, but for this, it is more common to use
Frama-C kernel options or axioms (we will present axioms in
Section~\ref{l2:acsl-logic-definitions-axiomatic}).


\levelThreeTitle{WP plugin vs. Dijkstra's WP calculus}


One might have notice from the beginning of this section that the $wp$ function
starts from the postcondition and, instruction after instruction, changes the
formula until it reaches the beginning of the function, when it generates the
verification condition (as briefly explained in
Section~\ref{l3:statements-basic-consequence}), that we must verify.


However, one might have also noticed that from the very beginning of this book,
in most examples, we do not have \emph{one} verification condition to verify,
but \emph{several}. On a theoretical aspect, this is not a crucial difference,
but on a practical aspect, it allows generating simpler verification conditions,
thus easier to verify using SMT solvers.


In fact, the WP plugin generates several verification conditions in parallel
during the WP calculus. Each time it meets a new instruction, the $wp$ function
is applied for this instruction on all verification conditions involved in the
program path the instruction belongs to (to guarantee this, the conditional
rule is also handled slightly differently, but we will not detail this aspect).
However, when this step includes the proof of a property (for example, an
\CodeInline{assert} annotation), instead of applying the rule by adding it as a
conjunction, it is simply added separately to the set of verification conditions
to prove. Intuitively: we start a new weakest precondition calculus in parallel
from this program point.


Let us illustrate this on a toy example:


\CodeBlockInput{c}{wp-plugin-calculus-toy.c}


We start (in 1) with two verification conditions:
\begin{itemize}
  \item \CodeInline{VC1: P(x)},
  \item \CodeInline{VC2: P(y)}.
\end{itemize}
We apply the transformer of assignment on each of them, thus we reach 2 with:
\begin{itemize}
  \item \CodeInline{VC1: P(x * y)},
  \item \CodeInline{VC2: P(y)}.
\end{itemize}
In 2, we have to deal with the conditional. Instead of directly applying the
standard WP rule, we split the analysis in two sets of verification conditions,
initially equivalent. Let us consider the \CodeInline{else} branch, then the
\CodeInline{then} branch.


From 2 to 3, we meet the \CodeInline{check} annotation, thus we add a new
verification condition, we get the set:
\begin{itemize}
  \item \CodeInline{VC1: P(x * y)},
  \item \CodeInline{VC2: P(y)},
  \item \CodeInline{VC3: Q(y)}.
\end{itemize}
Then, we apply the assignment rule, and we get in 4:
\begin{itemize}
  \item \CodeInline{VC1: P(x * (y+1))},
  \item \CodeInline{VC2: P(y+1)},
  \item \CodeInline{VC3: Q(y+1)}.
\end{itemize}


From 2 to 5, we meet the \CodeInline{assert} annotation, thus we add this
knowledge to all verification conditions, and we create a new one:
\begin{itemize}
  \item \CodeInline{VC1: Q(x) ==> P(x * y)},
  \item \CodeInline{VC2: Q(x) ==> P(y)},
  \item \CodeInline{VC4: Q(x)}.
\end{itemize}
Then we apply the assignment rule, and we get in 6:
\begin{itemize}
  \item \CodeInline{VC1: Q(x+1) ==> P((x+1) * y)},
  \item \CodeInline{VC2: Q(x+1) ==> P(y)},
  \item \CodeInline{VC4: Q(x+1)}.
\end{itemize}


Now we have to reconcile the two sets. Let us not enter into the details (just
notice the introduced primed variables, the trick is here) of how it is done, we
get, in 7, something like:
\begin{CodeBlock}{text}
  VC1:
    ((x <= 0 ==> y' == y+1 && x' == x) &&
     (x >  0 ==> y' == y   && x' == x+1 && Q(x+1))) ==>
       P(x' * y')

  VC2:
    ((x <= 0 ==> y' == y+1) &&
     (x >  0 ==> y' == y    && Q(x+1))) ==>
       P(y')
  VC3:
    (x <= 0) ==> Q(x+1)

  VC4:
    (x >  0) ==> Q(x+1)
\end{CodeBlock}


Finally, we meet the precondition, and we obtain in 8:
\begin{CodeBlock}{text}
  VC1:
    P(x) ==>
      ((x <= 0 ==> y' == y+1 && x' == x) &&
       (x >  0 ==> y' == y   && x' == x+1 && Q(x+1))) ==>
         P(x' * y')

  VC2:
    P(x) ==>
      ((x <= 0 ==> y' == y+1) &&
       (x >  0 ==> y' == y    && Q(x+1))) ==>
         P(y')
  VC3:
    P(x) ==> (x <= 0) ==> Q(x+1)

  VC4:
    P(x) ==> (x >  0) ==> Q(x+1)
\end{CodeBlock}


We can see that for each property that must be proved, we have built a separate
verification condition that gathers the knowledge available along the path that
leads to the program point where its verification is asked.


\levelThreeTitle{Exercises}



\levelFourTitle{A series of assignment}


Compute by hand the weakest precondition of the following program:


\begin{CodeBlock}{c}
/*@
  requires -10 <= x <= 0 ;
  requires 0 <= y <= 5 ;
  ensures -10 <= \result <= 10 ;
*/
int function(int x, int y){
  int res ;
  y += 10 ;
  x -= 5 ;
  res = x + y ;
  return res ;
}
\end{CodeBlock}


Deduce that the program is correct with respect to its contract using the
right rule.


\levelFourTitle{Empty ``then'' branch in conditional}


We have previously shown that when a conditional structure has an empty ``else''
branch, that means that the conjunction of the precondition and the negation
of the condition must be enough to verify the postcondition of the complete
conditional structure.
For both of the following question, we only need the inference rules and
no WP calculus.

Show that when, instead of the ``else'' branch, the ``then'' branch is empty,
the postcondition is verified by the conjunction of the precondition and
the condition (as only the ``else'' can change the memory state).

Show that when both branches are empty, the overall condition is just a
skip operation.


\levelFourTitle{Short circuit}


C compilers implement short circuit for conditions. For example, that
means that a code like this one (\textbf{without ``else'' block}) :


\begin{CodeBlock}{c}
if(cond1 && cond2){
  // code
}
\end{CodeBlock}



can be written as:



\begin{CodeBlock}{c}
if(cond1){
  if(cond2){
    // code
  }
}
\end{CodeBlock}



Show that on those two source code, the weakest precondition calculus
generates an equivalent weakest precondition for equivalent for any code
in the ``then'' block. Note that we assume the conditions to be pure
expressions (without side effects).



\levelFourTitle{A larger program}


Compute by hand the weakest precondition of the following program:


\begin{CodeBlock}{c}
/*@
  requires -5 <= y <= 5 ;
  requires -5 <= x <= 5 ;
  ensures  -15 <= \result <= 25 ;
*/
int function(int x, int y){
  int res ;

  if(x < 0){
    x = 0 ;
  }

  if(y < 0){
    x += 5 ;
  } else {
    x -= 5 ;
  }

  res = x - y ;

  return res ;
}
\end{CodeBlock}


Deduce that the program is correct with respect to its contract using the
right rule.

\begin{Information}
  This part is more formal than what we have seen so far. If the reader
  wishes to concentrate on the usage of the tool, they can skip the
  introduction and the first two sections (about the basic instructions
  and the bonus training) of this chapter. If what we presented so far has
  been difficult for the reader from a formal point of view,
  it is fine to save the introduction and 
  the first and last sections of this chapter for a later reading.

  The middle sections on loops, however, are indispensable. We will highlight
  the more formal parts of those sections.
\end{Information}


We will associate with every C programming construct:
\begin{itemize}
\item the corresponding inference rule,
\item its governing rule from the weakest precondition calculus and
\item examples that show its usage.
\end{itemize}

Not necessarily in that order and sometimes only with a loose connection
to the tool. Since the first rules are quite simple, we will discuss
them in a fairly theoretical manner. Later on, however, our presentation
will rely more and more on the tool, in particular when we begin dealing
with loops.



\levelThreeTitle{Inference rules}


An inference rule is of the form
$$\dfrac{P_1 \quad ... \quad P_n}{C}$$
and means that in order to assure that the conclusion C is true, first
the truth of the premises $P_1$, ..., and $P_n$ has to be
established. In the case where the rule has no premises
$$\dfrac{}{\quad C \quad}$$
then nothing has to be assured in order to conclude the truth of $C$,
and it is called an axiom.

In some cases, in order to prove that a certain premise is true, it
might be necessary to employ other inference rules which would lead to
something like this:
$$\dfrac{\dfrac{}{\quad P_1\quad} \quad \dfrac{\dfrac{}{P_{n_1}}\quad \dfrac{}{P_{n_2}}}{P_n}}{C}$$

This way, we obtain step by step the \emph{deduction tree} of our
reasoning. In our case, the premises and conclusions under consideration
will in general be \emph{Hoare triples}.



\levelThreeTitle{Hoare triples}


We are now returning to concept of a Hoare triple:
$$\{ P \}\quad  C\quad \{ Q \}$$


In the beginning of this tutorial we have seen that this triple
expresses the following: if the property $P$ holds before the
execution of $C$ and if $C$ terminates, then the property $Q$
holds too. For example, if we return to our (slightly modified)
program for the computation of the absolute value:



\CodeBlockInput{c}{abs-1.c}



The rules of Hoare logic tell us that in order to show that our program
satisfies its contract we have to verify the properties shown below in the
braces. (We have omitted one postcondition in order to simplify the
presentation.)



\CodeBlockInput{c}{abs-2.c}



Yet, Hoare logic does not tell us how we can automatically obtain the
precondition \CodeInline{P} of our program \CodeInline{abs}. Dijkstra's
\emph{weakest-precondition calculus}, on the other hand, allows us to
compute from a given postcondition $Q$ and a code snippet $C$ the
minimal precondition $P$ that ensures $Q$ after the execution of
$C$. We are thus in a position to determine for our example
\CodeInline{abs} the desired property \CodeInline{P}.




In this chapter, we present the different cases of the function $wp$
which, starting from a given program (or statement) and a postcondition,
computes the \emph{weakest} precondition that allows us to establish the
validity of the postcondition. We will use the following notation to
define the computation that corresponds to one or several statements:
$$wp(Instruction(s), Post) := WeakestPrecondition$$



The function \(wp\) will guarantee that the Hoare triple
$$\{\ wp(C,Q)\ \}\quad C\quad \{ Q \}$$
really is a valid triple.



We will thereby often use ACSL assertions in order to represent the
upcoming concepts:



\begin{CodeBlock}{c}
//@ assert some_property ;
\end{CodeBlock}



These assertions correspond in fact to possible intermediate steps for
the properties in our Hoare triples. We can, for example, replace the
properties of our function \CodeInline{abs} by corresponding ACSL assertions
(we have omitted here the property \CodeInline{P} because it is just
\CodeInline{true}):



\CodeBlockInput{c}{abs-3.c}

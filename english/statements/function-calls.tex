\levelThreeTitle{Calling a function}

\levelFourTitle{Formal - Weakest precondition calculus}

When a function is called, the contract of this function is used to determine
the precondition of the call. But one has to consider two important facts to
express the weakest precondition calculus.



First, the postcondition of the called function $f$ is not necessarily
directly the precondition that was computed for the instructions that follow
the call to $f$. For example, if we have a program: \CodeInline{x = f() ; c }
and $wp(\texttt{c}, Q) = 0 \leq x \leq 10$, whereas the postcondition of the
function \CodeInline{f} is $1 \leq x \leq 9$, we have to express some
weakening between the actual precondition of \CodeInline{c} and the computed
one. For this, we refer to the section~\ref{l3:statements-basic-consequence},
the idea is simply to verify that the postcondition of the function implies
the computed precondition.



Second, in C, a function can have side effects. Thus, the values of the
variables referenced in input are not necessarily the same as they were after the
call to the function, and the contract may express some property about those
values before and after the call. So, if we have labels in the postcondition,
we must correctly replace them.



In order to define the weakest-precondition calculus of function calls, let
us introduce some notation to make things clearer. For this, consider this
example:


\begin{CodeBlock}{c}
/*@ requires \valid(x) && *x >= 0 ;
    assigns *x ;
    ensures *x == \old(*x)+1 ; */
void inc(int* x);

void foo(int* a){
  L1:
  inc(a) ;
  L2:
}
\end{CodeBlock}




The weakest precondition of the function call asks us to consider the contract
of the function that is called (here, in \CodeInline{foo}, when we call the
\CodeInline{inc} function). Of course, before the call to the function we have
to verify its precondition, so it is part of the weakest precondition. But, we
also have to consider the postcondition of the function, else that would mean
that we do not consider its effect.




Thus, it is important to notice that in the precondition, the considered memory
state is the one where we compute the weakest precondition, whereas for the
postcondition it is not the case, the considered memory state is the one that
follows the call, while we need to explicitly mention the old state to speak
about the values before the call. For example, considering the contract of
\CodeInline{inc} when we call it in \CodeInline{foo}, \CodeInline{*x} in the
precondition is \CodeInline{*a} at \CodeInline{L1}, while \CodeInline{*x} in the
postcondition is \CodeInline{*a} at \CodeInline{L2}. Consequently, the pre and
the postcondition must be considered slightly differently when it comes to
mutable memory location. Note that for the value of the parameter \CodeInline{x}
itself, there is no such consideration: this value cannot be modified by the
call.




Now, let us define the weakest precondition of a function call. For this,
we denote:

\begin{itemize}
\item $\vec{v}$ a vector of values $v_1, ..., v_n$ and $v_i$ the $i^{th}$ value,
\item $\vec{t}$ the arguments provided to the function when we call it,
\item $\vec{x}$ the parameters in the function definition,
\item $\vec{a}$ the assigned values (seen from the outside, once instantiated),
\item $here(x)$ a value in postcondition
\item $old(x)$ a value in precondition
\end{itemize}

We name $\texttt{f:Pre}$ the precondition of the function, and $\texttt{f:Post}$
the postcondition:



\begin{center}
\begin{tabular}{rl}
  $wp( f(\vec{t}), Q ) :=$ & $\texttt{f:Pre}[x_i \leftarrow t_i]$ \\
  $\wedge$ & $\forall \vec{v}, \quad (
              \texttt{f:Post}[x_i \leftarrow t_i,
                              here(a_j) \leftarrow v_j,
                              old(a_j) \leftarrow a_j] \Rightarrow
              Q[here(a_j) \leftarrow v_j])$
\end{tabular}
\end{center}


We can detail a bit the reasoning for each part of this formula.


First, note that in both pre and postcondition, each named parameter $x_i$ is
replaced with the corresponding argument ($[x_i \leftarrow t_i]$), as we said
before we do not have to consider memory states there because those values
cannot be changed by the function call. For example in the contract of
\CodeInline{inc}, each \CodeInline{x} would be replaced by the argument
\CodeInline{a}.



Then, in the part of the formula that corresponds to the postcondition, we can
see that we introduce a $\forall \vec{v}$. The goal is here to model the fact
that the function can write any value in each memory location that is assigned.
So, for each of the assigned location $a_j$ (that is for our call to
\CodeInline{inc}, \CodeInline{*(\&a)}), we generate a value $v_j$ that is its
value after the call. But, if we want to check that the postcondition gives us
the right result, we cannot accept \emph{any value} for each assigned location,
we just want the ones \emph{that allows to satisfy the postcondition}.



So these values are used to transform the postcondition of the function and
verify that it implies the postcondition in input of the weakest precondition.
This is done by replacing, for each assigned location $a_j$, its value $here$
with the value $v_j$ that it is supposed to get after the call
($here(a_j) \leftarrow v_j$). Finally, we have to replace each $old$ value by
its value before the call, and for each $old(a_j)$, it is simply $a_j$
($old(a_j) \leftarrow a_j$).



\levelFourTitle{Formal - Example}



Let us illustrate this on an example by applying the weakest precondition
calculus to this short code, assuming the contract we previously proposed for
the \CodeInline{swap} function.



\begin{CodeBlock}{c}
  int a = 4 ;
  int b = 2 ;

  swap(&a, &b) ;

  //@ assert a == 2 && b == 4 ;
\end{CodeBlock}



We now compute the weakest precondition:


\begin{tabular}{l}
  $wp(a = 4; b = 2; swap(\&a, \&b), a = 2 \wedge b = 4) = $\\
  $\quad wp(a = 4, wp(b = 2; swap(\&a, \&b), a = 2 \wedge b = 4)) = $\\
  $\quad wp(a = 4, wp(b = 2, wp(swap(\&a, \&b), a = 2 \wedge b = 4)))$
\end{tabular}


Let us first consider separately:


$$wp(swap(\&a, \&b), a = 2 \wedge b = 4)$$



From this \CodeInline{assigns} clause, we know that the assigned values are
$*(\&a) = a$ and $*(\&b) = b$. (Let us shorten $here$ with $H$ and $old$ with
$O$).


\begin{tabular}{rl}
  $\quad \quad \texttt{swap:Pre}[x \leftarrow \&a,\ y \leftarrow \&b]$ & \\
  $\quad \wedge \forall v_a, v_b,(\texttt{swap:Post}$ & $ [ x \leftarrow \&a,\ y \leftarrow \&b, $ \\
                               & $H(*(\&a)) \leftarrow v_a,\ H(*(\&b)) \leftarrow v_b,$ \\
                               & $O(*(\&a)) \leftarrow *(\&a),\ O(*(\&b)) \leftarrow *(\&b)])$\\
  \multicolumn{2}{r}{$\quad \quad \Rightarrow (H(a) = 2 \wedge H(b) = 4)[H(a)) \leftarrow v_a, H(b)) \leftarrow v_b])$}
\end{tabular}


For the precondition, we get :
$$valid(\&a) \wedge valid(\&b)$$
For the postcondition part, let us first write the expression from which
we start before any term replacement (and without the syntax for the
replacement for the sake of conciseness):
$$H(*x) = O(*y) \wedge H(*y) = O(*x) \Rightarrow H(a) = 2 \wedge H(b) = 4$$
First we replace the pointers ($x \leftarrow \&a,\ y \leftarrow \&b$) :
$$H(*(\&a)) = O(*(\&b)) \wedge H(*(\&b)) = O(*(\&a)) \Rightarrow H(a) = 2 \wedge H(b) = 4$$
Then, the $here$ values, with the quantified $v_i$s ($H(a)) \leftarrow v_a, H(b)) \leftarrow v_b$):
$$v_a = O(*(\&b)) \wedge v_b = O(*(\&a)) \Rightarrow v_a = 2 \wedge v_b = 4$$
And the $old$ values, with the value before call
($O(*(\&a)) \leftarrow *(\&a),\ O(*(\&b)) \leftarrow *(\&b)$):
$$v_a = *(\&b) \wedge v_b = *(\&a) \Rightarrow v_a = 2 \wedge v_b = 4$$
We can now simplify this formula to:
$$v_a = b \wedge v_b = a \Rightarrow v_a = 2 \wedge v_b = 4$$


So, $wp(swap(\&a, \&b), a = 2 \wedge b = 4)$ is:
$$P: valid(\&a) \wedge valid(\&b) \wedge \forall v_a, v_b, \quad v_a = b \wedge v_b = a \Rightarrow v_a = 2 \wedge v_b = 4$$
Let us immediately simplify the formula by noticing that validity properties
are trivially true here (since the variable are allocated on the stack just
before):
$$P: \forall v_a, v_b, \quad v_a = b \wedge v_b = a \Rightarrow v_a = 2 \wedge v_b = 4$$


Let us now compute $wp(a = 4, wp(b = 2, P)))$, by first replacing $b$ with
$2$ by the assignment rule:
$$\forall v_a, v_b, \quad v_a = 2 \wedge v_b = a \Rightarrow v_a = 2 \wedge v_b = 4$$
and then replacing $a$ with $4$ by the same rule:
$$\forall v_a, v_b, \quad v_a = 2 \wedge v_b = 4 \Rightarrow v_a = 2 \wedge v_b = 4$$


This last property is trivially true, thus the program is verified.



\levelFourTitle{What should we keep in mind?}



Functions are absolutely necessary to modular programming, and the weakest
precondition calculus is fully compatible with this idea, allowing to reason
about each function locally and compose proofs just as we compose function
calls.


So as a reminder, we should just keep in mind the following general scheme:



\begin{CodeBlock}{c}
/*@
  requires foo_R ;
  assigns ... ;
  ensures foo_E ;
*/
type foo(parameters...){
  // Here we suppose that foo_R holds


  // Here we must prove that bar_R holds
  bar(some parameters ...) ;
  // Here we assume that bar_E holds


  // Here we must prove that foo_E holds
  return ... ;
}
\end{CodeBlock}


Note that for the last statement, with weakest precondition calculus, the idea
is more to show that our precondition is strong enough to ensure that the code
leads to our postcondition. However, first, this vision is simpler to
understand, and second the WP plugin does not actually perform a strict weakest
precondition calculus but a highly optimized one that does not follow exactly
the same rules.


\levelThreeTitle{Recursive functions}


Just as it is easy to prove anything about the postcondition of a function if it
contains an infinite loop, it is easy to prove anything about the postcondition
of a function that does an infinite recursion:


\CodeBlockInput{c}{trick.c}


\image{recursive-trick}


We can see that the function and the assertion are proved. And indeed the
proof is correct: we consider partial correctness, and we face a function
that does not terminate: anything that follows a call to this function would
be true.


Thus, the question is: what could we do in such a case? Again, we could use
some kind of variant to bound the depth of the recursive calls. In ACSL, it is
expressed thanks to the \CodeInline{decreases} clause:


\CodeBlockInput[1][6]{c}{decreases.c}


This clause expresses exactly the same idea as a \CodeInline{loop variant}.
The expression considered by a \CodeInline{decreases} clause is, when the
function is called again, a positive integer expression (or an expression
equipped with a relation) that strictly decreases. Note that it means that
when we reach the maximum recursion depth, the expression might be negative:


\CodeBlockInput[8][12]{c}{decreases.c}


We can see the generated verification condition for the previous example by
disabling goal simplifications (option \CodeInline{-wp-no-let}, we removed
redundant information in the screenshot):

\image{go_negative}


Here, the condition \CodeInline{n - 10 < n} is formulated
\CodeInline{n\_1 (= n) <= 9 + n} because of the normalization of the
formula.


Let us emphasize on the fact that the \CodeInline{decreases} clause really
bounds the depth of the recursion and not the number of calls:



\CodeBlockInput[14][20]{c}{decreases.c}


When proving the correctness of the \CodeInline{decreases} clause of a particular
function (and for a call to this same function), the expression is evaluated at
for two entities : the function under proof and the call instruction. The state
where the expression is evaluated for the function under proof is
\CodeInline{Pre}, the state the expression is evaluated for the call instruction
is \CodeInline{Here}. The value of the expression at call point must be less
than the value of the expression in \CodeInline{Pre} state.



\CodeBlockInput[22][30]{c}{decreases.c}



Of course, recursive functions can be mutually recursive thus the
\CodeInline{decreases} clause can be used to bound recursive function calls in
this situation. But, we only want to do so {\em only} for functions that are
indeed in the set of functions that are involved in the recursion. For this, WP
computes the strongly connected components from the set of functions, this is
called a {\em cluster}.

Thus, let us be more precise about how we proceed to verify of the
correction of a \CodeInline{decreases} clause. When a function is (mutually)
recursive, its specification must be equipped with such a clause to prove that
it terminates. Verifying that the \CodeInline{decreases} clause of a function
\CodeInline{f} gives a measure of the depth of recursion is done by checking for
each call to a function \emph{that belongs to the same cluster as
\CodeInline{f}}, that the expression is indeed positive and decreasing. And
thus, no verification condition is generated when calling a recursive function
that does not belong to the same cluster:


\CodeBlockInput[32][48]{c}{decreases.c}


\begin{Information}
  Note that the computation of the cluster is done syntactically, for now we do
  not cover function pointers in this tutorial, but one can refer to the option
  \CodeInline{-wp-dynamic} in the WP manual.
\end{Information}


Finally, if a function of a cluster does not have a \CodeInline{decreases}
clause, a \CodeInline{\textbackslash{}false} verification condition is generated
and a warning is emitted by WP.


\CodeBlockInput[50][63]{c}{decreases.c}


\image{fail-mutual-vc}


\levelThreeTitle{Specifying and proving function termination}



A desirable property for a function is often that it should terminate. Indeed,
in most programs, all functions must terminate and when it is not the case and
some functions can loop forever, it is highly common that there is a single
function that is allowed to loop forever (so almost all functions in this
program must terminate).



\levelFourTitle{Syntax and description}



ACSL provides the \CodeInline{terminates} clause to specify that a function must
terminate when some property is verified in precondition. The syntax is:



\begin{CodeBlock}{c}
//@ terminates condition ;
void function(void){
  // ...
}
\end{CodeBlock}



It states that when \CodeInline{condition} is verified in precondition, then
the function must terminate. For example, function \CodeInline{abs} must always
terminate:



\CodeBlockInput[3][8]{c}{terminates_abs.c}



while function \CodeInline{main\_loop} may not (note that with default options,
the variant is not verified, we will explain why later):



\CodeBlockInput[6]{c}{main_loop.c}



Let us emphasize on the fact that the function {\em may not} terminate, it is
not {\em forced to loop forever}. For example, in the following function, the
\CodeInline{terminates} clause is verified since
{\em whenever the termination condition is verified} (never), the function
terminates (always):



\CodeBlockInput[6][9]{c}{terminates.c}



\begin{Information}
  If one really wants to verify that some function never terminates, it can be
  done by specifying that the function never returns and never exits. That is:
  the post-conditions states associated to these kinds of termination are
  unreachable:
  \CodeBlockInput{c}{no_terminates.c}
\end{Information}



\begin{Information}
  In ACSL, it is specified that when a function does not have a
  \CodeInline{terminates} clause, the default is
  \CodeInline{terminates \textbackslash{}true}, however it is still not
  implemented by the Frama-C kernel. This behavior can be enabled in WP
  by using the options:
  \begin{itemize}
  \item \CodeInline{-wp-definitions-terminate} (for functions with definition),
  \item \CodeInline{-wp-declarations-terminate} (for functions without definition),
  \item \CodeInline{-wp-frama-c-stdlib-terminate} (for functions of the Frama-C stdlib),
  \end{itemize}
  to claim that, unless a clause is provided, they must terminate (and thus
  for functions with a definition, the verification must be performed).
\end{Information}



\levelFourTitle{Verification}



Verifying that a function terminates asks to verify that all reachable
statements of the function terminate. Assignments trivially terminate, thus
we do not have something particular to do for them. A conditional statement
terminates if all statements in the different (reachable) branches terminate,
thus we just have to verify that these statements terminate (or are unreachable).
The remaining statements are loops and function calls. Thus, we have to verify
that:
\begin{itemize}
  \item all loops have a (verified) \CodeInline{loop variant} clause,
  \item all called functions terminate with the given parameters,
  \item all recursive functions have a (verified) \CodeInline{decreases} clause,
  \item (or that there are no loops nor calls in the function).
\end{itemize}



However, we only have to do that when the termination condition of the function
is verified. So let us now explain what are the generated verification
conditions and when does WP generate them.



When a function has a \CodeInline{terminates} clause, WP visits all statements
and collects the loops that do not have a \CodeInline{loop variant} clause and
the function calls. If there are none of them, the \CodeInline{terminates}
clause is trivially verified.



\image{trivial_terminates}



When there are such statements, their termination must be verified {\em when}
the function must terminate (say, when \CodeInline{T}). Thus, the verification
conditions are of the form
\CodeInline{\textbackslash{}at(T, Pre)} $\Rightarrow$ \CodeInline{<statement termination>}.
One should note that the premise of the implication is evaluated at the
\CodeInline{Pre} state. Thus, in this code:



\CodeBlockInput[11][17]{c}{terminates.c}



Even if \CodeInline{r} has been decremented, the verification of the
termination of \CodeInline{call(r)} is done with
\CodeInline{\textbackslash{}at(r, Pre) > 0} as a premise. We will see in the
next section the verification condition generated for a call.



Note also that it means that when we reach a program point when \CodeInline{T}
is false, the verification condition is always verified:



\CodeBlockInput[19][25]{c}{terminates.c}



\paragraph{Function call}



Verifying that a function call terminates is done by verifying that when it is
called, its termination condition is true. For example, from the following
program:



\CodeBlockInput[27][36]{c}{terminates.c}


We get the following verification conditions (by using \CodeInline{-wp-no-let}
to disable simplifications):


\image{call_terminates_1}


\image{call_terminates_2}


Where the first call termination is indeed verified while the second is not.


When the called function does not have a \CodeInline{terminates} clause, it
is considered to be \CodeInline{\textbackslash{}false}, that is: unless a
\CodeInline{terminates} clause is provided, the function may not terminate.


For example with our previous \CodeInline{simple} function, we get the message
``Missing terminates clause on call to call, defaults to \textbackslash{}false''
and we fail to prove termination:


\image{simple_terminates_fails}


Note that the formula $\mathtt{r} > 0 \Rightarrow false$ is translated into
$\mathtt{r} <= 0$.


\paragraph{Loop variant}



When a function contains a loop that does not have a \CodeInline{loop variant}
clause, its termination cannot be verified, thus WP asks us to verify that when
the termination condition is verified, this loop is unreachable.



\CodeBlockInput[38][46]{c}{terminates.c}



In the previous code, the loop does not have a \CodeInline{loop variant}
clause, thus we have to verify \CodeInline{value > 0 ==> \textbackslash{}false}
at the loop location, which is OK: this code is unreachable when
\CodeInline{value > 0}.



Finally, when a loop has a \CodeInline{loop variant} clause, it must be
verified {\em only when} the function must terminate. So in the example
we presented at the beginning of this section:


\CodeBlockInput[6]{c}{main_loop.c}


We have to verify that the loop variant is a positive decreasing value only
when \CodeInline{debug\_steps} is not $-1$. However, this is not the default
behavior of WP (that always verify loop variants by default), this can
be enabled using the option \CodeInline{-wp-variant-with-terminates} and
in this case our function is entirely verified:


\image{main_loop_variant_terminates}


\paragraph{Recursion}


A recursive function should have a \CodeInline{decreases} clause when its
specification states that it terminates. If such a clause is missing, a
verification condition \CodeInline{\textbackslash{}false} is generated.


\CodeBlockInput[48][51]{c}{terminates.c}


Note that this code generates two verification conditions:


\image{missing_decreases_vcs}


The first one corresponds to the rule associated to a function call explained
earlier. The second corresponds to the fact that the \CodeInline{decreases}
clause is missing and is not verified.


Again in ACSL, the verification of the \CodeInline{decreases} clause
verification is required only when the termination condition is verified in
\CodeInline{Pre} state. The behavior of WP on this aspect is similar to the
\CodeInline{loop variant}. By default, the verification is always tried, the
ACSL specified behavior is enabled via the option
\CodeInline{-wp-variant-with-terminates}.


\CodeBlockInput[53][60]{c}{terminates.c}


\begin{Information}
  When a function under proof does not have \CodeInline{terminates}
  specification, whenever a verification condition needs it, the clause is
  considered to be \CodeInline{\textbackslash{}true} since it makes sure that
  we have more things to prove. For example, when the option
  \CodeInline{-wp-variant-with-terminates} is used and a function contains a
  \CodeInline{loop variant}, it guarantees that WP tries to prove it.
\end{Information}


\levelThreeTitle{Exercises}


\levelFourTitle{Explain proof failures}


In the following program, some verification conditions are not verified:


\CodeBlockInput{c}{ex-1-proof-failures.c}


Explain why they are not verified and propose a way to fix the specification so
that everything is verified.


\levelFourTitle{Explain termination proof results}


In the following program:


\CodeBlockInput{c}{ex-2-terminates.c}


Explain why termination clauses are verified or not. Modify the specification so
that all of them are verified.


\levelFourTitle{Search}


Specify and prove the following recursive search function:


\CodeBlockInput[5]{c}{ex-3-search.c}


The specification should include the termination condition.


\levelFourTitle{Sum integers}


The following program computes the sum of the integers between \CodeInline{fst}
and \CodeInline{lst}:


\CodeBlockInput[5]{c}{ex-4-sum.c}


Prove that this function terminates. Proving the correct behavior of the
function or the absence of runtime errors is not asked.


\levelFourTitle{Power}


The following program computes the power of \CodeInline{x} to \CodeInline{n}:


\CodeBlockInput[5]{c}{ex-5-power.c}


Prove that this function terminates. Proving the correct behavior of the
function or the absence of runtime errors is not asked.

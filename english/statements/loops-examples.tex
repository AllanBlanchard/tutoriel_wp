\levelThreeTitle{Writing loop annotations}


Writing loop annotations requires a lot of work during program proof. There is
no perfect way of writing them. In particular, finding the right invariant, and
the right way to express it, is mostly a matter of experience. Nevertheless, a
few simple ideas can help, so that at least we can avoid common mistakes that
would make us lose a lot of time during the process.


Before anything else, we should write the \CodeInline{loop assigns} clause. It is
generally easy to write (just look at the assignment instructions and function
calls), and if we forget this step we risk wasting time trying to prove incorrect
properties. When writing these clauses, we should not try to be too precise. WP cannot 
efficiently handle something like \CodeInline{array[x .. y]} when \CodeInline{x} or
\CodeInline{y} are themselves modified, so we generally prefer bounds that
include them while being constant during the execution of the loop, if we need
more precise information during the proof of the invariants, we will provide
some additional invariants for this.


Next, we bound the different variables that are assigned, in particular indexes.
These invariants are generally easy to guess, express and verify. We put these
invariants at the beginning of the list of \CodeInline{loop invariant} clauses,
since, as we explained in Sections~\ref{l3:statements-loops-multi-inv}
and~\ref{l3:statements-loops-inv-kinds}, the order of the invariants is important,
and these simple properties can be propagated by WP to the other loop invariants
to make verification condition simplifications.


For most loops, except the ones that rely on a complex condition, once this step
is done, it is easy to provide the \CodeInline{loop variant} clause. We can just
look at the variables that we have just bounded and at the value reached
at the end of the loop. For example, if in the loop some \CodeInline{i}
goes from 0 to \CodeInline{n}, \CodeInline{n - i} is a good candidate. For more
complex loops, one can rely on ghost code (that we present in
Section~\ref{l2:acsl-logic-definitions-ghost-code}) to make some measure
explicit and use it to provide a suitable loop variant.


And now, we add the ``main'' loop invariants of the loop, i.e., those that relate to
the expected postcondition of the loop (this might be the postcondition of the
function itself). For this, we can use the postcondition itself as a guide. If
we have something like \CodeInline{ensures P(n);} and a loop that iterates
\CodeInline{i} from 0 to \CodeInline{n}, \CodeInline{loop invariant P(i);} is
probably a good candidate. Notice that in some situations, the
\CodeInline{assumes} clause might also be interesting, typically when the result
is a simple value that depends on the precondition state, as we will see in a
later example. These invariants should be positioned at the end of the list of
loop invariants.


We may (optionally) need some ``glue'' to prove the most interesting ``main''
loop invariant. For example, if we deliberately provided an imprecise
\CodeInline{loop assigns} clause, we might need a loop invariant to explain that
some parts are not modified, or we may need to explain to the solver that
because some properties are true, then we can deduce that the ``main'' invariant
is verified, etc. These loop invariants should be provided before the ``main''
invariant, but after the simple invariants about variables bounds. Ordering them
might not be straightforward, so generally this part is mostly a
``trial and error'' process.


In the remainder of this section, we will illustrate this approach on some
examples. Although in the process, it is advised to run the proof between each
step, we will generally not go to this level of detail in the future examples.


\levelThreeTitle{Example with read-only arrays}
\label{l3:statements-loops-examples-ro}


The array is the most common data structure when we are working with loops.
It is therefore a good example base to practice with loops, and these
examples allow us to immediately demonstrate interesting invariants and
to introduce some important ACSL constructs.


Let's start with the search function, which allows one to find a
value in an array if the value is there. For now, let us focus on the contract of the function:


\CodeBlockInput[5]{c}{search-0.c}


There are enough ideas in this example to introduce some important syntax.


First, as we previously mentioned, the
\CodeInline{\textbackslash{}valid\_read} predicate (as well as
\CodeInline{\textbackslash{}valid}) allows us to specify not only the
validity of a readable address but also to state that a range of
contiguous addresses is valid. It is expressed using this syntax:


\begin{CodeBlock}{c}
//@ requires \valid_read(a + (0 .. length-1));
\end{CodeBlock}


This precondition states that all addresses \CodeInline{a+0},
\CodeInline{a+1}, ..., \CodeInline{a+length-1} must be valid readable
locations.


We also introduced two notations that are used almost all the time in
ACSL, the keywords \CodeInline{\textbackslash{}forall} ($\forall$) and
\CodeInline{\textbackslash{}exists} ($\exists$), the universal logic
quantifiers. The first one allows us to state that for any element, some
property is true; the second one allows us to say that there exists some
element such that the property is true. If we add comments to the
corresponding lines in our specification, we can read them this way:


\begin{CodeBlock}{c}
/*@
// for all "off" of type "size_t", IF "off" is comprised between 0 and "length"
//                                 THEN the cell "off" in "a" is different from "element"
\forall size_t off ; 0 <= off < length ==> a[off] != element;

// there exists "off" of type "size_t", such that "off" is comprised between 0 and "length"
//                                      AND the cell "off" in "a" equals to "element"
\exists size_t off ; 0 <= off < length && a[off] == element;
*/
\end{CodeBlock}


If we want to summarize the use of these two keywords, we would say that on a
range of values, a property is true, either about about all of them
(\CodeInline{\textbackslash{}forall}) or at least one of them
(\CodeInline{\textbackslash{}exists}). A common scheme is to constrain this set using
another property (here: \CodeInline{0 <= off < length}) and to prove the
actual interesting property on this smaller set. \textbf{But using
\CodeInline{\textbackslash{}exists} and \CodeInline{\textbackslash{}forall} is
fundamentally different}.


With \CodeInline{\textbackslash{}forall type a ; p(a) ==> q(a)},
the constraint \CodeInline{p} is followed by an implication.
For any element where a first property \CodeInline{p} is verified, we have
to also verify the second property \CodeInline{q}. If we use a conjunction,
as we do for ``exists'' (which we will later explain), that would mean that
all elements satisfy both \CodeInline{p} and \CodeInline{q}. In our previous
example, this is clearly not the case, as not all integers are comprised between
0 and \CodeInline{length}. Sometimes, a construction like this may be what we want to express, but
this construction would no longer correspond to the idea of constraining a set for which
we want to verify some other property.


With \CodeInline{\textbackslash{}exists type a ; p(a) \&\& q(a)}, the
constraint \CodeInline{p} is followed by a conjunction. We say there exists
an element such that it satisfies the property \CodeInline{p}, and at the same
time it also satisfies \CodeInline{q}. If, instead of a conjunction, we use an implication, as we do
for \CodeInline{\textbackslash{}forall}, such an expression will always be true
if \CodeInline{p} is not a tautology! Why? Is there an ``a'' such that p(a) implies
q(a)? Let us take any ``a'' such that p(a) is false, then the implication is true.


Notice that in this example, the \CodeInline{assumes} clause of the
\CodeInline{in} behavior is exactly the negation of the \CodeInline{assumes}
clause of the \CodeInline{notin} behavior, this is the reason why the
\CodeInline{disjoint} and \CodeInline{complete} clauses are proved, in fact it
could have been expressed with:


\begin{CodeBlock}{c}
  /*@ ...
    behavior in:
      assumes !(\forall size_t off ; 0 <= off < length ==> array[off] != element);
    ...
  */
\end{CodeBlock}


Let's turn now to adding the loop annotations. The first step is to add the
\CodeInline{loop assigns} clause. Here, it is simple: the loop only modifies the
variable \CodeInline{i}. Thus, we need to bound \CodeInline{i}, it goes from
\CodeInline{0} to \CodeInline{length}, this is our first loop invariant:
\CodeInline{0 <= i <= length}. That also gives us the loop variant:
\CodeInline{length - i}. Now, we can provide our ``main'' loop invariant.
In this example, the invariant is related to the \CodeInline{assumes} clauses, and not to the
\CodeInline{ensures} clauses. In particular, the interesting part of the loop is
the fact that unless we find the element we are searching for, it is not in the array.
Let's see if we can reuse part of the assumes clause from our \CodeInline{notin} behavior:
\begin{CodeBlock}{c}
  //@ \forall size_t off ; 0 <= off < length ==> array[off] != element;
\end{CodeBlock}
The variable that reaches \CodeInline{length} at the end of the loop is
\CodeInline{i}, thus:
\begin{CodeBlock}{c}
  //@ loop invariant \forall size_t off ; 0 <= off < i ==> array[off] != element;
\end{CodeBlock}
is certainly a good candidate. Thus, we have the following loop annotations:


\CodeBlockInput[20][25]{c}{search-1.c}


And indeed, our final loop invariant defines the treatment performed by our
loop, it indicates to WP what happens in the loop (or more precisely: what we
learn) along the execution. Here, this formula indicates that at each iteration
of the loop, we know that for each indexed memory location between indices 0 and the next
location to visit (\CodeInline{i} excluded), the memory location contains a
value different from the element we are looking for.


The verification condition associated with the preservation of this invariant is
a bit complex, and it is not interesting to closely examine at it;
however, the proof that the invariant is established before
executing the loop \emph{is} interesting:


\image{trivial}


We note that this property, while quite complex, is proved easily
by Qed. If we look at the parts of the program on which the proof
relies, we can see that the instruction \CodeInline{i = 0} is highlighted
and this is, indeed, the last instruction executed on \CodeInline{i} before
we start the loop. And consequently, if we replace the value of
\CodeInline{i} by 0 inside the formula of the invariant, we get:


\begin{CodeBlock}{c}
//@ loop invariant \forall size_t j; 0 <= j < 0 ==> array[j] != element;
\end{CodeBlock}


``For all j, greater than or equal to 0 and strictly lower than 0''; this
part of the formula is necessarily false, so our implication is 
necessarily true.


\levelThreeTitle{Examples with mutable arrays}


Let us present two examples with mutation of arrays. One with a mutation
of all memory locations, the other with selective modifications.


\levelFourTitle{Reset}


Let us have a look at the function that resets an array of integers to 0.


\CodeBlockInput{c}{reset.c}


This time, the loop does modify the content of the array, so we need to
state this in the \CodeInline{loop assigns} clause. Notice that we can use
the notation \CodeInline{n .. m} for this. Furthermore, we do not say that
the loop assigns the content from \CodeInline{0} to \CodeInline{i-1} (since
\CodeInline{i} is modified, and WP cannot exploit this) but from \CodeInline{0}
to \CodeInline{length-1}, this is less precise, but it can be used by WP outside the
loop. Finally, this time we directly relate the invariant to the postcondition.
The goal of the function is to reset the array from indices 0 to \CodeInline{length-1}, at
any given iteration, the loop has reset elements from indices 0 to \CodeInline{i-1}.


\levelFourTitle{Search and replace}
\label{l4:statements-loops-ex-search-and-replace}


The last example illustating how to prove functions
with loops is the ``search and replace'' algorithm. This algorithm
selectively modifies values in a range of memory locations. It is
generally harder to guide the tool in such a case, because on one hand
we must keep track of what is modified and what is not, and on the other
hand, the induction relies on this fact.


As an example, the first specification and loop annotations for
this function could be the following (which employs the same process we
used in the previous example):


\CodeBlockInput{c}{search_and_replace-0.c}


We use the logic function \CodeInline{\textbackslash{}at(v, Label)} that gives
us the value of the variable \CodeInline{v} at the program point \CodeInline{Label}.
If we look at the usage of this function here, we see that in the
invariant we try to establish a relation between the old values of the
array and the potentially new values:


\begin{CodeBlock}{c}
/*@
  loop invariant \forall size_t j; 0 <= j < i && \at(array[j], Pre) == old
                   ==> array[j] == new;
  loop invariant \forall size_t j; 0 <= j < i && \at(array[j], Pre) != old
                   ==> array[j] == \at(array[j], Pre);
*/
\end{CodeBlock}



For every memory location that contained the value to be replaced, it now must
contain the new value. All other values must remain unchanged. While we
previously relied on the \CodeInline{assigns} clauses for these kinds of properties,
here, we do not have this possibility because while the ACSL language provides
some facilities that we could use to write a very precise \CodeInline{assigns}
clause, WP would not perfectly exploit it. Thus, we have to use an invariant and
keep an approximation of the assigned memory location.


If we try to prove this invariant with WP, it fails, because the
\CodeInline{assigns} specification is not precise enough. Thus, in such a case,
we provide an additional loop invariant to detail in the assigned range what are
the locations that have not been modified yet by the loop in some iteration:


\begin{CodeBlock}{c}
for(size_t i = 0; i < length; ++i){
    //@assert array[i] == \at(array[i], Pre); // proof failure
    if(array[i] == old) array[i] = new;
}
\end{CodeBlock}


We can instead add this information as an invariant (the second from
the top):


\CodeBlockInput[13][26]{c}{search_and_replace-1.c}


And this time the proof succeeds.


\levelThreeTitle{Exercises}


For all these exercises, use the following command line to start verification:

\begin{CodeBlock}{bash}
frama-c-gui -wp -wp-rte -warn-unsigned-overflow -warn-unsigned-downcast your-file.c
\end{CodeBlock}


\levelFourTitle{Non-mutating: For all, Exists, ...}


Currently, function pointers are not supported by WP. Let us consider that
we have a function:


\CodeBlockInput[7][13]{c}{ex-1-forall-exists.c}


Write your own code and corresponding postcondition and then write the
following functions together with their contract and prove their correctness.
Note that, as \CodeInline{pred} is a C function, it cannot be used in the
specification of the functions you have to write, thus the property that you
have considered for this predicate should be inlined in the contract of the
different functions.


\begin{itemize}
\item \CodeInline{forall\_pred} returns true if and only if \CodeInline{pred}
  is true for all elements ;
\item \CodeInline{exists\_pred} returns true if and only if \CodeInline{pred}
  is true for at least one element ;
\item \CodeInline{none\_pred} returns true if and only if \CodeInline{pred}
  is false for all elements ;
\item \CodeInline{some\_not\_pred} returns true if and only if \CodeInline{pred}
  is false for at least one element.
\end{itemize}


The two last functions should be written by just calling the two first ones.


\levelFourTitle{Non-mutating: Equality of ranges}


Write, specify and prove the function \CodeInline{equal} that returns true if
and only if two ranges of values equal. Write, using the equal function, the
code of \CodeInline{different} that returns true if two ranges are different,
your postcondition should contain an existential quantifier.


\CodeBlockInput[7][13]{c}{ex-2-equal.c}


\levelFourTitle{Binary Search}
\label{l4:statements-loops-ex-bsearch}


The following function searches for the position of a particular value in an
array, assuming that this array is sorted. First, let us consider that the
length of the array is an int and use int values to deal with indexes. One
could note that there are two behavior: either the value exists in the array
(and the result is the index of this value) or not (and the result is -1).


\CodeBlockInput[5]{c}{ex-3-binary-search.c}


This function is a bit tricky to prove, so let us provide some hints. First,
this time the length is provided as an int, so we need to further restrict
this value to make it coherent. Second, one of the invariant properties
should state the bounds of \CodeInline{low} and \CodeInline{up}, but note
that for both of them one of the bounds is not needed. Finally, the second
invariant property should state that if some index of the array stores the
value, this index must be correctly bounded.


\textbf{Harder:} Modify this function in order to receive \CodeInline{len}
as a \CodeInline{size\_t}. You will have to slightly modify the algorithm
and the proof. As a hint, we advise making \CodeInline{up} an excluded
bound for the search.


\levelFourTitle{Mutating: Addition of vectors}


Write, specify and prove the function that adds two vectors into a third
one. Put arbitrary constraints to solve the integer overflow. Consider that
the resulting vector is entirely separated from the input vectors, however
the same vector should be usable for both input vectors.


\CodeBlockInput[7][9]{c}{ex-4-add-vectors.c}


\levelFourTitle{Mutating: Reverse}


Write, specify and prove the function that reverses a vector in place. Take
care of the unmodified part of the vector at some iteration of the loop.
Use the previously proved \CodeInline{swap} function.


\CodeBlockInput[7][11]{c}{ex-5-reverse.c}


\levelFourTitle{Mutating: Copy}


Write, specify and prove the function \CodeInline{copy} that copies a range
of values into another array, starting from the first cell of the array.
First consider (and specify) that the two ranges are entirely separated.


\CodeBlockInput[7][9]{c}{ex-6-copy.c}


\textbf{Harder:} The true copy and copy backward functions


In fact, a strong separation is not necessary. The actual precondition must
guarantee that if the two arrays overlap, the beginning of the destination
range must not be in the source range:


\begin{CodeBlock}{c}
//@ requires \separated(&src[0 .. len-1], dst) ;
\end{CodeBlock}


In essence, by copying the element in that order, we can shift elements from
the end of a particular range to the beginning. However, that means that we
have to be more precise in our contract: we do not guarantee anymore an equality
with the values of the source array but with the \emph{old} values of the
source array. And we have also to be more precise in our invariant, first by
also specifying the relation in regard to the previous state of the memory, and
second by adding an invariant that shows that the source array is not modified
from the \CodeInline{i}$^{th}$ we are visiting to the end.


Finally, it is also possible to write a function that copies the elements from
the end to the beginning, in this case, again, arrays can overlap, but the
condition is not exactly the same. Write, specify and prove the function
\CodeInline{copy\_backward} that copies elements in the reverse order.

\levelThreeTitle{Example with read-only arrays}
\label{l3:statements-loops-examples-ro}


The array is the most common data structure when we are working with loops.
It is then a good example base to exercise with loops, and these
examples allow to rapidly show interesting invariant and will allow us
to introduce some important ACSL constructs.


We can for example illustrate with the search function that allows to find a
value in an array. For now, let us focus on the contract of the function:


\CodeBlockInput[5]{c}{search-0.c}


There are enough ideas in this example to introduce some important syntax.


First, as we previously presented, the
\CodeInline{\textbackslash{}valid\_read} predicate (as well as
\CodeInline{\textbackslash{}valid}) allows us to specify not only the
validity of a readable address but also to state that a range of
contiguous addresses is valid. It is expressed using this syntax:


\begin{CodeBlock}{c}
//@ requires \valid_read(a + (0 .. length-1));
\end{CodeBlock}


This precondition states that all addresses \CodeInline{a+0},
\CodeInline{a+1}, \ldots{}, \CodeInline{a+length-1} must be valid readable
locations.


We also introduced two notations that are used almost all the time in
ACSL, the keywords \CodeInline{\textbackslash{}forall} ($\forall$) and
\CodeInline{\textbackslash{}exists} ($\exists$), the universal logic
quantifiers. The first one allows to state that for any element, some
property is true, the second one allows to say that there exists some
element such that the property is true. If we comment a bit the
corresponding lines in our specification, we can read them this way:


\begin{CodeBlock}{c}
/*@
// for all "off" of type "size_t", IF "off" is comprised between 0 and "length"
//                                 THEN the cell "off" in "a" is different from "element"
\forall size_t off ; 0 <= off < length ==> a[off] != element;

// there exists "off" of type "size_t", such that "off" is comprised between 0 and "length"
//                                      AND the cell "off" in "a" equals to "element"
\exists size_t off ; 0 <= off < length && a[off] == element;
*/
\end{CodeBlock}


If we want to summarize the use of these keywords, we would say that on a
range of values, a property is true, either about at least one of them
or about all of them. A common scheme is to constrain this set using
another property (here: \CodeInline{0 <= off < length}) and to prove the
actual interesting property on this smaller set. \textbf{But using
\texttt{exists} and \texttt{forall} is fundamentally different}.


With \CodeInline{\textbackslash{}forall type a ; p(a) ==> q(a)},
the constraint \CodeInline{p} is followed by an implication.
For any element where a first property \CodeInline{p} is verified, we have
to also verify the second property \CodeInline{q}. If we use a conjunction,
as we do for ``exists'' (which we will later explain), that would mean that
all elements verify both \CodeInline{p} and \CodeInline{q}. In our previous
example, it is clearly not the case as not all integers are comprised between
0 and \CodeInline{length}. Sometimes, it could be what we want to express, but
it would then not correspond anymore to the idea of constraining a set for which
we want to verify some other property.


With \CodeInline{\textbackslash{}exists type a ; p(a) \&\& q(a)}, the
constraint \texttt{p} is followed by a conjunction. We say there exists
an element such that it satisfies the property \texttt{p} at the same
time it also satisfies \texttt{q}. If we use an implication, as we do
for \CodeInline{\textbackslash{}forall}, such an expression will always be true
if \texttt{p} is not a tautology! Why? Is there an ``a'' such that p(a) implies
q(a)? Let us take any ``a'' such that p(a) is false, then the implication is true.


Notice that in this example, the \CodeInline{assumes} clause of the
\CodeInline{in} behavior is exactly the negation of the \CodeInline{assumes}
clause of the \CodeInline{notin} behavior, this is the reason why the
\CodeInline{disjoint} and \CodeInline{complete} clauses are proved, in fact it
could have been expressed with:


\begin{CodeBlock}{c}
  /*@ ...
    behavior in:
      assumes !(\forall size_t off ; 0 <= off < length ==> array[off] != element);
    ...
  */
\end{CodeBlock}


Now, let us talk about the loop annotations. The first step is to add the
\CodeInline{loop assigns} clause. Here, it is simple: the loop only modifies the
variable \CodeInline{i}. Thus, we need to bound \CodeInline{i}, it goes from
\CodeInline{0} to \CodeInline{length}, this is our first loop invariant:
\CodeInline{0 <= i <= length}. That also gives us the loop variant:
\CodeInline{length - i}. Now, we can provide our ``main'' loop invariant. Here,
it is related to the \CodeInline{assumes} clauses, and not to the
\CodeInline{ensures} clauses. In particular, the interesting part of the loop is
the fact that unless we meet the element that we search, it is not in the array,
we exploit this starting from our \CodeInline{notin} assumes:
\begin{CodeBlock}{c}
  //@ \forall size_t off ; 0 <= off < length ==> array[off] != element;
\end{CodeBlock}
The variable that reaches \CodeInline{length} at the end of the loop is
\CodeInline{i}, thus:
\begin{CodeBlock}{c}
  //@ loop invariant \forall size_t off ; 0 <= off < i ==> array[off] != element;
\end{CodeBlock}
is certainly a good candidate. Thus, we have the following loop annotations:


\CodeBlockInput[19]{c}{search-0.c}


And indeed, our final loop invariant defines the treatment performed by our
loop, it indicates to WP what happens in the loop (or more precisely: what we
learn) along the execution. Here, this formula indicates that at each iteration
of the loop, we know that for each memory location between 0 and the next
location to visit (\CodeInline{i} excluded), the memory location contains a
value different from the element we are looking for.


The verification condition associated to the preservation of this invariant is
a bit complex, and it is not interesting to precisely look at it,
on the contrary, the proof that the invariant is established before
executing the loop is interesting:


\image{trivial}


We note that this property, while quite complex, is proved easily
by Qed. If we look at the parts of the programs on which the proof
relies, we can see that the instruction \CodeInline{i = 0} is highlighted
and this is, indeed, the last instruction executed on \CodeInline{i} before
we start the loop. And consequently, if we replace the value of
\CodeInline{i} by 0 inside the formula of the invariant, we get:


\begin{CodeBlock}{c}
//@ loop invariant \forall size_t j; 0 <= j < 0 ==> array[j] != element;
\end{CodeBlock}


``For all j, greater or equal to 0 and strictly lower than 0'', this
part of the formula is necessarily false, our implication is then
necessarily true.


\levelThreeTitle{Examples with mutable arrays}


Let us present two examples with mutation of arrays. One with a mutation
of all memory locations, the other with selective modifications.


\levelFourTitle{Reset}


Let us have a look at the function that resets an array of integers to 0.


\CodeBlockInput{c}{reset.c}


This time, the loop does modify the content of the array, thus we have to
provide this in the \CodeInline{loop assigns} clause. Notice that we can use
the notation \CodeInline{n .. m} for this. Furthermore, we do not say that
the loop assigns the content from \CodeInline{0} to \CodeInline{i-1} (since
\CodeInline{i} is modified, and WP cannot exploit this) but from \CodeInline{0}
to \CodeInline{length-1}, this is less precise, but it can be used by WP out the
loop. Finally, this time we directly relate the invariant to the postcondition,
the goal of the function is to reset the array from 0 to \CodeInline{length}, at
a given iteration, the loop has done it from 0 to \CodeInline{i}.


\levelFourTitle{Search and replace}
\label{l4:statements-loops-ex-search-and-replace}


The last example we will detail to illustrate the proof of functions
with loops is the algorithm ``search and replace''. This algorithm
selectively modifies values in a range of memory locations. It is
generally harder to guide the tool in such a case, because on one hand
we must keep track of what is modified and what is not, and on the other
hand, the induction relies on this fact.


As an example, the first specification and loop annotations we can write for
this function could be this one (which follows basically the same process we
had for the previous example):


\CodeBlockInput{c}{search_and_replace-0.c}


We use the logic function \CodeInline{\textbackslash{}at(v, Label)} that gives
us the value of the variable \CodeInline{v} at the program point \CodeInline{Label}.
If we look at the usage of this function here, we see that in the
invariant we try to establish a relation between the old values of the
array and the potentially new values:


\begin{CodeBlock}{c}
/*@
  loop invariant \forall size_t j; 0 <= j < i && \at(array[j], Pre) == old
                   ==> array[j] == new;
  loop invariant \forall size_t j; 0 <= j < i && \at(array[j], Pre) != old
                   ==> array[j] == \at(array[j], Pre);
*/
\end{CodeBlock}



For every memory location that contained the value to be replaced, it now must
contain the new value. All other values must remain unchanged. While we
previously relied on the \CodeInline{assigns} clauses for this kind of properties,
here, we do not have this possibility because while the ACSL language provides
some facilities that we could use to write a very precise \CodeInline{assigns}
clause, WP would not perfectly exploit it. Thus, we have to use an invariant and
keep an approximation of the assigned memory location.


If we try to prove this invariant with WP, it fails, because the
\CodeInline{assigns} specification is not precise enough. Thus, in such a case,
we provide an additional loop invariant to detail in the assigned range what are
the locations that have not been modified yet by the loop at some iteration:


\begin{CodeBlock}{c}
for(size_t i = 0; i < length; ++i){
    //@assert array[i] == \at(array[i], Pre); // proof failure
    if(array[i] == old) array[i] = new;
}
\end{CodeBlock}


We can add this information as an invariant:


\CodeBlockInput[13][26]{c}{search_and_replace-1.c}


And this time the proof succeeds.


\levelThreeTitle{Exercises}


For all these exercises, use the following command line to start verification:

\begin{CodeBlock}{bash}
frama-c-gui -wp -wp-rte -warn-unsigned-overflow -warn-unsigned-downcast your-file.c
\end{CodeBlock}


\levelFourTitle{Non-mutating: For all, Exists, ...}


Currently, function pointers are not supported by WP. Let us consider that
we have a function:


\CodeBlockInput[7][13]{c}{ex-1-forall-exists.c}


Write your own code and corresponding postcondition and then write the
following functions together with their contract and prove their correctness.
Note that, as \CodeInline{pred} is a C function, it cannot be used in the
specification of the functions you have to write, thus the property that you
have considered for this predicate should be inlined in the contract of the
different functions.


\begin{itemize}
\item \CodeInline{forall\_pred} returns true if and only if \CodeInline{pred}
  is true for all elements ;
\item \CodeInline{exists\_pred} returns true if and only if \CodeInline{pred}
  is true for at least one element ;
\item \CodeInline{none\_pred} returns true if and only if \CodeInline{pred}
  is false for all elements ;
\item \CodeInline{some\_not\_pred} returns true if and only if \CodeInline{pred}
  is false for at least one element.
\end{itemize}


The two last functions should be written by just calling the two first ones.


\levelFourTitle{Non-mutating: Equality of ranges}


Write, specify and prove the function \CodeInline{equal} that returns true if
and only if two ranges of values equal. Write, using the equal function, the
code of \CodeInline{different} that returns true if two ranges are different,
your postcondition should contain an existential quantifier.


\CodeBlockInput[7][13]{c}{ex-2-equal.c}


\levelFourTitle{Binary Search}
\label{l4:statements-loops-ex-bsearch}


The following function searches for the position of a particular value in an
array, assuming that this array is sorted. First, let us consider that the
length of the array is an int and use int values to deal with indexes. One
could note that there are two behavior: either the value exists in the array
(and the result is the index of this value) or not (and the result is -1).


\CodeBlockInput[5]{c}{ex-3-binary-search.c}


This function is a bit tricky to prove, so let us provide some hints. First,
this time the length is provided as an int, so we need to further restrict
this value to make it coherent. Second, one of the invariant properties
should state the bounds of \CodeInline{low} and \CodeInline{up}, but note
that for both of them one of the bounds is not needed. Finally, the second
invariant property should state that if some index of the array stores the
value, this index must be correctly bounded.


\textbf{Harder:} Modify this function in order to receive \CodeInline{len}
as a \CodeInline{size\_t}. You will have to slightly modify the algorithm
and the proof. As a hint, we advise making \CodeInline{up} an excluded
bound for the search.


\levelFourTitle{Mutating: Addition of vectors}


Write, specify and prove the function that adds two vectors into a third
one. Put arbitrary constraints to solve the integer overflow. Consider that
the resulting vector is entirely separated from the input vectors, however
the same vector should be usable for both input vectors.


\CodeBlockInput[7][9]{c}{ex-4-add-vectors.c}


\levelFourTitle{Mutating: Reverse}


Write, specify and prove the function that reverses a vector in place. Take
care of the unmodified part of the vector at some iteration of the loop.
Use the previously proved \CodeInline{swap} function.


\CodeBlockInput[7][11]{c}{ex-5-reverse.c}


\levelFourTitle{Mutating: Copy}


Write, specify and prove the function \CodeInline{copy} that copies a range
of values into another array, starting from the first cell of the array.
First consider (and specify) that the two ranges are entirely separated.


\CodeBlockInput[7][9]{c}{ex-6-copy.c}


\textbf{Harder:} The true copy and copy backward functions


In fact, a strong separation is not necessary. The actual precondition must
guarantee that if the two arrays overlap, the beginning of the destination
range must not be in the source range:


\begin{CodeBlock}{c}
//@ requires \separated(&src[0 .. len-1], dst) ;
\end{CodeBlock}


In essence, by copying the element in that order, we can shift elements from
the end of a particular range to the beginning. However, that means that we
have to be more precise in our contract: we do not guarantee anymore an equality
with the values of the source array but with the \emph{old} values of the
source array. And we have also to be more precise in our invariant, first by
also specifying the relation in regard to the previous state of the memory, and
second by adding an invariant that shows that the source array is not modified
from the \CodeInline{i}$^{th}$ we are visiting to the end.


Finally, it is also possible to write a function that copies the elements from
the end to the beginning, in this case, again, arrays can overlap, but the
condition is not exactly the same. Write, specify and prove the function
\CodeInline{copy\_backward} that copies elements in the reverse order.

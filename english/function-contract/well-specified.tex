
\levelThreeTitle{Correctly write what we expect}


This is certainly the hardest part of our work. Programming is already
an effort that consists in writing algorithms that correctly answer to
our need. Specifying requests the same kind of work, except that we do
not try to express \emph{how} we answer to our need but \emph{what} is
exactly our need. To prove that our code implements what we need, we
must be able to describe exactly what we need.

From now, let us use another example, the \texttt{max} function:



\CodeBlockInput[4][6]{c}{max-1.c}



The reader could write and prove their own specification. Let us
use this one:



\CodeBlockInput[1][6]{c}{max-1.c}



If we ask WP to prove this code, it succeeds without any problem.
However, is our specification really what we need? We can try to prove this
calling code:



\CodeBlockInput[8][14]{c}{max-1.c}



There, it fails. In fact, we can go further by modifying the body of
the \CodeInline{max} function and notice that the following code is also
correct with respect to the specification:



\CodeBlockInput[1][8]{c}{max-2.c}



While being a correct specification of \CodeInline{max}, our specification
is however too permissive. We have to be more precise. We do
not only expect the result to be greater or equal to both parameters,
but also that the result is one of them:



\CodeBlockInput[1][7]{c}{max-3.c}


This specification can also be proved correct by WP, but now we can also prove
the assertion in our function \CodeInline{foo}, and we cannot prove anymore an
implementation that would just return the value \CodeInline{INT\_MAX}.


\levelThreeTitle{Inconsistent preconditions}


Producing a correct specification is crucial. Typically, by stating a
false precondition, we can have the possibility to create a proof of
false:


\CodeBlockInput[6]{c}{bad-precond.c}


If we ask WP to prove this function, it will accept it without any problem
since the property we ask in precondition is necessarily false.
However, we will not be able to give an input that satisfies the
precondition.


For this particular kind of inconsistencies, a useful feature of WP is
the ``smoke tests'' option of the plugin. It can be used to detect that
some preconditions are unsatisfiable. For example, here, we can run this
command line:


\begin{CodeBlock}{bash}
  frama-c-gui -wp -wp-smoke-tests file.c
\end{CodeBlock}


and we obtain the following result in the GUI:


\image{2-bad-precond}


We can see a red and orange bullet on the precondition of the function,
meaning that if there exists a reachable call to this function the
precondition will be necessarily violated, and a red bullet in the
list of goals, indicating that a prover succeeded in proving that this
precondition is inconsistent.


Note that when the smoke tests succeed, for example if we fix the
precondition like this:
\image{2-smoke-success}
it does not necessarily mean that the precondition is consistent, just
that the prover was unable to prove that it was inconsistent.


Some notions we will see in this tutorial can expose us to the
possibility to introduce subtle incoherence. So, we must always be
careful when specifying a program.


\levelThreeTitle{Pointers}


If there is one notion that we permanently have to confront with in C
language, this is definitely the notion of pointer. Pointers are quite
hard to manipulate correctly, and they still are the main source of
critical bugs in programs, so they benefit of a preferential treatment
in ACSL. In order to produce a correct specification of programs using
pointers, it is necessary to detail the configuration of the considered
memory.

We can illustrate with a swap function for C integers (this version can
only be verified without RTE checking for now):



\CodeBlockInput[5]{c}{swap-0.c}



\levelFourTitle{History of values in memory}


Here, we introduce a first built-in logic function of ACSL:
\CodeInline{\textbackslash{}old}, that allows us to get the old (that
is to say, before the call) value of a given element. So, our specification
defines that the function must ensure that after the call, the value of
\CodeInline{*a} is the old  value of \CodeInline{*b} and conversely.

The \CodeInline{\textbackslash{}old} function can only be used in the
postcondition of a function. If we need this kind of information
somewhere else, we use \CodeInline{\textbackslash{}at} that allows us
to express that we want the value of a variable at a particular program
point. This function receives two parameters. The first one is the variable
(or memory location) for which we want to get its value and the second one
is the program point (as a C label) that we want to consider.

For example, we could write:



\CodeBlockInput[2][6]{c}{at.c}



Of course, we can use any C label in our code, but we also have 6
built-in labels defined by ACSL that can be used:



\begin{itemize}
\item \CodeInline{Pre}/\CodeInline{Old}: value before function call,
\item \CodeInline{Post}: value after function call,
\item \CodeInline{LoopEntry}: value at loop entry
\item \CodeInline{LoopCurrent}: value at the beginning of the current step of
  the loop,
\item \CodeInline{Here}: value at the current program point.
\end{itemize}


\begin{Information}
  The behavior of \CodeInline{Here} is, in fact, the default behavior when we
  consider a variable. Its use with \CodeInline{\textbackslash{}at}
  generally allows us to ensure that what we write is not ambiguous, and is more
  readable, when we express properties about values at different program
  points in the same expression.
\end{Information}


Whereas \CodeInline{\textbackslash{}old} can only be used in function
postconditions, \CodeInline{\textbackslash{}at} can be used anywhere.
However, we cannot use any program point with respect to the type
annotation we are writing. \CodeInline{Old} and \CodeInline{Post} are only
available in function postconditions, \CodeInline{Pre} and \CodeInline{Here}
are available everywhere. \CodeInline{LoopEntry} and \CodeInline{LoopCurrent}
are only available in the context of loops (which we will detail later
in this tutorial).



Note that one must take care to use \CodeInline{\textbackslash{}old} and
\CodeInline{\textbackslash{}at} for values that make sense. This is why
for example in a contract, all values received in input are put into
\CodeInline{\textbackslash{}old} when used in postcondition, the ``new''
value of the input variables not make any sense for the caller of the
function as they are not accessible: they are local to the called function.
For example, if we have look at the contract of the swap function transformed
by Frama-C, we can see that in the postcondition, the pointers are enclosed into
\CodeInline{\textbackslash{}old}:


\image{2-old-swap}


For the built-in \CodeInline{\textbackslash{}at} function, we have to take care
of this more explicitly. In particular,
the specified label must make sense with respect to the scope of the
value. For example, in the following program, Frama-C detects that we ask
the value of the variable \CodeInline{x} at a program point where it does
not exist:


\CodeBlockInput[5][9]{c}{at-2.c}


\image{2-at-1.png}


However, in some other cases, we only reach a proof failure since determining
that the value does not exist at some particular label cannot be done by
a syntactic analysis. For example, if the variable is declared but undefined
or if we want the value of a pointed value:


\CodeBlockInput[11][23]{c}{at-2.c}


Here, it is easy to see the problem. However, the considered label is
propagated to sub-expressions. Thus, sometimes, terms that seem to be
innocent can have a surprising behavior if we do not keep this fact in mind.
For example, in the following example:


\CodeBlockInput[25][30]{c}{at-2.c}


The first assertion is proved, and while the second assertion seems to
express the same property, it cannot be proved. Because, it in fact does not
express the same property. The expression
\CodeInline{\textbackslash{}at(x[*p], Pre)} must be seen as
\CodeInline{\textbackslash{}at(x[\textbackslash{}at(*p)], Pre)}, since the
label is propagated to the sub-expression \CodeInline{*p}, for which we do
not know the value at label \CodeInline{Pre} (since it is not specified).


For the moment, we do not need \CodeInline{\textbackslash{}at}, but it can
often be useful, if not essential, when we want to make our
specification precise.



\levelFourTitle{Pointers validity}


If we try to prove that the swap function is correct (comprising the
verification of absence of runtime errors), our postcondition is indeed verified
but WP fails to prove that some runtime-error cannot happen, since we perform
access to some pointers that we did not indicate to be valid pointers in the
precondition of the function.

We can express that the dereferencing of a pointer is valid using the
\CodeInline{\textbackslash{}valid} predicate of ACSL which receives the
pointer in input:



\CodeBlockInput[3][11]{c}{swap-1.c}



Once we have specified that the pointers we receive in input must be valid,
dereferencing is assured to not produce undefined behaviors.



As we will see later in this tutorial, \CodeInline{\textbackslash{}valid}
can take more than one pointer in parameter. For example, we can give it
an expression such as: \CodeInline{\textbackslash{}valid(p + (s .. e))} which means
``for all \CodeInline{i} between included \CodeInline{s} and \CodeInline{e},
\CodeInline{p+i} is a valid pointer''. This kind of expression is
extremely useful when we have to specify properties about arrays in
specifications.



If we have a closer look at the assertions that RTE adds in the swap function
when we ask the verification of absence of runtime errors, we can notice that
there exists another version of the \CodeInline{\textbackslash{}valid}
predicate, denoted \CodeInline{\textbackslash{}valid\_read}. As opposed to
\CodeInline{\textbackslash{}valid}, the predicate
\CodeInline{\textbackslash{}valid\_read} indicates that a pointer can be
dereferenced, but only to read the pointed memory. This subtlety is due to the C
language, where the downcast of a const pointer is easy to write but is not
necessarily legal.



Typically, in this code:



\CodeBlockInput{c}{unref.c}



Dereferencing \CodeInline{p} is valid, however the precondition of
\CodeInline{unref} is not verified by WP since dereferencing
\CodeInline{value} is only legal for a read-access. A write access would
result in an undefined behavior. In such a case, we can specify that the
pointer \CodeInline{p} must be \CodeInline{\textbackslash{}valid\_read} and not
\CodeInline{\textbackslash{}valid}.



\levelFourTitle{Side Effects}


Our \CodeInline{swap} function is provable with regard to the specification
and potential runtime errors, however is our specification precise enough? We
can slightly modify our code to check this (we use \CodeInline{assert} to
verify some properties at some particular points):



\CodeBlockInput{c}{swap-1.c}



The result is not exactly what we expect:



\image{2-swap-1}


Indeed, we did not specify the allowed side effects for our function. In
order to specify side effects, we use an \CodeInline{assigns} clause which is
part of the postcondition of a function. It allows us to specify which
\textbf{non-local} elements (we verify side effects) can be modified
during the execution of the function.



By default, WP considers that a function can modify everything in
memory. So, we have to specify what can be modified by a function. For
example, our \CodeInline{swap} function can be specified to modify the
values pointed by the received pointers:



\CodeBlockInput[3][14]{c}{swap-2.c}



If we ask WP to prove the function with this specification, it is
validated (including with the variable added in the previous source
code).



Finally, we sometimes want to specify that a function is side effect
free. We specify this by giving \CodeInline{\textbackslash{}nothing} to
\CodeInline{assigns}:



\CodeBlockInput[3]{c}{assigns-nothing.c}



The careful reader will now be able to take back the examples we
presented until now to integrate the right \CodeInline{assigns} clause.



\levelFourTitle{Memory location separation}


Pointers bring the risk of aliasing (multiple pointers can have access
to the same memory location). For some functions, it does not cause any
problem, for example when we give two identical pointers to the
\texttt{swap} function, the specification is still verified. However,
sometimes it is not that simple:



\CodeBlockInput[5]{c}{incr_a_by_b-0.c}



If we ask WP to prove this function (let us ignore the verification of absence
of RTEs for this example), we get the following result:



\image{2-incr_a_by_b-1}


The reason is simply that we do not have any guarantee that the pointer
\CodeInline{a} is different of the pointer \CodeInline{b}. Now, if these
pointers are the same,



\begin{itemize}
\item   the property \CodeInline{*a == \textbackslash{}old(*a) + *b} in fact
  means \CodeInline{*a == \textbackslash{}old(*a) + *a} which can only
  be true if the old value pointed by \CodeInline{a} was $0$, and we do
  not have such a requirement,
\item
  the property \CodeInline{*b == \textbackslash{}old(*b)} is not validated
  because we potentially modify this memory location.
\end{itemize}


\begin{Question}
  Why is the \CodeInline{assigns} clause validated?

  The reason is simply that \CodeInline{a} is indeed the only modified memory
  location. If \CodeInline{a != b}, we only modify the location pointed by
  \CodeInline{a}, and if \CodeInline{a == b}, this is still the case:
  \CodeInline{b} is not another location.
\end{Question}


In order to ensure that pointers refer to separated memory locations,
ACSL provides the predicate
\texttt{\textbackslash{}separated(p1,\ ...,pn)} that receives in
parameter a set of pointers and is true if and only if these pointers are
non-overlapping. Here, we specify:



\CodeBlockInput[5]{c}{incr_a_by_b-1.c}



And this time, the function is verified:



\image{2-incr_a_by_b-2}


We can notice that we do not consider the arithmetic overflow here, as
we do not focus on this question in this section. However, if this
function was part of a complete program, it would be necessary to define
the context of use of this function and the precondition guaranteeing
the absence of overflow.


\levelThreeTitle{Writing the right contract}


Writing a specification that is precise enough can sometimes be a bit tricky.
Interestingly, a good way to check if a specification is precise enough is to
write tests. And in fact, this is basically what we have done for our examples
\CodeInline{max} and \CodeInline{swap}. We have written a first version of the
specification, and we have written some code with a call to the corresponding
function to determine whether we could prove some properties that we expected to
be easily provable from the contract of the function.



The most important
idea is to determine the contract without taking in account the content of the
function (at least, in a first step). Indeed, we are trying to prove the
function, but maybe it contains a bug, so if we write the contract taking in
account too directly its code, we have a risk to introduce the same bug, for
example taking in account an erroneous conditional structure. In fact, it is
generally a good practice to work with someone else. One specifies the function
and the other implements it (even if they previously agreed on a common textual
specification).



Once the contract have been stated, we work on the specifications that are due
to the constraints of our language and our hardware. That mostly concerns the
precondition of the function. For example, the absolute value does not really
have a precondition, this is our hardware that adds the condition we have given
in precondition due to the two's complement on which it relies. As we will see
in the chapter~\ref{l1:proof-methodologies}, verifying the absence of runtime
errors can also impact the postcondition. For now, let us leave this.



\levelThreeTitle{Exercises}


\levelFourTitle{Division and remaining}



Specify the postcondition of the following function, that computes the
results of the division of \CodeInline{a} by \CodeInline{b} and its
remaining and stores it into two memory locations \CodeInline{p} and
\CodeInline{q}:


\CodeBlockInput[5]{c}{ex-1-div-rem.c}

Run the command:

\begin{CodeBlock}{bash}
frama-c-gui your-file.c -wp
\end{CodeBlock}


Once the function is successfully proved to respect the contract, run:


\begin{CodeBlock}{bash}
frama-c-gui your-file.c -wp -wp-rte
\end{CodeBlock}


If it fails, complete the contract by adding the right precondition.



\levelFourTitle{Reset on condition}



Provide a contract for the following function that resets its first parameter
if and only if the second is true. Be sure to express that the second parameter
remains unmodified:


\CodeBlockInput[5]{c}{ex-2-reset-on-cond.c}


Run the command:


\begin{CodeBlock}{bash}
frama-c-gui your-file.c -wp -wp-rte
\end{CodeBlock}



\levelFourTitle{Addition of pointed values}


The following function receives two pointers as an input and returns the
sum the pointed values. Write the contract of this function:



\CodeBlockInput[5]{c}{ex-3-add-ptr.c}



Run the command:



\begin{CodeBlock}{bash}
frama-c-gui your-file.c -wp -wp-rte
\end{CodeBlock}



Once the function and calling code are successfully proved, modify the
signature of the function add as follows:



\begin{CodeBlock}{c}
void add(int* p, int* q, int* r);
\end{CodeBlock}


Now the result of the sum should be stored at \CodeInline{r}. Accordingly
modify the calls in the main function as well as the code and the contract
of \CodeInline{add}. \CodeInline{*p} and \CodeInline{*q} should remain
unchanged.



\levelFourTitle{Maximum of pointed values}



The following function computes the maximum of the values pointed by
\CodeInline{a} and \CodeInline{b}. Write the contract of the function:



\CodeBlockInput[5]{c}{ex-4-max-ptr.c}



Run the command:



\begin{CodeBlock}{bash}
frama-c-gui your-file.c -wp -wp-rte
\end{CodeBlock}



Once it is proved, modify the signature of the function as follows:



\begin{CodeBlock}{c}
void max_ptr(int* a, int* b);
\end{CodeBlock}


Now the function should ensure that after its execution \CodeInline{*a}
contains the maximum of the input value, and \CodeInline{*b} contains the
other value. Modify the code accordingly as well as the contract. Note that
the variable \CodeInline{x} is not necessary anymore in the \CodeInline{main}
function and that we can change the assertion on line 15 to reflect the
new behavior of the function.



\levelFourTitle{Order 3 values}



The following function should order the 3 input values in increasing order.
Write the corresponding code and specification of the function:


\CodeBlockInput[5]{c}{ex-5-order-3.c}


And run the command:


\begin{CodeBlock}{bash}
frama-c-gui your-file.c -wp -wp-rte
\end{CodeBlock}


Remember that ordering values is not just ensuring that resulting values
are sorted increasing order that each pointed value must be one of the original
ones. All original values should still be there after the sorting operation:
new values are a permutation of the original ones. To express this idea, one
can rely on the set datatype. For example, this property is true:


\begin{CodeBlock}{c}
//@ assert { 1, 2, 3 } == { 2, 3, 1 };
\end{CodeBlock}


We can use this to express that the set of original values and final values
is the same. However, that is not the only thing to consider, as a set only
contains one occurrence of each value. So, if \CodeInline{*a == *b == 1},
\CodeInline{\{ *a, *b \} == \{ 1 \}}. Thus, we also have to consider three
other particular cases:


\begin{itemize}
\item all original values equal
\item two original values equal and the last is greater
\item two original values equal and the last is lower
\end{itemize}


That should set one more constraint on the final values.


As an helper, one could use the following test program:


\CodeBlockInput[27][43]{c}{ex-5-order-3-answer.c}


If the specification is precise enough, each assertion should be proved.
However, that does not mean that all cases have been considered by our tests,
so do not hesitate do add other cases.

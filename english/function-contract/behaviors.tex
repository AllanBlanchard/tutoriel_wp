Sometimes, a function can have behaviors that can be quite different
depending on the input. For instance, a function can receive a pointer to
an optional resource: if the pointer is \CodeInline{NULL}, we have a
certain behavior, which is different from the behavior expected when
the pointer is not \CodeInline{NULL}.

We have already seen a function that has different behaviors: the
\CodeInline{abs} function. Let us use it again to illustrate behaviors. We
have two behaviors for the \CodeInline{abs} function: either the input is
positive or it is negative.

Behaviors allow us to specify the different cases for postconditions. We
introduce them using the \CodeInline{behavior} keyword. Each behavior
is named. For a given behavior, we have different assumptions about the
input of the function, they are introduced with the clause
\CodeInline{assumes} (note that since they characterize the input, the
keyword \CodeInline{\textbackslash{}old} cannot be used there).
The properties expressed by this clause do not \emph{have} to be verified before
the call; but they \emph{can} be verified, and in this latter case, the postcondition
associated with this behavior applies. The postconditions of a particular
behavior are introduced using an \CodeInline{ensures} clause. Finally, we can ask WP
to verify that behaviors are disjoint (to guarantee determinism) and
complete (to guarantee that we cover all possible input).

Behaviors are disjoint if for any (valid) input of the function, it
corresponds to the assumption (\CodeInline{assumes}) of a single behavior.
Behaviors are complete if any (valid) input of the function corresponds
to at least one behavior.

For example, for \CodeInline{abs} we can write the specification:



\CodeBlockInput{c}{abs.c}



Note that declaring behaviors does not forbid you from specify global postconditions.
For example, in the above code, we have specified that for any behavior, the function must
return a positive value.



Let us now slightly modify the assumptions of each behavior to illustrate
the meaning of \CodeInline{complete} and \CodeInline{disjoint}:
\begin{itemize}
\item
  replace the assumption of \CodeInline{pos} with
  \CodeInline{val \textgreater{} 0}, in this case, behaviors are
  disjoint but incomplete (we miss \CodeInline{val == 0}),
\item
  replace the assumption of \CodeInline{neg} with
  \CodeInline{val \textless{}= 0}, in this case, behaviors are
  complete but not disjoint (we have two assumptions corresponding
  to \CodeInline{val == 0}).
\end{itemize}


\begin{Warning}
  Even if \CodeInline{assigns} is a postcondition, indicating different assigns
  in each behavior is currently not well-handled by WP. If we need to specify
  this, we will:
  \begin{itemize}
  \item put our \CodeInline{assigns} before the behaviors (as we have done in our
    example) with all potentially modified non-local elements,
  \item add as a postcondition of each behavior those elements that are in fact
    not modified by declaring their new value to be equal to the
    \CodeInline{\textbackslash{}old} one.
  \end{itemize}
\end{Warning}


Behaviors are useful to simplify the writing of specifications when
functions can have very different behaviors depending on their input.
Without them, specification would be defined using implications
expressing the same idea but harder to write and read (which would be
error-prone). In addition, the translation of completeness and
disjointedness would need to be written by hand which would be
tedious and again error-prone.


\levelThreeTitle{Exercises}


\levelFourTitle{Previous exercises}


From previous sections, we revisit the examples:
\begin{itemize}
\item about the computation of the distance between to integers,
\item ``reset on condition'',
\item ``days of the month'',
\item ``Max of pointed-to values''.
\end{itemize}


Considering that the contracts were:


\CodeBlockInput{c}{ex-1-past.c}


Re-write them using behaviors.


\levelFourTitle{Two other simple exercises}


Complete the code and the specification of the two following functions
and prove them. The specification should make use of behaviors.


First, a function that returns if a character is a vowel or a consonant,
one should assume (and express) that the function receives a lowercase
letter.


\CodeBlockInput[5][9]{c}{ex-2-simple.c}


Then, a function that returns the quadrant of a given coordinate. When
the coordinate is on an axis, arbitrarily choose one of the quadrants.


\CodeBlockInput[11][13]{c}{ex-2-simple.c}


\levelFourTitle{Triangle}


Complete the following functions that receive the lengths of the different
sides of a triangle, and respectively return:
\begin{itemize}
\item if the triangle is scalene, isosceles, or equilateral,
\item if the triangle is right, acute or obtuse.
\end{itemize}


\CodeBlockInput[5]{c}{ex-3-triangle.c}


Assuming (and it must be expressed) that:


\begin{itemize}
\item the received parameters indeed define a triangle,
\item \CodeInline{a} is the hypotenuse of the triangle,
\end{itemize}


specify and prove that the functions do the right job.


\levelFourTitle{Max of pointed-to values, returning the result}


Let's look again at the example ``Max of pointed-to values'' from the previous section,
and more precisely, the version that returns the result. Considering that
the contract was:



\CodeBlockInput[1][10]{c}{ex-4-max-ptr.c}



\begin{enumerate}
\item Rewrite it using behaviors,
\item Modify the contract of 1 in order to make the behaviors non-disjoint,
  excepting this property, the contract should remain verified,
\item Modify the contract of 1 in order to make the behaviors incomplete,
  add a new behavior that makes the contract complete again,
\item Modify the function of 1 in order to accept \CodeInline{NULL} pointers
  for both \CodeInline{a} and \CodeInline{b}. If both of them are null pointers,
  return \CodeInline{INT\_MIN}, if one is a null pointer, return the value of
  the other, else, return the maximum of the two. Modify the contract accordingly
  by adding new behaviors. Be sure that they are disjoint and complete.
\end{enumerate}


\levelFourTitle{Order 3}


Let's go back to the example ``Order 3 values'' from the previous section. Considering
that the contract was:


\CodeBlockInput{c}{ex-5-order-3.c}


Rewrite it using behaviors. Note that you should have one general behavior
and 2 specific behaviors. Are these behaviors complete? Are they disjoint?

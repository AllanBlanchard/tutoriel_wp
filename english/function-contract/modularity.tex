For this last part, let us talk about function call composition,
and have a closer look at WP. We will also have a look at the way we can split
our programs in different files when we want to prove them using WP.

Our goal is to prove the \CodeInline{max\_abs} function, that returns
the maximum absolute value of two values:



\CodeBlockInput[6][11]{c}{max_abs.c}



Let us start by (over-)separating the declarations and definitions of the
different functions we need (and have previously proved) into header/source
files, that are \CodeInline{abs} and \CodeInline{max}. We obtain, for
\CodeInline{abs}:



File abs.h :

\CodeBlockInput{c}{abs.h}



File abs.c

\CodeBlockInput{c}{abs.c}



We can notice that we put our function contract in the header file.
The goal is to be able to import the specification at the same time as
the declaration when we need it in another file. Indeed, WP needs the
contract of the function when it is called. First to prove that the
precondition is verified (and thus that the call is legal), and second
to get in return the postcondition that is useful to prove the right
properties after the function call.

We can create a file using the same format for the \texttt{max}
function. In both cases, we can open the source file (we do not need to
specify header files in the command line) with Frama-C and notice that
the specification is indeed associated to the function and that we prove
it.


Now, we can prepare our files for the \texttt{max\_abs} function with
the header:



\CodeBlockInput{c}{max_abs_uns.h}



And its source file:



\CodeBlockInput{c}{max_abs.c}



We can open the source file in Frama-C. If we look at the side panel, we
can see that the header files we have included in \texttt{abs\_max}
correctly appear and if we look at the function contracts for them, we
can see some blue and green bullets:



\image{max_abs}


These bullets indicate that, since we do not have the implementation
of the function, the postcondition of the function is assumed to be
true. It is an important strength of the deductive proof of programs
compared to some other formal methods: functions are verified in
isolation from each other.

When we are not currently performing the proof of a function, its
specification is considered to be correct: we do not try to prove it
when we are proving another function, we only verify that the
precondition is correctly established when we call it. It provides very
modular proofs and specifications that are therefore more reusable. Of
course, if our proof relies on the specification of another function, it
must be provable to ensure that the proof of the program is complete.
But, we can also consider that we trust a function that comes from an
external library that we do not want to prove (or for which we do not
even have the source code).

The careful reader could specify and prove the \CodeInline{max\_abs}
function. A possible answer is provided there:



\CodeBlockInput[4][14]{c}{max_abs.h}



\levelThreeTitle{Exercises}


\levelFourTitle{Days of the month}
\label{l4:contract-modularity-ex-days-of-month}


Specify the function leap year that returns true if the year received
as an input is leap. Use this functions to complete the function
\CodeInline{days\_of} in order to return the number of days of the
month received as an input, including the right behavior when the year
is leap for February.



\CodeBlockInput{c}{ex-1-days-month.c}


\levelFourTitle{Alpha-numeric character}
\label{l4:contract-modularity-ex-alpha-num}


Write and specify the different functions used by
\CodeInline{is\_alpha\_num}. Provide a contract for each of them and
provide the contract of \CodeInline{is\_alpha\_num}.



\CodeBlockInput{c}{ex-2-alphanum.c}


Declare an enumeration with values \CodeInline{LOWER}, \CodeInline{UPPER},
\CodeInline{DIGIT} and \CodeInline{OTHER}, and a function
\CodeInline{character\_kind} that returns, using the different
functions \CodeInline{is\_lower}, \CodeInline{is\_upper},
\CodeInline{is\_digit}, the kind of character received in input. Use
behaviors to specify the contract of this function and be sure that
they are disjoint and complete.




\levelFourTitle{Order 3 values}
\label{l4:contract-modularity-ex-order-3}


Taking back the function \CodeInline{max\_ptr} that orders two values,
putting the maximum at the first location and the minimum at the second,
write a function \CodeInline{min\_ptr} that uses this function and
produces the opposite operation. Use these functions to complete the
four functions that orders 3 values. For each variant (increasing and
decreasing), write it once using only \CodeInline{max\_ptr} and once
using only \CodeInline{min\_ptr}. Write a precise contract for each of
these functions and prove them.



\CodeBlockInput{c}{ex-3-order-3.c}


\levelFourTitle{Give the change}


The goal of this exercise is to write a function that computes the change
on a purchase. The function \CodeInline{make\_change} receives the amount
to pay, the received amount of money, and a buffer to indicate how
many of each note must be returned to the client.


For example, for an amount of 410 and a received amount of 500, the array
should contain 1 in the cell \CodeInline{change[N50]} and 2 in the cell
\CodeInline{change[N20]} after the call to the function.


If the received amount is less than the price, the function should
return -1 (and oppositely, 0 on success).


\CodeBlockInput{c}{ex-4-change.c}


The function \CodeInline{remove\_max\_notes} receives a note value and
what remains to be changed (via a pointer), supposed to be greater than
0. It computes the maximum number of this note that can fit into the
remaining, decreases it accordingly and returns the number of notes.
The function \CodeInline{make\_change} should then make use of this function
to compute the change.


Write the code of these functions and their specification and prove their
correctness. Note that this should not use loop as we still do not know
how to deal with them.


\levelFourTitle{Triangle}


In this exercise, we want to gather the results of the functions we have
written in the previous section to get the properties of triangles into
a structure. The function \CodeInline{classify} receives three lengths
\CodeInline{a}, \CodeInline{b}, and \CodeInline{c}, assuming that
\CodeInline{a} is the hypotenuse. If these values do not correspond to a
triangle, the function should return -1, and 0 if everything is OK. The
properties are collected into a structure \CodeInline{info} received via
a pointer.


\CodeBlockInput{c}{ex-5-triangle.c}


Write, specify and prove all functions.


Note there are quite a lot of behaviors to list and specify. You can try
to write a version that does not require \CodeInline{a} to be the hypotenuse,
but it could be hard to finish the proof automatically with
Alt-Ergo because there are quite a lot of combinations to consider.

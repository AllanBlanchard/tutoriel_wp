The goal of a function contract is to state the properties of the input that are
expected by the function, and in exchange the properties that will be assured
for the output. The expectation of the function is called the
\textbf{precondition}. The properties of the output are called the
\textbf{postcondition}.



These properties are expressed with ACSL. The syntax is relatively
simple if one has already developed in C language since it shares most
of the syntax of boolean expressions in C. However, it also provides:
\begin{itemize}
\item
  some logic constructs and connectors that do not exist in C, to ease
  the writing of specifications,
\item
  built-in predicates to express properties that are useful about C
  programs (for example: a valid pointer),
\item
  as well as some primitive types for the logic that are more general
  than primitive C types (for example: mathematical integer).
\end{itemize}



We will introduce along this tutorial a large part of the notations
available in ACSL.



ACSL specifications are introduced in our source code using annotations.
Syntactically, a function contract is integrated in the source code with
this syntax:



\begin{CodeBlock}{c}
/*@
  //contract
*/
void foo(int bar){

}
\end{CodeBlock}



Notice the \CodeInline{@} at the beginning of the comment block, this
indicates to Frama-C that what follows are annotations and not a comment
block that it should simply ignore.



Now, let us have a look at the way we express contracts, starting with
postconditions, since it is what we want our function to do (we will
later see how to express precondition).



\levelThreeTitle{Postcondition}


The postcondition of a function is introduced with the \CodeInline{ensures} clause.
Let us illustrate its use with the following function
that returns the absolute value of an input value. One of its postconditions
is that the result (which is denoted with the keyword
\CodeInline{\textbackslash{}result}) is greater or equals to 0.



\CodeBlockInput[5]{c}{abs-1.c}



(Notice the \CodeInline{;} at the end of the line).

But this it is not the only property to verify. We also need to specify
the general behavior of a function returning the absolute value. That
is: if the value is positive or 0, the function returns the same value,
else it returns the opposite of the value.

We can specify multiple postconditions, first by combining them with an operator
\CodeInline{\&\&} as we do in C, or by introducing a new \CodeInline{ensures}
clause, as we illustrate here:



\CodeBlockInput[5]{c}{abs-2.c}




This specification is the opportunity to present a very useful logic
connector provided by ACSL and that does not exist in C: the implication
$A \Rightarrow B$, that is written \CodeInline{A ==> B} in
ACSL. The truth table of the implication is the following:



\begin{longtabu}{|c|c|c|} \hline
$A$ & $B$ & $A \Rightarrow B$ \\ \hline
$F$ & $F$ & $T$ \\ \hline
$F$ & $T$ & $T$ \\ \hline
$T$ & $F$ & $F$ \\ \hline
$T$ & $T$ & $T$ \\ \hline
\end{longtabu}



That means that an implication $A \Rightarrow B$ is true in two cases:
either $A$ is false (and in this case, we do not check the value of
$B$), or $A$ is true and then $B$ must also be true. Note that it means
that $A \Rightarrow B$ is equivalent to $\neg A \vee B$. The idea
finally being ``I want to know if when $A$ is true, $B$ also is. If
$A$ is false, I don't care, I consider that the complete formula is
true''. For example, ``if it rains, I want to check that I have an
umbrella, if it does not, I do not care, everything is fine''.



Another available connector is the equivalence $A \Leftrightarrow B$
(written \CodeInline{A <==> B} in ACSL), and it is
stronger. This is the conjunction of the implication in both ways
$(A \Rightarrow B) \wedge (B \Rightarrow A)$. This formula is true in
only two cases: $A$ and $B$ are both true, or both false (it can be seen
as the negation of the exclusive or). Continuing with our example, ``I
do not only want to know that I have an umbrella when it rains, I also
want to be sure that I have one only when it rains''.



\begin{Information}
  Let us give a quick reminder about all
  truth tables of usual logic connectors in first order logic
  ($\neg$ = \CodeInline{!}, $\wedge$ = \CodeInline{\&\&}, $\vee$ = \CodeInline{||}) :

\begin{longtabu}{|c|c|c|c|c|c|c|} \hline
$A$ & $B$ & $\neg A$ & $A \wedge B$ & $A \vee B$ & $A \Rightarrow B$ & $A \Leftrightarrow B$ \\ \hline
$F$ & $F$ & $T$ & $F$ & $F$ & $T$ & $T$ \\ \hline
$F$ & $T$ & $T$ & $F$ & $T$ & $T$ & $F$ \\ \hline
$T$ & $F$ & $F$ & $F$ & $T$ & $F$ & $F$ \\ \hline
$T$ & $T$ & $F$ & $T$ & $T$ & $T$ & $T$ \\ \hline
\end{longtabu}
\end{Information}


We can come back to our specification. As our files become longer and
contains a lot of specifications, it can be useful to name the
properties we want to verify. So, in ACSL, we can specify a name
(without spaces) followed by a \CodeInline{:} character, before stating the
property.
It is possible to put multiple levels of names to categorize our
properties. For example, we could write this:



\CodeBlockInput[7]{c}{abs-3.c}



In most of this tutorial, we will not name the properties we want to
prove, since they will be generally quite simple, and we will not have
too many of them, names would not give us much information.

We can copy and paste the function \CodeInline{abs} and its specification in
a file \CodeInline{abs.c} and use Frama-C to determine if the implementation
is correct with respect to the specification. We can start the GUI of Frama-C
(it is also possible to use the command line interface of Frama-C, but we
will not use it during this tutorial) by using this command line:



\begin{CodeBlock}{bash}
frama-c-gui
\end{CodeBlock}



Or by opening it from the graphical environment.



It is then possible to click on the button ``Create a new session from
existing C files'', files to analyze can be selected by double-clicking
it, the OK button ending the selection. Then, adding other files is
done by clicking Files > Source Files.



Notice that it is also possible to directly open file(s) from the
terminal command line passing them to Frama-C as parameter(s):



\begin{CodeBlock}{bash}
frama-c-gui abs.c
\end{CodeBlock}



\image{1-abs-1}


The window of Frama-C opens and in the panel dedicated to files and
functions, we can select the \CodeInline{abs} function. At each
\CodeInline{ensures} line, we can see a blue circle, it indicates that no
verification has been attempted for these properties.



We ask the verification of the code by right-clicking the name of the
function and ``Prove function annotations by WP'':



\image{1-abs-2}


We can see that blue circles become green bullets, indicating that the
specification is indeed ensured by the program. Furthermore, we can also prove
properties one by one, by right-clicking on them and not on the name of
the function. We also see that now, two additional annotations have been
generated by Frama-C for our function:
\begin{itemize}
  \item \CodeInline{exits \textbackslash{}false},
  \item \CodeInline{terminates \textbackslash{}true}.
\end{itemize}
The first one tells that the function must not call the C \CodeInline{exit}
function (or a function that might itself transitively call \CodeInline{exit}
at some point), and that it will thus normally finish its execution by
returning. The second tells that the function must not run forever. These
annotations are generated by default, however one can add  their own
\CodeInline{exits} or \CodeInline{terminates} clause to change this behavior.
We will explain this later, for now let us just ignore these annotations.



Is our code really bug free? WP gives us a way to ensure that a code
conforms to a specification, but it does not directly check the absence
of runtime errors (RTE) if we do not ask it. Another plugin of Frama-C,
called RTE, can be used to generate some ACSL annotations that can be verified
by the other plugins. Its purpose is to add, in the program, some controls to
ensure that the program cannot create runtime errors (integer overflow,
invalid pointer dereferencing, 0 division, etc.).



To activate these controls, we must activate this option in the configuration
of WP. This is done by first clicking on the plugin configuration button:


\image{plugin-options}


And then adding the option \CodeInline{-wp-rte} in the options related to WP:


\image{select-rte}


We can also ask WP to add the RTE controls in a function by
right-clicking on its name and then click ``Insert wp-rte guards''.


\begin{Information}
  By now, \CodeInline{-wp-rte} will always be activated to verify examples
  except if we explicitly indicate the opposite.
\end{Information}


Finally, we execute the verification again (we can also click on the
``Reparse'' button of the toolbar, it will delete existing proofs).

We can then see that WP fails to prove the absence of arithmetic
overflow for the computation of \CodeInline{-val}. And, indeed, on our
architectures, -\CodeInline{INT\_MIN} ($-2^{31}$) > \CodeInline{INT\_MAX} ($2^{31}-1$).



\image{1-abs-4}


\begin{Information}
We can notice that the overflow risk
is real for us, since our computers (for which the
configuration is detected by Frama-C) use the
\externalLink{Two's complement}{https://en.wikipedia.org/wiki/Two\%27s_complement}
implementation of integers, which does not define
the behavior of overflows.
\end{Information}


Here, we can see another type of ACSL annotation. By the line
\CodeInline{//@ assert property ;}, we can ask the verification of a
property at a particular program point. Here, the RTE plugin inserted it for
us, since
we have to verify that \CodeInline{-val} does not produce an overflow, but
we can also add such an assertion manually in the source code.



In this screenshot, we can see two new colors for our bullets:
green+brown and orange.



If the proof has not been entirely redone after adding the runtime error
checks, these bullets must still be green. Indeed, the corresponding
proofs have been realized without the knowledge of the property in the
assertion, so they cannot rely on this unproved property.



When WP transmits a verification condition to an automatic prover, it
basically transmits two kinds of properties : $G$, the goal, the
property that we want to prove, and $A_1$ \ldots{} $A_n$, the
different assumptions we can have about the state of the memory at the
program point where we want to verify $G$. However, it does not
receive (in return) the properties that have been used by the prover to
validate $G$. So, if $A_3$ is an assumption, and if WP did not
succeed in getting a proof of $A_3$, it indicates that $G$ is true,
but only assuming that $A_3$ is true, for which no proof has been established
so far.



The orange color indicates that no prover could determine if the
property is verified. There are two possibles reasons:
\begin{itemize}
\item the prover did not have the information needed to terminate the proof,
\item the prover did not have enough time to compute the proof and
  met a timeout (which can be configured in the WP panel).
\end{itemize}


In the bottom panel, we can select the ``WP Goals'' tab, it shows the list of
verification conditions, and for each prover the result is symbolized
by a logo that indicates if the proof has been tried and if it
succeeded, failed or met a timeout. Note that it
may require selecting ``All Results'' in the squared field to see all
verification conditions.


\image{1-abs-5}


In the first column, we have the name of the function the verification
condition belongs to. The second column indicates the name of the verification
condition. For example here, our postcondition is named
\CodeInline{postcondition 'positive\_value,function\_result'}, we can notice
that if we select a property in this list, it is also highlighted in the
source code. Unnamed properties are automatically named by WP with the
kind of wanted property. In the third column, we see the memory model
that is used for the proof, we will not talk about it in this tutorial.
Finally, the last columns represent the different provers available
through WP.



In the list of provers, the first element is Qed. This is not really a prover.
It is a tool used by WP to simplify properties before sending them to external
provers. Then we find Script, this is a way to finish some proofs by hand when
automatic solvers do not succeed in doing them. Finally, we find Alt-Ergo
which is one of the solvers we use. Notice that for the RTE property, some
scissors are indicated: it means that the solver was stopped due to timeout.



If we double-click on the property ``absence of overflow''
(highlighted in blue in the last screenshot), we can see the corresponding
verification condition (if it is not the case, make sure that the value ``Raw
obligation'' is selected in the blue squared field):



\image{1-abs-6}


This is the verification condition generated by WP about our property and our
program, we do not need to understand everything here, but we can get the
general idea. It contains (in the ``Assume'' part) the assumptions that
we have specified and those that have been deduced by WP from the
instructions of the program. It also contains (in the ``Prove'' part)
the property that we want to verify.



How does WP deal with these properties? In fact, it transforms them
into a logic formula and then asks different provers if it is
possible to satisfy this formula (to find for each variable, a value
that can make the formula true), and it determines if the property can
be proved. But before sending the formula to provers, WP uses a module
called Qed, which is able to perform different simplifications about it.
Sometimes, as this is the case for the other properties about the
\CodeInline{abs} function, these simplifications are enough to determine that
the property is true, in such a case, WP do not need the help of the
automatic solvers.


Often, automatic solvers do not give extensive feedback after attempting a
proof. When they succeed in proving a verification condition, the answer is
``yes'', which is fine because we do not really need more information most of
the time in this case. But they can fail, either because the property cannot be
proved (for example if it is simply wrong or if some hypotheses are missing) or
because the proof is beyond the capability of the solver. When they fail, they
generally return ``unknown'' or reach a ``timeout''. In such a case, it may be
hard to know why. Reading verification conditions can sometimes be helpful, but
it requires a bit of practice to be efficient. Exploring the proof with the
Coq Proof Assistant can also help, but one must be familiar with this tool, and
the encoding of the C semantics often makes the proof context hard to read.

In some situations, automatic solvers can be a better help. When they return a
timeout, we cannot really have more information, but when they return
``unknown'', some solvers are able to produce counter-examples provided that we
ask them to do so.  By default, WP does not enable this behavior since it may
reduce proof performances, but it can be enabled with the option 
\CodeInline{-wp-counter-examples}. Not all solvers can provide counter-examples,
but for example the CVC5 prover produces counter-examples that can be used to
understand proof failures. Let us run again Frama-C on our example with the
following options:
\begin{CodeBlock}{shell}
frama-c abs.c -wp -wp-rte -wp-counter-examples -wp-prover alt-ergo,cvc5
\end{CodeBlock}


Let us analyze this command line:
\begin{itemize}
  \item we have seen \CodeInline{-wp} and \CodeInline{-wp-rte}, they
        respectively enable the WP plug-in and the runtime error analysis,
  \item \CodeInline{-wp-counter-examples} enables counter-examples requests,
  \item \CodeInline{-wp-prover alt-ergo,cvc5} tells to WP to run both Alt-Ergo
        and CVC5 on all verification conditions.
\end{itemize}


Using this command line, we get:


\image{1-abs-2-counter-example}


In this screenshot (be sure to toggle on the CE selector), we see that for the
\textit{Probe} \CodeInline{val}, which represents the input of the function,
CVC5 has determined that if we set it to -2147483648 (\CodeInline{INT\_MIN}),
we have a counter example to the property ``absence of overflow''. And indeed,
we previously said that the property cannot be proved precisely because it is
not possible for the value \CodeInline{INT\_MIN}. We need to add some hypotheses
to guarantee that our function will behave correctly: a precondition.



\levelThreeTitle{Precondition}



Preconditions are introduced using \CodeInline{requires} clauses. As we
could do with \CodeInline{ensures} clauses, we can compose logic expressions
and specify multiple preconditions:



\begin{CodeBlock}{c}
/*@
  requires 0 <= a < 100;
  requires b < a;
*/
void foo(int a, int b){

}
\end{CodeBlock}



Preconditions are properties about the input (or about
global variables) that we assume to be true when we analyze the
function. We will verify that they are indeed true only at program
points where the function is called.



In this small example, we can also notice a difference with C in the
writing of boolean expressions. If we want to specify that \CodeInline{a} is
between 0 and 100, we do not have to write \CodeInline{0 <= a \&\& a < 100},
we can directly write \CodeInline{0 <= a < 100} and Frama-C will
perform necessary translations.



If we come back to our example about the absolute value, to avoid the
arithmetic overflow, it is sufficient to state that \CodeInline{val} must
be strictly greater than \CodeInline{INT\_MIN} to guarantee that the
overflow will never happen. Thus, we add it as a precondition of the
function (notice that it is also necessary to include the header where
\CodeInline{INT\_MIN} is defined):



\CodeBlockInput{c}{abs-4.c}



\begin{Warning}
  Reminder: The Frama-C GUI does not allow source code modification.
\end{Warning}


Once we have modified the source code with our precondition, we click on
``Reparse'' and we can ask again to prove our program. This time,
everything is validated by WP, our implementation is proved:



\image{2-abs-1}


We can also verify that a function that would call \CodeInline{abs}
conforms to the required precondition:



\CodeBlockInput[15]{c}{abs-5.c}



\image{2-foo-1}

Note that we can click on the bullet next to the function call to see the
list of preconditions and check which ones are not validated. Here, there is
only one precondition, but when there are multiple ones it is useful to check
what is exactly the problem.

\image{2-foo-1-bis}


We can modify this example by reverting the last two instructions. If we
do this, we can see that the call \CodeInline{abs(a)} is validated by WP if
it is placed after the call \CodeInline{abs(INT\_MIN)}! Why?



We must keep in mind that the idea of the deductive proof is to ensure
that if preconditions are verified, and if our computation terminates,
then the postcondition is verified.



If we give to a function an input that surely breaks the precondition, we can
deduce that everything can happen, including obtain false in postcondition.
More precisely, here, after the call, we just suppose that the precondition
is still true (as the function does not modify anything is memory), thus we
suppose that \CodeInline{INT\_MIN < INT\_MIN} which is obviously false.
Knowing this, we can prove absolutely everything because this ``false'' becomes
an assumption of every call that follows. Knowing false, we can prove everything,
because if we have a proof of false, then false is true, as well as true is true.
So everything is true.



Taking our modified program, we can convince ourselves of this fact by
looking at the verification conditions generated by WP for the bad call and the
subsequent call that becomes verified:



\image{2-foo-2}


\image{2-foo-3}


We can notice that for function calls, the GUI highlights the execution
path that leads to the call for which we want to verify the precondition.
Then, if we have a closer look at the call \CodeInline{abs(INT\_MIN)}, we
can notice that, simplifying, Qed deduced that we try to prove
``False''. Consequently, the next call \CodeInline{abs(a)} receives in its
assumptions the property ``False''. This is why Qed can immediately
deduce ``True''.



The second part of the question is then: why our first version of the
calling function (\CodeInline{abs(a)} and then \CodeInline{abs(INT\_MIN)}) did
not have the same behavior, indicating a proof failure on the second
call? The answer is simply that for the call \CodeInline{abs(a)} can we add
the assumption \CodeInline{a < INT\_MIN}, and while we do not have a proof
that it is true, we do not have a proof that is false neither. So, while
\CodeInline{abs(INT\_MIN)} necessarily gives us the knowledge of ``false'',
the call \CodeInline{abs(a)} does not, since it can succeed.


\levelThreeTitle{The particular case of the \CodeInline{main} function}


The main function is generally not called directly by the program. However,
it might have preconditions. In such a case, WP generates verification
conditions for them. But, the context that is provided to the proof is not
the same as what is provided for the other functions. Indeed, since the main
function is called before any other function of the program, at this point, the
global variables necessarily have the values that have been provided for their
initialization.

Let us illustrate this with the following code:


\CodeBlockInput[7]{c}{main.c}


If we start WP on this example, the \CodeInline{requires} clause of the
\CodeInline{main} function is proved, while no proof attempt has been done for
the precondition of the \CodeInline{f} function since it is not called by the
program. The precondition of the \CodeInline{main} function is proved because
after initialization \CodeInline{h} has value 42 which is obviously between
0 and 100. Notice that the assertion in the \CodeInline{main} function is also
proved because \CodeInline{h} is not changed before the assertion, while in the
\CodeInline{f} function, the assertion is not proved since WP does not make any
assumption about the values of the global variables.


\image{3-main-1}


Notice that WP makes this assumption about the \CodeInline{main} function
because it is the default behavior of Frama-C itself. This behavior can be
configured with the option \CodeInline{-lib-entry} that tells Frama-C to
consider any function (including \CodeInline{main}) to be callable in the
program. Thus, when we enable this option on command line for our example, no
verification condition is created for the \CodeInline{requires} clause of the
\CodeInline{main} function, and none of the assertions is proved.


\image{3-main-2}


Finally, Frama-C enables with the possibility to configure which function
must be considered as the main function. For example, if we call Frama-C with
the option \CodeInline{-main=f} on the command line, the \CodeInline{requires}
clause of \CodeInline{f} is proved as well as the assertion that it contains,
no proof attempt is generated for the \CodeInline{requires} clause of the
\CodeInline{main} function, and the assertion in \CodeInline{main} is not
proved.


\image{3-main-3}


\levelThreeTitle{Exercises}



While these exercises are not absolutely necessary to read the next chapters
of the tutorial we strongly suggest practicing them. Note that we also
suggest to, at least, read the fourth exercise that introduces a notation
that can be sometimes useful.



\levelFourTitle{Addition}


Write the postcondition of the following addition function:


\CodeBlockInput[5]{c}{ex-1-addition.c}


And run the command:


\begin{CodeBlock}{bash}
frama-c-gui your-file.c -wp
\end{CodeBlock}


Once the function is successfully proved to respect the contract, run:


\begin{CodeBlock}{bash}
frama-c-gui your-file.c -wp -wp-rte
\end{CodeBlock}


It should fail, adapt the contract by adding the right precondition.


\levelFourTitle{Distance}


Write the postcondition of the following distance function, by expressing
the value of \CodeInline{b} in terms of \CodeInline{a} and
\CodeInline{\textbackslash{}result}:


\CodeBlockInput[5]{c}{ex-2-distance.c}


And run the command:


\begin{CodeBlock}{bash}
frama-c-gui your-file.c -wp
\end{CodeBlock}


Once the function is successfully proved to respect the contract, run:


\begin{CodeBlock}{bash}
frama-c-gui your-file.c -wp -wp-rte
\end{CodeBlock}


It should fail, adapt the contract by adding the right precondition.



\levelFourTitle{Alphabet Letter}



Write the postcondition of the following function that return true if
the character received in input is an alphabet letter. Use the equivalence
operator \CodeInline{<==>}.



\CodeBlockInput[5]{c}{ex-3-alphabet.c}


And run the command:


\begin{CodeBlock}{bash}
frama-c-gui your-file.c -wp -wp-rte
\end{CodeBlock}


All verification conditions should be proved, including the assertions in the
main function.



\levelFourTitle{Days of the month}



Write the postcondition of the following function that returns the number
of days in function of the received month (notice that we consider that
they are numbered from 1 to 12), for February, we only consider the case
when it has 28 days, we will see later how to solve this problem:



\CodeBlockInput[5]{c}{ex-4-day-month.c}


And run the command:


\begin{CodeBlock}{bash}
frama-c-gui your-file.c -wp
\end{CodeBlock}


Once the postcondition is proved, run the command:



\begin{CodeBlock}{bash}
frama-c-gui your-file.c -wp -wp-rte
\end{CodeBlock}



If it fails, complete the precondition in order to solve the problem.



You might have notice that writing the postcondition can be a bit painful.
Let us try to simplify that. ACSL provide the notion of set, and the operator
\CodeInline{\textbackslash{}in} that can be used to check whether a value is
in a set or not.


For example:


\begin{CodeBlock}{c}
//@ assert 13 \in { 1, 2, 3, 4, 5 } ; // FALSE
//@ assert 3  \in { 1, 2, 3, 4, 5 } ; // TRUE
\end{CodeBlock}



Modify the postcondition by using this notation.



\levelFourTitle{Last angle of a triangle}



This function receives two values of angle in input and returns the value
of the last angle considering that the sum of the three angles must be 180.
Write the postcondition that expresses that the sum of the three angles is
180.



\CodeBlockInput[5]{c}{ex-5-last-angle.c}


And run the command:


\begin{CodeBlock}{bash}
frama-c-gui your-file.c -wp
\end{CodeBlock}


Once the function is successfully proved to respect the contract, run:


\begin{CodeBlock}{bash}
frama-c-gui your-file.c -wp -wp-rte
\end{CodeBlock}



If it fails, add the right precondition. Note that the values of the different
angles should not be more than 180 degrees, including the resulting angle.

Now that we have presented most of the important features of ACSL for program
proof, let us have a more general view of program proof with Frama-C and WP. We
will present some approaches that can be used depending on the target of
verification, the expected level of confidence and the kind of formalization
(and features of ACSL) we use.


%% \begin{levelTwo}
%%   {Data-structure API}
%%   {data-structure-api}
%% \end{levelTwo}

\begin{levelTwo}
  {Absence of runtime errors: Minimal contracts}
  {minimal-contracts}
\end{levelTwo}

\begin{levelTwo}
  {Guiding assertions and triggering of lemmas}
  {triggering-lemmas}
\end{levelTwo}

\begin{levelTwo}
  {More on ghost code: lemma functions and lemma macros}
  {lemma-functions}
\end{levelTwo}



\horizontalLine
\newpage

As we try to prove more complex properties, particularly when
programs involve loops, there is a part of ``trial and error'' in
order to understand what the provers miss establishing the proof.

It can miss hypotheses. In this case, we can try to add assertions to
guide the prover, or write ghost code with the right invariant that
allows to make a part of the reasoning by ourselves when it is too hard
for SMT solvers.

With some experience, we can read the content of the verification condition or
try to perform the proof with the Coq interactive prover to see whether
the proof seems to be possible. Sometimes, the prover just needs more
time, in such a case, we can (sometimes a lot) augment the timeout value.
Of course, the property can be too hard for the prover and ghost code
might be sometimes unsuitable, and in this case, we have to write the
proof ourselves with an interactive prover.

Finally, the implementation can be indeed incorrect, and in this case we
have to fix it. Here, we use test and not proof, because a test
allows us to prove the presence of a bug and to analyze this bug.

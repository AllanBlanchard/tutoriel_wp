\begin{levelTwo}
  {}
  {introduction}
\end{levelTwo}

\begin{levelTwo}
  {Basic concepts}
  {basic}
\end{levelTwo}

\begin{levelTwo}
  {Loops}
  {loops}
\end{levelTwo}

\begin{levelTwo}
  {More examples on loops}
  {loops-examples}
\end{levelTwo}

\begin{levelTwo}
  {Function calls}
  {function-calls}
\end{levelTwo}

\horizontalLine
\newpage


In this part, we have seen how assignment and control structures are translated
to a logic view of our program. We have spent quite a lot of time on loops
because they represent the main difficulty we have to face when we want to
specify and prove a program by deductive verification. The loop annotations
allow us to express their behavior as precisely as possible.



In the next part of this tutorial, we will look more closely at the logic
constructs provided by ACSL. They are important because they give us a way to
write more abstract specifications, which are easier to understand and to prove.

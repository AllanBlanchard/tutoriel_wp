\begin{Warning}
  Le code source de ce
  tutoriel est disponible sur GitHub, de même que les solutions aux
  différents exercices (incluant quelques preuves Coq de certaines
  propriétés).

  Si vous trouvez des erreurs, n'hésitez pas à créer une \textit{issue} ou
  une \textit{pull request} sur :

  \externalLink{https://github.com/AllanBlanchard/tutoriel\_wp}{https://github.com/AllanBlanchard/tutoriel_wp}

  ou à poster sur le sujet de la bêta sur Zeste de Savoir :

  \externalLink{https://zestedesavoir.com/forums/sujet/7725/introduction-a-la-preuve-de-programmes-c-avec-frama-c-et-son-greffon-wp/}{https://zestedesavoir.com/forums/sujet/7725/introduction-a-la-preuve-de-programmes-c-avec-frama-c-et-son-greffon-wp/}
\end{Warning}

\begin{Information}
Le choix de certains exemples et d'une partie de l'organisation dans le présent
tutoriel est le même que celui du
\externalLink{tutoriel présenté à TAP 2013}
             {https://frama-c.com/download/publications/tutorial_tap2013_slides.pdf}
par Nikolai Kosmatov, Virgile Prevosto et Julien Signoles du CEA List du fait de
son cheminement didactique. Il contient également des exemples tirés de
\textit{\externalLink{ACSL By Example}{https://github.com/fraunhoferfokus/acsl-by-example}}
de Jochen Burghardt, Jens Gerlach, Kerstin Hartig, Hans Pohl et Juan Soto du
Fraunhofer. Pour les aspects formels, je me suis reposé sur le cours
à propos de Why3 donné par Andrei Paskevich
\textit{\externalLink{à l'EJCP 2018}{https://ejcp2018.sciencesconf.org/file/441131}}.
Le reste vient de mon expérience personnelle avec Frama-C et WP.

\horizontalLine

Les versions des outils utilisés dans ce tutoriel sont les suivantes :
\begin{itemize}
\item Frama-C 30.0 Zinc
\item Why3 1.7.2
\item Alt-Ergo 2.6.0
\item Coq 8.16.1 (pour les scripts proposés, Coq n'est pas utilisé dans le tutoriel)
\item Z3 4.8.10 (utilisés dans un exemple, il n'est pas absolument nécessaire)
\end{itemize}
Selon les versions utilisées par le lecteur, quelques différences pourraient
apparaître avec ce qui est prouvé et ce qui ne l'est pas. Quelques fonctionnalités
ne sont disponibles que dans les versions récentes de Frama-C.

\horizontalLine

Le seul prérequis pour ce cours est d'avoir une connaissance basique du
langage C, au moins jusqu'à la notion de pointeur.


\end{Information}

\newpage


Malgré son ancienneté, le C est un langage de programmation encore largement
utilisé. Il faut dire qu'il n'existe, pour ainsi dire, aucun langage qui soit
disponible sur une aussi large variété de plateformes (matérielles et
logicielles) différentes, que son orientation bas-niveau et les années
d'optimisations investies dans ses compilateurs permettent de générer à
partir de programmes C des exécutables très performants (à condition bien sûr
que le code le permette), et qu'il possède un nombre d'experts (et donc une
base de connaissances) très conséquent.



De plus, de très nombreux systèmes reposent sur des quantités phénoménales de
code historiquement écrit en C, qu'il faut maintenir et corriger, car ils
coûteraient bien trop chers à re-développer.



Mais toute personne qui a déjà codé en C sait également que c'est un langage
très difficile à maîtriser parfaitement. Les raisons sont multiples, mais les
ambiguïtés présentes dans sa norme et la permissivité extrême qu'il offre au
développeur, notamment en ce qui concerne les accès à la mémoire, font que
créer un programme C robuste est très difficile même pour un programmeur
chevronné.



Pourtant, C est souvent choisi comme langage de prédilection pour la
réalisation de systèmes demandant un niveau critique de sûreté (aéronautique,
ferroviaire, armement, ...) où il est apprécié pour ses performances, sa
maturité technologique et la prévisibilité de sa compilation.



Dans ce genre de cas, les besoins de couverture par le test deviennent
colossaux. Et, plus encore, la question « avons-nous suffisamment testé ? »
devient une question à laquelle il est de plus en plus difficile de répondre.
C'est là qu'intervient la preuve de programme. Plutôt que tester toutes les
entrées possibles et (in)imaginables, nous allons prouver « mathématiquement »
qu'aucun problème ne peut apparaître à l'exécution.



L'objet de ce tutoriel est d'utiliser Frama-C, un logiciel développé au
CEA List, et WP, son greffon de preuve déductive, pour s'initier à la preuve
de programmes C. Au-delà de l'usage de l'outil en lui-même, le but de ce tutoriel
est de vous convaincre qu'il est possible d'écrire des programmes sans erreurs de
programmation, mais également de sensibiliser à des notions simples
permettant de mieux comprendre et de mieux écrire les programmes.



\begin{Information}
Merci aux différents bêta-testeurs pour leurs remarques constructives :

\begin{itemize}
\item \externalLink{Taurre}{https://zestedesavoir.com/membres/voir/Taurre/}
\item \externalLink{barockobamo}{https://zestedesavoir.com/membres/voir/barockobamo/}
\item \externalLink{Vayel}{https://zestedesavoir.com/membres/voir/Vayel/}
\item \externalLink{Aabu}{https://zestedesavoir.com/membres/voir/Aabu/}
\end{itemize}
Ainsi qu'aux validateurs qui ont encore permis d'améliorer la qualité de ce tutoriel :

\begin{itemize}
\item \externalLink{Taurre}{https://zestedesavoir.com/membres/voir/Taurre/} (oui, encore lui)
\item \externalLink{Saroupille}{https://zestedesavoir.com/membres/voir/Saroupille/}
\item \externalLink{Aabu}{https://zestedesavoir.com/membres/voir/Aabu/} (oui, encore lui aussi)
\end{itemize}

Un grand merci à Jens Gerlach pour son aide lors de la traduction anglaise du tutoriel.

Merci finalement aux reviewers occasionnels sur GitHub :
\begin{itemize}
  \item \externalLink{Alex Lyr}{https://github.com/AlexLyrr}
  \item \externalLink{Rafael Bachmann}{https://github.com/barafael}
  \item \externalLink{@charlesseizilles}{https://github.com/charlesseizilles}
  \item \externalLink{@Costava}{https://github.com/costava}
  \item \externalLink{Daniel Rocha}{https://github.com/danroc}
  \item \externalLink{@GaoTamanrasset}{https://github.com/GaoTamanrasset}
  \item \externalLink{André Maroneze}{https://github.com/maroneze}
  \item \externalLink{@MSoegtropIMC}{https://github.com/MSoegtropIMC}
  \item \externalLink{@rtharston}{https://github.com/rtharston}
  \item \externalLink{@TrigDevelopment}{https://github.com/TrigDevelopment}
  \item \externalLink{Quentin Santos}{https://github.com/qsantos}
  \item \externalLink{Ricardo M. Correia}{https://github.com/wizeman}
  \item \externalLink{Basile Desloges}{https://github.com/zilbuz}
\end{itemize}
pour leurs relectures et corrections.

\end{Information}



\levelTwoTitle{Assignment, sequence and conditional}


\levelThreeTitle{Assignment}


Assignment is the most basic operation one can have in an imperative
language (leaving aside the ``do nothing'' operation that is not particularly
interesting). The weakest precondition calculus associates the following
$$wp(x = E , Post) := Post[x \leftarrow E]$$


Here the notation $P[x \leftarrow E]$ means ``the property $P$ where
$x$ is replaced by $E$''. In our case this corresponds to ``the
postcondition $Post$ where $x$ is replaced by $E$''. The idea is
that the postcondition of an assignment of $E$ to $x$ can only be
true if replacing all occurrences of $x$ in the formula by $E$ leads
to a property that is true. For example:



\begin{CodeBlock}{c}
// { P }
x = 43 * c ;
// { x = 258 }
\end{CodeBlock}


$$P = wp(x = 43*c , \{x = 258\}) = \{43*c = 258\}$$


The function $wp$ allows us to compute, as weakest precondition of
the assignment provided our expected postcondition, the formula
$\{43*c = 258\}$, thus obtaining the following Hoare triple:


\begin{CodeBlock}{c}
// { 43*c = 258 }
x = 43 * c ;
// { x = 258 }
\end{CodeBlock}


In order to compute the precondition of the assignment we have replaced
each occurrence of $x$ in the postcondition by the assigned value
$E = 43*c$. If our program were of the form:
\begin{CodeBlock}{c}
int c = 6 ;
// { 43*c = 258 }
x = 43 * c ;
// { x = 258 }
\end{CodeBlock}
we could submit the formula $43*6 = 258$ to our automatic prover
in order to determine whether it is really valid. The answer would of
course be ``yes'' because the property is easy to verify. If we had,
however, given the value 7 to the variable \CodeInline{c} the prover's reply
would be ``no'' since the formula $43*7 = 258$ is not true.



Taking into account the weakest precondition calculus, we can now write
the inference rule for the Hoare triple of an assignment as
$$\dfrac{}{\{Q[x \leftarrow E] \}\quad x = E \quad\{ Q \}}$$


We note that there is no premise to verify. Does this mean that the
triple is necessarily true? Yes. However, it does not mean that the
precondition is satisfied by the program to which the assignment belongs
or that the precondition is at all possible. Here the automatic provers
come into play.



For example, we can ask Frama-C to verify the following line:
\begin{CodeBlock}{c}
int a = 42;
//@ assert a == 42;
\end{CodeBlock}
which is, of course, directly proven by Qed, since it is a simple
application of the assignment rule.



\begin{Information}
  We remark that according to the C standard, an assignment is in fact an
  expression. This allows us, for example, to write
  \CodeInline{if( (a = foo()) == 42)}.
  In Frama-C, an assignment will always be treated as a statement. Indeed,
  if an assignment occurs within a larger expression, then the Frama-C
  preprocessor, while building the abstract syntax tree, systematically
  performs a \emph{normalization step} that produces a separate assignment
  statement.
\end{Information}



\levelFourTitle{Assignment of pointed value}



In C, thanks to (because of?) pointers, we can have programs with aliases,
meaning that two pointers can point to the same memory location. Our weakest
precondition calculus should consider these cases. For example, let us consider
this simple Hoare triple:


\begin{CodeBlock}{c}
//@ assert p = q ;
*p = 1 ;
//@ assert *p + *q == 2 ;
\end{CodeBlock}



This Hoare triple is correct, since \CodeInline{p} and \CodeInline{q} are in
alias, modifying \CodeInline{*p} also modifies \CodeInline{*q}, thus both these
expressions evaluate to $1$ and the postcondition is true. However, let us apply
the weakest precondition calculus from the postcondition:



\begin{tabular}{ll}
$wp(*p = 1, *p + *q = 2)$ & $= (*p + *q = 2)[*p \leftarrow 1]$\\
                          & $= (1 + *q = 2)$
\end{tabular}



We get the weakest precondition: \CodeInline{1 + *q == 2}, and thus we could
deduce that the weakest precondition is \CodeInline{*q == 1}, which is true, but
does not allow us to conclude that the program is correct, since in our formula
we do not have anything that models that \CodeInline{p == q ==> *q == 1}. In
fact, here, we would like to be able to compute the weakest precondition like:



\begin{tabular}{ll}
$wp(*p = 1, *p + *q = 2)$ & $= (1 + *q = 2 \vee q = p)$\\
                          & $= (*q = 1 \vee q = p)$
\end{tabular}



For this, we have to take care of aliasing. A common way to do this is to
consider that the memory is one particular variable (let us name this variable
$M$) on which we can perform two operations: get the element at a particular
location $l$ in memory (which returns an expression) and set the element at a
particular location $l$ to a new value $v$ (which returns the new memory).


We denote:


\begin{itemize}
\item $get(M,l)$ with the notation $M[l]$
\item $set(M,l,v)$ with the notation $M[l \mapsto v]$
\end{itemize}


And basically, the get operation can be seen as follows:


\begin{tabular}{ll}
  $M[l1 \mapsto v][l2] =$ & if $l1   =  l2$ then $v$ \\
                          & if $l1 \neq l2$ then $M[l2]$
\end{tabular}


If there is no value associated to the location we use for a get, the value is
undefined (thus, the memory is partial function). Of course, at the beginning of
a function, the memory context can be populated with the memory locations for
which a value is known to be defined.


Now, we can change a bit the weakest precondition calculus for assignment
of pointed memory location. For this, we consider that we have an implicit
variable $M$ that models the memory, and we define the assignment of a memory
location as an update of the memory such that now the corresponding pointer points
to the written expression.
$$wp(*x = E, Q) := Q[M \leftarrow M[x \mapsto E]]$$


And evaluating a pointed value $*x$ in a formula now requires us to use the
get operator to ask the right value. Thus, we can for example compute the weakest
precondition of our previous program:
\begin{tabular}{lll}
  $wp(*p = 1, *p + *q = 2)$
  & $= (*p + *q = 2)[M \leftarrow M[p \mapsto 1]]$ & (1)\\
  & $= (M[p] + M[q] = 2)[M \leftarrow M[p \mapsto 1]]$ & (2)\\
  & $= (M[p \mapsto 1][p] + M[p \mapsto 1][q] = 2)$ & (3)\\
  & $= (1 + M[p \mapsto 1][q] = 2)$ & (4)\\
  & $= (1 + (\texttt{if}\ q = p\ \texttt{then}\ 1\ \texttt{else}\ M[q]) = 2)$ & (5)\\
  & $= (\texttt{if}\ q = p\ \texttt{then}\ 1+1 = 2\ \texttt{else}\ 1+M[q] = 2)$ & (6)\\
  & $= (q = p \vee M[q] = 1)$ & (7)
\end{tabular}
\begin{enumerate}
\item we have to apply the rule of assignment for pointers, but for this we need
  to introduce $M$,
\item we replace pointer accesses in the formula by a call to $get$ on $M$,
\item we apply the replacement asked by the assignment rule,
\item we use the definition of the $get$ operator for the expression about $p$
  ($M[p \mapsto 1][p] = 1$)
\item we use the definition of the $get$ operator for the expression about $q$\\
  ($M[p \mapsto 1][q] = \texttt{if}\ q = p\ \texttt{then}\ 1\ \texttt{else}\ M[q]$)
\item we perform some simplification to the formula ...
\item ... and finally conclude that either $M[q] = 1$ or $p = q$.
\end{enumerate}


Then in our program, since we know that $p = q$, we can conclude that the program
is correct.


The WP plugin does not exactly work like this. In particular, it depends on the
memory model chosen for the proof that will make different assumption about the
memory is organized. For the memory model we use, the typed memory model, in
fact WP creates multiple variables for memory. However, let us have a look at
the verification condition generated for the postcondition of the swap function:


\image{memory-model}


We can see, in the beginning of the verification condition, that a variable
\CodeInline{Mint\_0} representing a memory of values of integer types have been
created, and that this memory is updated and accessed using the operators we
previously introduced (see the definition of the variable \CodeInline{x\_2}).


\levelThreeTitle{Composition of statements}

For a statement to be valid, its precondition must allow us by means of
executing the said statement to reach the desired postcondition. Now we
would like to execute several statements one after another. Here the
idea is that the postcondition of the first statement is compatible with
the required precondition of the second statement and so on for the
third statement.



The inference rule that corresponds to this idea utilizes the following
Hoare triples:
$$\dfrac{\{P\}\quad S1 \quad \{R\} \ \ \ \{R\}\quad S2 \quad \{Q\}}{\{P\}\quad S1 ;\ S2 \quad \{Q\}}$$
In order to verify the composed statement $S1;\ S2$ we rely on an
intermediate property $R$ that is at the same time the postcondition
of $S1$ and the precondition of $S2$. (Please note that $S1$ and
$S2$ are not necessarily simple statements; they themselves can be
composed statements.) The problem is, however, that nothing indicates us
how to determine the properties $P$ and $R$.

The weakest-precondition calculus now says us that the intermediate
property $R$ can be computed as the weakest precondition of the second
statement. The property $P$, on the other hand, then is computed as
the weakest precondition of the first statement. In other words, the
weakest precondition of the composed statement $S1; S2$ is determined
as follows:
$$wp(S1;\ S2 , Post) := wp(S1, wp(S2, Post) )$$


The WP plugin of Frama-C performs all these computations for us. Thus,
we do not have to write the intermediate properties as ACSL assertions
between the lines of codes.



\begin{CodeBlock}{c}
int main(){
  int a = 42;
  int b = 37;

  int c = a+b; // i:1
  a -= c;      // i:2
  b += a;      // i:3

  //@assert b == 0 && c == 79;
}
\end{CodeBlock}



\levelFourTitle{Proof tree}


When we have more than two statements, we can consider the last
statement as second statement of our rule and all the preceding ones as
first statement. This way we traverse step by step backwards the
statements in our reasoning. With the previous program this looks like:


\begin{center}
\begin{tabular}{ccc}
  $\{P\}\quad i_1 ; \quad \{Q_{-2}\}$ & $\{Q_{-2}\}\quad i_2 ; \quad \{Q_{-1}\}$ & \\
  \cline{1-2}
  \multicolumn{2}{c}{$\{P\}\quad i\_1 ; \quad i\_2 ; \quad \{Q_{-1}\}$} & $\{Q_{-1}\} \quad i_3 ; \quad \{Q\}$\\
  \hline
  \multicolumn{3}{c}{$\{P\}\quad i\_1 ; \quad i\_2 ; \quad i\_3; \quad \{ Q \}$}
\end{tabular}
\end{center}

The weakest-precondition calculus allows us to construct the property
$Q_{-1}$ starting from the property $Q$ and statement $i_3$ which
in turn enables us to derive the property $Q_{-2}$ from the property
$Q_{-1}$ and statement $i_2$. Finally, $P$ can be determined from
$Q_{-2}$ and $i_1$.



Now that we can verify programs that consist of several statements it
is time to add some structure to them.



\levelThreeTitle{Conditional rule}


For a conditional statement to be true, one must be able to reach the
postcondition through both branches. Of course, for both branches, the
same precondition (of the conditional statement) must hold. In addition,
we have that in the if-branch the condition is true while in the
else-branch it is false.

We therefore have, as in the case of composed statements, two facts to
verify (in order to avoid confusion we are using here the syntax
$if\ B\ then\ S1\ else\ S2$):
$$\dfrac{\{P \wedge B\}\quad S1\quad \{Q\} \quad \quad \{P \wedge \neg B\}\quad S2\quad \{Q\}}{\{P\}\quad if\quad B\quad then\quad S1\quad else\quad S2 \quad \{Q\}}$$

Our two premises are therefore that we can both in the if-branch and the
else-branch reach the postcondition from the precondition.

The result of the weakest-precondition calculus for a conditional
statement reads as follows:
$$wp(if\ B\ then\ S1\ else\ S2 , Post) := (B \Rightarrow wp(S1, Post)) \wedge (\neg B \Rightarrow wp(S2, Post))$$
This means that the condition $B$ has to imply the weakest
precondition of $S1$ in order to safely arrive at the postcondition.
Analogously, the negation of $B$ must imply the weakest precondition
of $S2$.



\levelFourTitle{Empty \CodeInline{else}-branch}


Following this definition, we obtain for the case of an empty else-branch the
following rule by simply replacing the statement $S2$ by the empty
statement \CodeInline{skip}.
$$\dfrac{\{P \wedge B\}\quad S1\quad \{Q\} \quad \quad \{P \wedge \neg B\}\quad skip\quad \{Q\}}{\{P\}\quad if\quad B\quad then\quad S1\quad else\quad skip \quad \{Q\}}$$
The triple for \CodeInline{else} is:
$$\{P \wedge \neg B\}\quad skip\quad \{Q\}$$
which means that we need to ensure:
$$P \wedge \neg B \Rightarrow Q$$

In short, if the condition $B$ of \CodeInline{if} is false, this means
that the postcondition of the complete conditional statement is already
established before entering the else-branch (since it does not do
anything).



As an example, we consider the following code snippet where we reset a
variable $c$ to a default value in case it had not been properly
initialized by the user.



\begin{CodeBlock}{c}
int c;

// ... some code ...

if(c < 0 || c > 15){
  c = 0;
}
//@ assert 0 <= c <= 15;
\end{CodeBlock}



Let



$wp(if \neg (c \in [0;15])\ then\ c := 0, \{c \in [0;15]\})$



$:= (\neg (c \in [0;15])\Rightarrow wp(c := 0, \{c \in [0;15]\})) \wedge (c \in [0;15]\Rightarrow wp(skip, \{c \in [0;15]\}))$



$= (\neg (c \in [0;15]) \Rightarrow 0 \in [0;15]) \wedge (c \in [0;15] \Rightarrow c \in [0;15])$



$= (\neg (c \in [0;15]) \Rightarrow true) \wedge true$



The property can be verified: independent of the evaluation of
$\neg (c \in [0;15])$, the implication will hold.



\levelThreeTitle{Bonus Stage - Consequence rule}
\label{l3:statements-basic-consequence}


It can sometimes be useful to strengthen a postcondition or to weaken a
precondition. The former will often be established by us to
facilitate the work of the prover, the latter is more often verified by
the tool as the result of computing the weakest precondition.



The inference rule of Hoare logic is the following:
$$\dfrac{P \Rightarrow WP \quad \{WP\}\quad c\quad \{SQ\} \quad SQ \Rightarrow Q}{\{P\}\quad c \quad \{Q\}}$$

(We remark that the premises here are not only Hoare triples but also
formulas to verify.)

For example, if our postcondition is too complex, it may generate a
weaker precondition that is, however, too complicated, thus making the
work of provers more difficult. We can then create a simpler
intermediate postcondition $SQ$, that is, however, stricter and
implies the real postcondition. This is the part $SQ \Rightarrow Q$.

Conversely, the calculation of the precondition will usually generate a
complicated and often weaker formula than the precondition we want to
accept as input. In this case, it is our tool that will check the
implication between what we want and what is necessary for our code to
be valid. This is the part $P \Rightarrow WP$.

We can illustrate this with the following code. Note that here the code
could be proved by WP without the weakening and strengthening of
properties because the code is very simple, it is just to illustrate the
rule of consequence.



\begin{CodeBlock}{c}
/*@
  requires P: 2 <= a <= 8;
  ensures  Q: 0 <= \result <= 100 ;
  assigns  \nothing ;
*/
int constrained_times_10(int a){
  //@ assert P_imply_WP: 2 <= a <= 8 ==> 1 <= a <= 9 ;
  //@ assert WP:         1 <= a <= 9 ;

  int res = a * 10;

  //@ assert SQ:         10 <= res <= 90 ;
  //@ assert SQ_imply_Q: 10 <= res <= 90 ==> 0 <= res <= 100 ;

  return res;
}
\end{CodeBlock}



(Note: We have omitted here the control of integer overflow.)



Here we want to have a result between 0 and 100. But we know that the
code will not produce a result outside the bounds of 10 and 90. So we
strengthen the postcondition with an assertion that at the end
\CodeInline{res}, the result, is between 0 and 90. The calculation of the
weakest precondition of this property together with the assignment
\CodeInline{res = 10 * a} yields a weaker precondition
\CodeInline{1 <= a <= 9}, and we know that
\CodeInline{2 <= a <= 8} gives us the desired
guarantee.



When there are difficulties to carry out a proof on more complex code,
then it is often helpful to write assertions that produce stronger, yet
easier to verify, postconditions. Note that in the previous code, the
lines \CodeInline{P\_imply\_WP} and\CodeInline{SQ\_imply\_Q} are never used
because this is the default reasoning of WP. They are just here for
illustrating the rule.


\levelThreeTitle{Bonus Stage - Constancy rule}
\label{l3:statements-basic-constancy}


Certain sequences of instructions may concern and involve different
variables. Thus, we may initialize and manipulate a certain number of
variables, begin to use some of them for a time, before using other
variables. When this happens, we want our tool to be concerned only with
variables that are susceptible to change in order to obtain the simplest
possible properties.



The rule of inference that defines this reasoning is the following:
$$\dfrac{\{P\}\quad c\quad \{Q\}}{\{P \wedge R\}\quad c\quad \{Q \wedge R\}}$$
where $c$ does not modify any variable in $R$. In other words:
``To check the triple, let's get rid of the parts of the formula that
involve variables that are not influenced by $c$ and prove the new
triple.'' However, we must be careful not to delete too much
information, since this could mean that we are not able to prove our
properties.


As an example, let us consider the following code (here gain, we ignore
potential integer overflows):



\begin{CodeBlock}{c}
/*@
  requires a > -99 ;
  requires b > 100 ;
  ensures  \result > 0 ;
  assigns  \nothing ;
*/
int foo(int a, int b){
  if(a >= 0){
    a++ ;
  } else {
    a += b ;
  }
  return a ;
}
\end{CodeBlock}


If we look at the code of the \CodeInline{if} block, we notice that it does
not use the variable \CodeInline{b}. Thus, we can completely omit the
properties about \CodeInline{b} in order to prove that \CodeInline{a} will be
strictly greater than 0 after the execution of the block:



\begin{CodeBlock}{c}
/*@
  requires a > -99 ;
  requires b > 100 ;
  ensures  \result > 0 ;
  assigns  \nothing ;
*/
int foo(int a, int b){
  if(a >= 0){
    //@ assert a >= 0; // and nothing about b
    a++ ;
  } else {
    a += b ;
  }
  return a ;
}
\end{CodeBlock}



On the other hand, in the \CodeInline{else} block, even if \CodeInline{b} is
not modified, formulating properties only about \CodeInline{a} would render
a proof impossible for humans. The code would be:



\begin{CodeBlock}{c}
/*@
  requires a > -99 ;
  requires b > 100 ;
  ensures  \result > 0 ;
  assigns  \nothing ;
*/
int foo(int a, int b){
  if(a >= 0){
    //@ assert a >= 0; // and nothing about b
    a++ ;
  } else {
    //@ assert a < 0 && a > -99 ; // and nothing about b
    a += b ;
  }
  return a ;
}
\end{CodeBlock}



In the \CodeInline{else} block, knowing that\CodeInline{a} lies between -99 and
0, but knowing nothing about \CodeInline{b}, we could hardly know if the
operation \CodeInline{a += b} produces a result that is greater than 0.

The WP plugin will, of course, prove the function without problems,
since it produces by itself the properties that are necessary for the
proof. In fact, the analysis which variables are necessary or not (and,
consequently, the application of the constancy rule) is conducted
directly by WP.

Let us finally remark that the constancy rule is an instance of the
consequence rule
$$\dfrac{P \wedge R \Rightarrow P \quad \{P\}\quad c\quad \{Q\} \quad Q \Rightarrow Q \wedge R}{\{P \wedge R\}\quad c\quad \{Q \wedge R\}}$$


If the variables of $R$ have not been modified by the operation
(which, on the other hand, may modify the variables of $P$ to produce
$Q$), then the properties $P \wedge R \Rightarrow P$ and
$Q \Rightarrow Q \wedge R$ hold.


\levelThreeTitle{Assertion}


ACSL assertions needs to be proved (or used). Thus, they need their WP calculus
rules. There are three different kinds of assertions in ACSL:
\begin{itemize}
  \item \CodeInline{assert}
  \item \CodeInline{check}
  \item \CodeInline{admit}
\end{itemize}


Each kind of assertion has a different behavior.


Let us start with the one we have used so far: \CodeInline{assert}. An
annotation \CodeInline{assert P} has the following informal semantics: the
property \CodeInline{P} must be verified at this program point, and then the
fact that it is true at this program point is added as a hypothesis when proving
properties that come later. Note that it might not be true, but it is still
added to the proof context anyway. However, if it is not true, first, we cannot
to prove it, and second, the properties that we are proved admitting this
property are marked as ``valid under hypothesis''.


The annotation \CodeInline{check P} means that the property \CodeInline{P} must
be verified at this program point, but then it is not added to the proof context
for the proof of the annotations that follow it.


Finally, the annotation \CodeInline{admit P} means that from this program point,
the property \CodeInline{P} is considered to be true, thus added to the proof
context of the annotations that follow it, even if we do not verify it at this
program point.


Let us illustrate these different kind of annotations with this short example:


\CodeBlockInput{c}{annot-kinds.c}


\image{annot-kinds}


In the \CodeInline{check\_annot} function, neither the \CodeInline{check} nor
the \CodeInline{ensures} clauses are proved. Each one have been tried separately
(and failed).


In the \CodeInline{admit\_annot} function, the \CodeInline{admit} annotation
appear with a blue and green bullet (just like when we have a contract on a
function declaration without a body), meaning that from this point, it is
considered to be valid even it is not proved. Thus, the \CodeInline{ensures}
clause is proved, since \CodeInline{\textbackslash{}false} is assumed to be
true.


Finally, in the \CodeInline{assert\_annot} function, the \CodeInline{assert}
annotation is not proved, but from this point it is assumed to be true. Thus,
the \CodeInline{ensures} clause is proved. However, while in the
\CodeInline{admit\_annot} function it appears entirely valid (green bullet),
here, it appears valid under hypothesis: the validity depends on the validity
of a property that has not been proved yet (and will not).


We can encode these different behaviors with the following WP rules.


The rule for the \CodeInline{check} annotation is the following:
$$ wp(check\ A, Post) := A \wedge Post $$
which means that we add to the properties we must verify another property, which
is the property $A$.


The rule for the \CodeInline{admit} annotation is the following:
$$ wp(admit\ A, Post) := A \Rightarrow Post $$
which means that assuming $A$, we must verify the postcondition (of the
statement, which can be a property that has been computing using other WP
rules).


Finally, the rule for the \CodeInline{assert} combines the previous ones.
Indeed:
\begin{CodeBlock}{c}
  assert P;
\end{CodeBlock}
is equivalent to:
\begin{CodeBlock}{c}
  check P; // first, verify that P is true
  admit P; // now, we can assume that it is
\end{CodeBlock}
Thus:
$$ wp(assert\ A, Post) := wp(check\ A, wp(admit\ A, Post)) \equiv A \wedge (A \Rightarrow Post) $$


This allows to introduce some sort of cut in the proof, saying "OK, let us first
prove $A$ and once it is done, we will prove that when $A$ is true, $Post$ is
too".


One could think the \CodeInline{admit} annotation is dangerous. It is. But it
can be useful when one wants to debug a proof. In particular, in
Chapter~\ref{l1:proof-methodologies}, we will see that \CodeInline{assert}
annotations can be used to guide program proof. During the process,
\CodeInline{admit} can be useful to try annotations without having to
immediately prove them or to consider some annotations as valid to make the
design phase faster. More rarely, it can be useful to explicit some hypotheses
about hardware, or software platform, but for this, it is more common to use
Frama-C kernel options or axioms (we will present axioms in
Section~\ref{l2:acsl-logic-definitions-axiomatic}).


\levelThreeTitle{WP plugin vs. Dijkstra's WP calculus}


One might have notice from the beginning of this section that the $wp$ function
starts from the postcondition and, instruction after instruction, changes the
formula until it reaches the beginning of the function, when it generates the
verification condition (as briefly explained in
Section~\ref{l3:statements-basic-consequence}), that we must verify.


However, one might have also noticed that from the very beginning of this book,
in most examples, we do not have \emph{one} verification condition to verify,
but \emph{several}. On a theoretical aspect, this is not a crucial difference,
but on a practical aspect, it allows generating simpler verification conditions,
thus easier to verify using SMT solvers.


In fact, the WP plugin generates several verification conditions in parallel
during the WP calculus. Each time it meets a new instruction, the $wp$ function
is applied for this instruction on all verification conditions involved in the
program path the instruction belongs to (to guarantee this, the conditional
rule is also handled slightly differently, but we will not detail this aspect).
However, when this step includes the proof of a property (for example, an
\CodeInline{assert} annotation), instead of applying the rule by adding it as a
conjunction, it is simply added separately to the set of verification conditions
to prove. Intuitively: we start a new weakest precondition calculus in parallel
from this program point.


Let us illustrate this on a toy example:


\CodeBlockInput{c}{wp-plugin-calculus-toy.c}


We start (in 1) with two verification conditions:
\begin{itemize}
  \item \CodeInline{VC1: P(x)},
  \item \CodeInline{VC2: P(y)}.
\end{itemize}
We apply the transformer of assignment on each of them, thus we reach 2 with:
\begin{itemize}
  \item \CodeInline{VC1: P(x * y)},
  \item \CodeInline{VC2: P(y)}.
\end{itemize}
In 2, we have to deal with the conditional. Instead of directly applying the
standard WP rule, we split the analysis in two sets of verification conditions,
initially equivalent. Let us consider the \CodeInline{else} branch, then the
\CodeInline{then} branch.


From 2 to 3, we meet the \CodeInline{check} annotation, thus we add a new
verification condition, we get the set:
\begin{itemize}
  \item \CodeInline{VC1: P(x * y)},
  \item \CodeInline{VC2: P(y)},
  \item \CodeInline{VC3: Q(y)}.
\end{itemize}
Then, we apply the assignment rule, and we get in 4:
\begin{itemize}
  \item \CodeInline{VC1: P(x * (y+1))},
  \item \CodeInline{VC2: P(y+1)},
  \item \CodeInline{VC3: Q(y+1)}.
\end{itemize}


From 2 to 5, we meet the \CodeInline{assert} annotation, thus we add this
knowledge to all verification conditions, and we create a new one:
\begin{itemize}
  \item \CodeInline{VC1: Q(x) ==> P(x * y)},
  \item \CodeInline{VC2: Q(x) ==> P(y)},
  \item \CodeInline{VC4: Q(x)}.
\end{itemize}
Then we apply the assignment rule, and we get in 6:
\begin{itemize}
  \item \CodeInline{VC1: Q(x+1) ==> P((x+1) * y)},
  \item \CodeInline{VC2: Q(x+1) ==> P(y)},
  \item \CodeInline{VC4: Q(x+1)}.
\end{itemize}


Now we have to reconcile the two sets. Let us not enter into the details (just
notice the introduced primed variables, the trick is here) of how it is done, we
get, in 7, something like:
\begin{CodeBlock}{text}
  VC1:
    ((x <= 0 ==> y' == y+1 && x' == x) &&
     (x >  0 ==> y' == y   && x' == x+1 && Q(x+1))) ==>
       P(x' * y')

  VC2:
    ((x <= 0 ==> y' == y+1) &&
     (x >  0 ==> y' == y    && Q(x+1))) ==>
       P(y')
  VC3:
    (x <= 0) ==> Q(x+1)

  VC4:
    (x >  0) ==> Q(x+1)
\end{CodeBlock}


Finally, we meet the precondition, and we obtain in 8:
\begin{CodeBlock}{text}
  VC1:
    P(x) ==>
      ((x <= 0 ==> y' == y+1 && x' == x) &&
       (x >  0 ==> y' == y   && x' == x+1 && Q(x+1))) ==>
         P(x' * y')

  VC2:
    P(x) ==>
      ((x <= 0 ==> y' == y+1) &&
       (x >  0 ==> y' == y    && Q(x+1))) ==>
         P(y')
  VC3:
    P(x) ==> (x <= 0) ==> Q(x+1)

  VC4:
    P(x) ==> (x >  0) ==> Q(x+1)
\end{CodeBlock}


We can see that for each property that must be proved, we have built a separate
verification condition that gathers the knowledge available along the path that
leads to the program point where its verification is asked.


\levelThreeTitle{Exercises}



\levelFourTitle{A series of assignment}


Compute by hand the weakest precondition of the following program:


\begin{CodeBlock}{c}
/*@
  requires -10 <= x <= 0 ;
  requires 0 <= y <= 5 ;
  ensures -10 <= \result <= 10 ;
*/
int function(int x, int y){
  int res ;
  y += 10 ;
  x -= 5 ;
  res = x + y ;
  return res ;
}
\end{CodeBlock}


Deduce that the program is correct with respect to its contract using the
right rule.


\levelFourTitle{Empty ``then'' branch in conditional}


We have previously shown that when a conditional structure has an empty ``else''
branch, that means that the conjunction of the precondition and the negation
of the condition must be enough to verify the postcondition of the complete
conditional structure.
For both of the following question, we only need the inference rules and
no WP calculus.

Show that when, instead of the ``else'' branch, the ``then'' branch is empty,
the postcondition is verified by the conjunction of the precondition and
the condition (as only the ``else'' can change the memory state).

Show that when both branches are empty, the overall condition is just a
skip operation.


\levelFourTitle{Short circuit}


C compilers implement short circuit for conditions. For example, that
means that a code like this one (\textbf{without ``else'' block}) :


\begin{CodeBlock}{c}
if(cond1 && cond2){
  // code
}
\end{CodeBlock}



can be written as:



\begin{CodeBlock}{c}
if(cond1){
  if(cond2){
    // code
  }
}
\end{CodeBlock}



Show that on those two source code, the weakest precondition calculus
generates an equivalent weakest precondition for equivalent for any code
in the ``then'' block. Note that we assume the conditions to be pure
expressions (without side effects).



\levelFourTitle{A larger program}


Compute by hand the weakest precondition of the following program:


\begin{CodeBlock}{c}
/*@
  requires -5 <= y <= 5 ;
  requires -5 <= x <= 5 ;
  ensures  -15 <= \result <= 25 ;
*/
int function(int x, int y){
  int res ;

  if(x < 0){
    x = 0 ;
  }

  if(y < 0){
    x += 5 ;
  } else {
    x -= 5 ;
  }

  res = x - y ;

  return res ;
}
\end{CodeBlock}


Deduce that the program is correct with respect to its contract using the
right rule.



\levelTwoTitle{Loops}

Les boucles ont besoin d'un traitement de faveur dans la vérification déductive
de programmes. Ce sont les seules structures de contrôle qui vont nécessiter un
travail conséquent de notre part. Nous ne pouvons pas y échapper car sans les 
boucles nous pouvons difficilement prouver des programmes intéressants.



Avant de s'intéresser à la spécification des boucles, il est légitime de se 
poser la question suivante : pourquoi les boucles sont-elles compliquées ?



\levelThreeTitle{Induction et invariance}
\label{l3:statements-loops-invariant}


La nature des boucles rend leur analyse difficile. Lorsque nous faisons nos 
raisonnements arrières, il nous faut une règle capable de dire à partir de la
post-condition quelle est la pré-condition d'une certaine séquence 
d'instructions. Problème : nous ne pouvons \textit{a priori} pas déduire combien de 
fois la boucle va s'exécuter et donc par conséquent, nous ne pouvons pas non 
plus savoir combien de fois les variables ont été modifiées.



Nous allons donc procéder en raisonnant par induction. Nous devons trouver une
propriété qui est vraie avant de commencer la boucle et qui, si elle est vraie
au début d'un tour de boucle, sera vraie à la fin (et donc par extension, au 
début du tour suivant). Quand la boucle termine, nous gagnons la connaissance 
que la condition de boucle est fausse qui, en conjonction avec l'invariant,
doit nous permettre de prouver que la post-condition de la boucle est vérifiée.



Ce type de propriété est appelé un invariant de boucle. Un invariant de boucle
est une propriété qui doit être vraie avant et après chaque tour d'une boucle,
et plus précisément chaque fois que l'on évalue la condition de la boucle.
Par exemple, pour la boucle :



\begin{CodeBlock}{c}
for(int i = 0 ; i < 10 ; ++i){ /* */ }
\end{CodeBlock}



La propriété $0 \leq \texttt{i} \leq 10$ est un invariant de la boucle. La 
propriété  $-42 \leq \texttt{i} \leq 42$ est également un invariant de la boucle
(qui est nettement plus imprécis néanmoins). La propriété $0 < \texttt{i} \leq 10$
n'est pas un invariant, elle n'est pas vraie à l'entrée de la boucle. La propriété
$0 \leq \texttt{i} < 10$ \textbf{n'est pas un invariant de la boucle}, elle n'est
pas vraie à la fin du dernier tour de la boucle qui met la valeur $\texttt{i}$ à 
$10$.



Le raisonnement produit par l'outil pour vérifier un invariant $I$ sera donc :



\begin{itemize}
\item vérifions que $I$ est vrai au début de la boucle (établissement),
\item vérifions que si $I$ est vrai avant de commencer un tour, alors $I$ est vrai après (préservation).
\end{itemize}


\levelFourTitle{Formel - Règle d'inférence}


En notant l'invariant $I$, la règle d'inférence correspondant à la boucle est 
définie comme suit :
$$\dfrac{\{I \wedge B \}\ c\ \{I\}}{\{I\}\ while(B)\{c\}\ \{I \wedge \neg B\}}$$


Et le calcul de plus faible pré-condition est le suivant :
$$wp(while (B) \{ c \}, Post) := I \wedge ((B \wedge I) \Rightarrow wp(c, I)) \wedge ((\neg B \wedge I) \Rightarrow Post)$$


Détaillons cette formule :



\begin{itemize}
\item (1) le premier $I$ correspond à l'établissement de l'invariant, c'est 
vulgairement la « pré-condition » de la boucle,
\item la deuxième partie de la conjonction ($(B \wedge I) \Rightarrow wp(c, I)$)
correspond à la vérification du travail effectué par le corps de la boucle :

\begin{itemize}
\item la pré-condition que nous connaissons du corps de la boucle (notons $KWP$,
« \textit{Known WP} ») est ($KWP = B \wedge I$). Soit le fait que nous sommes
rentrés dedans ($B$ est vrai), et que l'invariant est respecté à ce moment
($I$, qui est vrai avant de commencer la boucle par (1), et dont veut 
vérifier qu'il sera vraie en fin de bloc de la boucle (2)),
\item (2) ce qu'il nous reste à vérifier c'est que $KWP$ implique la 
pré-condition réelle* du bloc de code de la boucle 
  ($KWP \Rightarrow wp(c, Post)$). Ce que nous voulons en fin de bloc, 
  c'est avoir maintenu l'invariant $I$ ($B$ n'est peut-être plus vrai en
  revanche). Donc 
$KWP \Rightarrow wp(c, I)$, soit $(B \wedge I) \Rightarrow wp(c, I)$,
\item * cela correspond à une application de la règle de conséquence expliquée
précédemment.
\end{itemize}
\item finalement, la dernière partie ($(\neg B \wedge I) \Rightarrow Post$)
nous dit que le fait d'être sorti de la boucle ($\neg B$), tout en ayant 
maintenu l'invariant $I$, doit impliquer la post-condition voulue pour la 
boucle.
\end{itemize}


Dans ce calcul, nous pouvons noter que la fonction $wp$ ne nous donne aucune
indication sur le moyen d'obtenir l'invariant $I$. Nous allons donc devoir 
spécifier manuellement de telles propriétés à propos de nos boucles.



\levelFourTitle{Retour à l'outil}


Il existe des outils capables d'inférer des invariants (pour peu qu'ils soient
simples, les outils automatiques restent limités). Ce n'est pas le cas de WP.
Il nous faut donc écrire nos invariants à la main. Trouver et écrire les 
invariants des boucles de nos programmes sera toujours la partie la plus difficile
de notre travail lorsque nous chercherons à prouver des programmes.



En effet, si en l'absence de boucle, la fonction de calcul de plus faible 
pré-condition peut nous fournir automatiquement les propriétés vérifiables de nos
programmes, ce n'est pas le cas pour les invariants de boucle pour lesquels 
nous n'avons pas accès à une procédure automatique de calcul. Nous devons donc 
trouver et formuler correctement ces invariants, et selon l'algorithme, celui-ci
peut parfois être très subtil.



Pour indiquer un invariant à une boucle, nous ajoutons les annotations suivantes
en début de boucle :



\CodeBlockInput{c}{first_loop-1.c}



\begin{Warning}
\textbf{RAPPEL} : L'invariant est bien : i \textbf{$\leq$} 30 !
\end{Warning}


Pourquoi ? Parce que tout au long de la boucle \CodeInline{i} sera bien compris entre
0 et 30 \textbf{inclus}. 30 est même la valeur qui nous permettra de sortir de la 
boucle. Plus encore, une des propriétés demandées par le calcul de plus faible
pré-conditions sur les boucles est que lorsque l'on invalide la condition de la
boucle, par la connaissance de l'invariant, on peut prouver la post-condition 
(Formellement : $(\neg B \wedge I) \Rightarrow Post$).



La post-condition de notre boucle est $\texttt{i} = 30$ et doit être impliquée par
$\neg \texttt{i} < 30 \wedge 0 \leq \texttt{i} \leq 30$. Ici, cela fonctionne 
bien : 
$$\texttt{i} \geq 30 \wedge 0 \leq \texttt{i} \leq 30 \Rightarrow \texttt{i} = 30$$
Si l'invariant excluait l'égalité à 30, la post-condition ne serait pas atteignable.



Une nouvelle fois, nous pouvons jeter un œil à la liste des buts dans « \textit{WP 
Goals} » :



\image{i_30-1}


Nous remarquons bien que WP décompose la preuve de l'invariant en deux parties : 
l'établissement de l'invariant et sa préservation. WP produit exactement le 
raisonnement décrit plus haut pour la preuve de l'assertion. Dans les versions
récentes de Frama-C, Qed est devenu particulièrement puissant, et l'obligation de
preuve générée ne le montre pas (affichant simplement « \textit{True} »). En utilisant 
l'option \CodeInline{-wp-no-simpl} au lancement, nous pouvons quand même voir 
l'obligation correspondante :



\image{i_30-2}


Mais notre spécification est-elle suffisante ?



\CodeBlockInput{c}{first_loop-2.c}



Voyons le résultat :



\image{i_30-3}


Il semble que non.



\levelThreeTitle{La clause « assigns » ... pour les boucles}


En fait, à propos des boucles, WP ne considère vraiment \textit{que} ce que lui 
fournit l'utilisateur pour faire ses déductions. Et ici l'invariant ne nous dit
rien à propos de l'évolution de la valeur de \CodeInline{h}. Nous pourrions signaler 
l'invariance de toute variable du programme mais ce serait beaucoup d'efforts. 
ACSL nous propose plus simplement d'ajouter des annotations \CodeInline{assigns} pour 
les boucles. Toute autre variable est considérée invariante. Par exemple :


\CodeBlockInput{c}{first_loop-3.c}



Cette fois, nous pouvons établir la preuve de correction de la boucle. Par contre, 
rien ne nous prouve sa terminaison. L'invariant de boucle n'est pas suffisant pour 
effectuer une telle preuve. Par exemple, dans notre programme, si nous réécrivons 
la boucle comme ceci :



\begin{CodeBlock}{c}
/*@
  loop invariant 0 <= i <= 30;
  loop assigns i;
*/
while(i < 30){
   
}
\end{CodeBlock}



L'invariant est bien vérifié également, mais nous ne pourrons jamais prouver
que la boucle se termine : elle est infinie.



\levelThreeTitle{Correction partielle et correction totale - Variant de boucle}


En vérification déductive, il y a deux types de correction, la correction 
partielle et la correction totale. Dans le premier cas, la formulation est 
« si la pré-condition est validée et \textbf{si} le calcul termine, alors la 
post-condition est validée ». Dans le second cas, « si la pré-condition est 
validée, alors le calcul termine et la post-condition est validée ». WP 
s'intéresse par défaut à de la preuve de correction partielle :



\CodeBlockInput{c}{infinite.c}


Si nous demandons la vérification de ce code en activant le contrôle de RTE,
nous obtenons ceci :



\image{infinite}


L'assertion « FAUX » est prouvée ! La raison est simple : la non-terminaison de
la boucle est triviale : la condition de la boucle est « VRAI » et aucune instruction
du bloc de la boucle ne permet d'en sortir puisque le bloc ne contient pas de code du
tout. Comme nous sommes en correction partielle, et que le calcul ne termine pas, nous
pouvons prouver n'importe quoi au sujet du code qui suit la partie non terminante. Si,
en revanche, la non-terminaison est non-triviale, il y a peu de chances que l'assertion
soit prouvée.



\begin{Information}
À noter qu'une assertion inatteignable est toujours prouvée comme vraie de cette 
manière :
\inlineImage{goto_end}

Et c'est également le cas lorsque l'on sait trivialement qu'une instruction
produit nécessairement une erreur d'exécution (par exemple en déréférençant 
la valeur \CodeInline{NULL}), comme nous avions déjà pu le constater avec l'exemple
de l'appel à \CodeInline{abs} avec la valeur \CodeInline{INT\_MIN}.
\end{Information}


Pour prouver la terminaison d'une boucle, nous utilisons la notion de variant de 
boucle. Le variant de boucle n'est pas une propriété mais une valeur. C'est une 
expression faisant intervenir des éléments modifiés par la boucle et donnant une
borne supérieure sur le nombre d'itérations restant à effectuer à un tour de la
boucle. C'est donc une expression supérieure à 0 et strictement décroissante d'un 
tour de boucle à l'autre (cela sera également vérifié par induction par WP).



Si nous reprenons notre programme précédent, nous pouvons ajouter le variant
de cette façon :



\CodeBlockInput{c}{first_loop-4.c}



Une nouvelle fois nous pouvons regarder les buts générés :



\image{i_30-4}


Le variant nous génère bien deux obligations au niveau de la vérification : 
assurer que la valeur est positive, et assurer qu'elle décroît strictement pendant
l'exécution de la boucle. Et si nous supprimons la ligne de code qui incrémente
\CodeInline{i}, WP ne peut plus prouver que la valeur \CodeInline{30 - i} décroît strictement.



Il est également bon de noter qu'être capable de donner un variant de boucle
n'induit pas nécessairement d'être capable de donner le nombre exact d'itérations
qui doivent encore être exécutées par la boucle, car nous n'avons pas toujours une
connaissance aussi précise du comportement de notre programme. Nous pouvons par
 exemple avoir un code comme celui-ci :



\CodeBlockInput{c}{random_loop.c}



Ici, à chaque tour de boucle, nous diminuons la valeur de la variable \CodeInline{i} par une
valeur dont nous savons qu'elle se trouve entre 1 et \CodeInline{i}. Nous pouvons donc bien 
assurer que la valeur de \CodeInline{i} est positive et décroît strictement, mais nous ne 
pouvons pas dire combien de tours de boucles vont être réalisés pendant une 
exécution.


Le variant n'est donc bien qu'une borne supérieure sur le nombre d'itérations 
de la boucle.


Notons aussi que le variant de boucle n'a besoin d'être positif qu'au début de l'exécution
du bloc de la boucle. Donc, dans le code suivant :


\begin{CodeBlock}{c}
int i = 5 ;
while(i >= 0){
  i -= 2 ;
}
\end{CodeBlock}


Même si \CodeInline{i} peut être négatif lorsque la boucle termine, cette valeur est
bien un variant de la boucle puisque nous n'exécutons pas le corps de la boucle à
nouveau.


\levelThreeTitle{Lier la post-condition et l'invariant}


Supposons le programme spécifié suivant. Notre but est de prouver que le retour
de cette fonction est l'ancienne valeur de \CodeInline{a} à laquelle nous avons ajouté 10.



\CodeBlockInput{c}{add_ten-0.c}



Le calcul de plus faibles pré-conditions ne permet pas de sortir de la boucle des
informations qui ne font pas partie de l'invariant. Dans un programme comme :



\begin{CodeBlock}{c}
/*@
    ensures \result == \old(a) + 10;
*/
int add_ten(int a){
    ++a;
    ++a;
    ++a;
    //...
    return a;
}
\end{CodeBlock}



En remontant les instructions depuis la post-condition, on conserve toujours les
informations à propos de \CodeInline{a}. À l'inverse, comme mentionné plus tôt, en dehors
de la boucle WP, ne considérera que les informations fournies par notre
invariant. Par conséquent, notre fonction \CodeInline{add\_ten} ne peut pas être prouvée
en l'état : l'invariant ne mentionne rien à propos de \CodeInline{a}. Pour lier notre
post-condition à l'invariant, il faut ajouter une telle information. Par 
exemple :



\CodeBlockInput{c}{add_ten-1.c}



\begin{Information}
Ce besoin peut apparaître comme une contrainte très forte. Il ne l'est en fait pas
tant que cela. Il existe des analyses fortement automatiques capables de 
calculer les invariants de boucles. Par exemple, sans spécifications, une 
interprétation abstraite calculera assez facilement \CodeInline{0 <= i <= 10} et 
\CodeInline{\textbackslash{}old(a) <= a <= \textbackslash{}old(a)+10}. En revanche, il est souvent bien plus difficile
de calculer les relations qui existent entre des variables différentes qui 
évoluent dans le même programme, par exemple l'égalité mentionnée par notre 
invariant ajouté.
\end{Information}


\levelThreeTitle{Terminaison prématurée de boucle}


Un invariant de boucle doit être vrai chaque fois que la condition de la boucle est
évaluée. En fait, cela signifie aussi qu'elle doit être vraie avant chaque itération,
et après chaque itération \textbf{complète}. Illustrons cette idée importante avec un
exemple.


\CodeBlockInput{c}{first_loop-non-term-1.c}


Dans cette fonction, quand la boucle atteint l'indice 30, elle effectue une opération
\CodeInline{break} avant que la condition de la boucle soit à nouveau testée. 
L'invariant que nous avons écrit est bien sûr vérifié, mais nous pouvons en fait le
restreindre encore.



\CodeBlockInput{c}{first_loop-non-term-2.c}



Ici, nous pouvons voir que nous avons exclu 30 de la plage des valeurs de
\CodeInline{i} et la fonction est correctement vérifiée par WP. Cette propriété
est particulièrment intéressante car elle ne s'applique pas qu'un l'invariant. 
Aucune des propriétés de la boucle n'ont besoin d'être vérifiées pendant l'itération
qui mène au \CodeInline{break}. Par exemple, nous pouvons écrire ce code qui est
également vérifié :



\CodeBlockInput{c}{first_loop-non-term-3.c}


Nous pouvons voir que nous pouvons écrire la variable \CodeInline{h} même si elle
n'est pas listée dans la clause \CodeInline{loop assigns}, et que nous pouvons
donner la valeur 42 à \CodeInline{i} alors que l'invariant l'interdirait, et aussi
que nous pouvons rendre l'expression du variant négative. En fait, tout se passe
exactement comme si nous avions écrit :


\CodeBlockInput{c}{first_loop-non-term-4.c}



C'est un schéma très pratique. Il correspond à tout algorithme qui cherche, à l'aide
d'une boucle, une condition vérifiée par un élément particulier dans une structure 
de données et s'arrête qunad cet élément est trouvé afin d'effectuer certaines
opérations qui ne sont finalement pas vraiment des opérations de la boucle. D'un
point de vue vérification, cela nous permet de simplifier le contrat associé à une
boucle: nous savons que l'opération (potentiellement complexe) ) réalisée juste 
avant de sortir de la boucle ne nécessite pas d'être prise en compte dans
l'invariant.



\levelThreeTitle{Exercice}


\levelFourTitle{Invariant de boucle}

Écrire un invariant de boucle pour la boucle suivante et prouver qu'il est respecté
en utilisant la commande :


\begin{CodeBlock}{text}
frama-c -wp your-file.c
\end{CodeBlock}


\CodeBlockInput[2][6]{c}{ex-1-invariant.c}


Est ce que la propriété $-100 \leq x \leq 100$ est un invariant correct ?
Expliquez pourquoi.



\levelFourTitle{Loop variant}


Écrire un invariant et un variant corrects pour la boucle suivante et prouver
l'ensemble à l'aide de la commande :

\begin{CodeBlock}{text}
frama-c -wp your-file.c
\end{CodeBlock}


\CodeBlockInput[2][6]{c}{ex-2-variant.c}


Si le variant ne donne pas précisément le nombre d'itérations restantes, ajouter
une variable qui enregistre exactement le nombre d'itérations restantes et l'utiliser
comme variant. Il est possible qu'un invariant supplémentaire soit nécessaire.



\levelFourTitle{Loop assigns}


Écrire une clause \CodeInline{loop assigns} pour cette boucle, de manière à ce
que l'assertion ligne 9 soit prouvée ainsi que la clause \CodeInline{loop assigns}.
Ignorons les erreurs à l'exécution dans cet exercice.



\CodeBlockInput[2][9]{c}{ex-3-assigns.c}


Lors que la preuve réussit, supprimez la clause \CodeInline{loop assigns} et
trouvez un autre moyen d'assurer que l'assertion soit vérifiée en utilisant des
annotations différentes (notez que vous pouvez avoir besoin d'un label C dans le
code). Que déduisez vous à propos de la clause \CodeInline{loop assigns} ?


\levelFourTitle{Terminaison prématurée}


Écrire un contrat de boucle pour cette boucle qui permette de prouver les
assertions aux lignes 9 et 10 ainsi que le contrat de boucle.


\CodeBlockInput[1][11]{c}{ex-4-early.c}



\levelTwoTitle{Loops - Examples}

\levelThreeTitle{Écrire des annotations de boucle}


L'écriture des annotations de boucles demande un effort important pendant la
vérification. Il n'y a pas de manière parfaite de faire ce travail. En
particulier, trouver le bon invariant et la bonne manière de l'exprimer, est
principalement une question d'expérience. Néanmoins, quelques idées simples
peuvent aider, afin d'éviter certaines erreurs communes qui peuvent faire perdre
beaucoup de temps pendant le processus.


Avant toute chose, nous devons exprimer une clause \CodeInline{loop assigns}
correcte. Elle est généralement facile à écrire (il suffit de regarder les
instructions d'affectation et les appels de fonctions), et si elle est
incorrecte nous pourrions prouver des propriétés fausses, ce qui nous ferait
perdre du temps. Il est plutôt déconseillé de chercher à fournir des clauses
\CodeInline{loop assigns} trop précises. WP ne peut pas utiliser efficacement
des clauses comme \CodeInline{array[x .. y]} lorsque \CodeInline{x} ou
\CodeInline{y} sont elles-mêmes modifiées, on préférera donc des bornes qui
les incluent tout en étant constante pendant l'exécution de la boucle. Si nous
avons ensuite besoin d'informations plus précises pendant la preuve des
invariants, il sera plus facile de fournir des invariants supplémentaires pour
cela.


Ensuite, nous bornons les différentes variables qui sont assignées, en
particulier les indices. Ces invariants sont généralement faciles à deviner,
exprimer et vérifier. Nous mettons ces invariants au début de la liste des
clauses \CodeInline{loop invariant}, puisque, comme nous l'avons expliqué
dans les sections~\ref{l3:statements-loops-multi-inv}
et~\ref{l3:statements-loops-inv-kinds}, l'ordre des invariants est important
et ces propriétés très simples peuvent être propagées par WP dans les autres
invariants pour simplifier la condition de vérification à prouver.


Pour la plupart des boucles, exceptées celles qui reposent sur une condition
complexe, une fois que cette étape est réalisée, il est facile de fournir une
clause \CodeInline{loop variant}. Les variables que nous venons de borner sont
une bonne piste : nous avons juste à regarder la valeur qu'elles vont atteindre
en fin d'exécution. Par exemple, si dans la boucle, on a une variable
\CodeInline{i} qui va de 0 à \CodeInline{n}, \CodeInline{n - i} est un bon
candidat. Pour des boucles plus complexes, nous pouvons utiliser du code
fantôme (que l'on présente en section~\ref{l2:acsl-logic-definitions-ghost-code})
pour rendre explicite une mesure utilisable pour un variant de boucle.


Ensuite, nous ajoutons nos invariants « principaux » à la boucle, c'est-à-dire
ceux qui sont liés à la postcondition de la boucle (qui peut être aussi la
postcondition de la fonction). Pour cela, nous pouvons utiliser la postcondition
elle-même comme un guide. Si nous avons quelque chose comme
\CodeInline{ensures P(n);} et une boucle qui itère \CodeInline{i} de 0 à
\CodeInline{n}, il y a fort à parier que \CodeInline{loop invariant P(i);} soit
un bon invariant de boucle. Notons que dans certaines situations, c'est la
clause \CodeInline{assumes} d'un comportement qui se révèle être intéressante,
typiquement quand le résultat de l'exécution est une simple valeur qui dépend
de l'état en précondition, nous verrons cela dans un exemple plus tard. Ces
invariants doivent généralement être positionnés en fin de liste des invariants
de boucle.


Nous pouvons (optionnellement) avoir besoin de « glu » pour prouver les
invariants « principaux ». Par exemple, si nous avons volontairement fourni
une clause \CodeInline{loop assigns} imprécise, nous pourrions avoir besoin
d'un invariant de boucle pour expliquer que certaines parties de la mémoire
n'ont pas été modifiées, ou nous pourrions avoir besoin d'expliquer aux
prouveurs que, parce que certaines propriétés sont vraies, nous pouvons déduire
que l'invariant « principal » est vérifié, etc. Ces invariants doivent être
placés avant les invariants « principaux », mais après les invariants simples
qui bornent les variables. Ordonner ces invariants pourrait ne pas être si
direct, généralement nous passerons par un processus d'essai et erreur.


Dans la suite de cette section, nous illustrons cette approche avec quelques
exemples. Même si dans ce processus, il est fortement conseillé de lancer la
preuve entre chaque étape, nous n'irons généralement pas jusqu'à ce niveau de
détail dans les exemples futurs.


\levelThreeTitle{Exemple avec un tableau en lecture seule}
\label{l3:statements-loops-examples-ro}


S'il y a une structure de données que nous traitons avec les boucles, c'est bien
le tableau. C'est une bonne base d'exemples pour les boucles, car ils permettent
rapidement de présenter des invariants intéressants et surtout, ils nous
permettront d'introduire des constructions très importantes d'ACSL.


Prenons par exemple la fonction qui cherche une valeur dans un tableau. Pour le
moment, concentrons nous sur le contrat de la fonction :


\CodeBlockInput[5]{c}{search-1.c}


Cet exemple est suffisamment fourni pour introduire des notations importantes.


D'abord, comme nous l'avons déjà mentionné, le prédicat
\CodeInline{\textbackslash{}valid\_read} (de même que
\CodeInline{\textbackslash{}valid}) nous permet de spécifier non seulement la
validité d'une adresse en lecture, mais également celle de tout un ensemble
d'adresses contiguës. C'est la notation que nous avons utilisée dans cette
expression :


\begin{CodeBlock}{c}
//@ requires \valid_read(a + (0 .. length-1));
\end{CodeBlock}


Cette précondition nous atteste que les adresses \CodeInline{a+0},
\CodeInline{a+1}, \ldots{}, \CodeInline{a+length-1} sont valides en lecture.


Nous avons également introduit deux notations qui vont nous être très utiles, à
savoir \CodeInline{\textbackslash{}forall} ($\forall$) et \CodeInline{\textbackslash{}exists} ($\exists$), les
quantificateurs de la logique. Le premier nous servant à annoncer que pour tout
élément, la propriété suivante est vraie. Le second pour annoncer qu'il existe
un élément tel que la propriété est vraie. Si nous commentons les deux lignes en
questions, nous pouvons les lire de cette façon :


\begin{CodeBlock}{c}
/*@
//pour tout "off" de type "size_t", tel que SI "off" est compris entre 0 et "length"
//                                 ALORS la case "off" de "a" est différente de "element"
\forall size_t off ; 0 <= off < length ==> a[off] != element;

//il existe "off" de type "size_t", tel que "off" soit compris entre 0 et "length"
//                                 ET que la case "off" de "a" vaille "element"
\exists size_t off ; 0 <= off < length && a[off] == element;
*/
\end{CodeBlock}


Si nous devions résumer leur utilisation, nous pourrions dire que sur un certain
ensemble d'éléments, une propriété est vraie, soit à propos d'au moins l'un
d'eux, soit à propos de la totalité d'entre eux. Un schéma qui reviendra
typiquement dans ce cas est que nous restreindrons cet ensemble à travers une
première propriété (ici : \CodeInline{0 <= off < length}) puis nous voudrons prouver la
propriété réelle qui nous intéresse à propos d'eux. \textbf{Mais il y a une
différence fondamentale entre l'usage de \CodeInline{exists} et celui de \CodeInline{forall}}.


Avec \CodeInline{\textbackslash{}forall type a ; p(a) ==> q(a)}, la restriction
(\CodeInline{p}) est suivie par une implication. Pour tout élément, s'il
satisfait une première propriété (\CodeInline{p}), alors il doit vérifier la
seconde propriété \CodeInline{q}. Si nous mettions un ET comme pour le « il
existe » (que nous expliquerons ensuite), cela voudrait dire que nous voulons
que tout élément vérifie à la fois les deux propriétés. Parfois, cela peut être
ce que nous voulons exprimer, mais cela ne correspond alors plus à l'idée de
restreindre un ensemble dont nous voulons montrer une propriété particulière.


Avec \CodeInline{\textbackslash{}exists type a ; p(a) \&\& q(a)}, la restriction
(\CodeInline{p}) est suivie par une conjonction, nous voulons qu'il existe un
élément tel que cet élément est dans un certain état (défini par \CodeInline{p}),
tout en satisfaisant l'autre propriété \CodeInline{q}. Si nous mettions une
implication comme pour le « pour tout », alors une telle expression devient
toujours vraie à moins que \CodeInline{p} soit une tautologie ! Pourquoi ?
Existe-t-il « a » tel que p(a) implique q(a) ? Prenons n'importe quel « a » tel
que p(a) est faux, l'implication devient vraie.


Notons que dans cet exemple, la clause \CodeInline{assume} du comportement
\CodeInline{in} est exactement la négation de la clause \CodeInline{assumes}
du comportement \CodeInline{notin}, c'est la raison pour laquelle les clauses
\CodeInline{disjoint} et \CodeInline{complete} sont prouvées, en fait nous
aurions pu l'exprimer comme suit :


\begin{CodeBlock}{c}
  /*@ ...
    behavior in:
      assumes !(\forall size_t off ; 0 <= off < length ==> array[off] != element);
    ...
  */
\end{CodeBlock}


Parlons des annotations de la boucle. La première étape est d'ajouter la clause
\CodeInline{loop assigns}. Ici, elle est simple : la boucle ne modifie que la
variable \CodeInline{i}. La valeur de cette variable doit donc être bornée, elle
va de \CodeInline{0} à \CodeInline{length}, c'est notre premier invariant :
\CodeInline{0 <= i <= length}. Ceci nous donne également le variant de boucle :
\CodeInline{length - i}. Maintenant, nous pouvons fournir notre invariant
« principal ». Ici, il est relié à la clause \CodeInline{assumes}, et pas à la
clause \CodeInline{ensures}. En particulier, la partie intéressante de la
fonction est qu'à moins que l'on rencontre l'élément recherché, il ne se trouve
pas dans le tableau, nous exploitons cela en partant de la clause
\CodeInline{assumes} du comportement \CodeInline{notin} :
\begin{CodeBlock}{c}
  //@ \forall size_t off ; 0 <= off < length ==> array[off] != element;
\end{CodeBlock}
La variable qui atteint \CodeInline{length} à la fin de la boucle est
\CodeInline{i}, donc :
\begin{CodeBlock}{c}
  //@ loop invariant \forall size_t off ; 0 <= off < i ==> array[off] != element;
\end{CodeBlock}
est certainement un bon candidat. Cela nous amène aux annotations de boucle
suivantes :


\CodeBlockInput[19]{c}{search-0.c}


Et effectivement, notre invariant de boucle final définit l'action de notre
boucle, elle indique à WP ce que la boucle fera (ou apprendra dans le cas
présent) tout au long de son exécution. Ici en l'occurrence, cette formule nous
dit qu'à chaque tour, nous savons que pour toute case entre 0 et la prochaine
que nous allons visiter (\CodeInline{i} exclue), elle stocke une valeur
différente de l'élément recherché.


Le but de WP associé à la préservation de cet invariant est un peu compliqué, il
n'est pour nous pas très intéressant de se pencher dessus. En revanche, la
preuve de l'établissement de cet invariant est intéressante :


\image{trivial}


Nous constatons que cette propriété, pourtant complexe, est prouvée par
Qed sans aucun problème. Si nous regardons sur quelles parties du programme la
preuve se base, nous voyons l'instruction \CodeInline{i = 0} surlignée, et c'est
bien la dernière instruction que nous effectuons sur \CodeInline{i} avant de commencer
la boucle. Et donc effectivement, si nous faisons le remplacement dans la formule
de l'invariant :


\begin{CodeBlock}{c}
//@ loop invariant \forall size_t j; 0 <= j < 0 ==> array[j] != element;
\end{CodeBlock}


« Pour tout j, supérieur ou égal à 0 et inférieur strict à 0 », cette partie est
nécessairement fausse. Notre implication est donc nécessairement vraie.


\levelThreeTitle{Exemples avec tableaux mutables}


Nous allons voir deux exemples avec la manipulation de tableaux en mutation.
L'un avec une modification totale, l'autre en modification sélective.



\levelFourTitle{Remise à zéro}


Regardons la fonction effectuant la remise à zéro d'un tableau.



\CodeBlockInput{c}{reset.c}


Cette fois, la boucle modifie le contenu du tableau, donc nous devons fournir
cette information dans la clause \CodeInline{loop assigns}. Notons que nous
pouvons utiliser la notation \CodeInline{n .. m} pour indiquer quelle partie du
tableau a été modifiée. De plus, nous ne disons pas que la boucle assigne le
contenu depuis \CodeInline{0} jusqu'à \CodeInline{i-1} (comme \CodeInline{i}
est modifiée, WP ne peut pas exploiter cette écriture) mais depuis \CodeInline{0}
jusqu'à \CodeInline{length-1}, c'est moins précis, mais cela peut être utilisé
par WP en dehors de la boucle. Finalement, cette fois, nous relions directement
l'invariant à la postcondition, le but de la fonction est de réinitialiser le
tableau de 0 jusqu'à \CodeInline{length}, à une itération donnée, la boucle l'a
déjà fait entre 0 et \CodeInline{i}.


\levelFourTitle{Chercher et remplacer}
\label{l4:statements-loops-ex-search-and-replace}


Le dernier exemple qui nous intéresse est l'algorithme « chercher et remplacer ».
C'est un algorithme qui modifie sélectivement des valeurs dans une
certaine plage d'adresses. Il est toujours un peu difficile de guider l'outil
dans ce genre de cas car, d'une part, nous devons garder « en mémoire » ce qui est modifié
et ce qui ne l'est pas et, d'autre part, parce que l'induction repose sur ce fait.


À titre d'exemple, la première spécification et boucle annotée, que nous pouvons
réaliser pour cette fonction ressemblerait à ceci (ce qui suit sensiblement le
même processus que dans l'exemple précédent) :


\CodeBlockInput{c}{search_and_replace-0.c}


Nous utilisons la fonction logique \CodeInline{\textbackslash{}at(v, Label)} qui
nous donne la valeur de la variable \CodeInline{v} au point de programme
\CodeInline{Label}. Si nous regardons l'utilisation qui en est faite ici, nous
voyons que dans l'invariant de boucle, nous cherchons à établir une relation
entre les anciennes valeurs du tableau et leurs potentielles nouvelles valeurs :


\begin{CodeBlock}{c}
/*@
  loop invariant \forall size_t j; 0 <= j < i && \at(array[j], Pre) == old
                   ==> array[j] == new;
  loop invariant \forall size_t j; 0 <= j < i && \at(array[j], Pre) != old
                   ==> array[j] == \at(array[j], Pre);
*/
\end{CodeBlock}


Pour tout élément que nous avons visité, s'il valait la valeur qui doit être
remplacée, alors il vaut la nouvelle valeur, sinon il n'a pas changé. Alors que
nous nous reposions sur la clause \CodeInline{assigns} pour ce genre de propriété
dans les exemples précédents, ici nous ne pouvons pas le faire. Même si ACSL nous
permettrait d'exprimer cette propriété de manière très précise, WP ne pourrait pas
vraiment en tirer parti, dû à la manière dont cette clause est traitée. Nous devons
donc utiliser un invariant et conserver une approximation des positions mémoire
affectées.


En fait, si nous essayons de prouver l'invariant, WP n'y parvient pas, parce que
la clause \CodeInline{assigns} n'est pas assez précise. Dans cette situation,
nous fournissons un invariant supplémentaire pour détailler dans la plage modifiée
quelles sont les positions en mémoire qui n'ont pas encore été modifiées par la
boucle à une itération donnée :


\begin{CodeBlock}{c}
for(size_t i = 0; i < length; ++i){
    //@assert array[i] == \at(array[i], Pre); // échec de preuve
    if(array[i] == old) array[i] = new;
}
\end{CodeBlock}


Nous pouvons donc ajouter cette information comme invariant :


\CodeBlockInput[13][26]{c}{search_and_replace-1.c}


Et cette fois, la preuve passera.


\levelThreeTitle{Exercices}


Pour tous ces exercices, utiliser la commande suivante pour démarrer la vérification :

\begin{CodeBlock}{bash}
frama-c-gui -wp -wp-rte -warn-unsigned-overflow -warn-unsigned-downcast your-file.c
\end{CodeBlock}


\levelFourTitle{Fonctions sans modification : For all, Exists, ...}


Actuellement, les pointeurs de fonction ne sont pas directement supportés par WP.
Considérons que nous avons une fonction :


\CodeBlockInput[7][13]{c}{ex-1-forall-exists.c}


Écrire un corps (au choix) pour cette fonction et un contrat l'accompagnant.
Ensuite, écrire les fonctions suivantes avec leurs contrats pour prouver leur
correction. Notons qu'il n'est pas possible d'utiliser une fonction C dans un
contrat, la propriété que choisie pour la fonction \CodeInline{pred}
devra donc être inlinée dans la spécification des différentes fonctions.


\begin{itemize}
\item \CodeInline{forall\_pred} retourne vrai si et seulement si \CodeInline{pred}
  est vraie pour tous les éléments ;
\item \CodeInline{exists\_pred} retourne vrai si et seulement si \CodeInline{pred}
  est vraie pour au moins un élément ;
\item \CodeInline{none\_pred} retourne vrai si et seulement si \CodeInline{pred}
  est fausse pour tous les éléments ;
\item \CodeInline{some\_not\_pred} retourne vrai si et seulement si \CodeInline{pred}
  est fausse pour au moins un élément.
\end{itemize}


Les deux dernières fonctions devraient être écrites en appelant seulement les deux
premières.


\levelFourTitle{Fonction sans modification : Égalité de plages de valeurs}


Écrire, spécifier et prouver la fonction \CodeInline{equal} qui retourne vrai
si et seulement si deux plages de valeurs sont égales. Écrire, en utilisant la
fonction \CodeInline{equal}, le code de \CodeInline{different} qui retourne vrai
si et seulement si deux plages de valeurs sont différentes, votre postcondition
devrait contenir un quantificateur existentiel.


\CodeBlockInput[7][13]{c}{ex-2-equal.c}


\levelFourTitle{Fonction sans modification : recherche dichotomique}
\label{l4:statements-loops-ex-bsearch}


La fonction suivante cherche la position d'une valeur fournie en entrée dans
un tableau, en supposant que le tableau est trié. D'abord, considérons que la
longueur du tableau est fournie en tant qu'\CodeInline{int} et utilisons des valeurs de ce
même type pour gérer les indices. Nous pouvons noter qu'il y a deux comportements
à cette fonction : soit la valeur existe dans le tableau (et le résultat est
l'indice de cette valeur) ou pas (et le résultat est -1).


\CodeBlockInput[5]{c}{ex-3-binary-search.c}


Cette fonction est un petit peu complexe à prouver, voici quelques conseils
pour en venir à bout. D'abord, la longueur de la fonction est reçue en utilisant
un type \CodeInline{int}, donc nous devons poser une restriction sur cette longueur en
précondition pour qu'elle soit cohérente. Ensuite, l'un des invariants de la
boucle devrait indiquer les bornes des valeurs \CodeInline{low} et
\CodeInline{up}, mais nous pouvons noter que pour chacune d'elles, l'une des
bornes n'est pas nécessaire. Finalement, la seconde propriété invariante
devrait indiquer que si l'un des indices du tableau correspond à la valeur
recherchée, alors cet indice devrait être correctement borné.


\textbf{Plus dur :} Modifier cette fonction de façon à recevoir \CodeInline{len}
comme un \CodeInline{size\_t}. Il faut modifier légèrement l'algorithme et
la spécification/les invariants. Conseil : s'arranger pour que \CodeInline{up}
soit une borne exclue de la recherche.


\levelFourTitle{Fonction avec modification : addition de vecteurs}


Écrire, spécifier et prouver la fonction qui ajoute deux vecteurs dans un
troisième. Fixer des contraintes arbitraires sur les valeurs d'entrée pour
gérer le débordement des entiers. Considérer que le vecteur est résultant est
spatialement séparé des vecteurs d'entrée. En revanche, le même vecteur devrait
pouvoir être utilisé pour les deux vecteurs d'entrée.


\CodeBlockInput[7][9]{c}{ex-4-add-vectors.c}


\levelFourTitle{Fonction avec modification : inverse}


Écrire, spécifier et prouver la fonction qui inverse un vecteur en place.
Prendre garde à la partie non modifiée du vecteur à une itération donnée de la
boucle. Utiliser la fonction \CodeInline{swap} précédemment prouvée.


\CodeBlockInput[7][11]{c}{ex-5-reverse.c}


\levelFourTitle{Fonction avec modification : copie}


Écrire, spécifier et prouver la fonction \CodeInline{copy} qui copie une plage
de valeur dans un autre tableau, en commençant pas la première cellule du
tableau. Considérer (et spécifier) d'abord que les deux plages sont entièrement
séparées.


\CodeBlockInput[7][9]{c}{ex-6-copy.c}


\textbf{Plus dur :} Les vraies fonctions \CodeInline{copy} et
\CodeInline{copy\_backward}.


En fait, une séparation aussi forte n'est pas nécessaire. Pour faire une copie
en partant du début, la précondition réelle doit simplement garantir que si les
deux plages se chevauchent en mémoire, le début de la destination ne doit pas être
dans la plage source :


\begin{CodeBlock}{c}
//@ requires \separated(&src[0 .. len-1], dst) ;
\end{CodeBlock}


Essentiellement, en copiant des éléments dans cet ordre, nous pouvons les
décaler depuis la fin d'une plage vers le début. En revanche, cela signifie que
nous devons être plus précis dans notre contrat : nous ne garantissons plus une
égalité avec le tableau source, mais avec les \emph{anciennes} valeurs du tableau
source. Nous devons également être plus précis dans notre invariant, d'abord en
spécifiant aussi la relation avec l'état précédent de la mémoire, et ensuite en
ajoutant un invariant qui nous dit que le tableau source n'est pas modifié entre
le \CodeInline{i}$^{ème}$ élément visité et le dernier.


Finalement, il est aussi possible d'écrire une fonction qui copie les éléments de
la fin vers le début. Dans ce cas, à nouveau, les plages de valeurs peuvent se
chevaucher, mais la condition n'est pas exactement la même. Écrire, spécifier et
prouver la fonction \CodeInline{copy\_backward} qui copie les éléments dans le
sens inverse.



\horizontalLine



In this part, we have seen how assignment and control structure are translated
to a logic view of our program. We have spent quite a lot of time on loops
because they represent the main difficulty we have to face when we want to
specify and prove a program by deductive verification. The loop annotations
allow us to express as precisely as possible their behavior.



In the next part of this tutorial, we will see more precisely the logic
constructs provided by ACSL. They are important because they give us a way to
express write more abstract specification, that are easier to understand and
to prove.

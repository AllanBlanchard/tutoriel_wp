We have seen that program proof allows verifying two main aspects of program
correctness: first that programs do not contain runtime errors, second that
programs correctly implement their specification. However, it is sometimes hard
to guarantee the latter, and the former is already an interesting step for
program correctness.

Indeed, runtime errors often cause the presence of so-called ``undefined
behaviors'' in C programs. These undefined behaviors are often vectors of
security breaches and thus, guaranteeing their absence already protects us of a
lot of attack vectors. The absence of runtime errors can be verified thanks to
an approach called minimal contracts.


\levelThreeTitle{Principle}


The minimal contracts approach is guided by the RTE plugin. Basically, the idea
is to generate the assertions related to absence of runtime errors for all the
functions of a module or a project and to write the minimal set of (correct)
contracts that are sufficient to prove that the runtime errors cannot happen.
Most of the time, we need far fewer lines of specification that what is usually
required to prove the functional correctness of the program.


Let us take a simple example with the absolute value function.


\CodeBlockInput[5]{c}{abs-1.c}


Here, we can generate the assertions required to prove the absence of runtime
errors, which generate this program:


\CodeBlockInput[5]{c}{abs-2.c}


Thus, we only need to specify as a requirement that the value of \CodeInline{x}
must be greater than \CodeInline{INT\_MIN}:


\CodeBlockInput[3]{c}{abs-3.c}


This condition is enough to prove that no runtime error can happen in the
function.


As we will see later, the function is however generally used in a particular
context. So this contract will likely not be enough. For
example, we often have global variables in our program and here we do not
specify what is assigned by the function. Most of the time the ``assigns''
clause cannot be ignored (which is expected in a language where everything is
mutable by default). Moreover, if one take the absolute value of an integer,
it is surely because they need a positive value. In reality, the minimal
contract of the absolute value function is more likely the following:


\CodeBlockInput[3]{c}{abs-4.c}


But this addition should only be guided by the verification of the context(s)
where the function is used once we have proved the absence of runtime errors in
the function itself.



\levelThreeTitle{Example: the search function}


Now that we have the principle in mind, let us work on more complex examples, in
particular with an example that involves a loop.


\CodeBlockInput[5]{c}{search-1.c}


Once we have generated the assertions related to runtime errors, we have the
following program:


\CodeBlockInput[5]{c}{search-2.c}


We have to prove that any cell visited by the program can be read, thus we need
to express as a precondition that the array is
\CodeInline{\textbackslash{}valid\_read} on the corresponding range of indices.
However, this is not enough to complete the proof since we have a loop in this
program, so we have to provide a suitable invariant. We also probably want to
prove that the loop terminates.


Thus, we get the following minimally specified function:


\CodeBlockInput[1][15]{c}{search-3.c}


This contract can be compared with the contract provided for the search
function in section~\ref{l3:statements-loops-examples-ro}, and we can see that it is
much more simple.


Now let us imagine that the function is used in the following program:


\CodeBlockInput[17][22]{c}{search-3.c}


We again have to provide a suitable contract for the function, again by having
a look at the assertion that RTE asks us to verify:


\CodeBlockInput[30][39]{c}{search-4.c}


Thus, we have to verify that:
\begin{itemize}
\item the pointer we received from \CodeInline{search} is valid,
\item \CodeInline{*p + 1} does not overflow,
\item we respect the contract of the \CodeInline{search} function.
\end{itemize}


In addition to the contract of \CodeInline{foo}, we have to provide some more
information in the contract of \CodeInline{search}. Indeed, we will not be able
to prove that the pointer is valid when it is not null if the function does not
guarantee that the pointer is in the range of our array in this case.
Furthermore, we will not be able to prove that \CodeInline{*p} is less than
\CodeInline{INT\_MAX} if the function can modify it.


This leads us to this complete annotated program:


\CodeBlockInput{c}{search-5.c}


\levelThreeTitle{Advantages and limitations}


The evident advantage of this approach is the fact that it can guarantee the
absence of runtime errors in any function of a module or a program in (relative)
isolation of the other functions. Furthermore, this absence of runtime errors
is guaranteed for any use of the function as long as the precondition is
verified when it is called. That allows to gain some confidence into a system
with a relatively low-cost approach.



However, as we have seen, when we use a function it can change the knowledge
we need about its behavior, requiring to make the contract richer and richer.
Thus, we can progressively reach a state where we basically proved the functional
correctness of the function.



Furthermore, even proving the absence of runtime errors is sometimes not trivial
as we have for example seen with functions like the factorial or the sum of
N integers, that require to give quite a lot of information to SMT solvers, in
order to prove that we cannot meet an integer overflow.



Finally, sometimes the minimal contracts of a function or a module basically
is the full functional specification, and thus formally verifying the absence
of runtime errors requires a full functional verification. This is commonly the
case when we have to deal with complex data structures where the properties
that are required for the absence of runtime errors depend on the functional
behavior of the function, maintaining some non-trivial invariant about the
data structure.



\levelThreeTitle{Exercises}


\levelFourTitle{Some simple examples}


Prove the absence of runtime errors in the following program using a minimal
contracts approach:


\CodeBlockInput[5]{c}{ex-1-simple.c}


\levelFourTitle{Reverse}


Prove the absence of runtime errors for the following \CodeInline{reverse}
function and its dependency using a minimal contracts approach. Note that the
\CodeInline{swap} function should also be specified with minimal contracts only.
Do not forget to add the options \CodeInline{-warn-unsigned-overflow} and
\CodeInline{-warn-unsigned-downcast}.


\CodeBlockInput[5]{c}{ex-2-reverse.c}



\levelFourTitle{Binary search}


Prove the absence of runtime errors for the following \CodeInline{bsearch}
function using a minimal contracts approach. Do not forget to add the options
\CodeInline{-warn-unsigned-overflow} and \CodeInline{-warn-unsigned-downcast}.


\CodeBlockInput[5]{c}{ex-3-binary-search.c}


\levelFourTitle{Sort}


Prove the absence of runtime errors for the following \CodeInline{sort}
function and its dependencies using a minimal contracts approach. Note that
these dependencies should also be specified with minimal contracts only.
Do not forget to add the options \CodeInline{-warn-unsigned-overflow} and
\CodeInline{-warn-unsigned-downcast}.


\CodeBlockInput[5]{c}{ex-4-sort.c}

As we will try to prove more complex properties, particularly when
programs involve loops, there will be a part of ``trial and error'' in
order to understand what the provers miss to establish the proof.

It can miss hypotheses. In this case, we can try to add assertions to
guide the prover. With some experience, we can read the content of the
proof obligations or try to perform the proof with the Coq interactive
prover to see whether the proof seems to be possible. Sometimes, the
prover just needs more time, in such a case, we can (sometimes a lot)
augment the timeout value. Of course, the property can be too hard for
the prover, and in this case, we will have to write the proof ourselves
with an interactive prover.

Finally, the implementation can be indeed incorrect, and in this case we
have to fix it. Here, we will use test and not proof, because a test
allows us to prove the presence of a bug and to analyze this bug.
